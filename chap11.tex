\chapter{}
„Vážený pane velvyslanče,
naše poslední setkání se odehrálo ve velmi nepřátelském duchu. Ne že bych mezitím k Vám nebo Vámi reprezentované totalitě pocítil sympatie, ale zjistil jsem, že Šanghajský lid je velmi ohrožený a toužím po tom mu pomoct. Rád bych se s Vámi setkal ještě jednou a tentokrát už jednal trochu konstruktivně.“

Asi tak nějak zněl email, který jsem po prostudování zprávy mnou podporovaných výzkumníků obratem poslal na Čínskou ambasádu v Lusace.

Velvyslanec odpověděl rychle: „Bránit Vám nebudeme, vyřešíme to za dva týdny ve středu v této  restauraci. Předem Vás ale upozorňuji, že na čínském lidu, jehož součástí jsou i agenti, které jsme za Vámi poslali, záleží i nám a že bychom velmi rádi věděli, zda jsou ještě naživu. Pokud ano, budu s Vámi popříští středu řešit i jejich návrat.“

To mi vyhovovalo, neměl jsem naprosto žádnou motivaci si ve sklepě nechávat sadistického a vzteklého vrahouna. Pak tu byl ještě Jie, ale v jeho případě stačilo říct velvyslanci pravdu. Jie zkrátka pochopil, že Čína si jeho energii nezaslouží, a tak se přidal ke mně. Velmi, bohužel velmi nemile, mě ovšem překvapilo zjištění, že můj syn to celé vidí dost jinak.

„Uvažuji nad svým návratem,“ přiznal se mi.

„Co tě to napadá. V Číně budeš moct leda chytat zločince, kteří navíc budou povětšinou mnohem lepší než ti příšerní prospěcháři, kteří tě zaměstnávají. Tady se mnou můžeš tvořit, tvořit velké a dobré věci.“

„To v Číně taky.“ Jieho v jeho rozhodnutí podporovala bohyně Mlking a já tak neměl šanci cokoliv zvrátit.

„Jakto?“

„Promluv si s Fetuem, ten mě o tom přesvědčil,“ poslechl jsem a okamžitě vyrazil za svým adoptovaným synem.

„Prý jsi ponoukal Jieho, aby se vrátil do Číny. Jak tě to napadlo?“

„Není to tak dávno, co jsem podnikl dovolenou na jih, do míst, ze kterých přišel kmen mého tehdy jen genetického otce a objevil něco, pro co jej již nazývám také otcem duchovním. 

Začnu tím, že můj a jeho kmen kdysi býval velmi vlivný a úspěšný. Všichni, kteří nad tím přemýšleli, usoudili, že to bylo díky jedné specifické strategii – podobně jako jiné negramotné kmeny si i on uchovával svá moudra a to jak o tom, co je správné a co ne, tak o tom, jak funguje svět  – co je nebezpečné, a co ne, v ústně tradovaných mýtech. Kmeny s touto ústní tradicí bývají většinou velmi konzervativní, ne tak kmen mých předků. V dobách míru a prosperity se mýty sice těšily posvátnosti, ale jakmile se kmen dostal do problémů a rada starších usoudila, že pokud svět funguje tak, jak jejich společenství ve svých mýtech traduje, je kmen odsouzen k zániku. A v těchto vyhrocených chvílích rada zbavila svou slovesnost posvátnosti a vyhlásila, že nastal čas „tance za milost“, tedy doba, v níž mohli lidé jednat proti všem dosavadním zvyklostem. Když Jihoafričtí vojáci přepadli mou vesnici, opustil můj otec svou rodinu, což by bylo v mírových dobách nepředstavitelné, a vyrazil varovat severnější vesnice. Kdyby svůj čin býval přežil a získal nám svou zprávou přízeň severnějších kmenů, vznikl by o něm mýtus oslavující posly, kteří v dobách šířící se krize opustí společenství vlastní, aby varovali ta cizí. Zkrátka rada starších v těžkých časech vyzvala své soukmenovce, aby jednali nikoliv podle dosavadních hodnot, ale podle toho, co přinese prospěch kmenu, a z jednání, které se osvědčilo, vyrobila hodnoty nové (staré přitom nezavrhla, tedy jen tehdy když se příčily těm novým). 

Ve světle objevů tebou placeného vědeckého týmu se domnívám, že v průšvihu je nejen Šanghaj, nýbrž celý svět. Ta archea se nachází v korálových útesech po celém světě a pokud bude okyselování moří pokračovat, můžou způsobit obrovskou katastrofu. Krom toho se jedná o dobrou ukázku toho, že naprosto nevíme, k čemu může klimatická změna vést. Ano, po celém světě se objevují problémy se záplavami, požáry, hurikány a suchem, ale to zdaleka nemusí být všechno. Opouštíme věk prosperity a vstupujeme do věku katastrof.

Přemýšlel jsem, co s tím dělat a došel k závěru, že je čas spustit strategii mého rodného kmene. Buďme radou starších, která vytvoří prostor pro vznik nových globálních hodnot. Ze všeho nejdřív však bude potřeba vychovat lidi k tomu, aby podle nějakých vůbec zvládli žít. K tomu by mělo posloužit „očkování“ Šanghajanů, kteří mají šanci stát se inspirací pro celý svět. Jie to ví a do Číny se chystá právě proto, aby dohlédl, že naplňují svůj potenciál. Já mám v plánu volit standardní cesty, jakými se dnes ideje předávají - začal jsem psát blog, natáčet videa, nahrávat podesty… prostě se chystám co nejvíc lidí přesvědčit o tom, že přichází čas jednat nikoliv podle toho, co bylo dosud považováno za dobré a správné, ale podle toho, co ochrání planetu a svobodu na ní.“

Smutně jsem se usmál. „Nu co, už jste dospělí, takže Jiemu bránit nebudu. Ostatně, co se týče té kmenové strategie, sám jsem se s ní setkal a myslím, že své vyznavače vede možná k vlastní zkáze, ale určitě také k prosperitě toho, co milují. Světu přijde jedině vhod, když se pokusíme překonat současný poživačný individualismus západu, slepou poslušnost dálného východu a lenivý šovinismus toho zbytku.“

Když byl za další dva týdny Gang vytažen ze své provizorní cely v mém sklepě, aby se umyl a převlékl, než bude vyslýchán, měl možnost po cestě vidět jinou provizorní celu a v ní Jieho, jednorázově ušpiněného, neoholeného a oblečeného do volných šatů, aby vypadal vyhuble a zuboženě.

Již o několik hodin později tedy ochotně potvrdil, že Jie byl stejně jako on vězněn ve sklepeních mého domu. Tento taktický krok však představoval pouhou čárkou ve velkém souvětí mého jednání s Číňany.
\vspace{0.75cm}

Ve stanovený den a stanovenou hodinu jsem přijel se stanovenými lidmi, svými syny a Gangem, na stanovení místo. Velmi mě překvapilo, když jsem vešel do restaurace a zjistil, že zdaleka nebudu jednat jen s velvyslancem.

S ambasadorem totiž u stolu seděl člověk ještě mnohem významnější – čerstvě zvolený (samozřejmě ne lidem, ale špičkami komunistické strany) prezident Čínské lidové republiky, Si Ťin-pching. 

„Jsem tu inkognito, ale předem vás upozorňuji, že jakýkoliv pokus o únos či jiné násilí by se vám velmi vymstil, v tomhle podniku nás bedlivě sleduje přes deset agentů,“ prohlásil stroze místo uvítání. Rozhlédl jsem se a musel uznat, že bych tolik Číňanů v zambijské restauraci rozhodně nečekal.

„Nuže tedy, prý jste si všiml, že Šanghajanům hrozí nějaké vážné nebezpečí.“
Přitakal jsem a pověděl mu, co výzkumnici objevili.

 „Takže do teplých moří se dostává smrtelný jed. Nejohroženějším místem je Šanghaj a vy s tím chcete něco dělat,“ shrnul prezident má slova, když jsem domluvil.
 
„Přesně tak.“

„Popište mi svůj plán prosím podrobněji. Proti jedu si člověk protilátky jen tak nevytvoří, takže bílé krvinky k boji s ním nějakou vakcínou vycvičíte jen těžko a pokud chcete lidem dávat slabé a pak silnější a silnější dávky rozložených chavezií, jak zajistíme, že to nepřeženete a neotrávíte půlku města, které je klíčové pro zemi, s níž celý život bojujete?“ otázal se mě.

„Žijeme v jednadvacátém století, už jsme schopni levně číst i upravovat lidské DNA. Tyto nové možnosti se chystám využít i při ochraně Šanghaje.

Týmy vědců, které za mé peníze onu ‚hrozbu zaplavených měst‘ tajně zkoumaly, objevily také protein, který, pokud je v krvi, zvládne mnoho jedu vstřebat. Mým plánem je implantovat co největšímu množství Šanghajanů kus kůže s genetickým rozšířením, díky němuž bude tento protein produkovat.“

Tato slova prezidenta Ťing-pchinga zvedla od stolu.

„Zbláznil jste se? Myslíte, že zrovna vás nechám hrabat se v osmi miliónech lidských bytostí?“ 

„Otec by takovou pravomoc opravdu dostávat neměl, ale já bych mohl,“ ozval se Jie.
„Kdo jste?“ zpražil ho prezident pohledem.

„Bývalý agent, před čtyřmi lety jsem byl poslán, abych svého otce zabil. To se mi nepovedlo a tak jsem strávil tři roky ve sklepě, než mě na nějaký čas začal pouštět ven, abych pomohl s přípravou toho očkovacího genu. Nyní bych se rád vrátil do Číny, podrobil komunistické moci a ručil za to, že je ona látka spolehlivá.“

„Mluví pravdu, má cela se nacházela hned vedle jeho,“ přitakal Gang.

Tato slova prezidenta vedla k delšímu zamyšlení. Asi minutu mlčel a pak nám stanovil své podmínky: „Je to absurdní, ale za možnost nám pomáhat budete muset zaplatit. Zaprvé vrácením agentů, které jste zadržel a zadruhé tím, že opustíte telefony uživatelů Huawei.“

Mohlo se zdát, že Si Ťin-pching tehdy opouštěl restauraci jako absolutní vítěz. Na čínském lidu mu záleželo dozajista méně než mě a mohl si tedy klást podmínky. To ovšem neměl tušení, co Šanghajany čeká…

S cloudovými službami jsem musel skončit, ale nejsladší ovoce bylo z nabouraných zařízení sklizeno již dříve. Nebýt výpočetní kapacity, kterou nám poskytly, bychom nikdy gen, který později změnil svět, nevyrobili.

Takto, využívajíce zařízení Huawei po celém světě, jsme ovšem mohli detailně simulovat jeho začlenění do genetické informace očkovaných lidí, následnou produkci proteinu a likvidaci rozložených archea.

Myslím, že to byl první od nuly naprogramovaný gen a proto pro něj platilo, co prý pro téměř každé jiné programování – nejvíc práce si vyžádalo odstraňování chyb, a ty bychom bez simulátoru nikdy neodhalili.

 Rozhodně byly i jednodušší cesty jak populaci naočkovat. Vyrobit protijed, který by se dával pouze otráveným, by nás stálo mnohem méně práce. Zvolená cesta však měla ještě jednu velkou výhodu. Protein se totiž produkoval neustále a již brzy ho brzy začalo v těle přebývat. Jakmile přebytky přesáhly určitou mez, začal se protein ukládat na receptorech neuronů a dělat zmatky při přenášení vzruchů. Jinými slovy, několik málo let po očkování začali pacienti zažívat stavy srovnatelné s těmi, které následují požití LSD či jiných halucinogenů.
 
A právě zfetovaná populace byla podle Fetua který chtěl vychovat lidi otevřené novým hodnotám, tou nejlepší možnou. Máloco člověka otevře novým perspektivám a dodá mu odvahu se vydat jejich směrem tolik, jako psychedelická droga. 

Předpokládali jsme, že se díky našemu očkování časem v Šanghaji utvoří nějaká nábožná a odvážná atmosféra, která bude mít za následek velké a dobré věci.

Velké byly. Zda dobré, to těžko říct. Každopádně prezident Si Ťin-pching by jistě svého rozhodnutí dát bez prodlení naočkovat všechny příslušníky bezpečnostních složek, veškerou mládež pod osmnáct let a obyvatele žijící do čtyř kilometrů od moře litoval a nebyl by sám, spolu s nimi by naši péči o Šanghaj odsoudili i miliony lidí po celém světě, které však konec námi spuštěného řetězu událostí usmrtil.

Zato bohyně Mlking by nás pochválila, pochybuji, že se v dějinách lidstva setkala s odvážnějším podnikem.
