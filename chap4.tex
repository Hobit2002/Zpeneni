\chapter{}
S velkými výčitkami jsem pozoroval, jak byl můj líný spoluvězeň spoután drátem připojený ke kladné elektrodě autobaterie. K elektrodě záporné pak Wuwangem pověřený dozorce připevnil drát druhý, který následně uchopil do své tlustou vrstvou hadrů obalené pravice.
	
 Elektricky nabitý drát proletěl vzduchem a dopadl na tělo ubohého vězně, který se na chvíli stal součástí elektrického obvodu. Ona chvíle stačila na to, aby zaúpěl a zoufale sebou zaškubal.
	
 Po ráně první následovala i rána druhá a třetí, po té nešťastný nemotora ztratil vědomí. Dozorce mu i přesto zasadil ještě další dvě rány. „Dost už! Vždyť má problémy se srdcem!“ zvolal náhle zdravotník, který celé věci přihlížel a hlídal, že trest bude provinilci výstrahou a nikoliv popravou. Když však ubožáka prohlédl, zjistil, že selhal. 
	
 Ovšem ještě dřív než stihl komukoliv vynadat, uslyšeli jsme střelbu a do popravčí chatky přiběhli dozorci svírající pokojného avšak cizího Afričana. 

„Stál v čele mouřenínů, kteří násilím pronikli dovnitř a dožadovali se jednání s Vámi, pane Shizhe. Některé jsme zastřelili, což další vylekalo natolik, že utekli. On ale zůstal, a tak jsme se rozhodli mu vyhovět,“ hlásil hned jeden z dozorců.
	
 Caifu Shizhe pokývnul hlavou a vyzval zajatce, aby promluvil.
	
 „Mocný pane, přicházím k vám s prosbou, abyste pomohl mým černým příbuzným zde v Zambii, brzy totiž budou napadeni bezohlednými bílými sviněmi z Jihoafrické republiky.
	
 Sám víte, jak moc představitelé Zambie s podporou celého národa kritizují zrůdný jihoafrický apartheid. A bílí utlačovatelé se nám za to mstí vojenskými intervencemi, přičemž ta poslední právě začala. Několik jednotek už překročilo hranice a vypálilo mou rodnou vesnici. Zanedlouho projdou okolo vašeho dolu a postřílí každého, koho potkají.
	
 Prosím nedopusťte, aby se tak stalo. Schovejte bezbranné v důlních prostorách a postavte se raubířům. Sám jsem viděl, že střelba Vašim lidem docela jde.
	
 Krom toho se o Vás mezi domorodci proslýchá, že jste docela dobrý člověk. Docela by mě zajímalo proč…“ 

Pohled vetřelého prosebníka spočinul na popáleném lenochovi a aparatuře, která ho zabila. Náš černý návštěvník byl zřejmě nejen člověk ušlechtilý, nýbrž také velmi bystrý. Rychle se dovtípil a zároveň byl vyděšen tím, k čemu zde došlo.
	
 „Omlouvám, že jsem vás otravoval. Zrovna jsem dostal skvělý nápad a už vás nepotřebuji vystavovat riziku,“ vykoktal ze sebe. S člověkem schopným popravovat tak barbarským způsobem své zaměstnance nechtěl mít nic společného.
	
 Caifua toto chování zmátlo natolik, že posunkem návštěvníkovi naznačil, ať tedy jde. Jakmile odvážlivec zmizel ze dveří, otočil se na Wuwanga a nešťastně se zašklebil: „Když ho nezastřelíme, rozkřikne se po celém okolí, že toto je tábor hrůzy a jestli se to dozví i inspektoři, udělají z nás vězně.“
	
 „Dobrá úvaha. Ten chlápek si jistě zaslouží mou péči,“ přitakal mu Wuwang. 
	
 Do tří minut dostal nezvaný host víc kovu než většina z nás za celý život. Nejprve kulku do kolena, poté želízka a nakonec vězení v plechové maringotce vybavené pouze železným nočníkem a postelí z kovových tyčí. 

Jelikož na maringotku několik hodin denně pražilo slunce, podlehl by zanedlouho každý v ní uvězněný nešťastník úžehu. Caifu si to uvědomoval a rozhodl se tedy našemu vězni dopřát popravu z milosti.
	
 Pokud by měl člověk na výběr mezi smrtí zastřelením a smrtí úžehem, dost možná by upřednostnil zastřelení. Rozhodně by si však nevybral nic, s čím by přišel Wuwang, avšak právě tomu byla poprava, jako každá jiná krvavá práce, svěřena.

	
 O chystané popravě jsem se doslechl ještě před jejím provedením a velmi se zhrozil.  Stejně jako já i zatčený host se stal pro svou odvahu miláčkem bohyně Jen a ta dbala toho, aby si sebe její děti navzájem všímaly. Už jsem se postavil na odpor masakru studentů a zbabělá poprava jednoho Afričana mi přišla jako menší zlo. Zbabělé přitakání bych si asi nikdy neodpustil.
	
 Dvě noci před popravou jsem se za tmy připlížil k zamčené maringotce a nakoukl do ní dovnitř zamřížovaným okýnkem.
	
 Uvnitř seděl odsouzenec a něco si vyřezával. 
	
 „Vy zemřete,“ oslovil jsem ho anglicky.
	
 „Jednou budu muset,“ odpověděl s úsměvem.
	
 „Vy ovšem zemřete dvakrát. A pozítří poprvé. “
	
 „Pochybuji, že se budu chtít z posmrtného světa vracet zpátky do téhle bídy. Kdybych byl Evropan, to by byla jiná…“ odvětil pobaveně.

Následovala konverzace, během níž jsem mu, Umlilovi, nastínil svůj plán, jak ho zachránit, se zájmem mi naslouchal a pak se zasmál: „Výborně, tak tenhle útěk si užiji“.
	
 Když jsme se loučili, otázal jsem se ho, co to vyřezává.
	
 „Ještě když jsem byl dítě, neznal můj kmen písmo a veškerá mravní i praktická poučení uchovával v ústně tradovaných mýtech a bajkách. Tyto příběhy byly nositelem kmenové duše a my si proto dávali velký pozor na to, aby nebyly ani pokřiveny, ani zapomenuty. Každý, kdo jednal v rozporu s příběhem, popřípadě si vymýšlel příběhy vlastní, ztratil respekt ostatních a nikdo s ním nechtěl nic mít. Tato pravidla však přestala platit, když se kmen dostal do problémů, jaké dosud nikdy nepoznal. V takových chvílích krajní nouze jsme se všichni stali vznešenými a získali tak právo nejen jednat v rozporu s mýty starými, ale také tvořit mýty nové.  

Třeba když ti bílí jihoafričtí šmejdi vypálili naši vesnici, nezůstal jsem se synem, jak mi staré mýty kážou, ale vyrazil jsem varovat jejich další oběti a teď, když mě tu zavřeli, nechal jsem si přinést několik kusů dřeva a začal z nich vyřezávat sošky svého dospělého syna, toho jediného, který přežil útok Jihoafričanů. Chci mu tím naznačit, že v tomhle světě může zanedlouho kdokoliv dokázat cokoliv. “

„Vy jste věřil, že ho potkáte?“

„Ne. Ale on by se mě po čase stejně vydal hledat a našel by je u nějakého ochotného člověka, jako jste vy.“

Způsob Umlilovy popravy si nikdo netroufl předvídat. Wuwang byl v páchání zla mimořádně kreativní a nepotřeboval své plány s nikým konzultovat. 
	
 Až v den popravy samotné jsme se tedy dozvěděli, že Umlil bude rukama připoután k jednomu důlnímu bagru a nohama k druhému. Následně do obou vozů vlezou řidiči, každý se rozjede jiným směrem a odvážný Afričan bude roztržen vejpůl.
	
 Umlil později litoval, že vyhlášení ortelu nikdy neslyšel a nemohl se Wuwangovi na místě vysmát. Nicméně tou dobou už tvrdě spal ve své maringotce.
	
 Když za ním přišli dozorci, našli ho, jak leží v posteli a nereaguje ani na sebesilnější třesení či sebehlasitější zvuky. Na zemi se povalovalo několik krabiček od morfiových tabletek.
	
 Okamžitě se dovtípili, že někde léky sehnal a otrávil se jimi. Chvíli po té už Wuwang prováděl inventuru ve věcech zdravotníka.
	
 „Nu, nevypadá to pro vás dobře. Ty krabičky, které byly u našeho negříka nalezeny, vypadají stejně jako ty s naším táborovým morfiem. Vysvětlete mi prosím, jak se k Umlilovi dostaly,“ zakončil svou prohlídku zdravotnických potřeb.
	
 „Dal jsem mu je,“ odpověděl zdravotník klidně.

„Proč?“ rozzuřil se Wuwang.
	
 „Protože jsem už mnohokrát viděl, jak hrozně to dopadá, když se práce s lidmi svěří vám. Tak jsem ji vzal do svých rukou,“ pokračoval zdravotník klidně.
	
 „Jestli jsi mi chtěl zkazit den, tak se ti to nepodařilo,“ procedil Wuwang mezi zuby, čapnul zdravotníka a začal ho táhnout směrem k připraveným bagrům.
	
 „Půjdu klidně sám. Caifu vás stejně nenechá mě popravit,“ozval se zdravotník.
	
 „Tuším to a o to víc se chci odreagovat tím, že vás na místo dotáhnu,“ odpověděl Wuwang.
	
 Když doklopýtali na popraviště, Caifu se svého předchůdce otázal, co dělá, a když slyšel, že Wuwang chce popravit zdravotníka, rázně ho zarazil.
	
 „Soudruhu, naším cílem není působit lidem co nejvíce bolesti. Toho černocha jsme potřebovali umlčet, ale rozhodně si nezasloužil být trhán na dvě půlky. Soudruh zdravotník udělal dobrou věc a myslím, že si zaslouží za ni být odměněn jednak větší svobodou a druhak funkcí táborového kata. Soudruhu zdravotníku, chápu, že možná jako odměnu nevnímáte, ale stál byste o takovou funkci?“
	
 „Pokud je Wuwang jediný konkurent, pak určitě“ odvětil zdravotník. 
Námezdní domorodí dělníci, kteří celou situaci pozorovali, začali jásat, zato Wuwang pocítil, že jeho život ztratil svůj smysl. Vrhnul se na Caifua a začal ho škrtit. Okamžitě k němu přiskočilo několik dozorců, kteří jej začali odtrhávat.
	
 Jakmile byl exsprávce dolu donucen se zvednout, klesl znovu k zemi. Tentokrát však bylo zřejmé, že sám už se nezvedne. V hrudi mu zela díra.
	
 Jeden z dozorců býval dříve voják a snadno poznal nejen že Wuwang byl zastřelen ale také to, že střela letěla nejspíš z maringotky. 
	
 Spolu s několika dalšími dozorci se k ní rozběhl, ale za chvíli už první z nich stihl stejný osud jako Wuwanga.

	
 Střelbu jsme si s Umlilem nedomluvili. Pušku jsme mu věnovali, protože zbraň se člověku uprostřed divočiny s nepřátelskou armádou v patách vždycky hodí.
	
 Jinak šlo ovšem doposud všechno podle plánu. Umlil dostal velikou dávku morfia a hromadu krabiček odpovídající dávce násobně větší. Uspal se a krabičky pohodil na zem. Dozorci si mysleli, že umřel, ale on se po chvíli probral a páčidlem, které jsme mu také dali (ne že by díry v okenní mříži byly tak velké, ale zdravotník měl klíče od vnějšího zámku, a tak mohl Umlila bez problému navštívit), vylomil dveře. Nu a pak měl prostě utéct.

On si však místo toho vzal pušku, poslední předmět, kterým byl obdarován, a začal střílet po dozorcích.
	
 Strážci nesvobody brzy poznali, že přestřelka s ním je beznadějná a nebezpečná, neb mu maringotka skýtala skvělou ochranu.
	
 Rozběhli se tedy směrem k bagrům, aby se v nich přiblížili až k jeho skrýši a rozdrtili ho.

	
 I já se rozhodl, že zkusím střelbu zastavit. Vsadil jsem na to, že jako komplic nebudu Umlilem zastřelen, zvedl ruce nad hlavu a vyrazil k maringotce.
	
 Přežil jsem. Umlil mě nechal dojít až k sobě.
	
 „Co blbneš?“ štěknul jsem na něj.
	
 „Dělám z tohoto tábora arsenál,“ odpověděl.
	
 Nechápavě jsem se na něj podíval a on v rámci odpovědi vykoukl z maringotky a nadlidsky mohutným hlasem cosi zařval. „Do večera tohle místo poslouží boji s Afrikánci,“ pokýval spokojeně hlavou.
