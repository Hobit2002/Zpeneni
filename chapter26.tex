\chapter{}

Úplně ze začátku tajná mise připomínala výlet čínské smetánky po Africe. Já, Jie, Lydie, Gang a jeho druzi rudí hrdlořezi (povětšinou pocházeli z periferií a do Šanghaje se dostali teprve nedávno díky své neobvyklé krutosti a loajalitě k režimu), jsme putovali po Zambii a tvářili se jako docela obyčejní turisté.

Ač jsem se bál, že mě nebo Lydii prokomunističtí rebelové podřežou, jakmile se jim k tomu naskytne příležitost, nestalo se tak. Rozhodně vůči nám neprojevovali žádné sympatie, ale chovali se slušně.

A tak jsme docela pokojně navštívili Viktoriiny vodopády i národní parky Kafue a Dolní Zambezi.
\vspace{0.75cm}

Ve Vděčné zemi se mezitím nový král Ping rozhodl učinit všechna opatření, aby jeho stát přežil. Svolal druhý summit, který se ovšem nyní nesl ve zcela pragmatickém duchu.

Ping chtěl do Šanghaje umístit co nejvíc vojáků západních států. S Putinovou smrtí totiž v Rusku do prezidentského křesla usedl jeho neodlučný politický druh Medvěděv, který oznámil, že okamžitě zruší ultimátum chránící Vděčnou zemi.

Naštěstí tak neučinil. Ani ne hodinu po té, co ve státní televizi své rozhodnutí sdělil veřejnosti, demonstrovalo už na Rudém náměstí několik tisíc fanoušků mystického státu, povětšinou duchovních. Zároveň Medvěděvovi okamžitě zavolal papež, kanadský premiér Trudeau, či jeho australský protějšek Morrison. Všichni tito vůdci se přimlouvali za to, aby Medvěděv zrušení ultimát o měsíc odložil a umožnil tak Vděčné zemi, aby si našla jinou ochranu. Medvěděv jejich přání vyhověl.

Oněch třicet dní Ping ani nikdo z jeho vlády neměl stání (snad s výjimkou nás, kteří jsme si užívali v Africe), jejich výkon tomu bohudík odpovídal.

Již po dvou týdnech seděl nový král u kulatého stolu s Trumpem, Macronem, Johnsonem a několika dalšími, kteří ho už nezajímali. Právě tito tři politici mu totiž slíbili, že do Šanghaje pošlou své vojáky a že pokud Čína na město zaútočí, budou to považovat za útok proti sobě samým.

Jak bylo slíbeno, tak se také stalo. Do několika málo dní se obyvatele města, jehož životní úroveň se za těch pár týdnů od svého založení propadla hluboko pod úroveň latinskoamerických velkoměst, při každé cestě do práce a nákup potkávali s americkými, francouzskými a britskými vojáky.

Aspoň v četnosti interakcí s vojáky jsme na tom my, doposud víc turisté než politici, nebyli pozadu.   
\vspace{0.75cm}

Onoho dne jsme vstali, nasnídali se, vyšli z hotelu a stejně jako každý jiný den nastoupili do půjčených aut. Tentokrát jsme ovšem nesměřovali do národního parku, nýbrž do bývalého měděného dolu.

Když jsme po betonové cestě přijeli až k bráně zasazené do vysokého plotu s ostnatými dráty, který celý důl obklopoval, vyšel nám naproti asi čtyřiadvacetiletý černoch neobyčejně mohutné a svalnaté postavy. 

„Nebyla nám hlášena žádná návštěva!“ prohlásil stroze.

„Jen se na mě podívej, chlape, máš pocit, že člověk mého postavení potřebuje svou návštěvu hlásit?“ otázal se ho Jie, který vystoupil z auta oblečený do mistrného plagiátu generálské uniformy.

„Kdykoliv dochází ke změně vrchního velení Čínské lidové osvobozenecké armády, jsme o tom informováni a musíme se naučit, jak který důstojník vypadá,“ nedal se vrátný.

„Došlo k nečekaným a tragickým událostem. Generální štáb byl záhadně otráven, pravděpodobně agenty Vděčné země, jsem ve své funkci teprve čerstvě. Jen se podívejte,“ lhal Jie dál a podal mu tablet, na kterém byla otevřena falešná napodobenina několika čínských mediálních stránek.

„Z našeho pohledu se stalo jen to, o čem jsme informováni pomocí našeho informačního systému, a ten, pokud vím, zatím žádnou změnu nehlásil,“ na ta slova vytáhl mobil a něco si v něm otevřel.

„Ne, žádnou mimořádnou událost tu nevidím…“ škrábal se vrátný na hlavě.

„To není možné!“ zvolal Jie a vytrhnul mobil vrátnému z ruky.

„Opravdu nic! Jak se mohl takhle kousnout? Nicméně zcela chápu, že nám nemůžete důvěřovat. Pokusím se chybu systému napravit a vrátím se teprve, až se to podaří,“ prohlásil rádoby zklamaně, načež nasednul do auta a vyrazil zpátky.

Ještě než jsme vyjeli z pralesa, předjelo Jieho  Gangovo auto, které se následně otočilo o devadesát stupňů a zatarasilo tak cestu. 

Jie se očividně pokusil vlastní vůz zabrzdit a zabránit srážce, ale měl smůlu. Naboural do kufru Gangem řízeného auta. Tehdy se však ještě nikomu nic nestalo.

O deset sekund později to bylo horší. Gang přiběhl k Jiemu, vytáhl ho z auta a popadl za krk. „Ty neschopný idiote! Všechno jsi zkazil!“ vrčel, vztekle exministrem financí lomcoval a táhl ho kamsi do lesa.

Třásl Jiem tak silně, že mu shodil brýle, které pak vzápětí rozšlápl. Jakmile si toho Jie všiml, dostal vztek. Vzepřel se a poprvé i naposled vrazil Gangovi pěst do tváře, načež ho ještě vší silou kopnul. Bývalého dozorce údery sice nepoložily, ale zato překvapily dost na to, abych ho s Lydií stihl chytnout ještě než se na Jieho vrhnul podruhé.

„Víš, kdo je tu idiot? Já rozhodně ne. Ty jsi idiot, Gangu. Ty brýle, které jsi rozšlápl byly chytré a nafilmoval jsem jimi podobu systému zprostředkovávajícího komunikaci Číny a mnohogenních jednotek. Kdybys mi je umožnil připojit na internet, odeslaly by video týmu hackerů, kteří by podle něj zjistili, o který čínský informační systém se jedná a po té by ho nabourali.

Ale tvá totální neschopnost to všechno zhatila. Místo videa jim budu muset systém popsat slovně a neviděl jsem dost na to, aby byl můj popis kvalitní,“

Gang neuměl přiznat chybu, a tak se nepřestával vzpouzet. Lydie mu tedy levačkou vytáhla zpoza pasu pistoli a zklidnila ho ťuknutím do hlavy. To neměla dělat. Ještě než se Gang sesul, vrhli se na nás jeho druzi rebelové, zkroutili nám ruce za zády a odvlekli nás do svých aut.

Odteď jsme byli jejich zajatci. Chvíle, ve které se chystali přebrat kontrolu, právě nadešla a my byli odsouzeni k smrti. Jie na tom byl jinak. Nejen, že se k nám úplně přestal hlásit, ale dokonce pravil i něco ve stylu: „Ty rozbité brýle nakonec zas až tak nevadí, když už jsou Dong s Lydií uklizení, můžeme mnohogenním vojákům říct pravdu“.
 
Jakmile jsme se vrátili do hotelu, zavřeli nás ti hrdlořezi do pokoje a zamkli zvenčí dveře. Asi hodinu jsme tam seděli v tichu a strachu, když jsme z vedlejšího pokoje zaslechli Jieho hlas. Nejspíše telefonoval, každopádně nemluvil ani čínský ani anglicky, nýbrž jakýmsi zambijským jazykem (jenž mu ušislyšně příliš nešel).

„Jsem přesvědčena, že to není až takový lotr,“ pronesla náhle Lydie. „Popisuje onen informační systém a snaží se o to místním jazykem, aby mu naši věznitelé nerozuměli.“

„Kdo jsou ti mnohogenní?“ otázal jsem se jí. „Prý tě zatkli, jelikož jsi o nich něco zjistila.“

„Ano. Věděla jsem, že Čína měla v Zambii měděný důl. Začátkem devadesátých let se v něm krátce vzbouřili dělníci a přebrali nad ním kontrolu, ale Číňané jej brzy získali zpátky. Následně v něm ještě rok těžili, ale pak s tím přestali. Nikdy se však dolu nezbavili, ba co víc, vykazované náklady určené na těžbu ještě vzrostly, to mi připadalo zvláštní, a tak jsem se vypravila do Zambie, abych té záhadě přišla na kloub.

Zjistila jsem, že důl začal sloužit armádně-vědeckým účelům. V roce 1995, tedy mnohem dřív, než se lidstvo naučilo rychle a levně přečíst vlastní DNA, se tu dělaly genetické pokusy.

Číňané si od zambijské vlády koupili právo unést deset tisíc těhotných zambijských žen (krom toho, že vláda za toto svolení dostala spoustu peněz, odpustili jí Číňané, že na svém území nechává podnikat jednoho extrémně protičínského byznysmena) a úplně naslepo nacpat do genetické informace jejich dětí DNA zcela jiných tvorů.

Tímto způsobem vzniklo mnoho příšerných a životaneschopných mutantů, pár výjimek však pokus přežilo a mnozí měli velmi unikátní schopnosti, ať už se jednalo o obrovskou sílu či schopnost regenerace. Mimořádně zajímavým se stalo jedno dítě, jehož genetická informace obsahovala hned několik scénářů vývoje.

Dokud vyrůstal v teplu a vlhku byl poslušný jako pejsek, když se však dostal do Číny, úplně se změnil a stala se z něj vlivem výrazně zvýšené produkce testosteron agresivní a ctižádostivá bestie podobná samci v říji. Tento jev Číňany velmi zaujal, a tak ke změně hormonové produkce (která sama o sobě nijak zvlášť žádoucí nebyla, ale asi nevěděli, jak se jí zbavit) přidali také změnu zbarvení kůže a metabolismu.

Právě tyto úspěšné mutace Číňané mnohokrát zopakovali. Přesvědčili tisíce afrických rodičů, že dají jejich dětem úžasnou budoucnost a následně geneticky upravili jejich ještě nenarozené děti, po tom co se vylepšení afričánci narodili, vzali si je k sobě a vychovali je právě v tomto dole, kde z nich vycvičili izolovanou a elitní armádu. Mezi lety 2020 a 2030 by měl být její výcvik dokončen.

Moc jsem si přála vidět důl a ty upravené vojáky naživo. A splnilo se mi to takhle. Nicméně na stopu mě přivedl právě Jie, který tehdy ještě nepracoval pro Čínu. Tuším, že mohl jednat ve službách zambijského miliardáře Cu-quiána, který kdysi onen měděný důl spravoval, ale nemám pro to důkazy. Každopádně to považuji za další z důkazů toho, že Jie není spojencem Číny“ posteskla si Lydie.
\vspace{0.75cm}

Již následující den nás pročínští hrdlořezi vyvedli ven. Znovu jsme usedli do aut a vyrazili do dolu. I nyní nám vyšel naproti vrátný, za naši stranu s ním ovšem vyjednával Gang.

„Včera jsme vám lhali, ale odpusťte nám to, byli jsme pod dozorem ministerských sviní Vděčné země. Nyní tyto svině čekají na zabijačku a my můžeme začít mluvit pravdu,“ na ta slova rychle vrátnému nastínil plán, který si dříve domluvil s Jiem, totiž připlout do Číny, zde nejprve vyrabovat pár bezvýznamných čínských obcí a následně se vrhnout na Šanghaj. Účelem akce bylo zničit Vděčnou zemi a zároveň nevypadat jako čínská jednotka.

Tento návrh vzal vrátný dost vážně na to, aby zavolal svého nadřízeného. Když dotyčný přišel, nechal si Gangem celý příběh dovyprávět ještě jednou, a pak prohlásil, že pro takovou akci bude nejprve potřebovat souhlas.
 
Jako včera jeho podřízený i on vytáhl smartphone a přihlásil se do informačního systému. Ale ouha! Informační systém byl zasekaný a vůbec nefungoval. Jak to velitel zřel, vyhlásil zvláštní stav, který obnášel i to, že se naši delegace ujali tři chlapci a čtyři dívky, kteří nám jasně dali najevo, že dokud se věci nevrátí do normálu, nesmíme odjet. Každý z těch lidí měřil více než dva metry a sílu měl nejspíše srovnatelnou s medvědem. Navzdory takřka nouzové situaci byli naši noví věznitelé velmi slušní, ohleduplní a snažili se v rámci možností vyjít vstříc všem našim prosbám o stín, vodu a heslo na wifi (v Gangovi toto chování budilo velkou nedůvěru, slyšel jsem, jak ho komentoval slovy „Jak mají tyhle slečinky masakrovat šanghajské děti je mi tedy záhadou.“). 

Zvláštní stav trval do večera. Jednotka jím byla zcela paralyzována, doposud se spoléhala výhradně na informační systém, který právě vypadl, aspoň tedy zaujala bojové postavení pro případ, že by sabotáž byla následována přímým útokem.

Večer si však všichni až na Ganga a jeho party oddechli, systém se zase nahodil a oznámil jednotce, že se Jie stal novým armádním generálem. Celý tábor nyní čekal na jeho rozkazy, toto čekání však bylo velmi krátké.

„Jste lidé mladí, silní, chytří a krásní, myslím, že by bylo zločinem vás zatahovat do válek. Ode dneška jste svobodní, ale jelikož jste svobodu ještě nikdy nezažili, nechte mě, abych vás ji naučil,“ těmito slovy analytik šokoval všechny přítomné. 

Gang se na Jieho už zase pokusil vrhnout, ale jen co se rozběhl, přišel o nohy. Mnohogenní vojáci mu je ustřelili. Já a Lydie jsme se na sebe akorát podívali s otevřenou pusou.

V následujících dnech se demobilizovaní mutanti přesunuli na opačný konec pralesa a dostali za úkol si postavit nové bydlení. Zároveň měli zakázáno se na cokoliv ptát svého generála a používat informační systém (to především). Jie v nové osadě nechal zřídit hernu a kino, obojí mělo vojákům pomoct objevit věci, jimiž běžní lidé tráví.

Gangův gang byl držen na jednom místě a ostražitě hlídán. Já a Lydie jsme se mohli svobodně pohybovat, i nás ale Jie nechal sledovat.

Byl jsem z exministra financí totálně zmatený. Očividně mu nešlo ani o existenci ani o zánik Vděčné země, hrál svou hru, jíž nikdo jiný nechápal. Další, co mi vrtalo hlavou, byly náklaďáky, které do nové osady občas přivezly léky, jídlo, nebo textil a neměly na sobě ani logo nějaké neziskové organizace ani státní vlajku. Kdo je posílal? Jak ho k tomu Jie přemluvil? To jsem naprosto netušil.

Jie si dával dobrý pozor, aby se se mnou nepotkával, a tak jsem se rozhodl začít s pátráním u Ganga. Tento chlap svět vždy viděl tak černobíle, a tak jednoduše, že by sotva mohl mít důvod něco zamlčovat, zvláště po té, co byl kvůli Jiemu zmrzačen.

„Prohnaný hajzlík je to. Ale co jiného se dá od úředníka čekat,“ prohlásil Gang bez přemýšlení, když jsem se ho na Jieho motivy zeptal „chce si prostě šplhnout u těch nahoře a ostatní lidi jsou mu úplně ukradení.“

S druhou částí jsem se ztotožňoval, překvapovalo mě však, že by obyčejný kariéristický šmejd tolikrát pomohl revoluci. „Proč ovšem neodvedl mnohogenní vojáky do boje proti Šanghaji?“

„Čína o to nestála. Byli by tak odhaleni a už by je nemohla použít jindy. Vždyť to ani nemá zapotřebí. Západní vojáci Šanghaj chrání jen před vojenským útokem, nikoliv však před ukončením dodávek vody a elektřiny. Jakmile ultimátum skončí, celé město zhasne a je jen otázkou času, kdy se samo odevzdá zpátky do čínských rukou. Byl jsem hlupák, když jsem se tím hadem nechal obalamutit a na tuto misi vyrazil a ještě větší, když jsem se ho pokusil napadnout a dal mu tak důvod, proč mi odepřít všechny zásluhy na celé misi. A že jich bylo! Vždyť jsem zabil Chun!“

Nejprve jsem jen šokovaně mlčel. Vážně by to Jie udělal? Vážně by zhatil revoluci, jíž i on sám pracně budoval? Čím déle jsem však nad touto otázkou přemýšlel, tím víc jsem si uvědomoval, že Jie nikdy ani nepůsobil jako člověk, který by ctil nějaké zásady a uznával, že jsou mantinely, které se nemají překračovat. A vzhledem k tomu, že Gang byl zničený a neměl důvod mi lhát, musel být ministr financí netvor, jehož v Božím království čekalo dlouhé soužení nemožností komukoliv výrazněji škodit. 

„Díky Gangu, pokud máš pravdu…“ začal jsem náš hovor zakončovat.

„Žádné pokud. Já ji mám a vím to,“ skočil mi bývalý dozorce do řeči.

„V tom případě je nyní Jie nepřítelem nás obou. Co říkáš na to, že bych ho vyhledal a udělal vše proto, aby ztratil veškerou důvěru, které se doposud ze strany čínských komunistů těšil?“ navrhnul jsem mu.

„Ten cíl schvaluji a sám udělám proti Jiemu vše, co bude v mých silách,“ zavrčel Gang. A bylo to. Měl jsem sice zlého a beznohého, ale přeci jen nějakého spojence.

Zprvu, jsem chtěl přijít za Jiem a promluvit mu do duše (tentokrát jsem byl odhodlán ho pro splnění tohoto cíle klidně probudit uprostřed noci) Přelétavý analytik ovšem kamsi odjel a nebyl k zastižení. Správcům tábora (bývalým důstojníkům) prý řekl akorát to, že odjíždí na delší dobu. Přítomnost, nepřítomnost, já nehodlal nechat své otázky nezodpovězené. Ještě, že jsem měl Ganga.
