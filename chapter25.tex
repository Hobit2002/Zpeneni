\chapter{}

Nedovedu si vysvětlit, čím to je, ale když asijští politici nejsou schopni zapsat se do učebnic ekonomie popřípadě válečného „umění“, prorazí si aspoň hlava nehlava cestu do těch dějepisných. Příkladem tohoto chování budiž Brežněv, Gorbačov, Mao Ce-tung, severokorejští diktátoři a v poslední řadě Chun.

Nebýt mimořádně náboženské povahy šanghajského lidu, zanikla by Vděčná země do několika málo dní.

Ani Putinovo ultimátum nově vzniklý městský stát neubránilo před totálním dopravním kolapsem, několika rozsáhlými výpadky elektřiny a internetu i nenadálou mírou kriminality.

Lidé však byli přesto šťastní. S širokými úsměvy si klekali na ulice a samou radostí ze života žehnali okolnímu světu. I Chun, jíž nepříjemně vyvedlo z míry, kolik toho jako královna musí dělat, se zvládla obšťastnit – provdala se za Pinga, který se tak stal jejím nástupcem.

Jen málokterá královská rodina by však slavila tak skromně. Obřad se konal v jednom z bytů, kde se dříve pořádaly tajné bohoslužby podzemní církve. Oddávajícím duchovním jsem byl já. Bránil jsem se tomu, necítil jsem se k tomu nikterak oprávněn, ale když se vdává královna…

Ač i tento počin měl vzdor své skromnosti velkou publicitu, získala Chun své místo v povědomí politiků i jejich voličů a nevoličů především uspořádáním Sněmu Lásky a Naděje, na který pozvala téměř všechny vrcholné představitele ostatních států.

Nedovedu si představit, že by se mezi sezvanými našel někdo, kdo si po přečtení, že „...je tímto pozván, aby otevřel své srdce Duchu Svatému a začal dělat politiku hodnou Božího služebníka“ nezaťukal na čelo, i přesto dorazili téměř všichni představitelé demokratických a křesťanských zemí. 

Myslím, že jsme za hojnou účast vděčili především sociálním sítím. Nadšení Šanghajané zaplavili internet zprávami, o tom, jak fantasticky se ve svém novém státě mají a Chun díky nim začala být online světovou populací milována a ctěna jakožto mimořádně talentovaná vůdkyně. Kdyby západní politici odmítli na její sněm vyrazit, vážně by si zahrávali s rizikem, že do pěti let už bude na podobné akce zván někdo jiný.  
	
Do Šanghaje přilétli první hosté. Povětšinou se nestačili divit, jak je možné že Šanghajané nový řád oceňují. Ulice byly plné odpadků, výpadky elektřiny, vody i internetu se doposud nepodařilo vyřešit a problémy mělo i školství a zdravotnictví. Krom toho celé město obléhala čínská armáda, která bránila, aby se Číňané nespokojení s komunistickým režimem zkoušeli stěhovat do toho svobodného chaosu Na prvnímu setkání byli přítomni představitelé téměř všech států latinské Ameriky a NATO, dorazili také všichni ministři Vděčné země. 

První den se nesl v duchu modliteb. Politici se museli účastnit víc jak deseti bohoslužeb, které vysluhovali duchovní z různých církví. Večer byl však o poznání příjemnější – večeře totiž probíhala nikoliv v uzavřených prostorách nýbrž na náměstí, kde za představiteli západních zemí co chvíli přišel někdo místní a aby ukázal, že je hoden být občanem Vděčné země, poděkoval jim za jejich účast na Sněmu nějakým dárkem.

Druhý den se už konalo cosi jako jednání. Politici nyní předstupovali před své protějšky a podobně jako na sněmu OSN mluvili o tom, co zrovna svět a jejich zemi trápí.

Třetí den před shromáždění vystoupil Jie a ukázal jim stručný seznam předchozí den obzvlášť často zmiňovaných témat. Následně politiky rozdělil do skupin a poslal je do několika restaurací, aby se domluvili na konkrétních řešeních. Inspiroval se přitom volbou papeže. Dokud se mocní nedomluvili, bylo jim znemožněno restaurace opustit. 

Po zbytek týdne ten hnisavý bolák v samém srdci církve pomocí skrytých kamer a diktafonů průběžně monitoroval politická jednání. Po většinu té doby si musel podpírat hlavu, aby mu zoufalstvím neklesla. Když se totiž dali dohromady populističtí představitelé západních demokracií, vzniklo něco, co svým idiotstvím daleko předčilo i jednání čínských komunistů.

Čím dál tím více času proto trávil u po revoluci nevyužívaných policejních superpočítačů s akademiky, kteří Sněm navštívili ze zvědavosti a chuti vyjádřit svůj odpor k největší totalitě v dějinách lidstva. O nic tím nepřišel.

Po pěti dnech totiž pouze dvě z deseti skupinek přišly s něčím, co Jiemu připadalo smysluplné, pouze dvě a co hůř, jednalo se o skupiny, ve kterých byli političtí představitelé Pobaltí a Skandinávie, tedy zemí, které fungovaly příkladně i bez summitu.

Nevím přesně, jaké myšlenky se mu tehdy honily hlavou, ale když jsem zpětně nahlédl do jeho osobních poznámek, našel jsem v nich následující slova: „otec i Fetu se pletou, to, co s politickou scénou provedli, je hrozné. Doposud jsem uvažoval, že budu pokračovat v jejich hře, ale je to fraška. Tihle politici měli lidi ponechat jejich osudu, ne se vžít do role univerzálních spasitelů. Je čas utnout všechny falešné víry a postavit se tváří tvář tvrdé pravdě“ Jaká škoda, že jsem nikdy nepotkal osobu, která by byla člověkem svého slova víc než Jie.

\vspace{0.75cm}
\textbf{Z Jieho deníku}
\vspace{0.75cm}

{\itshape
Jedné noci se ministr financí vydal do přístavu. Zde už na něj čekala Číňany vyslaná jachta, která ho odvezla až k nedaleko kotvící lodi, jež byla připravena Šanghaj kdykoliv zalít deštěm smrtících raket.

„Víte, o čem téměř bez ustání sním?“ otázal se Jieho kapitán křižníku, jakmile zrádný ministr opustil přepravní loďku.

„Že se budete plavit od Šanghaje a na této lodi nezůstane jediná raketa krátkého dosahu. Říkáte to, protože doufáte, že Vám s tím pomohu,“ odpověděl ministr znechuceně a předsevzal si krvelačnému důstojníkovi nevěnovat více pozornosti.

Namísto toho si hleděl námořníka, který jej zavedl až do důstojnické poradní místnosti, kde už seděl prezident Si Ťin-pching a s ním i hned několik maršálů a politiků.

„Říkal jste, že byste byl ochoten eliminovat jak Putina, tak Chun Čchenovou,“ uvedl prezident jednání. 

„Jen pokud mi Čína pomůže s novým způsobem, jak zbavit svět chavezie,“ odpověděl Jie.

„Mluvte,“ vyzval ho prezident.

„Předpokládám, že všichni, co tady sedíte, víte o tom, jak byla celá Šanghaj před lety naočkovaná. Jednalo se o velmi efektivní ale o to dražší způsob, kterak lidi před nebezpečnou archeou ochránit. V celočínském ba co více celosvětovém měřítku by takto lidi zachraňovat nešlo.

V posledních týdnech jsem však ve spolupráci s vědci z Jižní Koreje, Japonska, Taiwanu a Západu přišel na způsob, kterak chavézii plošně vymýtit a to jak levně, tak bez velké mediální pozornosti.“

„Jestliže se na tom podílel byť jen jeden Taiwanec, naši podporu určitě nečekejte,“ skočil Jiemu do řeči jeden z maršálů. „Není-liž pravda soudruhu prezidente,“ dodal a upřel tázavý pohled na Si Ťin-pchinga.

Si učinil krátké gesto, kterým naznačil, že se jednalo o velmi nevhodnou a hloupou poznámku. Maršál to pochopil a uznal, že už na sněmu nemá co dělat. Omluvil se, zvedl se a opustil místnost. Jie mohl pokračovat.

„Po šanghajské revoluci zcela osaměly policejní superpočítače dříve využívané k predikci podvratné činnosti a masové identifikaci lidí na základě jejich obličejů. Využil jsem je tedy k nalezení látky, která by zvládla zničit chavézii. Sezval jsem mikrobiology a datové vědce z celého svobodného světa a několik týdnů pro koordinování jejich výzkumu zanedbával své povinnosti ministra financí. 

Stálo to za to. Vědci počítači ukázali biologickou strukturu archeí, aby pak svěřili umělé inteligenci úkol najít z desetimiliónů známých chemických látek a mikroorganismů něco, co by se dalo využít v boji s chavézií. Uspěli. Našli virus napadající dýchací cesty drobných hlodavců, který by s jistými modifikacemi mohl účinně likvidovat také chavezii.

Moc bych si přál, kdyby mi čínská vláda poskytla zázemí, díky němuž bych mohl tento virus patřičně modifikovat a zároveň vytvořit v tak velkém množství, že bychom jím mohli promořit světový oceán a tím se archeí jednou provždy zbavit.“

„Je dost těžké vám věřit, soudruhu Jie. Nemáte problém kdykoliv kohokoliv zradit,“ poškrábal se Si Ťin-pching na bradě.

„Jsem hráč, nikoliv figurka. Tím se liším od vás všech, co tu sedíte a zcela se necháváte ovládat modlou vlastní moci. Jakožto hráč však nikdy nezradím své odhodlání něco dokázat a zrovna nyní chci vymazat ze světa životu nebezpečnou hrozbu pro miliony lidí v Číně a miliardy po celém světě,“ odpověděl Jie pohotově.

„Nemáme problém vašemu přání vyhovět, pokud se ovšem postaráte o to, že do měsíce z tohoto světa zmizí jak Putin, tak Chun Čchenová,“ prohlásil Si.

„Kdyby to šlo i bez vraždění…“ začal Jie.

„Šlo, ale víte, jak by to bylo drahé? Pokud byste nepohodlné vůdce dopravoval do rukou čínských tajných služeb a ty je pak roky schovávaly v utajení, stálo by nás to miliardy jenů. Zato když to s oběma dvěma rychle a elegantně skoncujete, nestojí nás to nic a my můžeme ušetřené miliardy investovat do Vaší virové farmy. Rozhodněte se tedy, buď můžete být vrahem s virovou farmou, nebo pouhým spoluúnoscem bez…“

„Budu ten vrah,“ skočil mu Jie do řeči.

„Výborně,“ zajásal komunistický prezident a jeho tvář se rozzářila úsměvem, jaký šlo jinak vidět pouze ve filmech s medvídkem Pů.

Když pak znovu vycházeli na palubu, zopakoval ještě prezident v metaforách, na čem se s Jiem dohodl:“Takže vy nám položíte na oltář pár oslů a poté rozpálíme holocaust, jaký svět dosud neviděl a nikdy neuvidí.“ 

„Tak tak, neviditelný holocaust provedený biologickými zbraněmi,“ pousmál se Jie.

„Asi chápete, že jsem zaneprázdněný a nemohu s vámi řešit všechny detaily tohoto podniku. Toho se ujme můj kolega Yang Ťao-chang, vojenský poradce ministra zdravotnictví,“ pokračoval prezident a ukázal na jednoho z maršálů.

„Už se nemohu dočkat, až oltář vzplane,“ ozval se důstojník a uklonil se Jiemu na pozdrav.
}
\vspace{0.75cm}

K spáchání ohavného činu mělo dojít během sněmu. Nikdo z účastníků přitom netušil, že všeobecná posvátná atmosféra může maskovat smrtelnou hrozbu. 

Stalo se to, když byl experiment ukončen. Jiemu, Lydii a několika dalším rozumným ministrům se podařilo Chun přesvědčit, že tahle snaha k ničemu nepovede. Alespoň se tedy na závěr uspořádala velkolepá oslava, která měla přebít negativní pocity hostů z toho, že byli po čtyři dny zavření.

Právě během těchto velkolepých oslav začali politici odlétat. Vládní letouny Spojených států, Kanady, Velké Británie… to všechno se odlepovalo od startovacích letištních drah a mířilo zpátky domů. Řada přišla i na osobní tryskáč ruského prezidenta.

Zpočátku šlo vše stejně, jako když odlétali ostatní politici. Za nadšeného potlesku šanghajských davů a žárlivého pohledu Pinga se jeden z nejmocnějších lidí políbil s Chun a pak nastoupil. Letadlo vzlétlo jako všechny před ním a podle očekávání vyrazilo podél pobřeží do Ruska.

Ke změně plánu došlo v blízkosti korejského poloostrova. Namísto toho, aby jej pilot oblétl, namířil si to přímo nad Severní Koreu. Právě v jejím vzdušném prostoru se doposud normální let změnil v delirické motání.

Putinovi ani jeho ochrance nejprve vůbec nedocházelo, co se děje, všichni byli pohroužení do děkovných a prosebných modliteb. Že je něco špatně si všimli až po té, co pilot opustil řídící kabinu a „vyrazil na záchod“. Jeho skutečným cílem však byl malý prostor v těsné blízkosti toalet, ve kterém se nacházely akorát padáky a dveře. Pilot dveře otevřel, jeden padák si nasadil, zbytek vyhodil a pak sám, dokud se pod ním ještě nacházela Čína, vyskočil.

Značný pokles tlaku i teploty, který otevření dveří způsobilo, vytrhnul ruského prezidenta i jeho ochranku z nábožného pohroužení. Na záchranu už ovšem bylo pozdě. 

Letadlo bylo v rukách Jieho lidmi přeprogramovaného autopilota a ochrance nezbylo než zavřít dveře a modlit se ještě urputněji.  

Hlídači severokorejského vzdušného prostoru se nestačili divit. Krom toho, že se nad jejich pevninu dostal neznámý a nikým nehlášený objekt, choval se tento objekt také mimořádně divně.

První hodinu si jen tak poletoval a kreslil svou trasou na mapy letových dispečerů kříž, po té ovšem zamířil k Pchjongjangu. A právě do toho se severokorejským bezpečnostním složkám ozval kdosi, kdo se představil jako „vrchní velekněz svaté revoluce Vděčné země“ a vzkázal jim, ať přimějí národ k modlitbě, neboť již brzy nastane severokorejské11. září. Jediné, co operátoři o dotyčném zjistili, bylo to, že opravdu volal z Šanghaje. 

Nekontrolovaný letoun se blížil k hlavnímu městu a Severokorejci nehodlali nic riskovat, a tak tryskáč sestřelili. Nepřežil nikdo.
\vspace{0.75cm}

Jak jsem později zjistil, Jie se se zrádným pilotem setkal, srdečně (na své poměry) mu pogratuloval a z vlastních peněz mu zajistil novou identitu a pohodlný zbytek života v luxusní karibské vile.

Stále byl ovšem ministrem financí Vděčné země a to s sebou obnášelo nutnost účastnit se vládního sněmu o katastrofě, kterou Putinova smrt pro Vděčnou zemi představovala.
	
Ona porada pro Jieho dopadla úspěšně, Chun i ostatní ministři při ní totiž odsouhlasili, že je nanejvýš vhodně posílit svou bezpečnost přivedením čínských geneticky vylepšených, mnohogenních vojáků.

Velké kontroverze však vyvolal Jieho návrh misi svěřit mimo jiné Gangovi a jeho jednotce, která měla sebe samu považovat za povstaleckou, a přitom sloužit vládě. Tak zodpovědný úkol tak nespolehlivé skupině? Jie však na jejím pověření trval a zdůvodňoval to tím, že nikdo jiný ono poslání nevezme tak vážně. Nakonec si svou prosadil ale jen s tou podmínkou, že bývalí zabijáci čínského režimu zůstanou pod kontrolou, respektive pod mým dohledem a že Jie bude situaci zpovzdálí sledovat a v případě jakýchkoliv problémů zasáhne. Lydii rada ministrů na cestu nepustila

Vyrazili jsme již další den. Na letišti jsem se krátce seznámil s pětatřiceti zarytými komunistickými lotry a pak vyrazil do státního letadla, které nás dopravilo až do Lusaky (Zambie byla dlouhodobě pročínskou zemí, a tak jsem vůbec nechápal, proč k nám byla od samého začátku vstřícná jako žádný jiný stát). 

A právě tehdy došlo k průšvihu. Jeden z těch lotrů, Gang, se při nasedání do letadla ztratil.

Co se dalo dělat? Informovali jsme Chun, pak se vznesli a po příletu do zambijské metropole se ubytovali v předem domluveném hotýlku. Bylo mi divné, že tito údajní rebelové dobře vědí, že peníze i cíle jim dala Vděčná země. Podařilo se mi zjistit, že považují Jieho za svého spojence, který tahá Chun a ostatní ministry za nos. Ministr financí pro mě byl člověkem natolik neprůhledným, že jsem si nemohl být jist, zda náhodou nemají pravdu. I tak jsem byl ovšem rád, když jsem ho následující ráno potkal.

Naše letadlo opustilo Šanghaj brzy ráno a už večer došlo k další katastrofě. 

Chun si večer po práci vyšla na střechu věznice, aby na samém okraji poklekla k modlitbě. A právě tehdy se na ni vrhnul maskovaný útočník a svrhnul ji dolů.

V Tilanquiau nebyla žádná ochranka, kamerové systémy už během summitu zkolabovaly a nová šanghajská policie fungovala všelijak. Šance, že by se případ podařilo v dohledné době vyšetřit, byla minimální.

Jednalo se bezpochyby o nejtragičtější smrt roku a celý svět jí byl vyveden z míry. Vděčná země se propadla do truchlení, údajně bylo nesmírně těžké potkat člověka, který by neměl zarudlé oči a třesoucí se hlas a vůbec bych se nedivil, kdyby to byla pravda, sám jsem nejprve málem dostal infarkt a po té se pláčem takřka dehydratoval.

Na Chunino místo nastoupil Ping, který smrt své ženy považoval za nezvratný důkaz o nedostatečné bezpečnosti svého státu. Naše mise, která měla tento nedostatek odstranit, se stala státní prioritou.

A právě z tohoto důvodu za námi přiletěl Jie, Lydie a Gang (Jie tvrdil, že se prý akorát seknul na záchodě, ale od svého nalezení byl pod bedlivým dohledem). Tak rád jsem své politické kolegy viděl, připomněli mi Chun a skutečnost, že Vděčná země dosud neumřela (Gangova přítomnost mě také potěšila, ovšem z jiného důvodu – měl jsem díky ní jistotu, že nedělá nějakou neplechu v naší vlasti)

V této sestavě jsme pak vyrazili na cestu.
