\chapter{}

Tento příběh má být víc o mém synu než o mně. Vrátím se tedy zpět do roku 1989, kdy jsem byl nucen Mei i Jieho opustit.

Manželka motivy mého stoupnutí před tank nikdy nepochopila, zato Jie jimi byl ovlivněn mimořádně hluboce. Vyprávěl mu o nich onen odvážný, tajuplný a moudrý cyklista, který mě provázel celým tankovým dobrodružstvím.

Shouwanga Źhe, tak se dotyčný jmenoval, jsem kdysi požádal jen, aby mé rodině odvezl nákup. Učinil tak a zdaleka ne jen jednou.

Jednalo se o člověka vskutku ryzí povahy. Pochopil, že na Mei by měl život matky samoživitelky nepříznivý vliv stejně tak jako na Jieho osud v bídě vyrůstajícího dítěte. Krom toho byla má ženuška stejně jako on mladá, neopotřebovaná… zkrátka taková, že nebyla žádná prohra mít ji za partnerku. 

Z manželského a ekonomického hlediska mě tedy Shouwang vystřídal (nikoliv nahradil, za bankovní přepážkou jsem dostával vyšší plat než on v knihovně), nicméně z hlediska rodičovského to už měl složitější.

Jie trávil drtivou většinu času hloubáním (podle svých vlastních vyprávění byl v tomto ohledu podobný Fetuovi). Mluvit se naučil v podobném věku jako jiné děti, se svými opatrovníky však prohodil prvních pár slov až po svých pátých narozeninách. Do té doby nahlas akorát přemýšlel a to jen když byl sám.

Drtivá většina rodičů by jeho chování pojala egocentricky a trpce by si pomyslela: „Buď je ten kluk idiot, nebo se mnou nemluví, protože mě nemá rád.“ Shouwang zvolil jiný přístup. Odposlouchal hodiny a hodiny Jieho žvatlání, aby pak jednoho večera přišel k jeho postýlce a knížkou a slovy: „Tak Jie, než půjdeš spát, přečtu ti kus z téhle knížky.“

Náš syn neprotestoval, a tak Shouwang začal: „Když bylo Zarathustrovi třicet let, opustil svou domovinu i jezero své domoviny a odešel do hor. Zde se kochal svým duchem a svou samotou a po deset let se jich nenabažil. Posléze však se proměnilo jeho srdce…“

Tu noc Jie neusnul. Shouwang mu do hlavy nasadil hned dva brouky. Prvním byla pohádka samotná. Pětiletý hošík ji vůbec nepochopil, nerozuměl tomu, proč se hlavní postava chovala, jak se chovala, nechápal ani význam jejích slov... nepochopil zkrátka vůbec nic a to mu vrtalo hlavou ještě víc. Zatím se mu nestalo, že by nějakému příběhu nerozuměl, a naprosto netušil, co tu změnu mohlo způsobit.

Následující dopoledne se s pláčem přišoural do ložnice za Mei a Shouwangem, našel onu knížku v knihovačce a když se Shouwang probudil, přišel za ním s onou knihou. „Já to nechápu…“ promluvil vůbec poprvé na svého opatrovníka.

A od té doby už Jie používal řeč k jejímu primárnímu účelu.  Pokročil dokonce tak daleko, až Mei nabyla mylného dojmu, že vychovává normální dítě. 

Navzdory tomuto omylu se Jiemu v rámci možností dostalo bezpochyby skvělého dětství. Ani Mei ani Shouwang neměli pocit, že komunistický režim umožňuje dát vlastnímu životu smysl jinak než výchovou dítěte. A nijak se tím před synkem netajili.

Shouwang Jiemu mnohokrát vyprávěl o tom, jak jsem si stoupnul před tank a zvládal tu historku interpretovat takovým způsobem, že v jeho podání nabývala takřka nábožné atmosféry.

To ovšem zrovna v případě Shouwanga nebyl žádný div. Mladý knihovník byl věřící hluboce i široce. Nejen, že i maličkému  Jiemu, čítával každý večer duchovní literaturu (sestávající celkem z čehokoliv od Bible až po Korán, Konfucia a Foxe), nýbrž se i aktivně angažoval v několika tajně se scházejících, podzemních, náboženských komunitách.

Právě v komunitě skrývajících se protestantů někdo dostal nápad, že by si mohli založit malé samizdatové nakladatelství a dostávat do společnosti náboženskou literaturu.

Shouwang byl všemi deseti pro. Akorát chtěl se začátkem činnosti vyčkat, než bude Jie přijat na internátní střední školu v Šanghaji.  

Prozíravě předvídal, že pokud by došlo k odhalení jeho činnosti, postihly by celou rodinu tvrdé perzekuce. Doufal však, že pokud začne až v době, kdy se už s Jiem nebude stýkat, mohl by být jeho svěřenec potrestán shovívavěji.

Shouwangovým úkolem bylo v knihovně skladovat již vytištěné a dosud nerozdané výtisky. Velmi se snažil, rozstrkal je do těch nejzapomenutějších skladů starých knih, a strčil do obalů od přípustných děl, ale nezabránil tak tomu, aby Jie dočasně všechno ztratil.

Stalo se to, když Jie končil sudium střední školy a byl již přijat na univerzitu (kam se jakožto člen velmi neprominentní rodiny dostal pouze pod podmínkou aktivního angažmá v komunistické mládeži) necelý rok.  Knihovna se měla přestěhovat a při té příležitosti byly vyneseny na světlo i ty nejzašantročenější knihy. Hlavní knihovník se velmi divil, kde se vzalo tolik Biblí. Netrvalo dlouho a Shouwang i Mei byli zatčeni.

Jie se to dozvěděl od jednoho z pedagogických pracovníků komunistické mládeže a to dost nepříjemným způsobem.

„Budeš raději utírat záchody, nebo vynášet popelnice?“ otázal se ho dotyčný po jednom mládežnickém setkání (které můj syn údajně celé pročetl).

„To už jsme ovládli celý svět, že se znalost mezinárodních vztahů nedá uplatnit lépe?“ podivil se žertem Jie, který se právě na tuto oblast zaměřoval.

„Pro lidi, jejichž rodiče podrývají režim nikoliv. Tví rodiče jsou zatčení za na narušitelskou činnost. Šířili literaturu, které má být mezi lidmi jen málo.“

Ta zpráva Jieho překvapila natolik, že vyrazil do Pekingu s tak malým prodlením, jaké jen bylo bez porušení školních předpisů možné. Avšak omezoval se zbytečně. Ještě než stihl požádat o dočasné uvolnění, bylo mu oznámeno, že se studiem končí.

Jen co přijel do rodného města, navštívil svůj byt. Byl už prázdný. Vyrazil tedy na policii a snažil se na Mei s Shouwangem doptat. Nikdo mu však nechtěl nic říct. 

To byly tedy časy, během kterých Jie neměl ani rodiče ani budoucnost, zkrátka nic.  Právě v těchto chvílích však lidi pevné vůle oslovuje bohyně Edison. Nejinak tomu bylo s Jiem, který se nehodlal vzdát a vzal věci do svých rukou. Jako kdysi mě i jeho tehdy bohyně políbila a usídlila se v jeho mozku.

Následující půl rok Jie tvrdě dřel, třetinu času prospal ve svém rodném bytě, třetinu dělal číšníka v podniku, kde se scházela hodně špatná společnost a kde se porušovaly téměř všechny čínské daňové zákony. Nikdy později prý nedělal nic, co by se mu tolik příčilo. Nejen, že ho práce svou jednoduchostí takřka zabíjela (což ovšem rozhodně neznamená, že by mu šla – Jie se co chvíli pohroužil do vlastních myšlenek a úplně přestal vnímat hosty, které měl obsluhovat), ale také byl nucen obsluhovat ty největší ubožáky, jaké kdy potkal – gangstery a podvodníky, lidi toužící po bohatství a luxusu ale zároveň příliš líné na to, aby se ho snažili získat způsobem, který bude svět činit krásnějším, zábavnějším, svobodnějším nebo lukrativnějším místem.

Když zrovna nespal popřípadě neobsluhoval mrzké lumpy, trávil Jie svůj čas nad knihami pojednávajícími o historii Číny a o jejím mezinárodním postavení. Na základě nabytých poznatků pak sepisoval knihu, jakou by on sám dobrovolně nikdy ani nerozečetl: „Čínský socialismus jako cesta lidstva“, doufal, že pokud ji vydá, mohl by tak očistit své jméno, setkat se s Mei i Shouwangem a pak odjet někam do ciziny.

Práce začala měnit jeho pohled na svět. Čím déle psal, tím více si uvědomoval, že bude schopen vytvořit něco upřímného i prorežimního zároveň. Aby mohl srovnávat Čínu se zbytkem světa, musel se seznámit i s jinými státy, přečetl si tedy mnoho knih o Evropě, Austrálii, USA, Kanadě i Novém Zélandu a čím hlouběji se západním světem zabýval, tím víc jím byl znechucen.

Populismus, obrovský vliv velkých firem na politiku a především pohodlí povýšené na hodnotu. Jestli Jiemu něco na západním světě opravdu vadilo, pak to bylo masové vyznávání idey, že člověk žije především pro vlastní blaho. Mému synovi přesvědčenému o tom, že bychom se měli vyždímat ve jménu něčeho smysluplného, se taková životní filosofie jevila zcela zvrácená.… To vše vedlo k tomu, že když byl v půli svého původního projektu, začal znovu.  Svou novou práci pojmenoval: „Západní hodnoty aneb jak se rozmazlit a chcípnout“, tentokrát psal od srdce. 

Po necelém roce tvorby Jieho kniha bez problémů vyšla, podobně jako jeho další plány. 
Jakmile se první výtisky objevily na pultech knihkupectví, Jie si jeden koupil a vyrazil s ním v podpaždí na policii, aby požádal o možnost u ní pracovat.

„Vaše vůle se polepšit je zřejmě silná, ale když váš vychovatel zkoušel šířit nevhodnou literaturu, nemůžeme vás jen tak přijmout do státních struktur,“ mnul si bradu náborem pověřený policista.

„Kdo jiný by měl napravit špatné, když ne ti nejlepší?“ zeptal se ho Jie a zašmátral ve své aktovce po druhém esu, na jehož základě, chtěl být do policie přijat.

„Zaměstnanci policie, volba policie. Každopádně bych ještě rád nahlásil jednu putyku v Ganjiakou. Je to neregistrovaný podnik, ve kterém kvete černý trh, hazard a prostituce. Bližší podrobnosti jsem zpracoval v tomto spise,“ řekl Jie a položil před policistů štos papírů, ve kterých chyby svého chlebodárce podrobně rozebíral.

„Tyhle věci se mě netýkají,“ upozornil ho policista.

„Většinou ne, ale tahle ano. Vy totiž rozhodnete o tom, zda spisů jako je tento, napíšu víc, nebo ne.“

Policista si povzdechl a vzal podané papíry. Už po krátké chvíli uznale pokýval hlavou.

„Dobrá práce, doufám, že vám to vydrží.“ 

A tak Jie získal své vysněné zaměstnání. Policie podnik v Ganjiakou i jeho provozovatele rychle zavřela a krátce po něm následovalo i mnoho dalších, ač pouze sedmnáctiletý, začal Jie rovnou jako vyšetřovatel organizovaného zločinu a vedl si velmi dobře. 

Byl v těch dnech šťastný. I práce v komunistické policii mu dávala smysl. Nelítostně a tvrdě pronásledoval všechny, kteří se podle něj odevzdali své touze co nejvíc se obohatit a to zcela na úkor společnosti jakožto celku.

Schopnosti i motivace, jedno větší než druhé, vedly k tomu, že mu jen za několik málo let bylo nabídnuto místo v zahraniční rozvědce.

Jie se s dosavadní prací loučil jen těžko, ale nabídku přijal, neboť správně tušil, že jako pracovník zahraniční rozvědky bude mít u policistů větší respekt a tedy i větší šanci se konečně setkat s Mei a Shouwangem.
	
Už když poprvé přišel do své nové kanceláře, potkal zde menšího svalnatého chlapa se zaschlou krví za nehty.
	
„Dobrý den, kolego,“ uklonil se mu. „Jmenuji se Gang Xie a přišel jsem Vám předat instrukce k Vašemu prvnímu úkolu,“ na ta slova vrazil překvapenému Jiemu do ruky desky s papíry. „Doufám, že uspějete. Ten chlap zabil mého učitele Wuwanga. Vlastně víc než učitele, mistra. Když jsem před lety popravoval kolaboranty v dole, který Číně troufale odcizil, doufal jsem, že byl mezi nimi. Ale on unikl, šmejd jeden.“

Jie spisem chvíli listoval a pak se zhrozil. Měl zabít podnikatele, který „byl dříve zatčen za vědomou snahu bojkotovat vojenské zákroky proti narušitelům veřejného pořádku“, jehož fotky Jiemu velmi připomínaly vlastní obraz v zrcadle a který se jmenoval stejně jako jeho biologický otec.

Ne snad, že by pro Jieho  něco znamenal fakt, že je mé dítě, ale vyděsilo ho, že i z hrdiny, který si stoupne před tank, se může stát kapitalista – a kapitalisté byli tehdy v Jieho očích lidé, kteří oddávají svůj život honbě za penězi, neboť si chtějí užívat pohodlí a nejen že sami opomíjí, ale i své bližní odvádí předváděním nabytých statků od vytrvalé cesty vzhůru, která byla podle Jieho smyslem lidské existence. Není tedy divu, že byl odhodlán svůj úkol splnit tak, jak pro něj bylo obvyklé – na výbornou. 

Následovalo několik týdnů, během nichž se na akci připravoval a během kterých se po letech potkal se svými rodiči, které synova práce pro stát nijak nepotěšila (a to zamlčel, že povýšil na agenta rozvědky). 

„Kolik lidí jsi už připravil o svobodu?“ tázal se ho vyčítavě Shouwang.

„Hodně, ale byli to samí šmejdi.“ Shouwang jen zavrtěl hlavou.

„Brzy odjedu do ciziny. Rád bych se tě ještě před odletem zeptal, jak se jmenovala ta kniha, kterou jsi mě kdysi přiměl s vámi mluvit.“

„Tak pravil Zarahustra, ale ty bys potřeboval spíš něco výchovného.“

Jie se jen smutně usmál a se slovy „Zkusím zajistit, aby vás odsud dostali,“ se s nimi rozloučil. Jakožto dítě bohyně Edison dokázal, co si zamanul, a tak, i když to vyžadovalo uchýlit se k vydírání některých úředníků, se mohl ještě na letišti nechat obejmout od Mei.
