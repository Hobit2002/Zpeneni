\chapter{}

Troufám si tvrdit, že v celé historii lidstva nebyl na zahraniční misi vyslán agent, který by měl k Jamesu Bondovi dál než můj syn.

Jie neuměl střílet, mnohokrát se to zkoušel naučit, ale na vzdálenost přesahující pět metrů nikdy nezasáhnul střed terče o poloměru pět centimetrů a to ani po dvaceti hodinách cviku. Krom toho měl sílu jak průměrná dospívající čínská dívka. Zkrátka zcela nepřicházelo v úvahu, aby byl jeho atentát nějak akční. 

Začalo to velmi prostě. Nechal si vyrobit pár falešných dokladů, sbalil si drona s kamerou a odletěl do Livingstonu. Psal se už rok 2010, takže mu nedalo velkou práci si na internetu zjistit, kde se nachází má rezidence. Jakmile vystoupil z letadla, půjčil si auto a rozjel se za mnou.

Když přijel do obce Isimangaliso, Fetuovy rodné vesničky, ze které má smaragdová těžba a další podnikatelské aktivity udělaly velkoměsto připomínající Tel Aviv, nestačil se divit. Mohu se pochlubit, že i pozitivně.

Když Jie na cestu vyrážel, předpokládal, že Isimangaliso je má soukromá kolonie, ve které dřou statisíce vykořisťovaných domorodců a my, já a mí nejvyšší manažeři, je ždímeme ve svých dolech a luxusních podnicích. Realita však byla jiná. Isimangaliso nebylo místem pro byznys pohodlných sviní, nýbrž živým a rozmanitým ekonomickým ekosystémem, ve kterému na úspěchu jednoho velké podniku vyrostly stovky jiných malých a středních. Má firma totiž například mnoho peněz investovala do místních škol a z těch pak vycházeli nejen inženýři pro mé doly, ale i mnoho schopných a podnikavých lidí tvořících něco vlastního. Krom toho se Isimangaliso stalo jakousi africkou Amerikou – snílci z celého kontinentu sem přicházeli, aby si vybudovali nový život. Zkrátka chrám bohyně Ai.

Jie byl překvapen, že i město řízené primárně na základě západních principů, tedy volného trhu a plnění občanských přání, může být o dost vznešenějším místem než Čína, která se podle něj také zvrhla, neboť zavedla hrůzný „státní kapitalismus“, tedy režim, ve kterém se člověk pro zbohatnutí musel zavázat poslušností straně a vzdal se tak toho, co člověka šlechtí ze všeho nejvíc – možnosti stoupat vzhůru svou vlastní cestou.

To v Isimangalisu člověk pro dosažení svých snů do žádného zadku lézt nemusel, stačilo tvrdě dřít a nebát se experimentů. I lidsky nevnímavý Jie vycítil atmosféru poctivé a všudypřítomné práce. Pohled na malé trhovce, setkání se skromným hoteliérem, v jehož penzionu se ubytoval… ta nesčetná drobná iniciativa Jieho v dobrém fascinovala. Jeho odhodlání mě zabít nicméně přetrvalo.

První den strávil tím, že  kvadrokoptérou s kamerou několikrát obletěl mou vilu a divil se, že nikde nevidí žádnou ochranku. Když byl s tímto průzkumem hotov, zaparkoval vozítko na střeše naproti vstupním dveřím, nechal si na notebook nadále přenášet obraz z kamery, vytáhl si knihu Tak pravil Zarahustra a občas mrknul na dveře, zda náhodou nevycházím ven.

Nietzscheho kniha Jieho velmi oslovila. ‚Nikoliv pohodlí, ale nadčlověk – to by měl být náš cíl. Kéž by se mezi politiky našel někdo, kdo by si to myslel také.‘ přemítal nad ní.

Navečer jsem vyšel z domu a můj syn mohl sklidit plody svého vyčkávání. Velmi nerad odložil knihu a jal se mě dronem opatrně pronásledovat. Doufal, že vyrazím do nějaké restaurace, kde by se se mnou mohl potkat a při té příležitosti mi otrávit pití, ale zklamal jsem ho. Šel jsem totiž z města ven, do lesa, na okraj dolu. Tam jsem se posadil a dlouhou chvíli nedělal nic (toto meditování jsem okoukal od Fetua).  To Jieho nepotěšilo, při této činnosti by mohl leda tak zkusit mě zastřelit a to nepřicházelo v úvahu, jelikož já sám u sebe nosil zbraň a můj syn správně usoudil, že s ní asi umím zacházet lépe než on.

Zkoušel to ještě týden, ale žádné další místo, kam bych pravidelně vyrážel, neobjevil. Akorát ho překvapil můj zvyk v neděli sezvat na svou velkou zahradu zaměstnance firmy, vysoké manažery a pak pár náhodných lidí z nižších pozic a poobědvat s nimi. Byl to další čin, kterým jsem podkopal jeho představu, že každý kapitalista je nemilosrdný vykořisťovatel.

Rozhodl se tedy jít hlouběji a zaměřil svého drona na okna mého domu. Po několika dnech zjistil, že do kuchyně skoro pokaždé brzy ráno přijde Fetu (z jeho tehdejšího pohledu jen jako náhodný černoch), a otevře asi na půl hodiny okna dokořán, aby vyvětral, dokud je ještě venku snesitelně. Během větrání Fetu většinou v kuchyni posnídal, ale někdy se z ní vzdálil.  

Jie se toho rozhodl využít. Přidělal na svého drona otvorem vzhůru zkumavku, kterou z většiny naplnil čpavkem, jenž po té zakapal benzínem, který měl menší hustotu než čpavek a utvořil tak víčko přes které nebyla smrdutá látka cítit. Celou zkumavku pak ještě obalil fólií, která odrážela sluneční světlo a bránila tak přehřívání a vypařování směsi.

Takto vybaveným dronem několik dní po sobě přilétával k oknu mé kuchyně, až se konečně dočkal rána, kdy Fetu místnost opustil. Jakmile se mu to poštěstilo, vletěl otevřeným oknem dovnitř a rychle se zavrtal do odpadkového koše (trocha smetí se přitom dostala i do zkumavky, ale nebylo to tolik, aby ji to významně vyprázdnilo, popřípadě narušilo její ochranu proti brzkému vypařování čpavku).

Když přišla noc, vylétl dron z koše, trochu se nad ním oklepal, a začal šmejdit po bytě. Poměrně rychle objevil místnost, ve které jsem spal. Vedle postele jsem měl sklenici s vodou a můj syn už věděl jak mě sprovodit ze světa – dát při nějaké další příležitosti do zkumavky jed, po té drona nad sklenicí naklonit a nepozorovaně uletět pryč.

Právě onen nepozorovaný úlet ještě bylo třeba naplánovat. Jie tedy pokračoval ve šmejdění a získal poměrně solidní vhled do mého života.

Nepotkal jsem člověka, kterému na osobních záležitostech svých bližních záleželo méně než Jiemu, ale to, co viděl v mé vile, mi zachránilo život. Celý dům byl polepený fotkami mých dělníků, jejich dětí a Fetua. To všechno stačilo k tomu, aby Jie ráno svou návštěvu ukončoval s pocitem, že pravděpodobně nejsem vykořisťovatel ani ubohý otrok vlastního majetku, ale člověk žijící tím, co tvoří a dává druhým, pročež mě nemá smysl zabíjet. 

Snaha najít elegantní únikovou cestu nebyla úspěšná, a tak Jie použil pro tento případ použitý čpavek. Ještě než se dron zase zahrabal do odpadků, otočil se nad košem o sto vertikálních stupňů a vylil do něj obsah zkumavky. Benzín okamžitě ztratil roli víčka a celá místnost byla během několika minut plná hrozného smradu.

Když čtvrt hodiny po této operaci přišel dovnitř Fetu, rozkašlal se, strčil si ruku před nos, otevřel okno a vyběhl ven.

Jie sice Fetua na kamerách v odpadcích zahrabaného dronu neviděl, ale ze svých předchozích pozorování věděl, kdy by k Fetuovu příchodu a následnému útěku mělo dojít. V pravý čas se tedy s dronem vznesl z koše a vyletěl oknem ven.

Po všech fotkách, které v mém domě viděl, se Jie rozhodl, že se se mnou seznámí. Večer po šmejdění tak na mě čekal u mé vyhlídky.

Poznal jsem ho. Však jsem také od svého zatčení strávil desítky hodin dumáním nad možným vzhledem svého syna.

„Jie? Jak ses sem dostal?“ 

„Letadlem, autem a pak pěšky,“ odpověděl stroze.

„Nevypadáš, že by ses chtěl vybavovat. Ale já potřebuji vědět, co tě sem přivádí a jak jsi zjistil, že mě tu potkáš?“

„Viděl jsem…“ v tu chvíli se zadrhnul.

„Jen mi klidně tykej,“

„Viděl jsem tě tu. Bližší podrobnosti ti sdělím, až mi povíš, co jsi dělal po svém zatčení.“

„Milerád!“ zvolal jsem a dal se do vyprávění.
