\chapter{}

„Co tu chcete?“

„Přišel jsem si pro svůj podíl.“

„Mluvte jasněji, idiote,“ osočil se na mě voják, který mě zadržel po té, co jsem přišel až k plotu objektu, svíraje Gangovu průkazku v ruce.

„Prostě chci mluvit s člověkem, který to tady řídí. Dostal tu funkci odměnou za náš společný podnik, ve kterém jsem já ostrouhal mrkvičku a rád bych si to s ním vyříkal.“

Voják pokynul dvěma ze svých kolegů, kteří mě následně vtáhli za plot a pořádně prohledali. „On se mstít nebude, nemá čím, ale dost možná je blízko odstřelovač,“ zhodnotil jeden z nich výsledek prošacování. Na ta slovy mě vojáci popadli a odtáhli do zcela prázdné betonové místnosti uvnitř střežené budovy.

Tento počin nám zhatil plán A, který spočíval v tom, že až se Jie objeví venku, začne Wobuciko házet do objektu kameny čímž spustí poplach a zmatek a já zatím proniknu někam dovnitř a zničím, co zničit půjde. (I v tomto ohledu byl Wobucikův vztah s Lydií přínosný, neboť na přímluvu své milé byl  voják ochoten jednat i proti pravděpodobným zájmům svého ‚generála‘.) Naštěstí tu byl ještě plán B, ale o tom až později.

„Kdo jste?“ otázal se mě jeden z vojáků.

Mlčky jsem mu podal Gangovu legitimaci. Voják si ji vzal, prohlédl a následně s ní odešel.

Chvíli po té se vrátil s pobaveně vyhlížejícím velitelem. Jaké bylo mé překvapení, když jsem zjistil, že se vůbec nejedná o Jieho.

„To by mě zajímalo, co byl ten podnik, ve kterém jste ostrouhal mrkvičku. Upřímně by mi nevadilo tuhle klícku na míle vzdálenou jakékoliv civilizaci někomu přenechat a neudělám to jen proto, že ti by mě šoupli do něčeho ještě horšího a nechali mě tam až nadosmrti.“

Nevěděl jsem, co říct a nakonec jsem ze sebe vykoktal pouze následující slova:“Ppardon, tto bylla mýlka.“

„Jak jinak, teď jste ovšem mým zajatcem a budete mi, kapitáne Gangu, muset o své, zřejmě dramatické, minulosti něco povědět. V téhle díře aby člověk chcípl nudou. Ještě, že mám dobrou VPN, cenzurovaný čínský internet by na dlouhé chvíle, jaké mě tu trápí, ani náhodou nestačil.“

Po těch slovech mě chytil za ruku a začal odvádět do své kanceláře. Po cestě jsem měl možnost přes několik prosklených zdí nahlédnout do vnitřní výrobní haly a zcela jsem se utvrdil v tom, že tenhle objekt vyhladí leda tak milovníky poctivé Coca-Coly. Na pohyblivých pásech totiž nejezdilo nic než stovky lahví čínské napodobeniny americké limonády, které byly plněny stroji a ukládány do beden.

„Jsme správná čínská fabrika. Naši lidé deset let pracovali pro Coca-Colu americkou, až se tu limonádu naučili dělat tak dobře, že mohli začít dělat Coca-Colu čínskou,“ dodal velitel objektu na vysvětlenou.

„A proč potřebujete tuhle fabriku na slazenou vodu hlídat desítkami vojáků, kteří by mohli někde měnit svět?“
„Od té doby, co se Donald Trump stal americkým prezidentem, to jde s čínskými zahraničními vztahy z kopce. Zatím vedeme jen několik obchodních válek, ale začínáme se už pomalu chystat i na střetnutí vojenské. Tento objekt byl postaven jako bludička, na kterou snad Američané v případě vojenského střetnutí zbytečně vyplýtvají pár bomb, které by jinak použili proti Třem soutěskám a krom toho jsme vůbec první ryze čínskou fabrikou na Coca-Colu, takže se nám dostává náležitého zabezpečení.“

„Na nic z toho ale snad nepotřebujete desítky vojáků?“

„Však jich tu máme jen minimum. Většina z těch strážců jsou akorát budižkničemové, kteří se k ničemu lepšímu než monotónnímu pochodování s airsoftovou zbraní nehodí.“

Když tento hovor skončil, byli jsme už v jeho kanceláři a já mu začal v první osobě vyprávět Gangův příběh, který byl pravdivý až do té části, kdy jsem měl přijít o nohy. Namísto toho jsem akorát řekl, že Jie nás po té, co přebral velení, nechal uvěznit a vrátil se do Číny. Celá mise byla v mém podání jen jeho cestou k tomu, jak nás šikovně uklidit, což se mu ovšem v mém příběhu nepovedlo, jelikož nechal ve vojenském táboře opuštěnou svou švédskou milenku, která se brzy začala cítit osamělá, a tak se důvěrně spřátelila se mnou. Když už jsem na ni měl nezanedbatelný vliv, seznámil jsem ji s dozorcem, který mě hlídal, a ona se mu podle očekávání zalíbila. Aby získal její srdce, rozhodl se splnit i její nejodvážnější přání, a tak mě nejen propustil, ale také doprovodil na cestě za pomstou. 

Tento příběh vojenského správce fabriky zaujal natolik, že mě požádal, abych ho seznámil i se svými společníky. Vyšel jsem tedy v doprovodu několika vojáků ven a Lydii i Wobucika si k sobě zavolal.

Cestou do kanceláře jsme mlčeli a veškerou pozornost věnovali našemu hostiteli, což byla dost možná osudová chyba. Kdybychom se aspoň v náznacích domluvili, co budeme (a hlavně nebudeme) uvnitř objektu dělat, mohly by být následující události podstatně jednodušší. Chybějící komunikace ovšem způsobila, že Wobucikovi nepřestal počítat s realizací útočného plánu B.

Jakmile nás správce objektu přivítal ve své kanceláři a vyhnal vojáky ven, přišel čas na řádný asijský pozdrav – úklonu. A právě tu Wobuciko přehnal a svou hlavou udeřil našeho hostitele do té jeho takovým způsobem, že dotyčný ztratil vědomí. 

Úplně jsem oněměl. Netušil jsem, že Wobuciko nevyhodnotil všudypřítomnou výrobu Coca-Coly jako důkaz toho, že tady se asi biologické zbraně nevyvíjí a tím pádem nemá smysl snažit se v pravou chvíli násilím zlikvidovat dozorce a pustit se do rozbíjení.

Lydie byla naštěstí pohotovější a Wobucika, který už otvíral dveře ven, aby se vrhnul i na tu druhou část, zastavila. „Počkej Wobíku, trochu se změnila situace, raději hlídej toho chlápka, kterého jsi už odrovnal.“

Ona sama se mezitím usadila k puštěnému počítači dozorce a začala se něčím zuřivě proklikávat. „Dongu, pojď sem! Já neumím psát čínsky,“ zasyčela po chvíli.

Rychle jsem přiskočil a zjistil, že Lydie vlezla do hostitelova WeChatu, našla v něm seznam jeho přátel a mezi nimi objevila čínského důstojníka, který na své úvodní fotce stál vedle Jieho a týmu lidí v bílých pláštích. „Pozvi je jménem našeho hostitele na dnešní večeři do některé z místních restaurací,“rozkázala mi a sama se chopila chytrého telefonu, který ležel na stole. 

Přihlásit se nezvládla, a tak ho aspoň resetovala, aby měl nebohý důstojník co na práci, než zjistí, že přes jeho WeChatový účet probíhala velmi podezřelá komunikace s jeho známými.

Ze stejného důvodu jsem měl i já, po té co jsem večeři domluvil (bohužel jsem ji musel posunout o den), rozeslat důstojníkovým přátelům zprávy, po jejichž přečtení by bez důkladného vyříkání přáteli dlouho nezůstali. (Lydie například navrhovala, ať nějaké zjevně mladé dívce napíšu následující zprávu: „Pokud vím, stále se někde s někým couráš a nemáš se k ničemu užitečnému. Jaké štěstí, že dnešní doba nabízí velmi lukrativní kariéru i takovým jako jsi ty“ a k tomu odkaz na neblaze proslulý nevěstinec a dovětek „nezapomeň se mi pochlubit, až tě přijmou.“) Doufala, že tak na nešťastného důstojníka přivolá mohutnou bouři rozzlobených reakcí, ve kterých se domluva večeře  úplně ztratí.

Tento příkaz jsem ovšem neposlechl. Nehodlal jsem ničit vlastně docela hodnému důstojníkovi jeho mezilidské vztahy. Zůstal jsem tedy pouze u zpráv typu: „Nevíš, co bych mohl dát tvému manželovi k narozeninám?“ a “Rád bych se s tebou někdy potkal, kdy by se ti to hodilo?“ 

V průběhu tohoto spamování se důstojník začal probouzet a zvedat hlavu, ale Wobuciko mu ji klidně a jistě přimáčkl k zemi takovým způsobem, že zvědavý důstojník znovu upadl do bezvědomí. „Už je čas jít, ať toho chudáka neumučíme,“ ozval jsem se, vida tento čin.

„Kolik jsi odeslal zpráv?“ otázala se Lydie.

„Asi padesát,“já na to.

„A kolik má přátel?“

„Sedm set.“

„Pak tedy netuším, kam chvátáš.“ Pokusila se mě u té ohavnosti udržet, ale to už byl počítač vypnutý a já na odchodu. Z komplexu jsme se dostali překvapivě snadno. Na chodbě se nás sice ujal voják a dohlédl na náš odchod, ale nikdo naštěstí nekontroloval svého nadřízeného.
	
Příští den večer jsme se, zase v rouškách a slunečních brýlích, vydali do smluvené restaurace a usedli do rohu. Generál s Jiem přišli právě na čas a usadili se tak blízko k nám, že bych po nich i já mohl hodit slánku a trefit se. 

Náš plán jak čínské papaláše odrovnat byl ovšem jiný a přiznám se, že jsem se kvůli němu celou noc modlil. Má modlitba nebyla jen tak ledajaká, mnohokrát jsem při ní užil stejná slova jako kdysi Ježíš v Getsemane:“Pane, odejmi ode mne tento kalich, ale staň se tvá vůle, ne má“. 

 „Takže až zavelím, vrhne se Dong zezadu na Jieho a začne ho škrtit. Wobuciko mezitím projde kolem toho důstojníka, omráčí ho a předá Dongovi jeho zbraň, následně sám uteče do míst, kde jsme spali. Dong bude zpovídat Jieho ohledně toho holocaustu a hodně nahlas opakovat, co řekne. Až přijede policie, začne vyhrožovat, že Jieho zastřelí, jakmile byť jen vytasí zbraň. Já budu poblíž odposlouchávat Jieho přiznání, pak se vytratím, Donga nechám na pospas policii. V lese se však setkám s Wobucikem a společně pak v noci Jieho plány překazíme, ať už to bude znamenat cokoliv,“zrekapitulovala Lydiie plán, který dala večer bez zvláštního přemýšlení dohromady (zajímal ji totiž jen Wobuciko a to, že potřebujeme dát dohromady nějaký plán, jsem jí musel mnohokrát připomenout).
 
„Dejte mi prosím ještě deset minut, chci se pomodlit,“ požádal jsem své druhy. Jejich plán se mi pochopitelně nelíbil, ale nic lepšího jsem nevymyslel a byl jsem jen prostý služebník Boží. Jie se prokázal jako extrémně nebezpečný šílenec a definitivní překažení jeho plánů za sebeoběť jednoznačně stálo. Krom toho jsem se neměl, čeho bát. Po smrti na mě čekalo Boží království a úžasná komunita, na níž už jsem se nemohl dočkat.

Měli jsme veliké štěstí, že k nám Jie se svým kolegou seděli z Boží milosti otočeni zády a jejich oči se obracely ke vchodu, kde vyhlíželi vojenského ředitele fabriky (kteréžto čekání jim nejspíš začalo být dlouhé, neboť si objednali k jídlu pár netopýrů) a nevnímali tedy ani zdaleka tolik námi vydávané zvuky – konkrétně Wobucikvo šeptání. Kdyby se do něj totiž zaposlouchali pořádně, došlo by jim, že tlumočí jejich hovor.

 „Tohle působí poněkud sovětsky, mám obavy, zda v tom není nějaký Stauffenberg.“ „To spíš něco mezi Eisenhowerem a Turingem“vyměňovali si své názory, když náhle člověk, na kterého čekali, přišel.

Zprvu jsme doufali, že se mu někdo z pozvané dvojky ozval, aby si ověřil, že je vše OK a námi omráčený si to vyložil jako iniciativu na druhé straně. Žel Bohu, nebylo tomu tak. Důstojník se nejprve pořádně rozhlédl po vnitřních prostorách restaurace, až uviděl nás.

Neváhal a svižnými kroky za námi vyrazil. „Tak je to Turing,“ zaznělo od Jieho stolu. To už na nás ale ještě včera velmi milý a slušný Číňan mířil pistolí a vyzýval nás, ať okamžitě vylezeme a vydáme se policii, která už venku čeká.
Ještě než jsme stačili cokoliv udělat, udeřil Wobuciko důstojníka pěstí a vyrazil mu jak zbraň, tak dech. Následoval další chvat, po kterém útočník proletěl několik metrů vzduchem a při pádu porazil stůl, u něhož zrovna seděla nějaká zamilovaná dvojice.

„Vůbec nemáš důvod jim pomáhat, jenom tě využívají. Ta holka s tebou vůbec nic mít nechce, vždyť jsi jen řadový voják s příšernou minulostí. Ve skutečnosti má vztah k tomu hubenému nekňubovi a tebe, stejně jako Jieho, jen zneužívá. A zkus se zamyslet nad tím k čemu, obrátila tě proti vlastnímu veliteli.“ Volal oficír, jakmile se začal po dopadu zvedat.

Tato slova Fetua, který se už chystal vrhnout na Jieho s generálem, zarazila a on začal střídavě házet tázavé pohledy tu na mě, tu na Lydii, tu na důstojníka a tu na Jieho.

„Nevěř mu, co o tom může vědět?“ vložila se do věci Lydie.

„Kdepak, má pravdu,“ začal jí oponovat Jie „jen zkus zapřít, co všechno jsi se mnou měla. Zkus zapřít cestu kolem světa, jíž jsem ti dopřál i luxusní vilu, v níž jsi díky vztahu se mnou mohla bydlet. Ale abys mi byla věrná, to ne. Už brzy po té, co jsem odletěl z Afriky plnit své povinnosti v Číně, dostalo se ke mně, jak se po večerech scházíš s Dongem a šuškáte si spolu.“

„To není pravda,“ vyhrkli jsme s Lydií naráz.

„Co jiného by teď mohli tvrdit?“ položil Jie Wobucikovi řečnickou otázku. „Jen tak mimochodem jaká je vaše hodnost?“ přidal k ní otázku druhou, tentokrát odpověď očekávající.

„Vojín, pane“ odpověděl Wobuciko a dokonce i zasalutoval.

„Dokázal jste ovšem velkou věc – dostat tyto proradné živly, které by mohly tropit veliké škody, až sem. To je úctyhodný výkon, a pokud jej dotáhnete do konce, zasloužíte si povýšení na majora. Ostatně už nyní si zasloužíte být minimálně nadporučíkem, takže vás prosím, nadporučíku, odveďte své zrádné přátele do bezpečí nedalekého hotelu, odkud je pak zítra deportujeme zpátky do Vděčné země.“

Velkorysost tohoto rozhodnutí všechny přítomné překvapila a to spíše nemile. Před restaurací už čekala policie, jíž jsme mohli být vydáni a Jie se rozhodl, že nás ubytuje v hotelu.

„Proč? Vždyť to nedává naprosto žádný smysl! “ rozčílil se okamžitě kdysi tak milý důstojník.

„Ti lidé jsou ministry Vděčné země, nad kterou teď drží ochrannou ruku nejeden ze států NATO. Byl bych strašně nerad, kdyby spravedlivý, avšak pomstychtivý hněv jednotlivců způsobil Číně problémy, sami jistě víte, jak to v místních věznicích chodí a víte i to, že není vhodné, aby to věděl celý svět,“ obhájil své stanovisko Jie.

Tento argument se mi zdál dost uvěřitelný na to, abych si přestal lámat hlavu s tím, jaké pikle na nás zas zrádný exministr financí kuje.

Hotel, do něhož jsme byli přivedeni, za mnoho nestál, ale ve srovnání s policejní vazbou, obvykle plnou pláče, práskajících obušků a skřípění zubů, to byl hotový luxus. 

„Zcela chápu, že mi nevěříš, Dongu, ale buď si jist, že mi z globálního hlediska nejde o nic než záchranu mnoha lidských životů. Támhle v hale,“ při těch slovech Jie ukázal z okna na velikou budovu postavenou zřejmě k tomu, aby sloužila jako skladiště “ve dne v noci poasijsku tvrdě dřou laboranti a vědci, kteří hledají konečné řešení pro mor, který může již brzy pohltit celý svět.“

„Mor?“

„Ten, který zabil tvou ženu. A pokud chceš lépe porozumět tomu, co se poslední dobou dělo, přečti si moje poznámky, které ti před chvílí přišly na email,“ po těch slovech a odešel a já si na mobilu otevřel textový soubor, který jsem od něj nedávno obdržel. Naším hlídáním byl pověřen Wobuciko.

 Já však zrovna takovou potřebu nepociťoval, neb jsem se začetl do Jieho „poznámek“. Jak jsem zjistil, jednalo se o spíše o velmi nepravidelně psaný deník. Jestli si nejpodlejší ze všech zrádců myslel, že si mě svým příběhem aspoň trochu udobří, tak se šeredně spletl. Ani Gangovo zacházení s vězni mě neznechutilo tolik jako myšlenky a činy toho hada. Celý Jieho život byl úplně zvrácený, pojmy jako zásada nebo svědomí pro něj nic neznamenaly. Zajímalo ho jedině to, zda se blíží k dosažení svého cíle. Zajisté, přírodní katastrofě by se zabránit mělo, ale člověk nesmí nikdy ztratit ze zřetele, že na první místo patří kvalita vztahů v mezilidské komunitě – církvi. Přírodní pohromy, války a revoluce, slouží k tomu, aby se v nich lidé semkli a byli si na konci krize bližší než na jejím začátku. Jie, který pro dosažení svého cíle vraždil, lhal a hrabal se v lidské nátuře, byl ukázkou kolosálně špatně pojatého světa.
 
Zatímco já se pohroužil do nejohavnějšího příběhu, s jakým jsem se kdy setkal, vydala se Lydie za Wobucikem, aby si ho usmířila. Nebylo to ovšem vůbec snadné. Wobuciko byl pobouřen tím, co se o ní od Jieho dozvěděl, a když se k němu přiblížila poprvé, kopnul ji tak silně, že se se zaúpěním složila na podlahu. Rozhodně se ovšem nevzdala, po předcích nucených přežít v chladné a neúrodné Skandinávii zdědila tuhý kořínek, díky kterému v sobě našla sílu zvednout se a pokračovat ve vyjednávání i po té, co byla sražena desetkrát.  

Sotva jsem mohl její snahy nevnímat a trýznilo mě pozorovat, jak Jie svým lhaním zvrátil kdysi krásný vztah do brutality. Ona zkáza pozorovaná na vlastní oči dovršila beztak hluboké znechucení jeho deníkem a vedla mě k rozhodnutí, že takového člověka nesmím v žádném případě nechat vyhrát, a to i kdyby všechny jeho křivárny měly nějaký bohulibý cíl. Zbývalo jen vymyslet, jak na to.

Bůh stál při mně. Cesta ven se ukázala, jakmile jsem zamyšleně vykouknul z okna – o patro níž se svítilo a dalo se tedy předpokládat, že bych se z bytu, který byl pode mnou, bez problémů dostal na chodbu.

Jelikož mě nic lepšího nenapadlo a už jsem si zvykl, že je můj život plný životu nebezpečných vylomenin, vlezl jsem do koupelny, chytil konec sprchy a vyskočil s ním z bytu ven (v koupelně byla velká poloprůhledná okna, která myjícím se umožňovala, aby se mohli kochat městským životem, aniž by je to stálo soukromí).

Bůhvíjak dobrý nápad to tedy nebyl. Hadice se napjala až k prasknutí a stejně byla příliš krátká na to, aby se na úroveň bytu o patro níž dostalo něco víc než špičky mých nohou. A tak mi nezbylo, než začít do okna vší silou kopat.

Milost Boží mě naštěstí neopustila – ve sprše někdo byl (což jsem poznal podle toho, že jsem ještě dřív než břinkot rozbitého skla uslyšel vyděšené zapištění), a tak mě už brzy dva k smrti vyděšení manželé pomohli na svou okenní římsu.

Poděkoval jsem jim, ale na vysvětlování nebyl čas. Wobuciko, který během mého útěku marně odrážel Lydiiny snahy o usmíření, už mi byl dost možná v patách. Okamžitě jsem tedy opustil byt svých zachránců a dobře udělal. Wobuciko se domníval, že se snažím schovat o patro níž. Zamkl tedy Lydii, vpadl do podezřelého bytu a jal se ho prohledávat. Já mezitím pádil dolů po schodech pryč z hotelu.

Jen co jsem se podruhé ocitl na čerstvém vzduchu, vyrazil jsem k hale. Bylo očividné, že se zde něco děje. U každého z několika různých vchodů byla sklopená závora a u ní několik ozbrojených strážců. Chvíli jsem přemýšlel, že do závory najedou autem, ale usoudil jsem, že i tak by byly mé šance na přežití srážky malé. Krádež a následné zneužití taxíku jsem tedy odsunul na pozici plánu B a vrátil se k hotelu.

„Možná mě tu bude hledat jeden mohutný černoch, tak mu prosím vyřiďte, že jsem šel támhle, do té haly,“požádal jsem recepční a znovu hotel opustil.

Učinil jsem tak na poslední chvíli, již o pár minut později jsem i ze vzdálenosti několika desítek metrů slyšel hřmět Wobucikův hlas: „Kde je ten lump?“. 

Recepční mě nezklamala. Již o pár vteřin později se Wobuciko řítil k nejbližšímu vstupu do haly a neuběhla ani minuta a statný superovják už se se strážci velmi ostře hádal. Dopadlo to tvrdě. Wobuciko ochranku objektu rozházel po parkovišti a nelítostně pokračoval hlouběji do objektu.

Já se zaradoval, že můj plán vyšel, chopil se pistole, která jednomu z hlídačů v letu vypadla a vběhl za Wobucikem dovnitř. Že by bylo uvnitř příjemně se tedy rozhodně říct nedá, Wobuciko se rval s několika dalšími bezpečáky a kolem panicky pobíhala spousta vyděšených pracovníků. Využil jsem zmatku a nikým nezastaven se dostal až k velkým bílým dveřím na vzdáleném konci vestibulu.

Když jsem je otevřel, octnul jsem se v obrovské hale plné laboratorních stolů a skleněných skříní s netopýry a jinými drobnými savci. „Takže i zvířata musí trpět Jieho zvráceností?“ řekl jsem si v duchu a vydal se k teráriu maje v úmyslu dopřát jim svobodu.

Abych Jieho plány překazil co nejvíc, shodil jsem vše z každého laboratorního stolu, který jsem po cestě minul, na zem. Nejspíš to ale velké škody nenadělalo, jednalo se o samé petriho misky a mé přání vyvolat požár reakcí náhodných chemikálií tak zůstalo nesplněno.

Zklamáním pro mě bylo i terárium. Vůbec jsem netušil, jak ho otevřít a lámal jsem si s ním hlavu tak dlouho, až bylo pozdě. Náhle totiž zazněl výstřel a má ruka zkoumající sklo terária byla děravá. Otočil jsem se a uviděl Wobucika jak stojí u vchodu a už podruhé mačká spoušť pistole. Okamžitě jsem se vrhnul na stranu a rána tak zasáhla akorát terárium. Už prvním výstřelem poškozené sklo se definitivně rozsypalo a do vzduchu vzlétlo celé hejno netopýrů.

Vzhledem k tomu, že mi šlo o život, jsem se však z tohoto úspěchu ani neradoval a místo toho se skrčen mezi stoly snažil utéct Wobucikovi. Brzy jsem však začal ztrácet naději. Zatímco já, shrbením zpomalen, pobíhal křížem krážem mez laboratorními stoly, Wobuciko skákal po nich a měl díky tomu poměrně dobrý přehled o celé hale. Každý laboratorní stůl na sobě měl položenou skřínku s pomůckami a preparáty a tato skřínka na něm po Wobucikově návštěvě nikdy nezůstala. Tímto způsobem zvládl můj žárlivý pronásledovatel nadělat během minuty řádově větší škody než já, když jsem cíleně shazoval obsah skříněk na zem. Za své vzalo i několik velkých skříní se zvířaty, které Wobucikovi překážely v cestě a hala se tak začala již brzy plnit netopýry (naneštěstí ty potvůrky brzy objevily cestu ven a přestaly Wobucikovi stínit ve výhledu).

Nakonec mě dostal. Nejprve mě střelil do nohou a znemožnil mi tak dále utíkat. Padl jsem k zemi a začal uvažovat, zda se nemám pokusit Wobucika zastřelit, ale usoudil jsem, že vražda se nedá ospravedlnit ničím, a tak jsem zbraň akorát zahodil, aby mě nepokoušela.

To už mě ale Wobuiko dopadl. Pevně mě chytil a začal mou hlavou rozbíjet skleněné tabule laboratorních skříní. To však jeho žárlivost nedokázalo ukojit a tak mě po chvíli prohodil stěnou jednoho z terárií. Obličej se mi zalil krví, a tak jsem pořádně neviděl, co se dělo dál.

Vím akorát to, že mě Wobuciko znovu zvednul a začal někam vláčet, když tu se ozvala další rána a on padl k zemi.
Bezvládně jsem lež a vyčkával, co se bude dít. Dočkal jsem se, nejprve jsem uslyšel kroky a po té se na mě snesla celá bouře vzteklých kopanců. „Ty idiote! To, co jsi provedl, bude stát životy stovky, ne-li tisíce lidí!“ začal na mě dotyčný řvát a já poznal, že se jedná o Jieho.

Nezbývalo mi než se pousmát. Misi jsem splnil. Svět jistě potřeboval ochranu před hrozbami všeho druhu, ale prohnaný a ostroloktý Jie nebyl způsobilý k tomu se na ní podílet. 

Po několika minutách kopání mě několik členů objektové ochranky zvedlo a setřelo mi krev z obličeje. A já viděl, že jsem zvítězil – Jie plakal.
.
