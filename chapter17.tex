\chapter{}

„Z Evropy moc pravověrných křesťanů neznám, ale vy to zřejmě berete dost vážně. Myslím, že snaha osvobodit nespravedlivě uvězněné by pro vás měla být stejně samozřejmá jako snaha pečovat o raněné,“ prohlásil můj odvážný pacient, jakmile jeho příběh skončil.

„To je na čínské poměry až příliš velké dobro. Zachránit vězně je moc těžké. Pokusy jako ten tvůj dopadají většinou tak, že je na konci v lepším případě jen o jednoho vězně víc.“

„Když se dá dohromady kompetentní tým s dobrým vůdcem, pak není těžké nic. Je ta vaše podzemní církev vůbec nějak organizovaná? “

„Ne bratře, církev je společenství lidí shromážděných kolem Krista, není třeba žádné organizace. Všichni jdou správnou cestou, a ta je úzká takže se na ní potkají.“

„Živoříte v pekle, na to, abyste žili, nemáte dostatek svobody. Komunistický režim vás likviduje, a i když se vám zrovna podaří uprchnout do ciziny, nebudou vás tam chtít. Přitom vás je tolik, že kdybyste se dali dohromady a začali pořádně snažit, mohli byste dosáhnout změny, o kterou bezpochyby prosíte ve svých modlitbách.“

Že řekl právě toto, jsem se dozvěděl až zpětně. Tehdy jsem však se svou mizernou angličtinou rozuměl pouze slovům peklo, komunistický režim, chtít a snažit. Odvětil jsem tedy: „Každý z nás dělá, co může, aby podryl moc toho ďábla, zosobněného prezidentem Si Ťing-pchingem.“

Sotva jsem větu dokončil, Jiří spustil nejdelší improvizovaný projev, jaký jsem kdy slyšel.

„Tenhle svět se řítí do záhuby a nikdo s tím nic nedělá! Všichni se bojí ztráty pohodlí, a tak se děje jedno velké nic.
Cesta však vede jinudy, a to skrze bláznovství a velké oběti. Lidé musí začít konat to nejlepší, co si dovedou představit. Vyrazil jsem do Číny, abych je k tomu inspiroval. To, že jste mi vyrazil na pomoc, vás řadí mezi první z nich. 

Vy křesťané se bojíte směřovat za velkými cíli, protože k jejich dosažení potřebujete přesně ty vlastnosti, jejichž opaky blahoslavíte. Já jsem sice z národa agnostiků, ale na střední škole jsme Bibli a evangelia probírali, takže vím, že vámi oslavovaný Ježíš ohlašoval příchod Božího království na tuto zemi. Jeho učedníci, pokud vím, doufali, že se dožijí jeho slavného sestupu z nebes, a když se ho nedočkali, usoudili, že naopak jejich duše vystoupají nahoru. A přitom ono slibované království, které je v tvém pojetí komunitou, v níž každý člověk tvoří něco, z čeho mají užitek i příslušníci mladší generace, pročež se jeho dílo, on sám, práce předává dál a dál a jemu se tak dostává nesmrtelnosti, může bez problémů existovat tady na zemi. Chce to, stejně jako každý jiný velký cíl, jen drzost, optimismus, chytrost a další záviděníhodné přednosti, které zavrhujete. My sami se sotva dožijeme chvíle, kdy již nebude jiné říše, ale můžeme být jejím výhonkem.

Abychom však tento cíl naplnili, musíme riskovat a vzepřít se Číně.“

Kdybych rozuměl tomu, co Jiří říká, byl bych zděšen jeho kacířstvím – především tím, jak zdůrazňoval důležitost tohoto světa a ten posmrtný posouval kamsi do nedůležitosti. Ovšem gesta, výrazy obličeje a tón hlasu, i přes nemalé odlišnosti evropské a čínské řeči těla Jiřího sdělení dodaly neuvěřitelné neodolatelnosti. 

Pochopil jsem, co pro něj jeho cíl znamená, a usoudil, že zápal, jako je tento, odjinud než od Boha pocházet nemůže. Ještě než domluvil, padl jsem na kolena, a když řekl své, dodal jsem, co jeho promluvě chybělo: „Amen“.

A amen znamená „staniž se“.
\vspace{0.75cm}

Ještě před obědem si pro evropského dobrodruha přijel Jie. „Tak to jste mi ušetřili práci,“ prohlásil, když zjistil, že už jsem se započetl mezi Jiřího kolegy a dokonce i sestavil seznam svých známých z podzemní církve, kteří by se ještě mohli zapojit.

Následoval Jiřího přesun do Šanghaje a budování týmu. O to první se postaral Jie, do druhého jsem se zapojil i já. Na týden jsem se uvolnil z práce a vyrazil na velkou cestu po Číně, během které jsem navštívil bezmála stovku svých známých a snažil se je přesvědčit ke spolupráci. Nepřidal se nikdo.

Had byl úspěšnější. Pomocníků sehnal asi padesát.

Vůbec netuším, jak to dokázal. Ale když jsem se vrátil do svého bytu, našel jsem v něm flešku s tabulku obsahující desítky kontaktů na sestry a bratry, kteří se rozhodli jít do akce s námi. Překvapilo mě, že všichni pocházeli z Šanghaje.

Několik následujících měsíců nám zabrala příprava. Upřímně nevím, jak moc se na ní podílel Jiří, ať už se totiž dělo cokoliv, byl za tím vidět akorát Jie. Ani teď, když vím, jak strašný člověk to byl, nemohu jeho přípravě téměř nic vytknout.

Samozřejmě, že veškerá domluva musela běžet přes sociální sítě. Naprosto nepřicházelo v úvahu, aby se sjely osoby, na které si režim často dával zvlášť velký pozor, a začaly řešit, jak osvobodit z vězení zahraniční novinářku obviněnou ze špionáže. Něco takového by četným kamerám a jiným sledovacím prostředkům samozřejmě neušlo. WeChat, čínská alternativa k Facebooku, ovšem lidem nenechával soukromí o moc víc.

Co na to Jie? Sepsal rozsáhlý manuál toho, která slova mají být v internetových konverzacích nahrazena kterými, aby pak příprava vniknutí do věznice vypadala jako domluva herních strategií do League of Legends. Tento manuál pak na své domácí tiskárně mnohokrát vytisknul a při bohoslužbách podzemní církve jednotlivé kusy rozdal tak, že z ruky do ruky doputovaly až k jeho kolegům.

A tak jsem se i já mohl poměrně brzy začít angažovat při plánování útoku. Prvních pár dní jsem strávil u WeChatu hodiny a snažil se do diskuze přispívat rozumnými návrhy, ale brzy jsem si všiml, že se do konverzace zapojuje hned několik analytiků Jieho formátu, vedle kterých jsem byl bezvýznamný prosťáček neschopný přijít s čímkoliv zajímavým.
Dlouho jsem tedy pro osvobození té novinářky nedělal vůbec nic, když se mi v prosinci roku 2019 ozval Jie a žádal mě, abych za ním vyrazil do Šanghaje.

Poslechl jsem, přijel jsem a zhrozil jsem se. To, s čím Jie a jeho analytický tým přišel, bylo ohavné. Plán, který se v tom démonickém mozku zrodil, nejlépe nastíním popisem jeho realizace.

V den akce mi Jie dal policejní uniformu a pokyn, abych mlátil do svých sester a bratří v Kristu. 

Skupina asi čtyřiceti křesťanů podzemní církve uspořádala protestní bohoslužbu přímo před branami věznice Tilanqiaa, kde byla Lydie uvězněná. Čínská policie se s demonstranty nikdy nemazlila a podle očekávání již velmi brzy a velmi tvrdě zasáhla. Několik jednotek z celého města se sjelo a nešetříce ranami, které bití přijímali víc než pokřesťansku vstřícně, začalo nahánět křesťany přímo na vězeňský dvůr.

A tehdy přišla má chvíle. Měl jsem se k násilnickým strážcům pořádku přidat a hnát nevinné a bezbranné křesťany až na vězeňský dvůr.

Jie předvídal, že ani tak velká věznice jako Tilanqiao, není připravená pojmout naráz padesát nových zadržených a že když dozorcům učiním nabídku (doplněnou lživým tvrzením, že mi to mí nadřízení dovolili) pomoci s tou nenadálou vlnou, vděčně ji přijmou.

 Nevím. Nestihl jsem to vyzkoušet. Do bití vězňů jsem se zapojoval jen s velkou zdrženlivostí, a tak jsem byl také jedním z posledních, kteří do vězeňské brány zamířili.
 
Dělilo mě od ní sotva deset kroků, když vězeň tažený dvěma policisty za ramena dovnitř náhle vykulil oči a začal se smíchem volat: „Anděl, támhle letí anděl!“ Jeho nadšení bylo tak velké, že dokonce vyprostil ruku a ukázal kamsi na protější střechu.

Vypadal tak přesvědčeně, že jsem se naznačeným směrem podíval. Nic jsem však neviděl, a to jsem se díval pořádně a dlouze, bohužel až moc dlouze, když jsem se totiž otočil zpátky, brána již byla zavřená.

Zoufale jsem se k ní vrhnul a zabušil. „Už jdeme!“ ozvalo se a ven vyšlo několik policistů, kteří nasedli do auta a odjeli. 

Proklouzl jsem dovnitř a šel někomu nabídnout svou pomoc. Pozdě. Pomocníků už se přihlásilo dost. A tak jsem se zase ocitl venku. Selhal jsem. 

Nepochyboval jsem, že vězeň, který ukazoval na anděla, nelhal a že ani nebyl šílený (i když o Šanghajanech se proslýchalo leccos). Bylo mi zcela jasné, proč jsem Božího posla neviděl. Nezasloužil jsem si to kvůli své zbabělosti, nedostatku síly říct „ne“ tomu příšernému plánu i nedostatku síly se vnutit k hlídání vězňů.

Přemýšlel jsem, co mě čeká nyní. Pravděpodobně budu zatčen, mučen, popraven a nakonec za své selhání odsouzen i k trestu Božímu. Můj neúspěch zcela jednoznačně svědčil o tom, že mi ráj nebyl předurčen.

Rozhodl jsem se tedy vynechat alespoň fázi světských muk a rozběhl se k řece Huangpu. Až když jsem byl u ní a rozmýšlel se, ze kterého místa skočit dolů, uvědomil jsem si, že jsem spolu s uniformou dostal také pistoli, nabitou. Pokusil jsem se ji odjistit a přiložil si ji ke spánku.

„Otče náš, jenž jsi na nebesích…“ zmáčknout spoušť se mi nechtělo, rozhodl jsem se nejdřív pomodlit. Už jsem vyčerpal všechny předem naučené a zrovna byl pohroužen do své vlastní, když mi někdo vrazil obrovskou ránu do břicha.
\vspace{0.75cm}

„Už do sebe nechávají mlátit. Opravdu nechceš vidět, co jsi vymyslel?“ kroutil Jiří hlavou při pohledu na displej Jieho foťáku, jenž mu analytik zapůjčil se slovy, která nyní v reakci na jeho komentář zopakoval.

„Nepotřebuji se dívat. Vím přesně, co se děje a vím, že je dobře, že se to děje.“ 

Čestný dobrodruh, který nebyl z plánu, jehož realizaci zrovna Jieho Fujinonem pozoroval, vůbec nadšený, zakroutil nešťastně hlavou a dodal: „Čím vás tak naštvali?“

„Ničím, jsou to mé děti, které miluji.“

Tomu se skaut podivil. „Vaše? Ne snad Boží?“

„Bůh je mizerný rodič,“ odpověděl Jie. 

„Tohle je oficiální stanovisko vašeho společenství?“

„Neoficiální společenství těžko může mít oficiální stanoviska a upozorňuji vás, že se odložitelným hovorem připravujete o neopakovatelnou akci venku.“

 Jiří poslechl a pohlédl na displej. Zhlédl tak zatažení vězňů do prostorů věznice a nakonec i mou osamělou a bezradnou postavičku stepující před zavřenou branou. Samozřejmě, že ho můj nedostatek asertivity naštval. Když viděl, že odcházím směrem k řece, rozlíceně se rozběhl za mnou.
 
Našel mě, jak stojím s pistolí u spánku a dívám se kamsi do vody. Rána pěstí do břicha mu přišla jako dostatečný trest, v tom jsem mu byl schopen dát za pravdu, způsobil mi tu nejhorší bolest, jakou jsem do té doby zažil.

Zlomil jsem se v pase a padl na zem. Z očí mi vyhrkly slzy bolesti, které po chvíli nahradily slzy žalu nad tím, že můj čin nejspíš bude stát Jiřího život.
 
„Nermuť se. Došlo k chybě, ale my ji napravíme,“ utěšoval mě Evropan, zatímco mi podával pomocnou ruku.

„Obávám se, že ne, bratře, schválně hádej, kolik kamer teď registrovalo, že cizinec, který vůbec nemá být v zemi, zaútočil na falešného policistu? Do večera budeme mrtví.“
