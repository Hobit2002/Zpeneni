\chapter{}
„Nuže tedy. Začalo to… (přeskočím všechno, co jsem již napsal dříve)… a tak jsem si koupil smaragdový důl a pan Musk mi poskytl zázemí, abych zde mohl rozjet těžbu. Byla to samozřejmě ohromná dřina, ale po několika měsících nedostatečného spánku se zdařilo. Důl vydělával a já mohl svého partnera již brzy štědře vyplatit.

Ale ještě dřív, než jsem se znovu stal milionářem, adoptoval jsem Fetua. Jistě jsi mě sledoval, a dost možná tedy znáš i takové detaily jako jeho vzhled… “ dal jsem Jiemu pauzu, aby se mohl vyjádřit, ale on to zřejmě nepochopil. Mlčel a pohledem mě vyzýval, abych pokračoval.

„Dlužil jsem mu to. Však sám víš, co se stalo s jeho otcem (při těch slovech můj syn obrátil oči v sloup). Fetu ti je asi hodně podobný, ani on se mnou prvních několik dní po našem setkání vůbec nemluvil, ale teď se, alespoň mně, docela otevřel. Vděčím za to především množství knih, které už byly vystaveny na internet. Fetu si je schopen celé dny pročíst. Když byl mladší, byla pro něj četba slastí číslo jedna. Dnes už ji však kombinuje se sepisováním svých vlastních postřehů. Když se mu podaří do svého dne dostat v hojné míře obojí, bývá mimořádně spokojený a večer se mě většinou vyptává na můj život. 

Zajímá ho, zda se mi nestýská po Číně, zda jsem v práci šťastný, zda se manažeři často hádají nebo jaká je má oblíbená kniha a proč zrovna ona. Jen a jen se ptá, on sám mi nikdy nic nevyprávěl.

A to jsem si jist, že by měl co. Samo sebou, že jsem měl v domě nejen čtečku ale i papírové knihy. Fetu přečetl všechny a během četby si do nich vkládal kusy papíru popsané svými postřehy. Poměrně často si je tam zapomněl.

Jednou jsem z knih utíral prach a můj zrak padl na svazek, který se jmenoval možná „Dialektika do hloubky“, už si to nepamatuji přesně, chvíli jsem si ho prohlížel a uvažoval, zda je o IT (od kolegů jsem často dostával celé kolekce kniha na určité téma), nebo o něčem jiném. Nakonec jsem se rozhodl svazek prolistovat.

Jen co jsem knihu otevřel, vypadla na mě Fetuova záložka. Byla to první, se kterou jsem se potkal, a tak jsem ji, zvedl a začal si ji číst.

‚Str. 272: přání bohužel tvůrcem myšlenky‘

‚Str. 184: pokud změnu světa definujeme změnou lidského vnímání, pak nelze vyloučit, že se mění pouze toto lidské vědomí. ‚Str. 400: Hegelova dialektika je přirozený výklad světa hloubavým a šťastným člověkem. Zkrátka psychologie, nikoliv metafyzika.‘

Chvíli jsem nad tím přemýšlel a usoudil, že už chápu, proč můj adoptivní syn svůj vnitřní život nesdílí. 

Fetu chodil do školy s ostatními dětmi. Drtivou většinu času pročetl a zcela jistě nabyl toho názoru, že lidi, které by zajímalo to co jeho, jinde než na univerzitě nepotká.

Asi měl pravdu, jednou jsem se s ním zkoušel o filosofii bavit a nebylo to nic moc. Fetu naprosto nezvládl o oblasti svého zájmu hovořit, zadrhával se, rozvíjel své myšlenky způsobem, který mi připadal zbytečný… víckrát jsem to nezkoušel.

Ale nemysli si, že jsem neměl odvahu, tehdy jsem totiž vyhlásil válku Číně.

Jistě sám víš, že naše rodná země se v současnosti stává ekonomickou velmocí, neboť zaplavuje všechny trhy svým levným zbožím. Tahle strategie může fungovat jen potud, pokud jsou Číňané chudí. Ale nelze předpokládat, že s jejich zbohatnutím přijde konec. Ne, akorát si troufnou na hi-tech. 

Sám víš, jaká zvěrstva se v naší domovině dějí, a jistě se nedivíš, že jsem nechtěl a stále nechci, aby takový režim prosperoval. A tak jsem se rozhodl ho maximálně zpomalit tím, že mu budu konkurovat tak moc, jak jen to půjde.

Číňané jsou chudí, ale ve srovnání se Zambijci se mohou cítit jako šlechta. Když jsem postavil první textilní továrny, poprali se o místa, kde si člověk vydělal při přepočtu na dolary méně, než kolik vyžebrá bezdomovec na Karlově mostě v Praze.

A již o pár let později, když se místní naučili patřičně disciplinované fordovské manufakturní práci, přešel jsem od textilií ještě dál a nechal své zambijské bratry, aby montovali iPhony, iMacy, iPady a jiné hračky. Věřím, že jednou se mezi dělníky objeví nějaký génius, kterého technologie zaujmou natolik, že přijde s něčím vlastním. A právě z tohoto důvodu, tedy nikoliv z chamtivosti, u sebe pár hodin týdně nechám pracovat i nějaké to dítě.

Zrovna toto byla pro Čínu těžká rána. Co by tamější komunisté dali za to, aby si Apple nechával své výrobky montovat právě u nich… Nebyl žádný div, že jsem těm pseudosocialistům začal hodně vadit. Zprvu se se mnou snažili jednat a mé podniky koupit, ale poslal jsem je nejnevybíravějšími slovy svého života zpátky do té pekelné díry, ze které vykoukli. Také zkoušeli uplatit zambijskou vládu a ta chvíli dokonce vypadala, že na mě uvalí nějaké hrozné daně. Nestalo se. Národ se za mě postavil, zvláště po té, co jsem mu postavil hned několik nemocnic a škol. Zvláště díky těm druhým institucím bych mohl být schopen do patnácti let založit plnohodnotnou technologickou firmu. Číňané mě již předběhli, ale dlouhodobě napřed nebudou. 

Já totiž Isimangalisanům dopřávám svobodu a nechávám je, ať si jdou vlastní cestou, i když se to ostatním lidem a institucím nelíbí. Jen v této svobodě ale může opravdu uspět jedinec tvůrčí, autentický a sebevědomý. A právě lidé tohoto typu navrhují tu krásnou část dějin. Čína, která lidi směřuje do centrálně vybraných kolejí, může být premiantem v již existujících trendech‚ ale jen sotva tu vznikne něco inovativnějšího než třeba sociální síť na sdílení patnáctisekundových videí. Díky svobodě a individualismu může být i Isimangaliso významnější než celá Čína a já dělám vše pro to, aby tento potenciál došel naplnění.“

„Buď opatrnější, otče“ přerušil mě v tu chvíli Jie. „Jsem agent zahraniční rozvědky a přijel jsem, abych tě zabil. Bylo by to velmi snadné, ale myslím, že světu prospěje, když to neudělám. “

„Ty pracuješ pro stát?“ zhrozil jsem se.

„Jsem lovec šmejdů, kteří se obohacují  nechávajíce za sebou spálenou zemi. Do Zambie jsem vyrážel s očekáváním, že budeš jako oni, ale byl jsem mile překvapen. Máš očividně tolik peněz, že je utrácíš i v cizím zájmu.“

„A co budeš dělat teď?“

„Pochybuji, že ti bude vadit, když zůstanu nějaký čas s tebou a pomůžu, s čím jen budu moci.“
\vspace{0.75cm}

Samozřejmě mi to nevadilo a následujících několik let jsem byl ze svého syna, či spíše nového kolegy, nadšený. Jeho oddanost bohyni Rock se projevovala každý den – odmítal spát více než šest hodin denně, neboť měl bytostnou potřebu zdokonalovat sebe i své okolí (což se ve spánku dělá těžko) a dny byly příliš krátké na to, aby v nich stíhal plnit svoje cíle. Začal tím, že během prvních dvou měsíců svého pobytu udělal z mého domu pomocí celé flotily dronů, bariéry bezpečnostních kamer, neprůstřelných oken, tajné únikové chodby do dolu a několika psů sídlo střežené úměrně mému majetku a geopolitickému významu.

Z pohledů mého, Jieho a globálního však byla mnohem víc než má vila důležitá láska, která se zrodila mezi Jiem a informačními technologiemi. Můj syn naplno využil volný čas, jehož se mu u mě dostalo, aby pronikl do tajů počítačových sítí, šifrování a strojového učení. (Už když vyrážel do Afriky, byl Jie schopen programovat své drony, ale až v Zambii se z něj stal technologický expert.) 

Ne snad, že bych ho nezvládal zaměstnat. Jie se stejně jako Fetu stal členem think-thanku posuzujícím, jak a jak moc rozvíjet různé projekty. Jeho nápady se osvědčily. Například nás přiměl, abychom si vyrobili veliký počítačový systém na sběr zpětných vazeb od našich zákazníků a následnou analýzu toho, v čem by se naše služby měly zlepšit. (Jie chtěl tento systém využít i na to, aby vyhodnocoval, kteří manažeři jsou schopní a zaslouží si povýšení, a které bychom vyhodit. Tento nápad ovšem ostatní analytici rezolutně odmítli – v roce 2011 lidstvo na podobné pokusy nebylo připraveno) 

Jeho další iniciativou byl také projekt, během něhož jsme vyvinuli umělou inteligenci, která na sociálních sítích, především Facebooku, sbírala data o uživatelích a po té jim vyhazovala takové pracovní nabídky v našich společnostech, které přesně odpovídaly jejich zájmům a kompetencím.

Už na základě těchto dvou zásluh jsem Jiemu důvěřoval dost na to, abych se mu svěřil se svými obavami o dění v mém bývalém dole. Jie je vyslechl a usoudil, že by stálo za to celou věc prozkoumat. Spojil se proto s jakousi nestudovanou ale o to houževnatější švédskou novinářskou, která se dlouhodobě zabývala především klimatickou změnou, a domluvil se s ní, že jí jím zastupovaná firma zaplatí letenky, techniku, kauce a všechno ostatní co by se mohlo novináři na nejodvážnější misi jeho života hodit. Mladá dobrodružka s nadšením přijala a ještě nám několik měsíců posílala děkovné emaily.

Plně se však Jie vyprofiloval až svým neodvážnějším návrhem - sice tichým, ale přímým útokem na čínskou Huawei. Můj syn chtěl, abych koupil, co nejvíc aplikací na Android a předělal je tak, aby se po instalaci na mobily od Huawei změnily ve viry, které jednak využívaly infikovaná zařízení jako čipy na strojové učení a druhak dělaly obrovské zmatky v datech, které Huawei mobily o svých uživatelích posílaly do Číny. (Jie například navrhoval, abychom vyvolali iluzi, že lidé ve velkém navštěvují čínské internetové služby jako Alibaba.com  popřípadě TikTok. Chtěl tak v prvé řadě narušit čínské šmírovaní celosvětové populace, ale také nemístně zvýšit sebevědomí čínské vládě a přivést ji ke dříve či později zhoubnému sebepřecenění)

Když Jie tuto ideu na poradě přednesl, vyvolala značný rozruch. Analytici se vůbec poprvé v historii think thanku začali překřikovat. Jedni volali, ať se do takového hazardního nesmyslu v žádném případě nepustím, jiní naopak tleskali a hvízdali, aby přerušili ty první.  A zbytek nevěděl, co si má myslet.

Ne, že bych Jiemu nedůvěřoval, ale mé podnikání bylo dost úspěšné na to, abych už poznal spoustu geniálních lidí a věděl, že pokud někdo má svůj životní příběh plný chyb, pak jsou to právě oni. Chtělo to sehnat experty a celou záležitost s nimi probrat. Jieho manipulační software posloužil k jejich výběru bezvadně.

Do měsíce v nedalekém hotelu bydlelo okolo třiceti akademiků vybraných špičkami v oblasti strojového učení (právě rozložení náročného trénování mezi spoustu málo výkonných zařízení, které navíc mohly každou chvíli přestat být k dispozici, představovalo jednoznačně největší technickou výzvu celého procesu) ze Stanfordu, Toronta a Berkley. Trvalo jim to pár týdnů, ale nakonec přišli s plánem jak ze spousty nespolehlivých buněk udělat poměrně použitelný mozek a jaká data odesílat do Číny, aby to její špionážní algoritmy zmátlo co možná nejvíc a nejtrvaleji. 

Následovalo několik dalších měsíců práce, pro kterou jsem musel po celém světě najmout skoro tisíc dalších programátorů. Dokonce jsem založil antivirovou firmu, jejímž účelem bylo jen a pouze zakrýt skutečný důvod tohoto shánění iťáků, skutečný produkt jsem si nechal kompletně vyvinout u Avastu, ale stálo to za to.

Přesně na nový rok 2013 jsem slavnostně udělal z firmy antivirové firmu, která poskytovala cloud computingové služby. Instituce z celého světa si u nás začaly trénovat algoritmy na rozpoznávání obličejů, předvídání poptávky po svých produktech či imitaci dávno zemřelých filosofů... a žádná z nich neměla tušení, že právě kvůli nim a jim podobným,  začínají být zařízení Huawei mimořádně pomalá a rychle se vybíjející.

Čínský gigant si samozřejmě uvědomil, že je něco špatně. Na internetu se začaly ve velkém objevovat stížnosti, že jinde výtečně fungující aplikace na jeho zařízeních dělají obrovskou neplechu. Ještě horší však byl pro firmu hněv čínské vlády, která shromažďovala data nasbíraná od uživatelů a předvídala na jejich základě výsledky voleb v demokratickém světě. Když jsme ovšem její algoritmy zahltili spoustou smetí, přestala tato umělá inteligence zcela fungovat, predikce vycházet a čínská vláda mimo pravidla hospodářské soutěže Huawei dotovat. To byl pochopitelně průšvih a firma byla odhodlána udělat cokoliv pro jeho odstranění.  

Věděl jsem, že k odhalení naší činnosti dříve či později dojde, ale to mi nikterak nekazilo radost z úspěchu. Uspořádal jsem velkolepou oslavu ve své vile a odměnil během ní vědecký tým, zahrnující i Jieho, týdenním pobytem na Bahamách, a zaměstnance cloud-computingové firmy prémiovou dovolenou. Fetu, který v softwarové firmě pracoval jako analytik, si volno ještě prodloužil a odcestoval kamsi na jih, aby mi odtud později přinesl zvěst o bohyni Rock.

Tím se však dlouhá řada triumfů přerušila. Tři dny po té, co vědci odletěli na Bahamy, mi přišel email od Jieho:

„Milý otče, kdyby mí bývalí kolegové věděli, co je zde čeká, vůbec by tu dovolenou nevyužili. Schválně si tipni, kolik z nich ji teď hodnotí dobře a kolik špatně. Zdravím, Jie“

„Co se stalo?“ odepsal jsem.

„Nejdřív si tipni čísla,“ Jie na to.

„20 nespokojených a jen 7 ano, třeba.“ (Nabídku na dovolenou nevyužil celý tým – například slavní profesoři, kteří jej sestavili, měli až nad hlavu práce na svých univerzitách)

„14 není při vědomí a možná už nikdy nebude, 13 z toho bylo velmi negativně velmi vzrušených.“

Došlo ke katastrofě. Netýkala se jen mých vědců, ale mnohem většího počtu návštěvníků bahamských ostrovů. Mezi lidmi, kteří dopoledne dováděli v moři a na oběd zašli do blízkých restaurací, se rozšířil nějaký patogen, který způsobil selhání celého organismu. Nakažení přišli o život a lékaři o sebevědomí.

Ona událost zažehla ve vědeckém světě paniku. Nikdo netušil, co nešťastníky usmrtilo, a odkud se to vzalo. Bez prodlení jsem badatelům nabídnul miliardy k důkladnému výzkumu oné hrozby. Bylo to dost peněz na to, aby usoudili, že bude lepší o další granty nežádat a nevyvolávat tak veřejnou paniku, které by se okamžitě chopili politici, kteří by vědce začali tlačit do populistických nesmyslů.

Výměnou za štědrou podporu jsem ovšem požadoval vhled do celého výzkumu. Dostalo se mi ho a já se tak čtením různých dokumentů a bezradným dumáním nad jejich implikacemi připravil o ohromné množství času a duševní energie. Když se se mnou pokusili spojit jednatelé firmy Huawei a přemluvit mě, abych jim prodal zavirované aplikace, ani jsem se s nimi nesnažil ujednat nějaký pořádný obchod, ale prostě jsem se s nimi odmítl bavit. Firmě tak nezbylo, než požádat o pomoc svou vládu, která za mnou, či spíše na mě, poslala agenta Ganga, aby udělal to, co před lety dostal za úkol Jie.

A aby toho nebylo málo, dostala se k nám zpráva, že švédská novinářka, která v rámci svého pátrání vyrazila do Číny, byla zatčena a odsouzena za špionáž.
\vspace{0.75cm}

I kdybych si to tehdy uvědomoval, netěšilo by mě to, ale skrze všechny tyto problémy se do světa zase začala vracet bohyně Rock.
