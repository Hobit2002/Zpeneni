Tým šedesáti špičkových biologů začal za mé peníze analyzovat souš na Bahamských ostrovech i vodu v okolním moři a já se mezitím se svými syny stáhl do sebe.

„Proč je svět takový? Vím, nemá smysl od něj cokoliv čekat, je velký chladný a nemilosrdný. Jak jsem dříve mohl věřit, že nad ním panuje řád, který jej neustále činí lepším a lepším? Jak jsem mohl být tak naivní?“

„Měl by ses k tomu vrátit,“ promluvil Fetu. Bylo to poprvé v životě, co jsem slyšel, že by někomu dal takhle přímočarý pokyn.

„Vracet se k nepravdě? Lhát si, že svět spěje k našemu dobru?“

„Ne k nepravdě, nýbrž k mimopravdě.“ To jsem nepochopil a dokonce ani Jie ne, ale znělo to zajímavě.

„Nemůžeme dokázat, že neexistuje žádný Bůh, který by průběžně zasahoval do chodu světa, zcela jistě ovšem víme, že tento Bůh hraje jinou hru než my a člověk ho nemůže nazývat spojencem. Lidé chtějí hlavně štěstí, pro sebe a pro své děti, kdyby Bohu šlo o to samé, svět by vypadal o dost jinak.

Ale on je, jaký je. Afričané jsou chudí, Číňané utlačovaní, několik skvělých vědců právě zahubil neznámý patogen. Stojí-li za tím vším všemocný a chtějící Bůh (jiného si člověk jen těžko představí, a když už, jen sotva mu uvěří), pak je i všechno toto zlo jeho vůlí
.
Ne, Bůh není zosobněné dobro. Ale to ještě neznamená, že není vůbec a že není dobré věřit, že je. Víra je umění, ke kterému se má přistupovat maximálně tvořivě. Je třeba si akorát vymyslet takového Boha, jehož vůle není přímo v rozporu s fungováním světa.“
„Taková víra se do reality téměř jistě nestrefí.

„Možností je tolik, že je každá víra téměř zcela jistě naprosto pomýlená.“

„A čemu se tedy dá věřit?“ otázal jsem se ho.

„Všemu, co přitakává tomuto světu i se vší jeho bídou. A to něco můžeš hledat v sobě. Pokud všemocný a chtějící Bůh stvořil tento svět podle své vůle, tak je naše přirozenost taková, jakou ji chce mít, a můžeme ho v ní hledat,“odpověděl Fetu klidně.

„Nesmysl! “vložil se do konverzace Jie “Záměrem s naší přirozeností může být její změna vnějšími činiteli a pak z jejího současného stavu nic vyčíst nemůžeme.“
Fetu se na chvíli zamyslel a pak přikývl „Máš pravdu Jie, tohle jsem nedomyslel, díky za připomínku.“

„Jsem z vás zmaten, kluci. Potřeboval bych slyšet nějaké stručné shrnutí, ne disputaci“ zašklebil jsem se.

„Vytvoř si svou víru, vytvoř si své hodnoty a ideály. Jediné meze, které nesmíš překročit, stanovuje logika (tou se nemusí řídit jen ti, kteří nemají na to vytvořit něco, co by s ní bylo v souladu) a smrt – nepočítej s žádným konkrétním přesahem svého současného života do toho, který jej bude následovat. Nemusíš se smrti bát, ale nenamlouvej si, že máš tušení, co bude po ní.“ splnil mé přání Fetu.

Chvíli jsem si ho prohlížel a uvažoval nad zázrakem, který způsobil, že se začal chovat jako nikdy dříve, a zcela přitom přehlížel bohyni Zarahustru, která ke mně jeho ústy hovořila.

„Zkusím to, hoši,“ řekl jsem nakonec a zvedl se. Následující dny jsem však na nějaké náboženství neměl ani pomyšlení, zkoušel jsem rutinně pokračovat v práci, ale stále mě to táhlo k mým emailům, ve kterých jsem zoufale, hledal nějaké zprávy o pokroku výzkumného týmu.

Nepřicházely, zato za mnou přišel Gang.
\vspace{0.75cm}

Zatímco Jie mě pečlivě šmíroval drony, jeho starší, jednodušší a krutější kolega si zkrátka vzal odstřelovací pušku (Zambie byla Číně vždy velmi nakloněná, takže lidem s VIP pasem dovolovala i převoz zbraní), vylezl na protější střechu a čekal, až se objevím v okně, aby mě sprovodil ze světa.

Jieho obranný val však zafungoval. Hlídací drony Ganga zaregistrovaly a poslaly mi upozornění, ať nechodím k oknům. Poslechl jsem je a zavolal policii. K domu, na jehož střeše Gang s puškou číhal, okamžitě přijelo několik strážníků, nahoru se jim ovšem nechtělo. Správně předpokládali, že atentátník má krom odstřelovací pušky také pistoli za pasem a nezdolné odhodlání splnit svou misi v srdci, pročež by střet jich, pseudovycvičený strážníků s ním, mohl skončit tragicky. 

I nyní však pomohl dron.

Co kdysi vyhnalo Fetua z kuchyně, dostalo tentokrát Ganga ze střechy. Jeden z dronů nenápadně obletěl dům a chrstnul mu z výšky do obličeje čpavkovou vodu. Gang sebou trhnul, upustil pušku a jal se vytírat si odporně páchnoucí látku z obličeje.

To už ale k němu přibíhali policisté a sápali se po něm. Gang byl tak zmaten, že je zprvu považoval za ochotné civilisty, kteří mu chtějí pomoct a svou situaci pochopil až po té, co ho policie odzbrojila a naložila do auta. 

Gang byl podobně jako jeho mistr Wuwang pouze hloupý primitiv, ale co mu chybělo na vlastních analytických schopnostech, to si nahrazoval podrobnou znalostí všemožných směrnic, příruček a manuálů. Právě v jednom z těchto materiálů se dočetl o mimořádných diplomatických vztazích Zambie a Číny. Jen co si na text vzpomněl, vytáhl pohotově z kapsy svůj agentský průkaz a vnutil jej strážcům zambijského zákona.

„Soudruzi policisté, jsem agent Čínské lidové republiky, velkého spojence Zambie v časech zlých i dobrých. Do vaší země jsem přijel kvůli misi nejvyšší důležitosti, proto bych vás poprosil, abyste přestali kazit klíčové zájmy svého největšího mezinárodního partnera,“ připomenul jim.

Jeden z policistů se k němu otočil a jeho průkaz si prohlédl. „Vskutku, je to agent. Mezinárodní smlouva o strategické spolupráci tajných služeb nám přikazuje, abychom takovým jako on v jejich práci nebránili a v případě, že jejich mise Zambii v žádném případě neuškodí, jim i pomáhali. Co je tedy vaším cílem, pane agente?“

„Zabít oligarchu, který tu vystavěl celé to odporné zazobané město.“

„To vám rozhodně nepomůžeme. Zambii jen máloco přináší větší prospěch než jeho práce. “
„Podle zambijských zákonů, vám musíme ve vraždě občana bránit,“ vložil se do hovoru druhý policista.

„Ano, ale podle smlouvy s Čínou jejím agentům bránit v ničem nesmíme a mezinárodní právo má navrch nad místními zákony,“ zamračil se první policista.

A tak policisty nenapadlo nic lepšího než nechat Ganga jít. Jakmile však atentátník opustil vůz, začali vášnivě diskutovat o tom, jak mě zachránit a přitom se neprovinit proti zákonům, které bránili a prosazovali.

„Ten Číňan pravděpodobně znovu zaútočí na vilu našeho mecenáše a pokusí se mu ublížit. Myslím, že bychom na to měli svého chlebodárce předem upozornit a vyzvat ho, aby nám zavolal zpátky s oznámením, že se do jeho domu snaží vniknout nebezpečný vetřelec se zbraní, a prosbou o zásah. My tam přijedeme a agenta zastřelíme. Zpětně nám už nikdo nedokáže, že jsme znali jeho poslání a tím pádem zakročit nesměli.“

Jak se domluvili, tak udělali.

Když Gang dorazil až k mé vile a zazvonil, neotevřel jsem mu. Přelezl tedy branku, ale to se rozezněl alarm, já znovu zavolal policii a na vetřelce poslal psy, aby náhodou nevyrazil dveře.

Můj dům hlídalo celkem pět zabijáckých šelem, první z nich však Gang zastřelil, jakmile se objevila. Dvě další se výstřelu lekly a utekly. Zbylí psi se vrhli do boje.  

Jeden skočil Gangovi po ruce ve, které svíral zbraň, ten ji ovšem zvedl a přetáhl psisko boxerskými prsteny, které měl na ruce druhé. Mezitím ho druhý pes kousl do nohy, za což ho Číňan zastřelil. Třetí šelmu smrt jejího druha rozlítila k ještě větší nepříčetnosti. Agentovi nezbylo nic než ji odkopnout a pak se hlavou i tělem rozběhnout proti dveřím. Nevyrazil je, ale kdyby byly jen o třídu méně bezpečné, podařilo by se mu to. Narazil do nich totiž takovou silou, že po nárazu ztratil vědomí. Ještě však ucítil, jak mu ramenem proletěla kulka z policejní pistole. Pes pochopil, že ho může nechat být.

„To nám neměl dělat,“ lamentoval už chvíli po té nad omráčeným tělem policista. „vzhledem k tomu, že neumřel, ho musíme odvézt do nemocnice, kde se probere a obviní nás z toho, že jsme mu zmařili misi.“ V tu chvíli jsem vyšel z domu. Trable mých zachránců mě vedlo k zamyšlení.

„Postarám se o něho sám,“ navrhnul jsem policistům, kteří už začali bezvládné tělo nešťastně nakládat do auta. „Z vašeho pohledu se celá tato událost nikdy nestala.“
„Chcete po nás, abychom porušili zákon ještě jednou?“

„Ano, rád bych toho útočníka vyzpovídal. Zároveň se mi teď při jednání s Čínou hodí rukojmí. Nu a vy se vyhnete obrovskému průšvihu.“ Tak jsme se tedy dohodli.  Já Ganga zavřel do sklepa a policisté odjeli. 

Když Číně zmizel první agent, ještě to nějak skousla, ale s druhým už to bylo na pováženou. Již velmi brzy mě navštívil čínský diplomat s pozváním na čínskou ambasádu, že prý tam se mnou chce někdo něco řešit.

Chtěl mě vlákat do pasti a já si to uvědomoval. Původně jsem se tedy chystal odmítnout, pak mě ale napadlo, že když se mohu o věcech radit s Jiem a Fetuem, dalo by se z té situace leccos vytěžit.

A tak jsem pozvání přijal a už za pár dní vstoupil na čínskou ambasádu v Lusace.
Byla to velmi nepohodlná návštěva. Přišel jsem na velvyslanectví oblečen do kabátu, ukrývajícím pár Jieho vychytávek, a ačkoliv diplomaté jednali rázně a nenechali mě čekat dlouho, vypotil jsem ještě před „jednáním“ snad až desetinu své hmotnosti.

Jen co jsem zaklepal na kancelář samotného velvyslance, ochotně mi otevřel a zval mě dovnitř. Vešel jsem velmi opatrně, i tak mě však mladý Číňan, který okamžitě vystartoval ze svého úkrytu, překvapil. Ještě větší dojem však na mě udělala rychlost a obratnost, s níž mě srazil na kolena a zkroutil mi ruce za zády.

 „Co pod tím kabátem schováváte?“ otázal se mě velvyslanec.
 
„Podívejte se sami,“ odpověděl jsem mu, a tak mi mladík teplý háv konečně sundal.
Pod kabátem jsem měl klasickou slušnou košili, přes tu však byla navlečena ještě jedna, jemná a drátěná, která byla zapojená do několika baterií v mých kapsách. 

Velvyslancův poskok se mi okamžitě pokusil drátěnku strhnout, ale jen co se jí dotknul, dostal elektrický šok, po kterém upadl do bezvědomí.

 Mně se nestalo nic. Pod košilí se totiž nacházela, jak ostatně brzy zjistil i sám velvyslanec, neprůstřelná a nevodivá uhlíková vesta. Když totiž jeho druh padl k zemi, vytáhl pistoli a napálil mi to párkrát do břicha.
 
Ještě než zvládl namířit na hlavu, přiběhl jsem k němu a dotkl se loktem jeho zbraně. Mohutný a rychlý proud elektronů se okamžitě přenesl z mého ohozu, přes pistoli až do jeho ruky, která zbraň okamžitě upustila.

„Tak, a teď si můžeme promluvit férově,“ vyzval jsem ho. „Předpokládám, že už jste si nějak zavolal posily. Považuji tedy za slušné vás upozornit, že mám chytré brýle, které nahrávají vše, co se kolem mě děje a zároveň jsou spárované s mými chytrými hodinkami. Tyto hodinky sledují můj tep a když registrují, že jsem zemřel, pošlou o tom brýlím informaci a ty automaticky zveřejní videozáznam na několika sociálních sítích,“ dodal jsem. (Tohle byl pro změnu Fetuův nápad.)

Velvyslanec pochopil, že je v šachu, a tak jen bezmocně pokrčil rameny: „Tak se do toho dejme…Především vás varuji, že pokud nepřestanete házet klacky pod nohy našim tajným slu…, chci říct firmě Huawei, klidně vybombardujeme Isimangaliso. Víme, že za útoky stojíte vy, ale těžko s nimi něco naděláme. Aplikace prodat odmítáte, ale my máme výjimečně na své straně také pravdu a pokud ji veřejnosti sdělíme, zničíme vás.“

„Na tuto možnost už jsem připraven. V okamžiku, kdy to uděláte, odškodním poškozené uživatele a vyrobím samostatnou aplikaci pro sdílený cloud computing, jejíž uživatelé budou dostávat za poskytnutý prostor zaplaceno a která bude dostupná nejen na Huawei. Mnou koupené aplikace budou nadále sloužit pouze k matení tajných služeb.“

Velvyslanec přemýšlel, jak mi odpovědět patřičně diplomaticky, ale vzhledem k tomu, že se o žádnou špičku oboru nejednalo (to by byl jinde než v Zambii), zůstal tento záměr nenaplněn. „Pak tedy nezbývá, než se uchýlit k tomu bombardování.“

„Z těchhle výhrůžek si velké obavy nedělám. Chybí vám páka. Malým útokům se ubráním sám a po velkých by Čína upadla do mezinárodní izolace.“

„Jen si nevyskakujte, pro největší národ na zemi jste se stal tak ostrým trnem v oku, že si pro jeho vyndání klidně zlomí ruku.“

„To mě těší a až pár vašich agentů zpacká další pokusy o atentát, budu mít radost ještě větší. Ale buďte si jisti, že si spíš zaplatím soukromý Iron Dome s rozšířenou působností na letadla, než abych vám ustupoval,“ po těchto slovech jsem se zvedl, vyzval ho, aby se s případnými diváky kompromitujícího videa rozloučil poklonou a doprovázen jeho zdviženým prostředníčkem odešel.
\vspace{0.75cm}

Mé názory se ovšem již brzy změnily. Pár dnů po návštěvě ambasády mi vědecké týmy zkoumající záhadné onemocnění poslaly poměrně podrobnou dokumentaci svých dosavadních objevů.

Podařilo se jim zjistit, že patogen způsobující vážné selhání bezmála všech částí organismu se šíří teplou mořskou vodou. 

Usmrcení ubožáci měli tu čest setkat se s nějakou dosud neznámou archeí, která pravděpodobně doposud přebývala v korálových útesech.  V posledních letech ovšem stouplo pH mořské vody, pročež mnoho korálů uhynulo, právě z nich se pak tato archea, chavezia, uvolnila a putovala mořskou vodou až k břehům a plážím. Zde přišla do kontaktu s lidskou kůží a pokusila se uchytit v jejich spórách. Když člověk vylezl z vody a trochu uschnul, archea se buď vyplavila s potem, nebo uhynula (jako v případě vědců, kteří po koupání vyrazili do klimatizované restaurace).

Po své smrti se mikroorganismy velmi rychle rozložily a to byl průšvih. Zbytky jednobuněčných těl těchto drobných stvoření byly pro lidi extrémně toxické.

Pro tyto vlastnosti vědci archea pokřtili pracovním názvem chavezia. Nejednalo se o jméno dokonale výstižné, ale to hlavní, poselství „přijdu, když je veselo, a způsobím zkázu“, zůstalo zachováno. Naštěstí toho výzkumníci objevili i trochu víc, a tak bylo možné začít pomocí superpočítače poskládaného ze zavirovaných mobilů Huawei hledat látky, které by infikovanou osobu zachránily.

Krom podrobného rozebrání vlastností archeí dokumentace obsahovala i statistiku dosavadních obětí. S převahou nejvíc lidí chavezia zabila v Šanghaji. Následkem tohoto zjištění jsem se ocitl na vážkách. Mé svědomí mi kázalo Šanghajanům pomoct, ale rozum mi radil nedělat nic, co by bylo totalitní supervelmoci po chuti. Toto dilema bylo tak velké, že mě dovedlo až k úvahám nad smyslem života a mého dřívějšího rozhovoru s Fetuem.

 Byl jsem sice dekamiliardář, ale to rozhodně neznamenalo, že bych se nesetkal s hrůzami a problémy. Střet s tankem, vyhnanství v Africe, nájemní vrazi poslání z Číny… to všechno byly výzvy, které jsem překonal, a nyní za ně byl dokonce až tak vděčný, že mě představa života bez nich upřímně děsila. Člověk žije od toho, aby si vybral svou cestu a překonával překážky, které na ní potká. Však jsem si také až doposud své velké cíle vybíral podle toho, jak moc mě lákaly, a ne podle toho, jak naléhavé se problémy, které mě obklopovaly, zdály vědcům a odborníkům. Tento přístup prospíval nejen mě, ale lidskému společenství obecně, protože jsem byl díky silné vnitřní motivaci podstatně výkonnější a je lepší vyřešit malý problém úplně, než problém dvakrát větší jen ze čtvrtiny. Nyní jsem se však ocitl v situaci, kdy bych nevybočením z cesty, jíž jsem sobě samému vytyčil, ponechal miliony lidí napospas hrozné smrti a to jsem nehodlal dopustit. Toto vybočení však žádalo nejen to, abych si zadal nové dobré cíle, ale abych spoluprácí s Čínou, která byla pro ochranu Šanghajanů nezbytná, popřel mnoho své dosavadní práce.  
 
 Snad tedy není divu, že jsem velmi dlouho a velmi intenzivně přemýšlel nad tím, jak se k celé věci postavím a už jen jednoduchá filosofická úvaha, podle níž jsem se nakonec rozhodnul, si vyžádala až absurdně velké množství času. Uznal jsem, že jedinou věcí, která je cennější než úplné vyždímání svých vlastních životních možností, je zvětšování nejen těch vlastních, ale i cizích. A pokud má člověk na výběr mezi další dramatickou kapitolou vlastního příběhu a záchranou příběhu cizího, měl by zvolit tu druhou možnost. Rozhodl jsem se tedy Šanghajanům pomoct. Zbývalo už jen vymyslet, jak to udělat, aby z toho čínský stát nevyšel posílen. Ještě, že jsem měl své dva syny… 
