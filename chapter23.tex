\chapter{}
\textbf{Tankerenova poznámková kapitola}
\vspace{0.75cm}
{\itshape
Mnohé z událostí, ke kterým doposud v Dongově příběhu došlo, nemusí být ani ve světle mého předchozího vyprávění zcela jasné. Rád bych všechen zmatek ve čtenářských hlavách vymýtil popsáním svého posledního hovoru s Jiem.
\vspace{0.75cm}

Bylo to setkání po téměř pěti letech. Pracoval jsem zrovna ve své vile, když na ni můj syn zcela nečekaně zazvonil.  Tak rád jsem ho zase viděl, i když mi nebylo vůbec jasné, jak svou cestu za mnou vysvětlil úřadům.

„Vítej zpátky, Jie, copak tě sem přivádí?“

„Když pozoruji dění ve světě, vidím, že se s Fetuem činíte, ale zatím mi není jasné, k čemu je to vaše činění dobré. Dosadili jste do vládnoucích pozic několik populistů. Proboha proč?“

Zcela chápaje jeho nechápavost pozval jsem svého syna dovnitř a zde mu řekl všechno o projektech, které napomohly nejen Trumpovi k moci a Brexitu k úspěchu, ale také Macronovi k prezidentskému křeslu a odvážným společnostem jako Tesle ke slávě.

„Mám-li být upřímný, Jie, ani já nemám pocit, že by se za tu dobu, co působíš v Číně, událo něco dobrého.“

„Nebylo toho mnoho, ale něco by se našlo. Například očkování Šanghaje proběhlo velmi úspěšně. Původně měl být onen kus uměle vypěstované kůže, která díky upravenému DNA vylučovala absorpční proteiny, implantován jen pár obyvatelům pobřeží, ale nakonec operací prošla většina šanghajské populace.“

„Na vedlejší účinky nikdo nepřišel?“

„Jistěže ano. Policejní lékaři, kteří dávku dostali jako jedni z prvních, si už dva roky po očkování všimli, že to s nimi i dalšími pacienty je nějaké divné, a ve spolupráci s několika neurology a genetiky velmi přesně identifikovali proč. 

Zjištění, že byli doživotně zfetováni, je pochopitelně velmi pobouřilo a okamžitě mi volali, aby to se mnou vyřešili.
Pozval jsem je na poradu do jedné z nejluxusnějších restaurací v Šanghaji, ale zároveň trval na tom, že ještě před ní vyrazíme na koncert. Když dozněly libé tóny a číšnici nám podali chutné a hlavně drahé jídlo, začal jsem se věnovat tématu.

‚Jste unešení, pánové,‘ oznámil jsem jim.

‚To ano, nezažít líbanky, byl tohle nejlepší večer mého života,‘ odpověděl mi jeden.

‚Musím souhlasit a ještě dodat, že mě to udivuje, neboť můj život byl doposud velmi dobrý‘

‚V následujících letech bude ještě mnohem lepší. Usazování proteinů v prostorech mezi neurony nebylo naším záměrem, ale myslím, že to nemusíme nazývat ani omylem. Právě díky němu budete zbytek svého života prožívat násobně hlouběji, než kdyby vaše nervová soustava fungovala normálně a jelikož lidé mají životy spíše dobré než špatné, je tento vedlejší účinek velmi pozitivní.‘ 

Lékaři chvíli přemýšleli a pak mi přitakali. Podobné setkání jsem musel zopakovat ještě několikrát, neboť bystrých duchů bylo v Číně víc a pokaždé jsem oslavil úspěch, nikdo mě nezavřel – a to navzdory tomu, že mé argumenty vůbec nebyly pravdivé.

Aby proteiny lidem život zlepšily, museli se dotyční chtít radovat z pěkných drobností kolem sebe, což většina věčně uspěchaných konzumentů nedělá. Lékaři si díky mým radám oproti střízlivým fázím svého života velmi přilepšili, ne tak drtivá většina populace. Ta upadla do existenciálních problémů pramenících z nedostatku životního naplnění.“

„Proč to nikdo neřešil s tvými nadřízenými?“

„Z nekalých úmyslů mě nikdo nepodezříval. K čemu by mi ostatně takový terorismus byl? Čínský režim mě za mnohou práci, jíž jsem pro něj odváděl, odměňoval obrovským množstvím peněz a zároveň nevídanou svobodou, díky níž jsem tě například mohl navštívit. 

Jen tak mimochodem Gang dopadl o dost hůř. Ze zahraniční rozvědky ho vyhodili za neschopnost a skončil jako vězeňský dozorce. Naštvalo ho tolik, že během prvního měsíce ubil v různých záchvatech vzteku asi šest vězňů.

Ale zpátky k tématu, kdo mě v Číně zná, myslí si, že se mám skvěle a že není důvod mě podezřívat z terorismu.“

„To si tě komunisté tolik hýčkají jen za to, že jsi naočkoval Šanghaj?“

„Ne, ale byl jsem zatažen i do dalších projektů. Například do vytváření systému, který by hodnotil, jak spořádaně čínští občané žijí, a na základě vypočtených výsledků omezoval, popřípadě posiloval jejich možnosti podnikat, cestovat a využívat internetové služby.“

„To je zrůdnost! Ty ses na ní podílel taky?“ zděsil jsem se a vzápětí se zastyděl, neboť jsem si uvědomil, že jsem škod nadělal ještě mnohem víc.

„Ano. Nemám pro to omluvy. Přidal jsem se do odpovědného týmu, neboť to ode mě všichni očekávali a chtěl jsem jim dokázat, že jsem opravdu na straně Číny.“

„Ty jsi ale přeci dost chytrý, abys do toho systému dokázal vložit něco dobrého, něco čeho si nikdo nevšimne.“

„Ano. Vložil jsem do toho Rusy.“

„Proč?“

„Zpracovávat informace jim jde. Mají jednu z nejlepších tajných služeb na světě, což by nebylo možné bez špičkových iťáků. Ostatně ani v jiných kybernetických oblastech nikterak nezaostávají za zbytkem světa a zároveň neznají demokracii a nevadí jim Číně pomáhat. Tak jsem jejich pozvání zdůvodnil čínským komunistům.

Reálným účelem však bylo, aby uskutečnili jeden objev – aby zjistili, že jsme v Šanghaji provedli jakési hromadné očkování. ‚Omylem‘ jsem několik složek s protokoly o očkování Šanghajanů zařadil mezi materiály používané k vývoji šmírovacího systému. Rusové je objevili, přečetli si je a poskytli je svým nadřízeným.

Tak se informace dostala až na nejvyšší místa, což jsem chtěl (a což jsem nijak příměji zajistit nemohl, neboť čím méně mi režim diktoval, kam smím a nesmím chodit, tím bedlivěji mé kroky sledoval.)“

„Jak ti tedy může projít setkání se mnou?“

„Uvidíš sám. Ale zpátky k Rusům. Když Putin zjistil, že obyvatelům této planety může něco hrozit a že se před tím něčím dá chránit, zděsil se. Pocit, že by mohl být ohrožen něčím, před čím je obyčejný prodavač z šanghajské večerky v bezpečí, mu nedal spát. 

Již několik týdnů po tom, co se zpráva o hromadném očkování k ruskému prezidentovi donesla, dostal jsem od ministra zdravotnictví zprávu, že se ruské tajné služby dozvěděly o šanghajském očkování a nyní by ho chtěly pod hrozbou, že utajovanou akci prozradí celému světu, také pro svého prezidenta.

Podobně jako my, ani čínská vláda nechtěla žádnou paniku, a tak mi dala pokyn, abych Putinovi očkování zajistil. Milerád jsem poslechl a dokonce pro něj nechal připravit specielní látku, jejímž následkem ochranného proteinu v těle vznikalo násobně víc. Násobně dřív se tedy dostavily také vedlejší účinky.

O těle ruského prezidenta jsem měl spolu s čínskou tajnou službou poměrně přesné informace, neboť jsme do očkovací látky přidali několik nanobotů, kteří stav prezidentova těla monitorovaly a informovaly nás o něm. 

V polovině roku 2018 už, soudě podle množství proteinu v jeho krvi, z diktátora musel být mystik trávicí většinu svého duševního života ve stavech, jaké navozuje LSD.

Díky zkušenostem se spoustou svých dávno očkovaných kolegů, kteří téměř neustále vzdychali nad tím, jak je svět prázdné, bezútěšné a nudné místo bez lásky a citu, jsem věděl, že pokud někde ve světě vznikne cosi jako 2. Hippies, společenství lidí naplněných naivními sny, Putin se k němu přidá. 

A tehdy přišel čas na Šanghajskou revoluci. Potřeboval jsem akorát hrdinu, nějakého rytíře, který by inspiroval sice mimořádně citlivé ale nijak zvlášť iniciativní Šanghajany. Z toho důvodu jsem se rozhodl proniknout do komunity lidí, jejichž duše se opíraly o sice extrémně iracionální ale zároveň silně motivační pilíř – náboženství, rozhodl jsem se získat kontakty v podzemní křesťanské církvi (svým kolegům a nadřízeným jsem to vysvětloval jako špionáž  - a abych svým slovům dostál, udal jsem během pár měsíců přes třicet lidí, o jejichž odvaze a užitku pro revoluci jsem silně pochyboval). Tento krok se mi vyplatil, ale svého hrdinu jsem našel jinde. 

Na jaře roku 2019 začala na severovýchodě Číny řadit zvláštní lupička. Vyhazovala lidi z aut a odškodňovala je za to auty dražšími, byť bohužel kradenými. 

Její zásadovost mě zaujala natolik, že jsem ji chytil a zjistil, že sice není žena, zato že jako rozbuška bude dokonalá. Cílem, se kterým vyrazila na cestu, bylo osvobození jedné novinářky, kterou jsme zatkli, neboť strkala nos do státních tajemství.

Pomohl jsem rozbušce vybudovat velký tým, který se pokusil dotyčnou zachránit. Doufal jsem, že úspěch akce zažehne revoluce. Ale tým, tvořený členy podzemní církve, selhal.

Několik desítek křesťanů bylo zatčeno, ale rozbuška unikla. Dal jsem ji tedy dohromady se zatím neznámou šanghajskou policistkou, jejíž nervová soustava byla mimořádně silně vyvedená z míry očkovacími proteiny. Nebylo to úplně lehké, policistka se totiž snažila rozbušku zatknout, a ač jsem se jejich setkání rušičem signálu v policejní čepici a slepými náboji v pistoli snažil maximálně prodloužit, rozbuška to stejně nepřežila.

Den její smrti byl tím nejhorším mého života, ale naštěstí se podařilo předat jiskru dál. Na oné policistce už jsou patrné změny chování, které svědčí o tom, že se brzy postaví režimu na odpor. Jeví mimořádný zájem o zatčené křesťany, dokonce se stala dozorkyní, aby s nimi mohla trávit víc času. Podrobnosti se snaží zjistit i ode mě, večer co večer chodí k mému domu. Vím to, neboť i v Zambii mám přístup k záznamům z bezpečnostních kamer.

Mým současným cílem je maximálně zhoršit její vztah s čínským režimem, a proto jsem přiletěl sem. 

Rád bych natočil deep fakové video, ve kterém fiktivní přítel rozbušky prozradí čínským tajným službám její skutečné záměry a zapříčiní tak, že zatčení křesťané budou odsouzeni k smrti za velezradu. Bývalou policistku to přiměje k akci. Vím, že v sobě najde odvahu, udělali jsme z ní blázna a právě blázni jsou největší hrdinové.“

Ač jsem Jieho po letech viděl rád, svým příběhem mě tedy rozhodně nenadchnul. Co se však dalo dělat? 

Fetu si stoupnul před kameru, přeříkal Jiem předepsaný text a počítač z něj rychle a snadno udělal bělounkého Evropana.

Následně jsme nabourali osobní počítač a email náhodného občana té země pohodlných vykuků, z níž rozbuška pocházela, a z něj pak Číně video nabídli. Zabralo.

„Už to hoří! V největší čínské věznici Tilanqiao se vzbouřili vězni a přebrali kontrolu nad objektem,“ oznámil nám asi dva týdny po odeslání Jie (a řekl to docela bez emocí). „Vrátím se do Číny, abych revoluční oheň maximálně rozdmýchal.“

„A co uděláš s nimi?“ otázal jsem se ho a ukázal z okna. Už od Jieho příjezdu postávali na ulici dva Číňané a ostražitě hlídali mou vilu.

V tu chvíli se Fetu pousmál a Jie skoro také, byť se mi to možná jen zdálo, taky alespoň my odpověděl: „Žijeme v době technologií, v době, která přeje lidem s velkou představivostí. Čína mě za tuhle cestu pochválí.“ Na ta slova vytáhl tablet a ukázal mi druhé deep fake technologií vylepšené video. Pravděpodobně jej sehrál s Fetuem, nicméně počítače z mého adoptovaného syna udělaly mě.

„Máš tedy poslední šanci. Buď přestaneš troubit do světa protičínské lži na sociálních sítích, nebo zveřejním vše, co o tvých podnikatelských aktivitách vím,“ vyhrožoval mi Jie v oné rádoby skrytou kamerou natočené scénce.

„Ne. Šířím pravdu a nehodlám se od toho odradit,“ oponoval jsem mu.

„Buď zavoláš odpovědnému manažerovi a řekneš mu, ať stáhne ze sociálních sítí vše, co se týká Číny, nebo tě zastřelím,“ přitvrdil Jie.

„To bys neudělal.“ Následoval výstřel a velmi realistický pád k zemi. Tím video skončilo.

„Až se ti to bude hodit, vystřelím z pistole a zavolám sanitku, která akorát přijede, záchranáři vyskočí a odnesou něco na nosítkách, nu a pak zase odjedou. Tou dobou však už uteču z domu, půjdu přímo naproti té ostraze. Předám jí flešku s videem a pochlubím se, že jsem se vetřel do tvé přízně a pak se tě pokusil přimět, abys stáhl ze sociálních sítí protičínskou propagandu, což jsi neudělal, a proto jsem tě zastřelil.

Věřím, že z toho budou dost zmatení na to, aby mě nechali odletět do Šanghaje, kde se naplno zapojím do revoluční činnosti.“

Byl to Jie, takže co řekl, to udělal a věci se vyvinuly plus minus tak, jak zamýšlel.}
