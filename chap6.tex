\chapter{}

Švábům se smí lhát a po tom, co Umlil provedl, pro mě nikým víc než švábem nebyl. 
Jeho hordu jsem dohnal brzy. Putovala od dolu směrem k velké prales křížící cestě.
„Popravovat dozorce mi přišlo nechutné, ale myslím, že váš boj je spravedlivý. Chci se bít po vašem boku,“ to byla ta lež.

„Zvláštní přání, ale budiž ti splněno,“ zasmál se Umlil.

Po dvou dnech cesty jsme skutečně dorazili k velké štěrkové cestě, která byla postavena teprve nedávno a zambijská armáda na ni zatím nezvládla umístit pořádné hlídky.

 „Pokud ty svině chtějí proniknout hlouběji do vnitrozemí, budou muset použít právě tuto cestu. A na tom si vylámou zuby. My přes ni svrhneme pár stromů, skryjeme se za ně i kolem nich a pak vetřelce zasypeme takovým množstvím olova a koktejlů, že se jim nebude chtít pokračovat,“ oznámil nám Umlil spokojeně, když si místo prohlédl.
 
I já zjistil vše, co jsem potřeboval, a tak jsem už první noc utekl a vyrazil směrem na jih.

Snažil jsem se jít svižně, ale nikdy jsem nebyl velký výletník, a tak mé zásoby ubývaly výrazně rychleji než vzdálenost, která mě dělila od jihoafrických jednotek.

A i když jsem zkoušel nahradit chůzi vytrvalostním během, došlo jídlo a voda dřív než já. Co mi však zbývalo? Nic než pokračovat dál.
 
Zvládl jsem to ještě den. Když ovšem začalo pršet, zastavil jsem se a pak se s velkým elánem pustil do chlemtání vody přímo z cesty. Moc efektivní způsob pití to nebyl, a tak jsem chlemtal a chlemtal mnoho minut.
 
Cesta dešťové vodě dodala nijak zvlášť libou prachovou chuť. Kdybych za sebou neměl už noc a den žízně, nedokázal bych ji polknout. Ale ani takto jsem nebyl schopen pozřít cokoliv.

Lokal jsem a lokal jsem, když jsem náhle polkl něco mimořádně nechutného. Rychle jsem se zvednul a podíval se do strůžku, který ke mně to odporné svinstvo poslal.
	
 Na hladině jsem uviděl duhu, jaká vzniká, když se někde rozlije olej. To byl zázrak. Nechal jsem chlemtání a rozběhl se po cestě proti proudu stroužku.
Má naděje se naplnila, za několik set málo metrů, už mě prohledával jihoafrický voják.    

\vspace{0.75cm}

„Kolik jich je?“ tázal se mě asi hodinu po prohledání poddůstojník pověřený komunikací s místními.

„Něco mezi padesáti a šedesáti, jsou to samí domorodci. Žádný cvičený voják. Ozbrojení jsou důlní technikou, molotovovými koktejly, puškami a loveckým náčiním.“

Poddůstojník se rozesmál. „Mnohokrát děkuji, že jste nám to přišel říct. Byla by to velká škoda se na takovou švandu pár dní netěšit.“   
 
Jeho dobrá nálada se již brzy přesunula i na dvě stě dalších vojáků, kteří intervenční jednotku tvořili. Byl jsem z nich znechucený. Vůbec nepřemýšleli nad tím, že to, co dělají, může jedině uškodit všem zúčastněným. Navíc jsem pojal obavy, že převaha Jihoafričanů nad Umlilem a jeho bojovníky bude taková, že se po střetu nezastaví a připraví další nevinné domorodce v nejlepším případě o obydlí, v horším o příbuzné a v nejhorším o život. Musel jsem jihoafrické nájezdníky přesvědčit, že už Zambii zasadili dost tvrdou ránu.
 
„Proč tu vlastně jste?“ otázal jsem se proto jednoho večera poddůstojníka, který mě jako první vyzpovídal.
 
„Prostě plácáme Zambii přes prsty. Učíme její politiky, že lhát o našem režimu se jim nikdy nevyplatí.“

„A jaký je váš konkrétní cíl?“

„Kdyby nějaký byl, nesměl bych ho prozradit náhodnému čínskému vězni, který se jen tak potlouká po pralese, ale to že žádný není, před vámi tajit nemusím. Prostě jdeme, jdeme a jdeme, dokud neuděláme nějakou pořádnou lumpárnu. Samozřejmě, že něco jako plán existuje, ale zatím jsme vždy nadělali dostatečné škody ještě dřív, než jsme k slibovanému cíli dorazili.“
 
Tohle mi znělo nadějně, začal jsem doufat, že zvládnu potrestat Umlila, zabránit expanzi, uštědřit ránu Číně a zařídit si mimořádně žádoucí život. Touha, díky níž se celý plán zrodil, však nepocházela z konfuciánských Nebes a mého nejlepšího svědomí, nýbrž z temného po moci a slávě toužícího nitra mé osobnosti. Váhal jsem proto, zda bych ji neměl potlačit, ale krom svého původu nebyla ničím špatná a navíc se jevila jako příliš silná na to, abych se jí vzepřel. Začínal jsem být skutečným dítětem bohyně Let.

\vspace{0.75cm}

Vojáci postupovali o dost rychleji, než já, když jsem jim šel naproti. Byli cvičení a všechna těžší zavazadla si vezli v neprůstřelné dodávce.

Po třech dnech jízdy jsem je upozornil, že do večera se střetnou s Umlilem. Ten den byla rekordně dobrá nálada a navečer jsem mohl smutně pozorovat, jak moc byla opodstatněná.
	
Díky kvalitním vojenským dalekohledům si Jihoafričané prohlíželi stromovou barikádu Zambijců ještě dřív, než je mohl kterýkoliv domorodec vidět.

Když zhodnotili situaci, vyložili z dodávky zásoby jídla a rozdělili se na tři skupiny. První, sestávající přibližně z třiceti mužů, naskákala do dodávky. Druhá, sto sedmdesátičlenná vyrazila do lesa a třetí, dvoučlenná hlídala vyložená zavazadla, jako jediná nedisponovala žádným fotoaparátem.

Dodávka jela a jela, dokud někdo dva metry za ní nepohodil molotovovův koktejl.  Směs se na chvíli vznítila a pak zase zhasla, ke škodě přišel pouze vrhač, kterého v rychlosti postřelil jeden voják z druhé útočné skvadry, která už také dorazila na místo.

Koktejl byl následován velkým náklaďákem, který vyjel z lesa a pokusil se do dodávky najet, ta ovšem rychle zacouvala a řidič důlního stroje musel začít svůj vůz pod lehkou palbou z obou bočních stran otáčet. Přitom se otřel o stromovou barikádu a vyrobil v ní dlouhý ale poměrně široký průjezd.

Když obě vozidla opět stála čumák proti čumáku. Rozjel se mohutný stroj znovu. Dodávka nejprve lehce zacouvala a pak ho ladně objela a průjezdem, který vytvořil, se dostala na druhou stranu barikády.   

Umlilovi muži za kládami ukrytí se sotva stihli otočit a z dodávky už vyskakovali profesionální vojáci Jihoafrické republiky, kteří z nich do několika minut udělali hromadu mrtvých těl.

Mezitím už jejich kolegové v pralese rozvrátili boční křídla. Koho nezastřelili, ten před nimi utekl do pralesa. Umlil ztratil většinu svých mužů, zato afrikánci jen tři. (Z toho jeden byl alkoholik, který uviděl u padlého Zambijce pivní lahev a knot neknot se z ní napil).

Umlil padl. Jsem přesvědčen, že nijak netrpěl, neboť bil se tak urputně, že se ani nelekl vojáka, který mu zezadu prostřelil hlavu.

Zato Caifu přežil. Zambijci ho akorát přivázali ke stromu a nacpali mu do pusy roubík.  Osvobodil jsem ho od obojího a následně jej zasvětil do svého plánu.

Po triumfální likvidaci Umlilovy domobrany měli Jihoafričané v plánu pokračovat nerušeně dál. Já však s Caifuem navštívil velícího důstojníka a přesvědčil ho, aby se nejprve stavil v našem dolu.

„Tohle místo má obrovský význam pro čínsko-zambijské vztahy a Čína je pro Zambii nejdůležitější mezinárodní partner, jeho vyřazení by tuto zemi poškodilo víc než vypálení nějakého městečka,“ uvedl jsem ho. 

Řekněte mi, co konkrétně chcete,“ vyzval mě velící důstojník.

„Privatizovat ho. Já a Caifu se staneme správci a většinovými vlastníky, to už je jasná věc a svými rozhodnutími to nijak nezměníte. Jihoafrické republice a Vám osobně můžeme dát určité podíly ze zisků, pokud ukončíte intervenční misi a naopak nám pomůžete důl zabezpečit před případnou čínskou snahou o znovuobsazení svého dolu. “

Důstojník dlouho přemýšlel a nakonec se pomocí radia ozval svým nadřízeným, kteří si vynutili pár dalších podílů i pro sebe, na nabídku kývli a poslali nám několik armádních inženýrů s úkolem „maximálně pozdvihnout úroveň strategického dolu“.

\vspace{0.75cm}

A tak začalo jedno z nejnáročnějších období mého života. Důl totiž po revoltě námezdních pracovníků vůbec nezůstal v dobrém stavu. Nejen že byla zničena, ztracena, nebo poškozena většina důlní techniky, ale mnoho vězňů také odmítalo pracovat, a to buď proto, že již nebylo Wuwanga, který by je za lenost mučil, anebo proto, že se báli čínského návratu do dolu a tvrdého trestu pro všechny, kteří by se na privatizaci podíleli.

Aby se práce znovu rozjela, musel jsem osobně promluvit s několika desítkami lidí, poznat jejich strachy a přání, vymyslet pro nově nabytý podnik takovou strategii, aby se ona přání na rozdíl od strachů naplnila a pak ho představit demotivovaným. Například těm, kteří měli strach z čínské odplaty, jsem slíbil, že jakmile důl začne vydělávat, začnu pro ně zařizovat letenky do Tchaj-peje a jiných měst svobodného světa.
	
Další problém spočíval v tom, že v průběhu střetu Afrikánců s Umlilovou domobranou vězni zcela vyrabovali zásoby jídla, zdravotnických potřeb i oblečení a vzhledem k tomu, že jsem si je nechtěl hned na začátek znepřátelit tím, že bych jim věci s pomocí jihoafrických vojáků zase bral, musel jsem zajistit doplnění zásob. A tady se objevil další problém – důl samotný měl jen minimální zásoby hotovosti, protože drtivou většinu zisků posílal do Číny. Půjčit nám nikdo nechtěl, protože nevěřil, že se udržíme déle než pár měsíců. 
 
Nakonec jsem vzal zavděk jednou lichvářskou půjčkou, díky níž jsme mohli opravit důlní stroje. Dále jsme zrušili centrální stravování a přídělový systém toaletního papíru, zubní pasty, pracovního oblečení atd. a zavedli táborovou měnu. Všichni lidé dostali na začátek rovnou částku a další peníze se do oběhu přidávaly prostřednictví každodenních odměn pro ty, kteří pracovali. Od samého začátku jsme také bývalým vězňům nabízeli možnost zaplatit si spánek v komfortnějších chatkách dříve obývaných dozorci, popřípadě výjezd afrikánskými vojenskými auty do některých blízkých měst, kde si mohli užít kontakt s civilizací. Ze všeho nejhodnotnější však byla letenka do kterékoliv země světa, jíž si bylo možné za obrovskou částku pořídit i před tím, než se důl stal výdělečným. Tyto služby daly penězům od samého začátku reálnou hodnotu, a tak se již brzy v táboře zrodila malá, ale funkční tržní ekonomika.

Ukradené věci byly již brzy prodávány potřebným a my nemuseli řešit jejich kritický nedostatek. Zprvu na kapitalistickém experimentu vydělávali ti největší sobci a hamouni, kteří v době bitvy nakradli nejvíc, ale to bylo nutné zlo, které navíc časem odeznělo. Když se totiž vyčerpalo kradené zboží, bylo třeba začít s velmi náročným importem. Naštěstí se našlo pár podnikavých pracantů, kteří si tak moc přáli odletět pryč, že tvrdě pracovali pro v okolí žijící domorodce a nechávali se za svou práci odměňovat dobytkem, materiály na výrobu oděvů a dalších surovin, která pak povýšili na produkty, jež za nemalé peníze prodávali ostatním. Vzhledem k vysoké poptávce a tedy i cenám si mohli začít brzy najímat pomocníky a platit jim mnohdy víc než my v dole.

 Díky tomuto opatření se důl nastartoval a mnozí pracovali přes čas, aby si vydělali dost na nedostatkové produkty popřípadě přímo letenku. Jejich tempo ještě vzrostlo, když těžba znovu začala být zisková a my nahradili umělou táborovou měnu skutečnými penězi. Zároveň bylo také třeba začít plnit slib, že začneme zájemcům kupovat letenky do zemí jejich snů. Abychom udrželi morálku, přišli první na řadu ti nejbohatší, kteří tak mohli vlastní prací získané kwachy využít k jiným účelům. Méně pracovití a podnikaví vězni si stěžovali, že je tento systém nespravedlivý, ale mýlili se. Tohle nebyl východní blok podporující líné ovce bez nápadů ale chrám bohyně Let.
 
Když jsme se zbavili lichváře, který nám kdysi půjčil na stroje (poslal jsem na něj pár afrických výrostků, kteří ho jednou v supermarketu unesli, odvlekli do prázdného skladiště a udělali mu „za jeho mrzkou živnost“ takové věci, až mě z toho dodnes mrazí), začal jsem bohatnout i já. O zisky dolu jsem se musel dělit se Zambií, Jihoafrickou republikou, domorodci v okolí dolu a spoustou jednotlivců, kteří nám podnik umožnili vůbec rozjet, takže mé příjmy nebyly nijak zázračné, i tak jsem se cítil velmi úspěšný a byl rád, že jsem měl v samých počátcích víru, která mi umožnila pracovat dlouho do noci a s klidem řešit nepředvídatelné problémy, které se na nás hrnuly ze všech stran. 
 
Nechal jsem si postavit pěknou vilu poblíž vesnice domorodců, celé dny v ní trávil jednáním s manažery a občas si udělal volno a vyrazil na výlet.

Během jednoho z těch výletů jsem znovu navštívil místo, na kterém kdysi došlo k boji mezi Umlilem a Afrikánci.

I po dvou letech bylo stále patrné, že zde někdo umřel. Aniž bych je hledal, narazil jsem v porostu na asi čtyři kosti.  Vzpomínky, které se mi při té příležitosti vybavily, nebyly veselé. 

Pocítil jsem nenadálé výčitky. Vše, co jsem v posledních letech získal, pocházelo ze zrady.  To, co provedl Umlil v táboře, bylo hrůzostrašné, ale jihoafričtí vojáci byli ještě horší. Trestat čerta ďáblem je stejně špatné, jako snažit se ho jím vyhnat. A já si to uvědomil až teď! Ze sebe samého se mi zvedl žaludek. Usoudil jsem, že bych měl svůj čin nějak odčinit.

Při pohledu na kosti roztroušené na zemi jsem si vzpomněl na dřevěné figurky, které Umlil vyřezával pro svého syna. Ztratily se. Já byl ovšem dost bohatý na to, abych jeho dítěti nedal jen inspirativní figurky, nýbrž skutečnou budoucnost.
	
Následující den jsem vyrazil směrem na jih. V batohu jsem si nesl vše, co se do divočiny hodí, ale jinde je postradatelné, například boty. Bylo zač se postit.

Chůze po rozpáleném betonu (nechali jsme dříve prašnou cestu zpevnit a byznysu to velmi pomohlo) na boso, mě přiměla, abych trochu přemýšlel nad účelem svého postu.

Čím víc jsem dumal, tím víc jsem si uvědomoval obrovský nesoulad mezi tím, co bylo správné na základě mých na Konfuciově učení založených názorech a vnitřní intuicí určující, co chci a co mi připadá dobré, která už byla plně oddaná bohyni Let. Na jednu stranu jsem si uvědomoval, že mé konání ani zdaleka neodpovídalo Konfuciánským normám, ale na druhou cítil, že to, co díky mým svévolím a hříchům vzniklo, je zcela úžasné.

Do cíle jsem dorazil ještě dřív, než byly tyto otázky zodpovězeny. Lidí, které jsem potkal po té, co jsem vylezl z pralesa, jsem se doptal na vesnice, které před dvěma lety vydrancovali vojáci Jihoafrické republiky, následně ony vesnice obešel a od všech se snažil zjistit něco o Umlilovi popřípadě jeho synovi.

Po měsíci a půl, jsem uspěl. Dorazil jsem do úplně obyčejné africké pospolitosti, kde mi řekli, že sirotek Fetu, jehož otec Umlil odešel a nikdy se nevrátil, a jehož matka zemřela na nějakou chorobu, nejspíš pase kozy u nedaleké skály. 

Vyrazil jsem tam a vskutku zde našel černouška, jemuž mohlo být kolem deseti, tedy podobně jako Jiemu. Měl u sebe sice pár koz, ale mnohem více se soustředil na skálu před sebou.

„Dobré odpoledne chlapče, před lety mě tvůj otec Umlil vyzval, abych ti přinesl budoucnost. Tak jsem tedy tu.“

Hoch se ke mně otočil, zkoumavým pohledem mě změřil, ale nic neodpověděl.
	
 „Byl to odvážný muž, velký bojovník. Trochu jsem se s ním o některých věcech nepohodl, ale zaslouží si, abych jeho přání splnil. Víš, co jsou to počítače?“

Hošík mlčky zavrtěl hlavou.

„Takové hračky, které lidem, co to s nimi umí, umožňují čarovat. Ty jsi ještě mladý a zvládneš se naučit s nimi pracovat. Nechtěl bys takový počítač?“

Žádná, ani mimická odezva. Nevěděl jsem, co říct, a tak nastala chvíle trapného ticha.

Sundal jsem si batoh a vytáhl z něj nepromokavý pytlík. „Hele, já chápu, že jsi zmatený a ještě se moc neznáme, ale myslím to s tebou dobře,“ řekl jsem a sáček mu podal. Bylo v něm padesát tisíc zambijských kwach. 

Fetu sáček otevřel a překvapeně vykulil oči. „Díky,“ hlesnul nesměle. Dál už ovšem neříkal nic.

Bezmocně jsem se rozhlédl kolem. Můj pohled padl na Fetuovu oblíbenou skálu. Skutečně byla zajímavá, na několika místech se zeleně třpytila. Má práce už mě naučila poznávat leckteré kameny, a tak jsem bezpečně poznal, že má hodnotu mnoha miliard kwachů, byla totiž plná smaragdů. 

Objev smaragdové skály vedl k tomu, že jsem byl odhodlaný se do oněch míst vrátit a začít zde na vlastní pěst těžit, ale to hlavní se nepovedlo. Fetu si mě nehodlal pustit k tělu a už vůbec nebyl ochotný se mnou někam jít.

Vrátil jsem se tedy sám. Nyní už v botách. Má víra, že když se budu rozhodovat tak, abych svým jednáním svět přibližoval ideálním vztahům, dosáhnu něčeho dobrého, byla značně otřesena a po návratu do bývalého dolu zcela padla.

Číňané se po dvou letech rozhodli ukončit soukromé podnikání ve svém dole. 

Když jsem kráčel po pralesní cestě, vyskočili na mě z křoví tři vojáci, namířili na mě puškami a donutili mě zahodit zbraň, jíž jsem si na cestu vzal.
	
Naštěstí byli úplatní. Dal jsem každému z nich čtyři tisíce dolarů, otočil se čelem vzad a pak se rozběhl do nejbližšího jižního města.

Zde jsem zjistil, že Číňané proti podniku zakročili podobně tvrdě jako kdysi proti demonstrantům.Všichni bývalí vězňů doposud v dole pracující (naštěstí už jich mnoho nezbývalo a jednalo se o ty nevětší lenochy) byli popraveni způsobem, ze který by se ani Wuwang nestyděl – vojáci je nejprve mučili petrolboardingem a po té zapálili. Byl jsem přesvědčen, že kdybych v dolu zůstal, podařilo by se mi s Číňany vyjednat něco lepšího, anebo bych alespoň osazenstvo dolu zmotivoval k ozbrojenému odporu končícímu alespoň důstojnější smrtí. 

 A právě toto přesvědčení definitivně zabilo mou víru, že „správné vede k dobrému“ a že existuje nějaký svatý řád, kterým se člověk může řídit. Mílovým skokem jsem se tak přiblížil bohyni Let.
 
Můj život však pokračoval dál. V bance jsem si vybral peníze a vyrazil do Jihoafrické republiky najít někoho, kdo by investoval do mé těžby smaragdů. Sám jsem měl peněz dost, ale na vybudování nového dolu to ani náhodou nestačilo.

Hledal jsem dlouho, pak se mi ovšem poštěstilo potkat podnikatele Errola Muska, který již s těžbou smaragdů zkušenosti měl a jeho majetek tomu odpovídal. Společně jsme se domluvili, že mi za pětatřicetiprocentní podíl půjčí lidi, know how a techniku. Přijal jsem a znovu se pustil se do práce.

Rozjet podnikání bylo napodruhé mnohem lehčí. Už jsem znal firmy, se kterými stálo za to spolupracovat, a osvojil si čich na nalezení správných lidí do týmu. Když jsem k tomu přičetl kontakty a jiné prostředky, které mi přinesla spolupráce s panem Muskem, nacházel jsem se v mnohem žádoucnější situaci než před třemi lety. Ani tentokrát však byznys nebyl lehký a nemít nezlomnou víru a vůli, sotva bych se s jeho pomocí učinil jedním z nejbohatších lidí na světě. Ještě než jsem začal pořádně těžit, musel jsem přesvědčit zambijskou vládu, že se kvůli mně nezopakuje čínská intervence na její území, a to podepsáním prohlášení, že se nebudu nijak protičínsky angažovat ve světovém dění. Neměl jsem zrovna dost peněz na to, abych politiky uplatil, a tak nezbývalo než požadavek podepsat. Ani minutu jsem však nepočítal s tím, že jej budu respektovat.

 Další problém, se kterým se kterým jsem se musel vyrovnat, byla malá vzdálenost dělící smaragdová ložiska a africkou vesnici. Přestěhování lidí žádné nadstandardní kroky nevyžadovalo, potíže nám však způsobily posvátné baobaby. Domorodci absolutně nezvládali uvěřit, že by tak majestátní stromy bylo možné přesadit, a tak značná část z nich trvala na tom, že si zemi předků vzít nenechají, a to i když jsme pro ně už postavili nové domy s přívodem vody a solárními panely na výrobu elektřiny.
 
Nakonec nezbylo než si najmout oddíl konžských žoldáků a udělat z nich eskortu chránící několik lesních inženýrů, kteří ignorujíce nejprve k smrti vyděšené a posléze agresivní domorodce stromy přesadili do nové vesnice. Tímto činem jsme porušili zambijské právo a zbožnější obyvatelé vesnice nás oprávněně předvolali před soud. Afrika je ovšem kontinentem společenského zla všeho druhu – válek, epidemií, pověrčivosti a v neposlední řadě také korupce. Pro úspěšnou firmu tak nebyl problém všechny soudy vyhrát (a ještě si hrát na hodnou, když za chudé vesničany uhradila soudní výdaje).

 V zemi, konkrétně v mém bývalém dole, se však děly i podstatně zlověstnější věci. Koncem devadesátých let přivezlo několik amerických turistů z dovolené fotky, které si pro svou hrůzostrašnost získaly velkou publicitu. Na oněch záběrech byla vidět těla na první pohled života neschopných lidských mutantů, kterak se válí v pralese nedaleko od čínského dolu.
 
Po krátké senzaci západní veřejnost usoudila, že se jedná o podvrh, a snímky upadly v zapomnění. Já však už dříve registroval, že Číňané jeví mimořádný zájem o těhotné matky a sváží je právě na území měděného dolu, a tak tyto snímky jen prohloubily mé podezření, že se zde děje něco nekalého.

Ať už ale Číňané v Africe podnikali cokoliv, můj byznys to nijak nepoškozovalo. Znovu jsem si mohl v blízkosti dolu postavit přepychovou vilu a právě do té za mnou jednoho rána přišel Fetu.

„Dobré ráno, chlapče“ přivítal jsem ho.
 
„Dobré ráno. Máte ještě počítač?“ hlesnul nesměle. 

 „Ano,“

„Tak to máte jistě také knihy.“

„Mám spoustu knih.“

„Mohl bych si nějaké půjčit?“

„Jistě. Jaké by sis přál?“

„Nějaké, nad kterými se musí přemýšlet ale ne počítat.“

Chvíli jsem přemýšlel a pak vytáhl z poličky zaprášenou a ošoupanou brožuru, která mě doprovázela na všech mých dobrodružstvích. Nezvládli mi ji sebrat ani ve vazbě ani v táboře. „Jmenuje se Hovory, Konfuciovy Hovory, a klidně si je nech, já už z ní vyrostl.“
