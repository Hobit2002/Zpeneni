\chapter{}

Zpátky do Jieho bytu jsme běželi. Ten zmijištír ho ovšem zamknul a akorát před dveře položil papír s následujícím vzkazem: „Bratři, selhali jsme. Strávili jste příliš mnoho času na ulicích. Policie už nejspíš ví nejen, že se Šanghají toulá jeden neregistrovaný cizinec a zdravotník, který se strojí jako policista, nýbrž má i poměrně přesné informace o vaší poloze. Mým velkým selháním je, že jsem tomu nezabránil. Nadále si přeji, aby byla Lydie a s ní i naši bratři a sestry z podzemní církve osvobozeni a mám pár plánů, jak toho dosáhnout. Žádný z těchto plánů by ovšem nevyšel, kdybych byl zatčen, což by se zcela jistě stalo, pokud by vás policie našla v mém bytě.

Věřím, že i pro vás plápolá plamínek naděje. Když se na delší čas schováte do divočiny, snad se mi podaří některé informace odstranit z policejních databází a vy se pak budete moct, třeba s nějakou falešnou identitou, vrátit a dílo dokončit.

A abyste se oprávněně necítili zrazeni, vyhodil jsem z východního okna plechovku od Coca-Coly. V té plechovce jsou klíče od mého auta. Kdyby vás náhodou chytli, nezapírejte, že jsem vám ho půjčil. I nyní mám plán.“

Ten vzkaz vykouzlil na mých rtech největší úsměv za posledních pár let. Jiří už znal čínskou mimiku dost dobře na to, aby věděl, že se jedná o projev velmi koncentrované směsi strachu a vzrušení. On jakožto Evropan situaci reflektoval verbálně: „Do háje s Jiem a hlavně do háje s námi, nějakého hodně dalekého háje.“

Na delší promluvy než odtušení ovšem nebyl čas. V rychlosti jsem ze sebe strhal policejní uniformu, vyběhl před dům, tam ji rychle hodil do odpadkového koše a ze silnice sebral plechovku od limonády.  Aspoň, že v něčem démon nelhal – plechovka skutečně obsahovala klíče, se kterými jsme odemkli Jieho auto, naskočili dovnitř a vyrazili. Jiří ani nelezl do kufru, ale zkroutil se pod sedadlem spolujezdce.

A vůz vystřelil plnou rychlostí vpřed směrem od moře. Sice jsem si netroufl porušovat dopravní předpisy, ale jinak si dával jsem si pozor, abych náhodou nejel nižší než nejvyšší povolenou rychlostí.

Zrovna když jsme opouštěli Šanghajskou aglomeraci a dostávali se do obce Shihudang, něco v zadní části kufru zavrzalo. Jiří se okamžitě vztyčil, aby zjistil, co to bylo a ještě ve zlomku sekundy zabránil čínské policistce, která zezadu vyskočila, aby do mě vrazila a zcela jistě tak nepřímo způsobila manévr, který by se nejspíš nevešel ani do silnice ani do množiny těch, které čínské silniční zákony akceptovaly.

Policistka si prošla výcvikem a uměla pár bojových umění, ale Jiří byl muž na vrcholu svých fyzických sil, který za sebou měl pár let aikida, takže v zápase zpočátku bodoval. Než se mu podařilo přimáčknout strážkyni tyranských čínských zákonů, která se do auta nějak záhadně dostala, na podlahu, dostal sice pár úderů pěstí do obličeje a pár ran kolenem do břicha, dokonce i auto následkem jejích kopanců provedlo několik hrůzostrašných tanečních figur, ale byl to právě on, kdo soupeře složil na lopatky. 

Doufal, že má vyhráno. Tiskl její ruce k zemi a klečel na jejím břiše. Policistka nejprve chvíli vypadala jako přemožená, ale šlo jen o zdání. Trvalo to jen pár sekund, než tvrdě potrestala jednu docela malou Jiřího chybu. Můj druh si nechal příliš nízko hlavu.

 Policistka tedy zvedla tu svou a vší silou ho udeřila do čela. Před očima se jistě zatmělo oběma, ale Číňanka s tím počítala, a tak se zvládla hned po svém úderu vyprostit a kopnout ho oběma nohama do břicha. Jiří se svalil na řadicí páku a pedály. Já panicky otočil volantem a auto nabouralo do vozu ve vedlejším pruhu.
 
Všemi, kteří jsme v autě byli, to hodilo směrem k policistce. Airbagy se však postaraly o to, že nikdo nepřišel ani o život ani o vědomí. Když se Číňanka vzpamatovala z šoku, namířila na Jiřího pistolí, kterou si zpoza pasu vytáhla ještě během kopance a zmáčkla spoušť.

Usoudil jsem, že setrváním v autě bych si akorát přivodil jistou smrt, a tak jsem vyskočil ven. Ozvala se rána a nakonec i víc ran.  Podíval jsem se předním oknem dovnitř a zjistil, že ani první ani žádný další výstřel neudělal vůbec nic. Policistka byla slepotou svých nábojů překvapená očividně podobně jako my její přítomností. A já pochopil, že má smysl vrátit se do boje.

Když Číňanka usoudila, že snaha Jiřího zastřelit je marná, vytáhla obušek a začala Evropana bít. Stihla ovšem jen dva údery. Zrovna, se napřahovala ke třetímu, když jsem zvenčí otevřel dveře, o které se opírala, a ještě vší silou trhnul za její vlasy. Chudinka spadla zády na asfalt.

Jiří byl šeredně potlučený, ale plně si uvědomoval vážnost situace. S mou pomocí vylezl z auta a pak se kulhavě rozběl. K bouračce došlo, když jsme zrovna jeli po mostě, jedinou únikovou cestu tedy nabízela silnice.

Policistka, která se v rámci svých možností velmi rychle, tedy opravdu pomalu, zvedla, sáhla po vysílačce. Posily se snažila zavolat, už když se schovávala v kufru auta, ale něco rušilo její signál (což byl taky důvod, proč se rozhodla zaútočit přímo). Nyní se však navázat spojení podařilo a krom toho přijížděly první vozy, aby řešily autonehodu. Ti dva pochroumaní běžci tedy měli jisté, že je zanedlouho bezpečnostní složky obklíčí. Ona sama by už nemusela dělat nic, to se však příčilo její činorodé čínské nátuře.

Sehnula se pro čepici, která se jí při pádu asi dva metry odkutálela a vyrazila za námi.

Jiří uběhl jen pár desítek metrů, když klopýtl. Pomohl jsem mu se zvednout a poznal, že už pokračovat v běhu nezvládne. On si to uvědomoval taky.

 „Nečekej na mě!“ přikázal mi. Tento čin ovlivnil dějiny. Policistka byla v tu chvíli tak blízko, že jeho slova slyšela, a ta nesobeckost ji u lidí, které měla za zločince, velmi překvapila.

K druhému nevědomky prorockému činu došlo jen o tři minuty později. Tehdy policistka Jiřího dohnala. Už ho měla takřka na dosah obušku, když se náhle doposud kulhající ubožáček rozběhl s vervou mladého buvola. Pronásledovatelce nezbylo než přidat také. 

Ovšem Jiří neměl na to běžet, onen sprint byl pouhým zdáním. Ve skutečnosti udělal dva kroky a pak se vrhnul dopředu. Předem však věděl, že a jak dopadne. Tím získal obrovskou výhodu proti policistce, která si sotva všimla, že narušitel neběží, ale leží, a už na tom nebyla o nic lépe. Jiří ji chytil za nohu, takže už podruhé ten den letěla hlavou na asfalt. Nyní si aspoň stihla před obličej strčit ruce, takže si akorát po dopadu překousla jazyk. Oči se jí zalily slzami bolesti, a když se vzpamatovala, nebyla už ani na zemi ani v Jiřího náruči (kde si pár sekund pobyla), nýbrž ve vzduchu. Evropský dobrodruh ji hodil z mostu do řeky.

Čína se o své policisty stará dobře. Zvláště pak, když jdou do akce v zimě. To je nezapomene vybavit teplým, pevným a neprůstřelným oblečením, které je ovšem trochu těžší než oblečení běžné a rozhodně se v něm velmi špatně plave. Policistka, která přeci jen nebyla zcela bez bolístky, by měla s plaváním potíže i bez toho, že několik vrstev jejího oblečení nasáklo vodou a začalo ji táhnout ke dnu.

‚Ona se topí?‘ pochopil její stav Jiří. ‚To jsem tedy nechtěl. Nehodlám mít na svědomí žádné zabití.‘ Neváhal. Rychle si sundal bundu i mikinu a vrhnul se do řeky za policistkou.

Ač byl sám polámaný a vyčerpaný, dokázal ji za vlasy dotáhnout až ke břehu a zde ji přivést k životu. Zachráněná se zrovna probírala a s obrovským údivem si uvědomovala, kým to byla zachráněna, když už se Jiří s jejím obuškem vrhal dohrát svou roli proroka až do úplného konce.
\vspace{0.75cm}

Policie byla rychlá. Jakmile dostala od policistky zprávu, poslala nám naproti dva vozy a právě tyto policejní vozy mě během Jiřího kříšení zachráněné tonoucí potkaly. Jak jsem je uviděl, rozběhl jsem se opačným směrem zpátky k Jiřímu. To se ví, že mě chytli. Zkoušel jsem se bránit, ale proti osmi policistům byl můj vzdor bez šance.

Zrovna mě spoutaného vraceli do auta, když mi Jiří přiběhl na pomoc. Dokázal se nepozorovaně připlížit k policejnímu autu, kterým jsem měl být přepraven do vězení, a v pravý čas tomuto vozu vylezl na střechu. Z té se pak vrhnul na záda jednomu z policistů, kteří mě drželi. Přistál mu na ramenou a hbitě ho přetáhl obuškem přes hlavu tak silně, že dotyčný ztratil vědomí. Ještě se napřahoval k další ráně, když rovnou dva policisté vytáhli pistole a vyprázdnili do něj zásobníky. ´ 

Obětavý Evropan se k mému smrtelnému zděšení zhroutil z policistových ramen na zem. Jeden z policistů došel k řece Dahzeng pro svou kolegyni, usadil ji na místo původně určené pro Jiřího a vyrazil s ní do nejbližší nemocnice. Policistka by se ho moc ráda vyptávala, o co se její zachránce pokoušel, ale přední část jazyku zřejmě ztratila v řece, a tak se jí mluvilo špatně.

Svůj poslední pohled Jiří věnoval mně. Pokud z něj šlo něco vyčíst, bylo to asi toto: „Vida, světlo. Třeba nejste zcela mimo.“
