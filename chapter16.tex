\chapter{}

Ani Jie nevěděl o všem a dokonce i on někdy přehlédl věci zcela zásadní. Například to, že lesy v okruhu šedesáti kilometrů od základny slouží každou noc jako cvičiště pro čínské vojáky, zjistil až na poradě, tedy ještě později než Jiří.
\vspace{0.75cm}

První setkání skauta s Čínskou lidovou osvobozeneckou armádou, mohlo proběhnout tragicky. Nestalo se tak. Jiří popošel kus od silnice do lesa, tam si sednul ke stromu a snažil se vzpomínkami na všeliké noční hry a přepady cizích skautských táborů zaplašit obavy, že by zde mohl potkat vojáka či jiného příslušníka čínských bezpečnostních složek.

Nicméně právě to stalo. Po necelé půlhodině Jiří uslyšel šramot. Předpokládal, že se jedná pouze o nějakého hlodavce, pak ovšem uviděl siluetu vojáka v plné zbroji, jak se rychle a soustředěně plazí kamsi dopředu. 

To Jiřího vyděsilo. Ačkoliv si byl na základě četných zkušeností jistý, že vojáka nyní zajímají jen vlastní ruce, nohy a země pod ním, na faktu, že sotva mohl být vyložen na méně bezpečném místě, se nic neměnilo.

Chvíli po prvním vojákovi, se kolem něj proplížil také druhý a třetí. Čtvrtý prolezl mezi Jiřím a silnicí (tedy dobrodruhovi za zády) a stejně tak i osmý a desátý. Evropan byl zkrátka zcela obklopen vojáky.

Naštěstí jej všichni soustředění rekruti přehlédli, a pokud ho přeci jen nějaký z nich registroval, považoval ho nejspíše za důstojníka pozorujícího, jak dobře cvičení zvládají. Dokonce i v Číně se však našli neohrabaní flákači.

Se značným odstupem po většině svých kolegů, přibatolil se k Jiřímu jeden obtloustlý vojín, jehož styl pohybu byl takový, že se i doposud na smrt vyděšený Jiří, který navíc ve tmě viděl jen část pohybů onoho nemehla, takřka rozesmál.

Pobavení ho ovšem brzy přešlo. Voják totiž náhle něco zlostně vykřikl, posadil se, zašátral v kapse pro hroznový cukr a pak rozsvítil čelovku. Kužel světla ozářil celé Jiřího tělo. Baculatý mladíček něco překvapeně zvolal.

„Tuto noc jsem tu pro tebe,“ oslovil jej pevným hlasem Jiří jednou z naučených frází a postavil se. Zdálo se, že voják přinejmenším porozuměl, stále se tvářil nedůvěřivě, ale když k němu Jiří přistoupil blíž, nezastřelil ho, ba dokonce ani nevykřikl.

„Jen se mi svěř. V autě nám bude nejlíp,“ zkombinoval vetřelec rovnou dvě fráze, ve snaze nechat se rekrutem odvést k nějakému vozu, kterým by mohl ujet. Na mladíka ovšem příliš přesvědčivě nezapůsobil. Číňan udělal pohyb, který baterkou oslněný skaut vyhodnotil jako snímání samopalu z ramen.

„Neboj. Tuto noc jsem tu pro tebe,“ začala docházet odhalenému slovní zásoba. To už na něj ale Číňan namířil zbraň a cosi mu poručil. Jiří usoudil, že pokud ho poslechne, bude zatčen a posléze popraven, nikoliv však sám nýbrž s Jiem a Lydií, které do té doby udá. Rozhodl se tedy bojovat a na rekruta se vrhnul.

Buclatý vojáček bleskurychle přišel o svou zbraň a výměnou za ni dostal roubík a kanadská pouta. Pak už se jen nechal Jiřím bezmocně vláčet lesem podél silnice dál a dál od své základny.

Asi po hodině a půl kdy oba bezmála běželi, Jiří usoudil, že už jsou dost daleko a donutil své rukojmí, ať na kraji silnice poklekne. Tam mu pantomimicky předvedl, že pokud se zvedne, zastřelí ho.

Jinak byl ovšem mimořádně velkorysý, rozvázal vojákovi ruce, dal mu je před břicho, vložil mezi ně vysílačku a zase je zavázal. Pak mu ještě jednou pohrozil a schoval se do lesa.

Zmatený mladíček se ozval svým důstojníkům a tak naplnil vše, co od něj Jiří očekával. Velení, mezi kterým už po jeho zmizení nastal poplach, okamžitě na místo poslalo rekruty cvičící v okolních lesích a jeden obrněný transportér, který měl po likvidaci nebezpečného vetřelce odvézt všechny zpátky na základnu.

Vojenské auto na místo dorazilo o dost dřív než první pěšáci. Řidič, přijel a zastavil těsně vedle klečícího vojáčka. Jiří to chvíli pozoroval a žasnul, že i v tak efektivní zemi, jako je Čína, se někdo může dopustit takové hloupé chyby. Vyběhl z lesa, ale to už řidič nezávisle na něm začal couvat. Jiří přesto stihl naskočit na stupínek před dveřmi spolujezdce a pažbou samopalu vysklít okno.

Řidič reagoval tak, že zrychlil, český dobrodruh se pokusil vlézt otvorem do auta, ale nezdařilo se. Zato řidič tlačítkem na palubní desce odemknul zamčené dveře a pak je kopnutím otevřel. Doufal, že se mu tak podaří evropského soka setřást, ale ten se díky tomu akorát dostal dovnitř. Jen, co měl Jiří řidiče na dosah ruky, pokusil se ho vyhodit tak, jako už mnoho lidí, ale voják byl silný muž a ukázalo se, že má nad exstudentem práva značnou převahu, byť musel nejprve zastavit auto, aby se svému hostu mohl věnovat naplno.

Jakmile ovšem vůz stál, natáhl se po pistoli. To se Jiřímu zdálo velmi neférové, on sám už ho mohl dávno zastřelit samopalem, ale neudělal to, protože nechtěl mít na rukou ničí krev. Situace ovšem byla natolik kritická, že si dovolil, mít ji na hlavni cizí zbraně. Ještě než ho tedy řidič stihl zastřelit, vrazil mu hlaveň ukradeného samopalu do zubů a to takovou silou, že na podlahu auta dopadly čtyři řezáky, čtyři špičáky i pár stoliček.

Řidič vykulil oči, tenhle způsob boje se samopalem ještě nezažil. Jiří tím ovšem nekončil, začal na pušku čím dál tím víc tlačit a tím ji zarážet hlouběji a hlouběji do řidičova hrdla. Silný Číňan pustil pistoli, jednou rukou bleskurychle odemkl dveře na své straně, druhou je otevřel a pádem z vozu unikl.

Jiří se usadil na jeho místo a vyrazil plnou rychlostí vpřed. Projel dvě zatáčky a málem ze silnice smetl osobní auto. Uvědomění, že svou jízdou ohrožuje jiné účastníky silničního provozu, vedlo Jiřího, beztak sžíraného vědomím, že musel svému sokovi vyrazit zuby, k tomu, že vozovku opustil a vyrazil přímo ze svahu dolů.
\vspace{0.75cm}

Vojáci ukončili poradu a Jie byl asi jediný, kdo z nečekané situace mohl mít jiný než jen zcela negativní pocit (i když těžko říct, zda ten sociopat vůbec kdy něco cítil). Tehdy se mu ještě Jiří hodil živý, a tak nejspíše ocenil, že jeho střet s ČLOA dopadl nejméně špatným možným způsobem.

Jakmile čerstvě bezzubý řidič nahlásil krádež svého vozu, byly do terénu poslány čtyři helikoptéry každá s infračervenou kamerou a šesti kvadrokoptérami obloženými výbušninami a navigovanými teplem. Právě z obrazu, který se z kamer v helikoptérách přenášel na interaktivní tabuli v poradní místnosti, Jie pochopil, že se Jiří řítí přímo ze svahu dolů a že tedy již zanedlouho skončí v řece Xiliao, kde drony ukradené auto dostihnou a zničí.

Byl by velmi nerad, kdyby přitom jeho „hrdina“ zahynul, zvlášť když ho napadalo, jak mu pomoct. Omluvil se tedy, že jde na záchod a tam mi zavolal. Nebudu již opakovat slova, kterými mě do celé té historie zapletl. Můj čas i dech se krátí až příliš rychle.
\vspace{0.75cm}

Jiří se řítil z kopce dolů, když náhle uslyšel někde vzadu ránu a ucítil, jak se auto otřáslo. Podíval se ven a zjistil, že zadek jeho transportéru hoří. Také si všiml také několika létajících a světélkujících objektů, které se k jeho vozu nebezpečně přibližovaly. Rychle pochopil, s kým má tu čest a zoufale šlápl na plyn, aby využil maximum svých pohonných hmot, ještě než vyletí se vším všudy do povětří.

Věci se však měly jinak, než očekával. Všiml si, že vyjel z lesa, ale rychlost vozu byla taková, že si novou krajinu ani nestihl prohlédnout a transportér už nejel, nýbrž ležel na boku. Jiří se ho rozhodl urychleně opustit. Otevřel dveře u sedadla spolujezdce a vyskočil z nich.

Zrovna, když dopadl do vody, vrazily do auta další dva drony a Jiří usoudil, že pokud ho pronásledují robůtci navádění teplem, bude nejlepší, když se před nimi schová do vody. Tak tam tedy čekal, pozoroval, jak další a další drony postupně vůz likvidují a ač omdléval zimou, nedovolil si vylézt, akorát si podepřel hlavu kameny, aby se v případě skutečné ztráty vědomí neutopil. Moudře učinil. 

Naše příběhy se tak mohly již za pár hodin spojit.