\chapter{}

V následujících měsících se jen minimum věcí mohlo vyvíjet hůř. Nemoc z vypuštěných netopýrů se skutečně přenesla na lidi a ve Wuhanu tak začala epidemie, kvůli které muselo být celé město uzavřeno do karantény.

Bohužel však k tomuto uzavření došlo příliš pozdě a to i když se Jie (a zdaleka nejen on), respektive Wen-liang, snažil varovat úřady tak moc, až mu hrozila ztráta svobody (či spíše její čínské napodobeniny) a jemu nezbylo než podepsat „že přestane šířit lživé zprávy“.

Nemít ohromnou smůlu, našel by si Jie jistě jiný způsob, jak na virus upozornit, ale jeho nepřítel, který už se čile šířil mezi lidmi, jej předběhl. Při jednom očním vyšetření se můj syn nakazil lékem, který kdysi sám vyvíjel, a už se neuzdravil. 7. 2. 2020 zemřel ve Wuhanské nemocnici na nemoc Covid-2019.

Jeho smrt vyvolala značné pobouření a dokonce i lidé v Číně si nahlas stěžovali na zhoubný nedostatek svobody, bohužel virus se šířil dál. V době uzavření Wuhanu už virus promořoval Irán a severní Itálii. Trvalo jen pár měsíců a lidé po celém světě byli nuceni zůstat zavření doma.

Podle oficielních statistik pandemii nejhůře odnesly Spojené státy a Evropa. Byl to ovšem pouhý klam. Ve vyspělých a svobodných zemích se testovalo víc než kde jinde, díky čemuž bylo případů nákazy diagnostikováno také víc než kde jinde. Nikdo se zároveň nesnažil tato čísla zatajovat popřípadě všemi možnými způsoby vykreslovat v co nejlepším světle. Většině nakažených se dostalo zdravotní péče, a když zemřeli, byli do statistik započítáni jako oběť, i když měli kromě koronaviru i jiné, často výrazně vážnější, problémy. 

Mimořádně špatná byla i situace ve Vděčné zemi. Její královna Lydie Olofssonová (král Ping, jehož imunita byla značně oslabená stálým stresem a nedostatkem spánku, chorobu nepřežil) se pro velké množství nakažených a totální kolaps nejen veřejných i soukromých služeb rozhodla vydat zemi zpátky do pod čínskou správu a vrátila se do své švédské vlasti. 

Skutečné peklo začalo jinde – ve slumech a favelech Latinské Ameriky, Asie a samozřejmě taky Afriky, kde v bídných podmínkách na sebe namačkané živořily mnohdy až statisíce lidí. Tito chudáci neměli ani čistou vodu ani osobní prostor, aby se mohli před virem chránit, ani peníze, aby si mohli dovolit zdravotní péči, ani zásoby jídla, aby mohli uposlechnout vládní nařízení a nevycházet ven.

Za ty roky, co jsem podnikal v Zambii, vytvořil jsem si k Africe zvláštní, takřka otcovské, pouto. Miliony lidských životů, které by mohly trvat podstatně déle, kdyby Čína nebyla odpornou diktaturou a pár potřeštěných fanatiků používalo hlavu víc než srdce, mě neskutečně mrzelo.

Skoro jsem upadl do deprese, ale byznys mě naučil, že jediná cesta z problémů vede skrz ně, a tak jsem odolal a vynaložil maximální úsilí, abych pandemii i její dopady co nejvíc zmírnil. Spolu s Billem a Melindou Gatsovými jsme vrazili obrovské množství peněz do vývoje vakcíny a když už za sebou měla první klinické testy, nechali ji vyrobit v obrovském množství, aby jí bylo již hned po schválení k dispozici co nejvíc.
 
Dalším z mých kroků bylo okamžité vypnutí umělé inteligence, která pomáhala populistům a střeleným politikům obecně vyhrávat volby. Volby v roce 2020 tedy Donald Trump prohrál. „Poněkud ztrácím víru v tvou filosofii, Fetu“ vysvětlil jsem tento krok svému synovi „pandemie a neschopnost světa, především Spojených států, ji konstruktivně řešit, je způsobena především tím, co jsme v posledních letech podporovali – bezmyšlenkovitou odvahou. Globalizovaný svět není africká savana a co fungovalo tvým předkům křovákům, je již mimo hru. 

Naopak vidím, že všude na světě nad krizi poráží zástupy lidí, kteří se nesnaží vyniknout a dělají, co se jim řekne.“

Fetu na mou výtku nezareagoval ani tehdy, ani následujících pět let. V roce 2025 mě však znenadání pozval na Tchaj-wan, že prý našel startup, který by si zasloužil investici. Byl jsem poněkud skeptický ohledně Fetuových schopností poznat dobrý podnik – chyběly mu totiž jakékoliv zkušenosti. Nicméně syn je syn, a tak jsem jednoho letního večera vešel do tchajwanské restaurace a ihned se mě ujalo několik velmi zdvořilých mladíků.

„Prý jste vynaložil obrovské množství peněz, abyste uspíšil vývoj vakcíny a pomohl tak světu v boji s covidovou pandemií. Jsme proto velmi poctěni, vaší pozorností, i my bychom se totiž rádi postavili jednomu globálnímu problému.“

„A to jakému?“ otázal jsem se jich.

„Klimatické změně. Máme řešení, které by mělo definitivně zastavit globální oteplování a jediným důvodem, proč jsme si přáli, abyste přijel vy za námi a ne my za vámi, jak by bylo přirozené, je to, že bychom vám jej rádi předvedli.“ 

Byl jsem sice velmi nedůvěřivý, ale přesto jsem se nechal usadit do létajícího taxíku a vyrazil směrem k místu, kde mi měl být onen zázrak předveden. Dron se řídil sám, a tak se mi mohli podnikaví mladíci naplno věnovat.

„Jsme z Honkongu, kdysi skvělého města. Když jej ovšem Čína začala v roce 2019 poutat k sobě, usoudili jsme s bratrem, že nás zde nic dobrého nečeká a přestěhovali se na Tchaj-wan. Zpětně to rozhodnutí oba dva hodnotíme jako nejlepší v našem životě. Tahle země je plná pracovitých a vynalézavých lidí, kteří si cení toho nejzákladnějšího - jedince.

Když jsme opouštěli Honkong, cítili jsme se hrozně špatně. Ten pocit, že úžasné město plné života již brzy sevře tvrdá ruka čínských komunistů a na dlouhé roky tak zadusí nejen zdejší poctivou podnikavou atmosféru ale i tu nejzákladnější osobní iniciativu nás přiváděla k zoufalství. Shodli jsme se, že bychom měli udělat něco pořádného, ať jsou naše životní příběhy jednou dobré. A tak vzniklo něco, co by nás asi v Číně stálo život. A k tomu něčemu jsme právě dorazili.“

Chvíli po těch slovech přistál taxík na střechu poměrně nevzhledné a zchátralé budovy. „Sám vidíte, že nám máte co dát,“poznamenal jeden z mladíků, když zvětřil mé znechucení jejich pracovištěm.

Když jsme byli uvnitř v prostorné pokusné hale, předvedli mi chlapci několik sice obyčejných, ale zároveň výkonných a velkých infračervených zářičů, které namířili proti několika plechovým buňkám zvenčí podobných těm, které obývají dělníci na stavbách. Ovšem ještě než byly lampy rozsvíceny, obletěl jeden z mladých inženýrů buňky dronem, ze kterého stoupal dým.

Když se zářiče rozsvítily, pozvali mě mladí podnikatelé dovnitř. Buňka byla velmi přepychově zařízená, a tak jsme zde strávili příjemnou hodinu hovorem o Honkongu, Tchaj-wanu i Zambii‘. (Původně jsem se snažil vyzvědět, čím přesně se mi bratři inženýři chtějí pochlubit, ale vyprosili si právo mi vše předvést názorně).

„Jak to, že tu není horko? Vždyť zářiče by měly toto místo změnit v totální peklo“ otázal jsem se svých hostitelů, když za námi bylo sedmdesát minut vzpomínání. Tento dotaz je zjevně velmi potěšil.

„To je právě ten vynález. Jen se pojďte podívat ven,“ vyzval mě jeden a já ho uposlechl. V hale bylo skutečně k zalknutí, a to i když člověk nestál mezi zářiči a buňkami. Inženýři mě vyvedli ven z budovy a vysvětlili mi, jakou roli ve fyzikálních překvapeních tohoto večera hrál dron.

Ukázali mi, že se uvnitř nacházela nádrž plná nadýchané a namodralé pěny, která se snadno vypařovala ve formě plynu odrážejícího obrovské množství tepelného záření.

„Tím bychom rádi obalili planetu,“ vysvětlil mi jeden z mladíků.

Zarazilo mě, s jakou suverenitou to prohlásil. Hodinové odrážení infračerveného záření z několika reflektorů přeci jen klade na produkt výrazně nižší požadavky než celoplanetární tepelný štít. 

Se svými pochybnostmi jsem se před chlapci nijak netajil. Naštěstí na ně byli schopni patřičně reagovat.

„Máte pravdu. To, co jste zkusil, by rozhodně k masovému využití termopěny nestačilo, ale můžeme vám ukázat i více jejích vlastností,“ na ta slova se se mnou mladíci vrátili do dílny a začali mi předvádět pokus za pokusem.

Mohl jsem vidět, jak vložili pěnu do platinové nádoby, nechali ji, aby se vypařila a pak ji intenzivně ozařovali ji UV i IR zářením a na počítačích mi ukazovali, že se štěpí především na kyslík, vodík a pak ve velmi malém množství dlouhou řadu sloučenin, z nichž nezanedbatelná část byla poměrně nebezpečná, což ovšem pro malé objemy nemělo představovat problém.

Krom ukázky toho, jak je plyn z pěny uvolněný stabilní a v případě rozštěpení poměrně bezpečný, mi bylo názorně předvedeno, že pro živé organismy není sice dýchatelný, ale jinak jim neškodí, i to, že je dost lehký na to, aby se udržel i ve vysokých vrstvách atmosféry.

„Jsme nevýslovně rádi, že žijeme ve svobodné zemi. Kvůli tomuhle projektu jsme předčasně ukončili svou vysokoškolskou docházku, a kdyby se to stalo v Číně monitorované hodnotícími algoritmy, sotva by nám banka na náš projekt půjčila peníze a už vůbec by se nám nepodařilo koupit si za ně pokročilou techniku, na jejímž používání vývoj závisí,“ zakončil prohlídku jeden z inženýrů.

Tehdy jsem jim žádné peníze nedal. Vůbec by nebylo těžké mě, v technické oblasti totálního laika, oblafnout. Prvním penězům musela předcházet druhá návštěva v doprovodu inženýra a ekonoma, kteří by zkontrolovali, že si kluci na nic nehrají.

Ona druhá návštěva se uskutečnila již o měsíc později a skončila skvěle. Chlapci ode mě měli přislíbeno několik miliónů dolarů. Když jsme se vraceli z jejich dílny, položil mi Fetu následující otázku: „Před pandemií se svět posouval díky dravcům s ostrými lokty, během ní byl zachráněn díky tichým a pracovitým mravencům a teď, a nejen kvůli téhle termopěně, se protagonisty stávají tiché bouře. Co mají tyhle duše světa společného?“

Podíval jsem se na noční oblohu, kde se hvězdy mísily s čerstvě vypuštěnými satelity, v tu chvíli jsem si uvědomil, že po letech začínám věřit a to nikoliv starým mudrcům a jejich slovům, nýbrž tomu, co jsem měl nad sebou – aspiracím a věčnému dítěti v každém člověku, které věří, že bude každý den dokonalejší než ten předchozí. A snad i nějaké vyšší, neprokazatelné a nejspíš neexistující moci, bohyni Jen, Mars, Eu… na jménu nesejde, která toto dítě miluje a dělá vše proto, aby mělo dostatek svobody pro své hry.

Fetuovi jsem však takto vzletně nezodpověděl: „Vůli, vlastní vůli. Autentické odhodlání člověka dostane přes každou velkou překážku, ať už je to chavézie, pandemie nebo klimatická změna. Právě díky ní bude pozítřek takový, jaký bychom si přáli zítřek. 

Aby někde bylo autentické odhodlání, musí zde být svoboda. Svobodní lidé nejsou svázáni ani totalitou, jaká drtí čínský lid, ani iracionálními zásadami, které ty blázny z Vděčné země vedly k tomu, že rozpoutali pandemii. 

Svoboda neochrání člověka. Jie by nezemřel, kdyby přijal za svá jak pravidla komunistické Číny, tak alespoň základní zásady slušeného chování. On byl ovšem svobodný a jeho špatné stránky – naprostá absence respektu k čemukoliv a komukoliv a krajní nepoctivost, se rozvinuly tak moc, že ho to stálo život. Jeho dobré vlastnosti – obrovské odhodlání až do sebezničení sloužit lidstvu a nesmírná otevřenost novým myšlenkám, však mohly rozkvést taky a díky tomu dnes lidstvo není decimováno chavézií. Takto ze svobody jedince profituje celá společnost.

A nyní je to tu znovu. Další příběh, který dokazuje, že jedině svoboda může přinést vykoupení. 

Jedině ve svobodě může skrze iniciativu zezdola dojít ke Zpěnění.“
