\chapter{}

A tak se Chun stala křesťankou. Skokem z mostu se sama pokřtila, ale už když třesoucí se zimou a nadšením vylézala z veliké a neskutečně špinavé řeky Huangpu, bylo jí jasné, že ke skutečnému duchovnímu životu bude potřebovat také bratry a sestry. Naštěstí jich pořád měla ve vězení víc než dost. 

Za svého duchovního mentora si vybrala mě a začala na mé cele trávit tolik času, až se její kolegové dostali strach, zda se do mě nezamilovala.

Ona se však stala mou sestrou a družkou v duchovních rozhovorech, nikoliv milenkou.

„Kristus je autentická radost ze služby bližním, když tě tato radostná láska prodchne, prodchnul tě Kristus,“ poučoval jsem ji například.

„A je Kristus nějak přítomen i pro nešťastné?“ tázala se Chun podezřívavě.

„Ano, v podobě chuti a vůle si vždy najít něco, co dává smysl pro zlepšování lidského společenství – církve, obecné církve. Pokud chceš věřit, musíš v sobě tuto vůli najít a poté plně prožít svou cestu za službou.“

A tak se Chun učila prožívat. Velmi se jí zalíbila idea každodenního duchovního rituálu a rozhodla se, že si taky nějaký vytvoří. 

Pozdní večer co pozdní večer tedy vyrážela na most a zastavovala se u zábradlí, aby mohla jít do sebe a naslouchat svým vnitřním hlasům, které byly zřejmě s křesťanstvím velmi spokojené. Ona nekontrolovatelná část její mysli nechávala nad řekou Huangpu poletovat anděly a dopřávala Chun tu čest vést snově nesmyslné diskuze s proroky. Jejich identita se však nezměnila a Chun se po okamžicích šílenství nemohla zbavit pocitu, že se událo něco divného a šíleného. Avšak Kristus neměl být odtrženost, nýbrž intenzivní láska k úplně všem. A skutečné setkání s takovýmto Kristem Chun teprve čekalo. 

Někdy se dívala na řeku, jindy směrem do silnice a opakováním naučených postupů v sobě zkoušela Krista vyvolat. 

Bohužel, žehnání okolnímu městu ne a ne probudit tu hledanou lásku ke světu a štěstí jak z nebeského společenství, přesto se nevzdávala, učila se mít ráda lidi a přinejmenším v tom činila jisté pokroky. Jednou večer, když kolem ní procházela povědomá osoba, pocítila, že se v ní konečně vzedmula láska k bližním. Aniž by přemýšlela nad tím, koho před sebou má, otočila se ke kolemjdoucímu, rozpřáhla ruce k požehnání a řekla „Miluji tě,“.

Dotyčný se zastavil, zvedl hlavu a stroze odpověděl. „Já tebe ne. Ale společně strávené noci bych se nebránil,“ Chun vyděsila ani ne tak odpověď samotn, jako její autor - Gang. Vůbec netušila, co si teď psychopatický dozorce kolega myslí o ní a o jejím vztahu k němu.

„Myslela jsem to vážně, ale jinak než se domníváte,“ pokusila se věci uvést na pravou míru.

„Jako ti velezrádci, které hlídáš?“

„Asi ano,“ přitakala Chun, kterou pojem „velezrádci“ nepříjemně překvapil.

„Škoda, nevadilo by mi s tebou spát,“ odvětil Gang skoro příjemně a odkráčel.

Série překvapení, kterých se od Ganga dočkala, tím ovšem zdaleka neskončila.

Dva dny po té, co Gang přijal Chunino vyznání sesterské lásky, rozhodl se navštívit mou celu.

Když otevřel dveře, našel čtyři vězně a svou kolegyni, jak klečí v kruhu na podlaze a pološeptem se společně modlí. „Družíš se s těmi podsvinčaty a dopouštíš se tak vážného prohřešku. Ale jelikož jsi mi nedávno vyznala lásku, předpokládám, že aspoň tvé srdce zůstává na straně pravdy a komunismu,“ napomenul Chun.

„Bratře Gangu, křesťané milují obzvláště své nepřátele,“ namítl jsem.

„Mlč!“ zařval Gang a bleskově mě udeřil obuškem do čelisti. „Tohle jsem se už beztak dozvěděl od tajných služeb.“

Zmínkou o tajných službách vyvedl z míry všechny přítomné.

Chvíli jsem mlčel a po té odvětil: „Tím spíše je nyní její jméno zapsáno v Beránkově knize. Bůh vám to odpusť.“ (V rukách tajných služeb většinou končili zvlášť „nebezpečné živly“, kteří podle komunistické strany potřebovali také tajný konec)

„Lháři!“ zařval Gang a ohnal se po mě znovu obuškem. Za pár minut jsem měl hlavu na zemi a krev na podlaze. „Ty tu lítost jen předstíráš! Dobře víš, že Bo Zhao je už dávno na svobodě! Ty tu kvůli ní ani nejsi! Podívej se na tohle!“ pokračoval a vytáhl z kapsy mobil, chvíli na něm otvíral různé složky a pak mi ho strčil před nos. 

Na obrazovce jsem uviděl Evropana, který poměrně plynulou angličtinou (jež měla z nějakého záhadného důvodu africký přízvuk) přednesl následující slova.

„Jsem si jist, že jste v nedávné době lapili tohoto člověka (na ta slova se vedle řečníka začal objevovat bezpočet Jiřího fotek). Býval to můj přítel, ale časy našich sympatií jsou pryč. Vím, že měl železnou vůli a že se vám z něj možná doposud nepodařilo získat účel jeho cesty a jeho snahy proniknout do některé z Vašich věznic. Pro jistotu Vám ho tedy povím já, chtěl pomoci na svobodu této osobě. (A okolo mladého muže se zase začaly objevovat obrázky mladé slečny nordického vzezření). Pokud vím, jedná se o novinářku, která vážně ohrozila čínské zájmy. O tom, že se ti dva znali, svědčí například tyto fotky (a původně nevinné snímky z Jamboree, na kterých spolu Jiří a Lydie komunikovali, se staly nepřímým důkazem, že jsme se dopustili velezrady) Jako důkaz pravdivosti mých slov nechť poslouží konverzace, ve kterých jsem se snažil svému, tehdy už jen známému, jeho záměr vymluvit (na ta slova se objevily printscreeny z Messengeru a E-mailu).		

Potrestejte toho zmetka maximálně tvrdě a postarejte se o to, aby všichni, kteří s ním spolupracovali, dopadli stejně jako on.“

Byli jsme usvědčeni. Nevěděl jsem co dál, a tak jsem jen sklesle hleděl na podlahu.

Chun dostala mobil hned po mně a rozplakala se ještě dřív, než video skončilo. Později mi řekla, že ji v tu chvíli strašně zklamal celý svět. Jak mohl tohle někdo svému bližnímu udělat? Jak mohl někdo tak odporně zradit svého přítele? Náladu jí ještě zhoršil Gangem vyřčen verdikt: „Nejen, že jste nám lhali, nýbrž jste napomáhali pokusu o velezradu. Jistě víte, že když zradíte svou zemi, není v ní pro vás místo. Mohu vám předem oznámit, že budu vaším katem.“
 
Příliš mnoho špatných zpráv na jednu hodinu. Chun byla totálně rozhozená, slzy jí zamlžily zrak a čím méně toho viděla očima, tím víc obrazů začala produkovat její nekontrolovatelná intuice. Svět se jí změnil v jedno peklo, kde lidé vynakládali obrovské úsilí, aby mohli zradit své přátele a kde se každý aspoň trochu mocný třásl touhou po krvi svých bližních. Její bláznivá povaha všechny tyto pocity mocnila do extrémů pro lidi s obyčejnou myslí nepředstavitelných. Nedbajíc na pracovní dobu utíkala z toho hrůzného doupěte pryč, na střechu.	

Když se tam dostala a osaměla, začal nesmírnou hrůzu střídat jiný pocit, kupodivu klid. Už se stmívalo (a to skutečně, nejen v Chunině tvořivé mysli), ale dozorkyně spatřila obrovské zlaté Slunce, kterak vychází na západě a postupně stoupá vzhůru. Vizuální vjem byl následován slovy pronesenými zvučným shůry přicházejícím hlasem: „Chun, má dcero, já jsem s tvými sestramii bratry, stejně tak jako s Tebou. Nedopustím, aby byl komukoliv z vás byť jen zkřiven vlas na hlavě. Věř mi, žij mou láskou a dělej to nejlepší, co tě napadne. Všichni jste v mé péči.“

„Mám pro tebe důležité poslání!“ tato slova už však pronášel Gang, který za Chun na střechu vylezl.

Chun se zvedla z kleku, spustila k tělu rozpražené ruce a otočila se. V tu chvíli jí bylo jasné, co po ní Bůh chce – obrátit Ganga na pravou víru.

Mohlo se to zdát nemožné. Stačilo s ním pobýt jen pár minut a člověk si uvědomil, že pokud někdo není kompatibilní s dobrou komunitou, pak je to právě Gang. Ale Chun věřila a její nadšení pro nový cíl bylo tak intenzivní, že za Gangem přiběhla a políbila ho.

„Opravdu by mi nevadilo s tebou spát,“ reagoval na to Gang.

„Mé lásky k tobě velmi nabylo, avšak stále se jedná jen o lásku sesterskou,“ odpověděla Chun.

Nestihla to ani doříct a Gang jí vlepil bolestivý pohlavek. „Je-li tomu tak, pak tomu tak již brzy být nesmí. Byla jsi nakažena, ale já tě vyléčím. Budeš mi asistovat při popravě. “

Chun z toho přeběhl mráz po zádech, ale věděla, že aby naplnila své poslání, musí být Gangovi tak blízko, jak to jen půjde a tak zajásala: „Díky, díky moc!“
\vspace{0.75cm}

Termín popravy přišel asi týden po těchto událostech. Gang, silně inspirován svým dávným mistrem Wuwangem, se rozhodl masovou popravu pojmout patřičně autorsky. 

Nechal všechny vězně přivést na dvůr a připoutal je vedle sebe podél zdí. Sám si stoupl doprostřed a na zem pečlivě vyskládal nabité zbraně, na každého vězně jednu, zapůjčené od svých kolegů.

Ti postávali okolo a celou scenérii pozorovali. Specielní roli dostala Chun, která byla pověřena podáváním zbraní Gangovi. Její kolega doufal, že tak lépe pochopí, která strana je ta její. Gang si od ní vzal první zbraň, namířil ji na prvního vězně a rychle střelil, střela však byla slepá, určená jen k vyděšení. Druhá, ostrá rána už šla do země. Chun Ganga zezadu objala a tím strhnula jeho ruku dolů. 

„Opravdu tě miluji Gangu a chci být v tvé blízkosti nejen nyní, ale i po smrti v Božím království.“ křikla na nadřízeného, ten se pokusil vyprostit, ale Chun ho mačkala téměř nadlidskou silou. 

„A Bůh Tě miluje, bratře Gangu ještě víc než já, víš, jak bys ho tímhle zklamal?“ svá slova doplnila polibkem. Trochu doufala, že sadistický dozorce omdlí, on se však bránil dál, až se Chun začala obávat, že ho už dlouho neudrží. Nezbylo než zužitkovat jednu z jeho oblíbených praktik – hlazení pažbou pistole na klidný spánek. Fungovalo! Mírné klepnutí a Gang už ležel.

Sice s těžkým srdcem, ale preventivně, vystřelila Chun párkrát směrem k svým bezbranným kolegům, kteří se okamžitě rozutekli, a pak už se jala osvobozovat své spoluvěrce.
