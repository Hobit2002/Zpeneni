\chapter{}

S výjimkou omráčeného Ganga se všechno dalo do pohybu. Osvobození křesťané začali odpoutávat ty, na které se dosud nedostalo. Šokovaní dozorci běželi pro ozbrojené posily, osvobození vězni si pro všechny případy vzali Gangovy zbraně a ozbrojené posily přiběhly na dvůr. To už ovšem na plácku žádní křesťané nebyli.

Zatímco se po zuby ozbrojení dozorci dohadovali, co podniknout a kam se podívat, proletěla jim nad hlavou policejní helikoptéra.

„Centrálo, tak zlé to není, zvládneme to i bez techniky,“ ozval se okamžitě velitel komanda do vysílačky, a aby dal svým slovům za pravdu, poslal ihned polovinu jednotky prohledat skladiště náhradního oblečení.

„Budete muset,“ zněla strohá odpověď. „a nejen bez techniky nýbrž proti technice.“

„Ten vrtulník není od vás?“ otázal se velitel.

„Ztratili jsme kontrolu. Zatímco jste si brali zbraně, ti velezrádci nám ukradli techniku, budete muset…“ začalo se ozývat, ale helikoptéra se snesla níž a hluk, který dělala, přehlušil dosavadní komunikaci s nadřízenými.

Shodou okolností jsem v tom vrtulníku seděl (jakmile nás Chun osvobodila, rozběhli jsme se do garáží a na heliport, abychom získali nějaké dopravní prostředky, v nichž by se dalo rychle utéct někam dost daleko). Když jsem uviděl těžkooděnce pod sebou, pocítil jsem velkou lítost, že neznají světlo víry. Proto jsem pilota vyzval, aby se kousek snesl.

S jistou nevolí poslechl. Když jsem si viděl s členy komanda do obličeje, začal jsem jim pantomimicky naznačovat, ať zahodí zbraně a nahradí je Biblí.

Velitel dal na chvíli vysílačku od ucha a gestem rozkázal podřízeným, aby po nás začali střílet. Můj pilot však dříve pracoval jako záchranář, a tak se naše helikoptéra se poměrně elegantně vznesla zpátky do vzduchu. To do ní ale narazila jiná, potřeštěná, kterou zřejmě pilotoval nějaký nadutec slepě přesvědčený o svých leteckých schopnostech.
Naštěstí jsme srážku ustáli. O jejích vinících se to bohužel říct nedalo. Pro všechny zúčastněné bylo smutné vidět, kterak se tento možná náfukou možná zoufalcem řízený vrtulník vymkl kontrole už dokonale. Po svém ťuknutí do naší olízla vrtulí stěnu věznice, načež začala pomalu ale nezadržitelně klesat. 

Stroj byl tedy zničený, ale jeho posádka nejspíše všechno ve zdraví přežila a komandu by se tedy mohlo podařit aspoň někoho chytit. 

Velitel zavelel a ta půlka, která neplýtvala čas prohledáváním skladu oblečení, se rozběhla areálem věznice na místo, kde předpokládala zřícenou helikoptéru. 

Číňana zvyklého, že všechno má svůj řád, jakým velitel specielního komanda byl, skutečně zranilo, když ho po cestě zastavili odzbrojení dozorci s oznámením, že Chun a její spoluvěrci nejen, že ukradli helikoptéry a pár aut, nýbrž také otevřeli velkou část dalších cel. Řešit zdrhající zlodějíčky a násilníky už však přesahovalo velitelovu mentální kapacitu, a tak jen zavrčel a běžel dál.

„Obraťte se bratři“ ozvalo se dozorcům i komandu náhle za zády. Bezpochyby se jednalo o nejúčinnější misii v dějinách lidstva, jelikož během vteřiny se vskutku obrátili všichni oslovení. Zatímco dosud jejich oči směřovaly na záda velitele komanda, nyní provrtávaly Chun s megafonem. Každý člen komanda by jí s chutí zastřelil, ale to by v cestě nesměli překážet dozorci, kteří se na svou rebelskou kolegyni bez váhání vrhli.

Jakmile se první dostal do vzdálenosti jednoho metru, přetáhl Chun obuškem a srazil ji k zemi. Chun se zvedla, a rozhodla se bojovat způsobem, který jsem při našich duchovních sezeních vždy vychvaloval, nastavila druhou tvář. 

„Bůh Ti odpusť, neboť nevíš, co činíš,“ vypravila ještě ze sebe ještě, než ji zasáhla druhá rána.	

„Složte zbraně bratří, prosíme“

„Ona vás miluje!“

Chodbou přibíhali další vězni, mávali rukama a křičeli, aby dozorce zastavili, těžko říct, zda to byli křesťané, nebo osvobození zloději. Jisté však je, že veliteli komanda ruply nervy, a tak si sňal ze zad samopal a pokropil celou chodbu olovem. První zasažení byli dozorci a jeho podřízení, pak přišlo na první vězně a konečně i na velitele samotného.

Jeden z osvobozených vězňů na velitele skočil zezadu a vytrhnul mu zbraň. Odvážlivce okamžitě zpacifikoval dosud nezraněný člen komanda, ale k pomoci svému nadřízenému se příliš nehlásil vzhledem k tomu, že důstojník před chvílí zastřelil jeho kamaráda. Mezitím chodbou přibíhali další a další vězni. Měli ruce nad hlavou a volali po míru. Nikdo neměl srdce proti nim použít samopal, a tak vojáci složili zbraně.

V následujících hodinách se křesťané pod mým vedením starali o raněné nehledě na to, k jaké straně se ještě před chvílí ranění hlásili. Vydesinfikovat rány, obvázat je, uklidňovat dotyčného příběhy o Ježíši, nakonec ho pokřtít a vyrazit za dalším raněným… takto několik obětavých sester trávilo celý večer. 

Před věznicí se už tou dobou shromažďovali ozbrojení těžkooděnci čekající na povel vpadnout dovnitř a získat zpátky kontrolu nehledíce na prolitou krev a ztráty na životech.	

Chun si to všechno uvědomovala. Seděla na kraji střechy, právě na tom místě, kde se byla nedávno svou bujnou myslí ubezpečena o Boží ochraně, a pozorovala helikoptéry, které kroužily nad věznicí. Vojáci pověření reconquistou Tilanqiaa povstalcům vzkázali, že o půlnoci zaútočí a zabijí každého, kdo nebude na své cele.

Jestli se však někomu zvlášť nechtělo zpátky k vězeňskému životu, pak to byli křesťané, na které čekala spíše vězeňská smrt. A tak se nakonec každý, pro koho byla Bible svatá, rozhodl, že kanalizace je mu milejší než cela. Chun považovala za Boží zázrak, že bezpečnostní složky zapomněly hlídat kanály a utéct z vězení tak bylo mimořádně snadné. Krom několika málo výjimek však k zbrklému a neorganizovanému úprku dosud nedošlo.

Opuštěním areálu problémy ani náhodou nekončily. Čínský režim byl díky svým všudypřítomným sledovacím prostředkům víc než kterýkoliv jiný schopný lokalizovat a následně zatknout kteroukoliv osobu na svém území. Pokud by se nyní křesťané jen tak rozutekli, po čase by je policie stejně znovu pochytala. Chtělo to nějakou vizi…    

„V Associated Press News už jsem zveřejnila zprávu o našem povstání, myslím, že budeme celému západnímu světu velmi sympatičtí“ ozvalo se náhle za ní velmi dobrou angličtinou.	

Chun se otočila a poznala slečnu, kterou zrádný Jiřího přítel ve svém videu označil za skandinávskou špiónku, jejíž osvobození bylo skutečným cílem evropského hrdiny a tedy i jeho kompliců z podzemní církve.

„O tom nepochybuji. Ale pokouší mě pochybnosti, zda budeme taky živí. Vím, že Bůh se o nás postará, ale zatím vůbec netuším, jak.“

„Velmi dobrá otázka. Obávám se, že jediné, co by nás mohlo zachránit, je revoluce,“ odvětila Lydie.

Chun se na dlouho chvíli odmlčela, neboť upadla do jednoho ze svých blouznivých stavů. Měla pocit, že se jí před očima míhají různé scénáře toho, co může přijít. Když vidění skončilo, zvedla se a s myšlenkou ´Revoluce. To je ono!´ šla zorganizovat exodus.
