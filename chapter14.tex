\chapter{}

Když jsem konečně k řece dorazil, objevil jsem z části u a z části v ní jen plápolající zbytky něčeho, co asi kdysi bylo vozem.

Zastavil jsem a rychle vyskočil, abych se podíval, zda z Jiem slibovaného charismatického hrdiny něco zbylo.

A zachovalo se toho docela hodně! Jen malý kus od auta ležel v řece mladý muž, zřejmě ze západního světa. Hlavu měl podloženou kamenem a nereagoval. Voda ho podchladila natolik, že ztratil vědomí. Od toho však Jie povolal mě, zdravotníka, aby podobné nepříjemnosti nepřerostly do problémů.

Vytáhl jsem mladíka z řeky, strčil ho do auta, sundal mu jeho promočené oblečení a pak se chytil za hlavu. Neměl jsem s sebou žádné deky, kterými bych ho přikryl. Svlékl jsem se tedy ze svých šatů, natáhl je na něj a sám si nasadil jeho mokré svršky. Nu, a pak stačilo zapnout topení a co nejrychleji překročit dva prahy. První rychlostní a druhý domovní.  

Zachráněného mladíka jsem uložil do své postele, a pak napsal do nemocnice, že jsem se právě dozvěděl o nenadálé smrti svého kamaráda z dětství a že potřebuji na pár dní zmizet z práce. Šéf nebyl rád, ale vyhověl mi.

Následovalo chystání teplé snídaně sobě i svému svěřenci. Jiemu jsem napsal, ať se na večer staví a toho „svatého rytíře“ si odveze.
\vspace{0.75cm}

Východ Slunce jsem strávil pravidelnou modlitbou na střeše, a když jsem se vrátil, můj host už byl vzhůru.

„Vzbuzen?“ otázal jsem se anglicky (tedy jazykem, který mi nikdy nešel), vida, že má otevřené oči.

„Ano,“ odpověděl můj host.

„Cítíte se… dobře?“ otázal jsem se neschopen najít přesnější výraz.

„Zotavitelně,“ odpověděl mi prý, já tomu ovšem vůbec nerozuměl, a tak jsem se jen nervózně zazubil.

„Ne, teď se dobře necítím, ale vím, že v budoucnu se dobře cítit budu.“ užil jednodušších slov. „Jak jsem se sem dostal?“ dodal na závěr.

A tak jsem mu v rychlosti pověděl to, co se událo na začátku této kapitoly. Pohledný mladík pochopitelně naslouchal velmi pozorně avšak překvapivě zcela nevzrušeně.

„Děkuji mnohokrát,“ řekl s překrásným úsměvem, když jsem své vyprávění dokončil. „A teď je asi řada na mě?“ dodal po chvíli ticha. 

Jiří, tak se Jiem za rytíře označený mladík jmenoval, se narodil a vyrůstal v České republice. Patřil mezi ty ambiciózní chlapce, kteří celý život chtěli dokázat něco víc.

Tak jako většina z těchto lidí ani Jiří dlouho netušil, kterak svou touhu naplní.  

Velmi citlivě vnímal vážné obavy o klima, které se však ke konci desátých let jednadvacátého století začaly šířit mezi blahobytnou západní společností, a vyložil si je tak, že pro výjimečné činy přinejmenším bude prostor.

Netrvalo dlouho a přesvědčení, že se svět nachází v průšvihu, ze kterého se dostane ne pomocí společné vůle mas, nýbrž velkým a výjimečným úsilím jednotlivce, se stalo těžištěm jeho osobnosti.

Příčinou této velké osobnostní přeměny byla událost celospolečensky bezvýznamná.

Jiří byl aktivním členem mládežnické organizace zvané skaut. V roce 2018 už v tomto svobodném pionýru dělal vedoucího a musel zajišťovat takové věci jako třítýdenní letní tábory. To s sebou obnášelo také domluvení louky, na níž by se tábor postavil.

Jeho oddíl měl své tradiční místo, kdesi na severu Jiřího rodné zemičky.

Na této louce stála dřevěná základna a u ní se nacházela studna, z níž oddíl po mnoho let čerpal pitnou vodu. Během teplého a suchého jara roku 2018 však tato studna vyschla.

Tábor to nijak neohrozilo. Jiří zavolal hasičům, ti na místo přivezli velkou nádrž pitné vody, a když byla v půli tábora skoro prázdná, znovu ji doplnili.
 
 Děti si ty tři týdny užily. Téměř během nich nepršelo a chataři, kteří jinak v blízkosti tábořiště trávili začátek prázdnin a věčně si stěžovali na hluk a neklid, letos nedorazili, protože jim také vyschly studně. Rachejtle, kapslíky, akční noční hry… to vše se mohlo rozjet naplno.

Citlivý Jiří se však i z tohoto tábora, který mnozí ocenili jako výborný, vracel otřesený. ‚Kam se to sypeme?‘ Ptal se sám sebe.

Kdysi, roku 2015 se při celosvětovém setkání skautů, Jamboree, v Japonsku seznámil s tehdy šestnáctiletou Švédkou, která si už tehdy vedla blog a na něm publikovala zprávy o změnách klimatu. 

Ač byl její přístup velmi zaujatý, Jiří neznal nikoho, kdo by se touto problematikou zabýval důkladněji, a tak, když se rozhodl si po suchém táboře nastudovat, jak se to tedy s klimatem má, pokusil se ozvat právě jí.

Lydie, tak se Švédka jmenovala, ovšem zmizela. Toto zmizení bylo dramatické dost na to, aby o něm napsala i některá skandinávská média. 

Jakmile Lydie dokončila střední školu, oddala se své novinářské dráze naplno. Začala jezdit po světě. Navštívila Taiwan, Vietnam, Thajsko, Hong Kong a nakonec i pevninskou část Číny, odtud se však nevrátila. Čínské úřady akorát zveřejnily informaci, že se dopustila vážné špionáže.

Tento nový poznatek Jiřího počínající frustraci nad stavem světa pouze posílil.

‚Co s tím mohu dělat? Můžu to vůbec nějak změnit?‘ Tázal se sebe sama. Odpovědět si ovšem dát neuměl a tak začal hodně číst, hodně poslouchat i se hodně dívat.

„…a díky tomu se stali hegemonem stepi. Zkrátka ve chvílích krize úplně vyškrtli ze svých slovníků slovo ‚správné‘ a vší jeho autoritu převedli na pojmy ‚potřebné‘ a ‚účelné‘.“

Těmito slovy skončil jednak pro inspiraci sledovaný TED Talk, ve kterém africký filosof popisoval, za co jeho rodný kmen vděčil své prosperitě, a druhak Jiřího hledaní.  

Člověk je silný, musí se akorát zcela oddat jedné vizi a nepochybovat o tom, že může být prosazena, ať to stojí, co to stojí.

Frustrace, ambice a mravní povzbuzení, to Jiřího učinilo konečně připraveným začít jednat.
