\chapter{}
Už je to přes třiatřicet let, co básník (tehdy jsem to ještě nebyl já, ale ubohý ufňukaný melancholik) seděl na břehu řeky Xiliao, psal si úvahy do deníku a pozoroval nedávno postavená panelová města. Tušil, že až dokončí vojnu, bude právě v jednom z těchto šeredných panelových měst jakožto zcela průměrný občan bydlet.

Básník svůj život považoval za tak průměrný, že dokonce slovem Průměrnost pojmenoval vlastní deník, do něhož pravidelně škrábal své stesky. Nikdy ničím nevyniknul. Narodil se do průměrné rodiny, chodil do průměrné školy, patřil mezi průměrné studenty a obecně nijak nevyčníval.

Na samém konci tlustého sešitu, který každodenně obohatil o nějakou svou úvahu, měl ovšem velikými znaky napsané jiné slovo, které považoval za dokonalý opak „Průměrnosti“, to slovo bylo „Zpěnění“. Básník si pod ním nedokázal představit nic konkrétního a právě ten jeho potenciál znamenat cokoliv, ho na něm uchvacoval. Doufal, že konec svého životního příběhu bude moct pojmenovat právě „Zpěnění“.

 Nyní při pohledu na odporné domy, které pro lidi jako on čínská vláda budovala, pocítil vztek. ‚V tom nechci bydlet! Takhle nechci žít! Proč mi podsouváte zrovna tuto cestu? Já chci Zpěnění! Raději umřu, než abych se dál topil v průměrnosti. A ještě raději umřu v boji za Zpěnění!‘ 

Takové záchvaty nespokojenosti, způsobené většinou představou toho, čím vším by mohl být, a čím vším ho komunistický režim mít nechtěl, prožil již mnohokrát, tentokrát se ovšem rozhodl bezprostředně jednat. Ani nedokončil zrovna rozepsanou větu a začal pracovat na své poslední básni, manifestu Zpěnění.

Pustil se do díla tak zhurta, že se mu zlomila tužka. Jeho vůle začít hned a naplno však byla tak silná, že vytáhl svou služební kudlu a řízl se do tváře.

Když se pak sklonil nad sešit, aby začal psát, „inkoust“ mu docela sám kapal pod špičku nože, který využil jako perko.

Na této písni, ódě na vzdor a touhu přetrhat veškerá pouta, pracoval několik měsíců, ale stále to nebylo ono. A právě tehdy si přiznal, co předtím sám sobě roky zatajoval snůškou lží; že nechtěl být básníkem, jelikož by ho skládání veršů nějak těšilo, nýbrž proto, že si na nic náročnějšího nevěřil. 

Nebyl to příjemný pocit. Básník pochopil, že jako básník nebude stát za nic. Stále byl mladý a své budoucí zaměření si mohl vybrat, rozhodl se tedy, že se stane narušitelem a podvratným elementem, o kterého si i totalitní režim Číny se vší svou brutalitou vyláme zuby.

Nejprve se však vrátil na břeh řeky Xiliao, aby se zde rozloučil s minulostí. Klekl si k vodě, vytáhl ze své aktovky čtyři listy papíru, na kterých byla úhledným písmem zaznamenána poslední podoba jeho básně, a poslal je po proudu.

Ambiciózní mladík byl možná znechucen nedokonalostí svého díla, ale matku světa i života, bohyni Eu, zaujalo odhodlání a vůle, které z něj čišely. Celá její péče o svět spočívala právě v tom, že podobné lidi podporovala. Jak dlouhé byly dějiny lidského druhu, ohýbala realitu tak, aby nejsilnější vůle sklidila nejvíc ovoce.

Básník se však ke své minulosti otočil zády, a tak si vstřícného gesta bohyně nevšiml. A právě proto se jí mimořádně zalíbil. Takových lidí, odhodlaných, chtějících a bojujících, si cenila. Neudržela se a svého nového oblíbence pohladila.

Co krásnějšího člověk může zažít než pohlazení od bohyně? A co jiného je život, než marná snaha vrátit se do toho nejkrásnějšího, co jsme zažili? Veškerá básníkova vůle se bít a jít si vlastní cestou v tu chvíli zintenzívněla na desetinásobek. Už to nebyl ubohý uplakaný básník.

 Byl jsem to já.

\vspace{0.75cm}

 I já jsem byl po svém zrození velmi průměrný. Čína konce osmdesátých let byla plná mladých lidí, kteří si uvědomovali, že jsou pouhými vojáky, studenty, inženýry atd. a kteří by raději byli sami sebou.

Komunistická strana zrovna sesazením svého nejvyššího tajemníka Chu Jao-panga, prozápadního reformisty, ukončila období liberalizace a nás, mladé lidi, tak proti sobě poštvala.

Doufali jsme, že se jedná o konec pouze dočasný a nadále velkou naději spatřovali, kde to jen šlo.  Třeba v Gorbačovovi a to aniž bychom si uvědomovali, že naše země už dávno přitakává kapitalismu a má blíž k USA než Sovětskému svazu.  Anebo v Reganovi, nevěda, že tomu šlo hlavně o uchránění světa před jaderným konfliktem, kterým Čína nikomu nehrozila.
	
 Zkrátka, naivně jsme doufali v brzký nový začátek.
Stal jsem se aktivním členem Pekingské studentské autonomní federace, trávil mnoho času se sobě podobnými a aktivně usiloval o to, aby se naše řady rozrostly.

V té době také začal můj náboženský život. Ve studentských kruzích bylo velmi populární se v rámci revolty proti ubíjejícímu maoismu vracet k tomu, co ho předcházelo, například tradičnímu čínskému náboženství konfucianismu. Tak jako mnoho jiných studentů i já si sehnal jsem knihy jak od Konfucia, tak i od jeho žáků a téměř fanaticky přitakával všemu, co kdy dávný učitel byť jen naznačil.	

Myslím, že v oddanosti svému mistru jsem se vyrovnal prvním křesťanům. Právě nad jeho Hovory jsem čínský režim začal vnímat nejen jako nepříjemně omezující nýbrž také jako zcela zvrácený.
Zvláště Mao Ce-tung se přinejmenším z perspektivy tradičního čínského náboženství jevil jako potměšilec ve všech ohledech.

Konfucius zdůrazňoval, že fungování světa je postaveno na mnoha hierarchických vztazích, ve kterých ten vyšší, mistr, vládce či otec, pečuje o nižšího, žáka, poddaného, syna, který se mu odvděčuje úctou a poslušností.

Komunisté svým důrazem na dokonalou rovnost a likvidaci mezilidských rozdílů, nesměřovali k ničemu méně strašnému než vybudování pekla na zemi! Ač sám Konfucius před žádnou ďábelskou silou tohoto formátu nevaroval a metafyzično omezil na vzdálená Nebesa, já jsem si byl zcela jist, že Mao Ce-tung a všichni, kteří stejně jako on usilovali o vyhlazení a rozbití nepostradatelných společenských struktur, představovali zmocněnce nějaké hrozné avšak vědomé síly chtějící zvrátit svět do tupého chaosu.
A stejně tak jsem pochopil, že mé rozhodnutí vzdorovat komunistické stránce komunistického režimu ze mě udělá taktéž bojovníka proti této nebezpečné síle. 

Povzbuzen svatostí svého poslání, podnikl jsem několik cest po Číně a při každé mluvil s lidmi z venkova a menších měst o tom, proč bychom měli maximálně podporovat každé liberalizující rozhodnutí současného vedení a naopak se zuřivě bránit veškerému omezování naší svobody

Nezdá se to jako velké hrdinství. Dělal jsem to, co tehdy mnoho mladých lidí po celém světě a přesto jsem cítil, že právě toto si po mně Nebesa žádají.

Mnohokrát jsem si připadal, jak by se nade mnou rozevřela a na mě, klidně v domě mezi svými kamarády z fakulty a uprostřed noci, dopadla zář pradávného bezvadného císařství.

Měl jsem však velké štěstí, že na mě ve společenských místnostech svobodomyslných studentů dopadlo ještě něco dalšího - pohled Mei, sličné mladé a hlavně nezadané studentky.

I ona byla ještě hodně naivní, a tak se nechala učarovat antikomunismem mého filosoficko-náboženského žvatlání. Lhal bych, kdybych to nazýval vznešeněji.

Strávili jsme spolu mnoho skvělých hodin ve studentské komunitě a ještě víc na společných agitačních cestách. Nejdůležitější však, jak už to v životě bývá, nebyly krásné květy našeho vztahu, nýbrž jeho plody.

Mezi ty krom mnoha nyní už hlavně nostalgických vzpomínek patřilo i jedno dítě – Jie.
