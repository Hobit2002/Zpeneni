\chapter{}

A bylo! Vychytralost byla poražena a já měl dokonce to štěstí, že mě Číňané jakožto významného politika západem chráněného státu ani neuvěznili ani nepopravili, ale vrátili spolu s Lydií, která po své snaze usmířit se s Wobucikem potřebovala ošetřit několik zlomenin, do Vděčné země.

Wobuciko, kterého ochranka nezastřelila, ale pouze uspala, zůstal v Číně a mám obavu, že se Jiemu podařilo ho nadále udržet na své straně. Jist si tím ovšem nejsem, protože jsem o tom podlém padouchovi už nikdy neslyšel.
	
Po návratu mě začalo trápit zdraví. Eskorta asi věděla, co dělá, když nás ze své země vyprovázela v respirátorech. Chvíli jsem měl strach, že jsem se mohl nakazit stejnou chorobou jako kdysi má žena. Mé obavy se však nepotvrdily, byla to spíše nějaká chřipka.

Ze začátku neprobíhala choroba nijak zle, ale asi po týdnu jsem začal mít tak vážné potíže s dýcháním, že jsem skončil v nemocnici, kde mě lékaři připojili na plicní ventilátor.

Tato rozhodně ne příjemná epizoda mého života ještě neskončila. Ve Vděčné zemi každou chvíli vypadává a elektrický proud a výpadky bohužel postihují i nemocnici. Nemoc už postoupila tak daleko, že bez ventilátoru takřka nejsem schopen dýchat. Mám podezření, že se blíží konec a snažím se tedy do jednoduchých poznámek zachytit svůj příběh.

Lydie, která má podstatně lehčí průběh než já, mi slíbila, že pokud bych zemřel, přepíše mé body do něčeho souvislejšího a tím mě zbavila veškerého strachu, který jsem ze smrti měl. Zažil jsem mnohonásobné vítězství dobra nad zlem a je mi dána možnost tento příběh uchovat i po té, co už tu nebudu. Co víc si lze od života přát?

Díky za všechno, Pane a jsem připraven posunout se zase o kus dál