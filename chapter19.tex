\chapter{}

Jiří zemřel, ale krásné snažení, které později zastavilo velkou a arogantní temnotu, pokračovalo dál. Na scénu se dostala nová hrdinka. Jelikož jsem ji mohl na rozdíl od Jiřího opravdu dobře poznat, budu její část příběhu vyprávět z její perspektivy.
\vspace{0.75cm}

Chun přemýšlela ráda, ale neuměla se soustředit. Velmi často se jí při zahloubání začaly v hlavě ozývat hlasy a před očima vyvstávat neidentifikovatelné ornamenty. Nyní se přistihla, že místo poctivého hledání odpovědí na své otázky pozoruje otazník, který se před ní vznáší ve vzduchu a snaží se udělat grimasu, při níž by ho její ústa svým tvarem co nejvíc připomínala. ´Do háje, už zase. Tohle se má žít, ne pozorovat!´ zanadávala na svůj duševní stav.

Od té doby, co jí byly ošetřeny rány na pahýlu jazyka, chtěla usilovat jen a jen o pochopení podivných událostí, ke kterým na mostě došlo.

Jednání muže, se kterým se před týdnem porvala, pro ni bylo ve všech ohledech záhadné. Chápala by například, kdyby ji shodil do řeky a následně zachránil, aby se tak svým nepřátelům aspoň trochu zalíbil a oni ho pak třeba i pustili domů. Takový záměr by jí připadal velice inteligentní. Ale to, že se jen s obuškem vrhnul proti přesile pistolemi ozbrojených policistů, pochopit nemohla. Zvláště pak proto, že její zabitý soupeř všechna tato rizika podstupoval kvůli člověku, se kterým nebyl ani příbuzný. Ještě nikdy nepotkala osobu podobných hodnot a motivací. Dotyčný musel být blázen.

Ani nesvobodný čínský režim nedokázal zabít přirozenou lidskou solidaritu, a právě ta umožnila Chun být tak blízko tajemné komunitě, která za celou akcí stála. Když totiž ještě ležela v nemocnici, navštívil ji její nadřízený s otázkou, zda by si za svou obětavost přála nějaký benefit. Chun chvíli váhala, a pak ho požádala, aby mohla nějaký čas strávit jako dozorkyně v Tilanqiau a hlídat lidi, při jejichž stíhání se jak zranila, tak vyznamenala. Vzhledem k tomu, že vězňů se zmařením Jieho plánu znatelně přibylo a věznice sháněla posily, nebyl žádný problém její prosbě vyhovět.

Několik dní strávila nepříliš objevnými meditacemi před mou celou. I já jsem ji svým jednáním velmi udivoval. Od svého zatčení, tedy již sedm dní, nejen že jsem nepozřel jediné sousto, ale také neřekl jediné slovo. 

Chun si to vykládala jako exces, podobně jako její kolegově, které jsem dováděl k nepříčetnosti zvláště u výslechů.

Dnes večer přišel čas na další pokus. Chun měla tu čest být přítomná, ale nijak se na to netěšila. Krom toho, že ji děsil můj zcela zubožený zjev, měl být u výslechu přítomen nejkrutější dozorce v celé věznici, Gang. Ganga na vyslýchání nasadili už při třetím pokusu a od té doby jsem přišel o polovinu svých zubů, byl mi zlomen nos a měl jsem modřiny po celém těle.

„Jestli dnes tu svoji hubu neotevřeš, vypíchnu ti obě oči!“ pozdravil mě Gang, když přišel čas, abych byl odveden k dalšímu výslechu.

Mlčky jsem přikývl a pokojně se nechal odvléct. Chun mě držela za druhé rameno než Gang a po chvíli si všimla, že její mysl se už zase přepnula do myšlenkového módu, který silně zkresloval realitu. Nejen že v tu chvíli viděla svět „mýma“ očima, ale celá chodba se na okamžik zaplnila poskakujícími démony. Podobné věci se jí stávaly odmala a už se naučila, že mnohé dojmy nesmí brát příliš vážně. Většinou, když si uvědomila, že to co vidí, je vlastně hodně zvláštní, stačilo, aby se na chvíli zastavila a zamyslela, čímž se znovu vrátila do světa, kde do sebe vjemy zapadaly. Tato kůra zafungovala i nyní. 

„Pojďte! Ať to prase konečně umlátíme k výpovědi,“ zavolal na ni Gang. Chun se rozešla a v tu chvíli ji napadlo, že se možná její souboj se zastřeleným Evropanem vůbec neodehrál. Ta úvaha ji vyděsila, znamenala by totiž, že vůbec nemůže věřit svým vzpomínkám a jelikož zrovna probíhající stavy své mysli hodnotila podle toho, jak moc odpovídaly jejím vzpomínkám, nemohla by věřit vůbec ničemu. To už však přicházeli do místnosti připravené na výslech.

„Nás mučení nebaví, takže to aspoň pro začátek zkusíme po dobrém,“ uvedl mě ale hlavně Ganga do situace detektiv, který už čekal na místě „proč jste…“

„S bratry a sestrami z šanghajské podzemní církve, jsme chtěli osvobodit naši společnou sestru v Kristu z Tilanqiaa,“ skočil jsem mu do řeči.

„Koho?“

„Bo Zhao. Byla zatčená za svou aktivitu v podzemní církvi. Chtěli jsme jí pomoct na svobodu a stejně tak všem dalším zatčeným ze stejného důvodu.“

„A co ten Evropan?“

„Ten se k nám přidal docela náhodou. Sám byl horlivě věřící, a když slyšel, jak hrozná je naše situace v téhle zemi, nazdařbůh vyrazil do Číny, aby s tím něco udělal.“

A tak výslech pokračoval. Stejně jako dříve zatčení, opakoval jsem předem připravenou výpověď (jíž mnozí níže postavení považovali za pravdivou), která díky množství informací, jimiž disponoval Jie, zněla policistům věrohodně. Bo Zhao skutečně byla své doby uvězněná v Šanghaji, sice ne v Tilanqiau nýbrž v Baimaolingu, nicméně během přípravy akce jí režim umožnil pod podmínkou, že změní jméno, bydliště, podrobí se policejnímu dozoru a nebude se již nadále stýkat s žádnou církví, vrátit se do normálního života. Snad všem zaměstnancům věznice, s výjimkou Ganga, tak bylo zadržených upřímně líto.

I přes to, že jsem velmi lhal, vzbudila má slova v Chun větší zájem než všechny detektivky, které kdy četla a thrillery, které viděla během posledního půl roku. Proč křesťané dávali dohromady tak velkou organizaci jen kvůli jedné ze svých řad?  Chápala by, kdyby se jednalo o společnou příbuznou, jenže dotyčná byla pro většinu povstalců jen známá a pro mnohé, například toho Evropana, ani to ne. Co je to za lidi, kteří nasazují život pro člověka, jehož ani neznají? V tom nebyla žádná moudrost, museli to být blázni.

„Proč jste nám to neřekl dřív?“ zakončil mimořádně spokojený detektiv výslech.

„Postil jsem se, abych uctil památku bratra, který nasadil svůj život za mou svobodu,“ odpověděl jsem.

Tato odpověď všem zástupcům čínského režimu přišla dost zvláštní, ale pochopili, že důvod pro ono sebetrýznění byl zřejmě morální, a tak se už v něm nehodlali hrabat. Detektiv řekl Chun, aby mě odvedla do cely k několika dalším bratřím (dosud jsem byl držen na samotce). Ganga, na kterém byl patrný hluboký zármutek, že mi nemohl vypíchnout oči,  nezaúkoloval záměrně.

Po cestě Chun s velkým potěšením usoudila, že její souboj s Jiřím byl zřejmě skutečný. Nasvědčovalo tomu úplně všechno od mé výpovědi, přes její práci ve věznici po chybějící špičku jazyka. Skutečný byl tedy zřejmě i onen bláznivý ovšem fascinující vztah, kteří tito podivní lidé, my křesťané, chovali k jiným lidem. 
 	
Den na to měla Chun za úkol hlídat naše cely a nosit nám jídlo. Jednalo se o další potvrzení toho, že práce vězeňského dozorce je neskutečně nudná. Opřela se a zase zkusila nad něčím přemýšlet. Marně, na její mysl v mžiku udeřil příboj šeptajících hlasů. K těm se po chvíli přidaly i zpěvy. Původně nepovažovala za reálné ani jedno, ale když byly zpěvy podezřele dlouho smysluplné, napadlo ji, že možná vychází z našich cel. Nahlédla tedy do jedné z nich a s úlevou si ověřila, že tohle se jí nezdálo. Vězni ze sousední místnosti seděli vždy uprostřed, dávajíce dohromady nechutnou vězeňskou stravu, kterou sice dostali všichni stejnou, ale asi měli pocit, že když bude všechno všech, prospěje jim to, a nad tím vším zpěvem chválili svého Boha. Blázni.

„Po cestě domů se Chun chtěla stavit u Jieho. Poprvé o něm slyšela už poměrně dávno, když ho její nadřízený zmiňoval jako vzor, který dává pod sebou pracujícím analytikům. Chun tedy donedávna předpokládala, že se bude nejspíše jednat o někoho částečně velmi chytrého a částečně zcela vygumovaného komunistickou ideologií. Dojmy, které měla z jejich setkání obraz uhlazeného patolízala, jemuž to ovšem velmi dobře myslí, značně zbořily.

 Začalo to tak, že Jie zavolal na její oddělení s prosbou, aby policie urychleně přijela a zatkla dva povstalce, z nichž jedním měl být a taky byl v zemi ilegálně pobývající cizinec. Vůz s pěti policisty byl vyslán okamžitě. Jie je uvítal před svým domem a v rychlosti jim sdělil, co o jejich poloze věděl. Chun se tehdy moc nesoustředila, Jie mluvil dost potichu, fakticky šeptal a v ní to vyvolalo obrovskou erupci jejích vlastních vnitřních hlasů. Po několik desítek sekund vůbec nevnímala okolní svět, a když se vzpamatovala, byla totálně zmatená. Její kolegové, kteří tento Chunin neduh znali, mimo jiné proto, že byl v Šanghaji extrémně častý, svěřili svou kolegyni do Jieho péče a sami vyrazili do akce.
 
Jie Chun odvedl do bytu, dal jí něco k pití a ponechal v klidu s trochou decentních stimulů, které jí pomohly se vrátit zpátky do reality. Rozloučil se s ní pokynem, aby si vlezla do kufru jeho auta, přikryla se dekou a počkala tam na stíhané zločince. Moudrá z toho Chun nebyla ani náhodou, ale takové široce uznávané kapacitě, jako Jiemu si netroufala vzdorovat. I tak ji velmi překvapilo, když dvě osoby přesně odpovídající jeho popisu, do auta skutečně nastoupily a rozjely se s ním.

Chun neměla tušení, jak mohl Jie toto setkání předvídat, a tak v duchu přitakala k veškeré chvále, kterou na něho kdy slyšela. Ovšem po té, co zjistila, že něco v autě ruší signál její vysílačky a že má zbraň plnou slepých nábojů, začala chovat podezření, že slovutný analytik nebyl tak úplně na její straně. Vůbec ovšem nechápala, čí hru by pak hrál, žádné dílčí vysvětlení naprosto nedávalo smysl. Nejpravděpodobnější jí připadala taková verze celé události, ve které si pistoli i vysílačku zapomněla u Jieho doma a onu historku s rušeným signálem jen vytvořila její nevyzpytatelná mysl. 

Nu a právě na to, jak se to všechno seběhlo, se chtěla nyní analytika zeptat (přes moderní komunikační kanály jí neodpovídal). Ale ani když klepala na dveře jeho bytu, neotvíral jí. Měla na něj číslo, tak zkusila ještě jednou zavolat, ale marně.
\vspace{0.75cm}

Dny plynuly, vězení nebylo nekonečné a nikdo příliš netoužil po tom v něm nechávat lidi příliš dlouho jen proto, že demonstrovali ze solidarity k osobě, která už stejně byla propuštěná. Některé propustili, jiné vykoupil nějaký příbuzný a další se zas podařilo obvinit z těžších zločinů, za které byli poslání do drsnějších věznic. Tak či tak, vždy když některý opouštěl Tilanqiao a uviděl Chun, či jiného pracovníka věznice na chodbě, nezapomněl mu požehnat. To Chun taky nechápala, tihle křesťané neměli důvod jí přát cokoliv dobrého, a přeci… Museli to být blázni.

Po celou tu dobu se snažila kontaktovat Jieho ale marně. Po analytikovi se slehla zem. Nejlogičtější vysvětlení bylo, že se skutečně provinil proti čínskému režimu a byl za to potrestán.   
\vspace{0.75cm}

Poslední pokus. Když v Číně někdo zmizel, stála za tím téměř vždy buď přírodní katastrofa, nebo policie. A k přírodní katastrofě v Šanghaji už hodně dlouho nedošlo. Jestliže byl tedy Jie zatčen, mohla by Chun být za svou snahu navázat s ním kontakt potrestána. Čínský šmírovací aparát bedlivě sledoval, kdo se s kým stýká, a když se někdo družil s osobami nízkého hodnocení, brzy se to projevilo i na jeho vlastních osobních možnostech.

Od příštího dne měla v plánu chodit z věznice přímo domů a tohle tedy byla její poslední večerní cesta přes most Yangpu. Na chvíli se zastavila u jeho zábradlí a pozorovala rozzářené budovy i lodě kotvící v přístavu. Přesně tento pohled vídala den co den, když ještě chodila s Pingem a trávila u něj většinu nocí. To však byla minulost, Vnitřní mocný hlas ji rozkázal, aby se vztahem skončila, a tak ho poslechla.

Bylo jí to líto, ale zpětně uznala, že jí ona svébytná část její mysli radila dobře. Ping byl strašně sobecká osoba, měl jen malé pochopení pro Chunin vnitřní život a chtěl ji u sebe mít až moc často. Ona to ovšem, když pracovala jako policistka, měla do jeho bytu přes hodinu cesty autem, což v kombinaci s dlouhou dobou strávenou únavnými úkoly způsobovalo, že k Pingovi přijížděla totálně vyčerpaná a nechtěla nic než spát. Pingovi se však takové pasivní trávení času příčilo, a tak ji ještě před ulehnutím popostrkával do všelikých aktivit. Jednou se u něj Chun uprostřed noci probudila a uslyšela mimořádně silný chraplavý hlas opakující slova: „Nebuď labutí.“

Převzala si to po svém a již příští noc i všechny následující trávila u sebe v bytě (kam sama Pinga raději nikdy nepozvala, neboť praskal ve švech výtvory vzniklými při jím nechápaných úletech Chuniny duše).

A nyní, když se znovu rozešla se záměrem nechat upadnout teskné vzpomínky s mostem spojené v zapomění, se onen vnitřní hlas ozval znovu. „Vodní prasopes.“ opakoval stále dokola.

„Máš rozkaz dnes v noci spát,“ ozval se náhle hlas jiný, který byl, bohužel, reálný. Snad čirou náhodou Chun zrovna minula Ganga, který se opíral o zábradlí, koukal do velkého bahnitého proudu a kouřil.

„Co bych dělala jiného?“ zívla Chun.

„Chodila pozdě večer po mostě,“

„Tak tomu bývalo, když jsem měla kluka, co mě tudy po večerech tahal“.

„Jen blázen by měl za partnera někoho, kdo po něm něco chce,“ (nutno dodat, že Gang sám byl svobodný).

„Byl tak extravertní, stále si chtěl povídat a někam vyrážet. Bývala jsem vyčerpaná i beztak…“ pokračovala Chun ve vzpomínkách.

„Mě to ale nezajímá, já chci prostě vyspalé kolegy,“

´A já chtěla mrtvé ticho, ve kterém bych si mohla povídat se svými vnitřními hlasy, zapomínat na realitu a ztvárňovat úlety svého mozku´ přemýšlela už potichu Chun.

Avšak „Možná by bylo dobré se trochu zbláznit,“ řekla zcela mimoděk nahlas.

„Nebylo,“ stroze odvětil Gang, pustil vajgl do velké bahnité řeky a odešel.

Chun ovšem její vlastní slova zcela pohltila. Zvláště inspirovány byly zřejmě ty části její mysli, které nekontrolovala. Hlavou jí začal pulzovat líbezný hlas prozpěvující: „Vzleť do vězení. “ a chvíli měla pocit, že vidí některé ze svých vězňů, jak chodí sem tam po hladině řeky Huangpu a světélkují.

Přes všechny tyto excesy jí bláznivé zapomínání na sebe sama, které se pro ni poslední dobou stalo synonymem k životnímu přístupu křesťanů, připadalo jako lákavé, i když se nad ním zamyslela konvenčně. Během těch několika dní, co hlídala a se zájmem sledovala své křesťanské vězně, se o jejich víře stihla leccos dozvědět a rozhodně nebyla znechucená. Ve srovnání s mrakodrapy Šanghaje se často cítila jako postradatelný článek samoúčelného ekonomického stroje, kterým však nebyla. Tedy alespoň z křesťanské perspektivy. Byla nepostradatelným článkem Božích plánů, či spíš jím být mohla, šlo jen o to se stát jedním z těch bláznů, přestat se bát a začít budovat dokonalou komunitu - církev.

Začala tím, že se pokřtila. S výkřikem „Přijmi mě“ skočila z mostu dolů do vody, a když z ní potom vyplavala, třásla se zimou a vzrušením. Od této chvíle byla bláznem.
