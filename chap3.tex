\chapter{}

Policisté v civilu mě přes několik ulic odvlekli do skladu plného prázdných přepravek.

Jeden z tajných někam odběhl, pravděpodobně najít nejbližší telefonní budku a zavolat policii. Druhý si ke mně přikleknul.
„Co máš na světě nejraději?“ otázal se mě.

„Svého syna, svou ženu a Konfuciovo učení,“ odpověděl jsem.

„Vida,“ zaradoval se policista. „S výjimkou toho Konfucia jsem na tom podobně. Řekněte mi prosím, normální člověče, co vám dalo odvahu k tak riskantnímu činu? Přál bych si mít stejnou kuráž jako vy. Zanedlouho mi nastanou krušné časy.“

„Právě ten Konfucius. Ukázal mi, že komunistické rovnostářství je zosobněním zvrácenosti. Musel jsem se mu postavit.“

Tajný policista se smutně usmál. „Utečme,“ navrhl mi. „Za chvíli se vrátí Gang a to bude zkáza pro nás oba.“ 

„Pročpak?“

„Nemyslete si, že komunistická strana je jednotná. Mocenské boje v ní probíhaly už od jejího vzniku a vždy, když byl někdo vyřazen ze hry, skončil spolu s ním ve vězení, ba co hůř na popravišti, bezpočet jeho sympatizantů. 

Je to sotva měsíc, co z kola vypadl  Čao C‘-jang, fantastický politický idealista, kterému šlo víc o nás a o vás než o své vlastní postavení. Zašel tak daleko, že vyjádřil sympatie demonstrantům a vyzval je, aby se zklidnili. Předvídal, co se stane, když ho neposlechnou. 

Ale vy jste ho vážně nevzali. Na vás poslali tanky a on byl zavřen do domácího vězení. Jeho jméno mizí z médií a jeho stoupenci ze svých pozic. Já jsem jedním z nich. 

Nedělám jen tajného policistu, to je jen koníček na zlepšení kádrových posudků, jsem také úředník a to docela vysoko postavený. Tomu bude brzy konec. Doufám, že aspoň zůstanu na svobodě.

Uvažuji, že bych se zvednul a opustil Čínu, dokud vláda řeší stále ještě víc demonstrace než nepravověrné komunisty. Jestli se sem ovšem vrátí Gang s policií a zatknou vás, už tu šanci nedostanu. Proces s vámi se nějaký ten čas povleče a já v něm budu hrát důležitou roli. Budou se mi ozývat soudci a policisté, stále budu kontrolován, zkrátka už nedostanu šanci uniknout a kdo ví, zda mě nezavřou hned po vás.“

„Tak utečte hned! Máte poslední minuty.“

„Bojím se. Co tomu řekne žena? Ostatně, ihned mě začnou stíhat mí kolegové, zvlášť když utečete taky.“

„Já utíkat nebudu. Na mé straně je prastarý řád dávného císařství, co by mi mohl udělat člověk?“

„Přejet vás tankem. Zastřelit vás či vás zbít a pak nechat shnít ve vězení.“

„Vše z toho už jste vy, popřípadě tankista udělat mohli, ale ne, lidé jsou dobrý druh. Akorát si občas hodně ublíží vlastními systémy.“

„Dobří lidé? To ještě neznáte Ganga.“

„Na tyhle řeči nemáte čas, pokud chcete žít, tak konejte a vypadněte. A pokud chcete vypadat méně nedbale, tak mě před tím spoutejte nebo omračte. Vaším kolegům řeknu, že se vám udělalo hodně špatně a musel jsem na chvíli zmizet ze zdravotních důvodů.“

„Mohl bych vás vystřídat, na rozdíl od vás nemám rodinu.“ Ozval se náhle ode dveří skladu moudrý demonstrant. Když mě tajní policisté čapli a vláčeli ulicemi, on na kole vyrazil za nimi a posledních několik minut zřejmě strávil odposlechem našeho hovoru. Nyní se do něj přidal a svými slovy potvrdil, že to, co kdysi říkal, myslel vážně. 

„Dobrý člověče, jsem otec a to znamená, že nesu nemalou odpovědnost za své dítě. V prvé řadě mu musím ukázat, že se zlým se bojuje až do posledního dechu a poslední kapky krve.

Pro syna bude mnohem lepší, když bude žít sice bez fyzicky přítomného otce, ale s vědomím, že se tomu zlu, které nás obklopuje, dá vzdorovat.

V druhé řadě bych ho ovšem měl šatit a krmit. Proto vás prosím, odvezte do mého domu ty nákupní tašky.“

Cyklista se usmál a učinil, jak jsem žádal.

Dosud rozpačitý policista, se také zvednul. „Vidím, že svět je úplně šílený, ale zřejmě taky stále ještě docela dobrý. To jsou vhodné podmínky pro nenadálé opuštění země.“

A i on se zvedl a učinil, jak řekl.

Ještě několik minut jsem osaměle čekal v prázdném skladišti, než do něj vtrhnul druhý tajný policista s doprovodem několika kolegů, kteří již tajní nebyli.

Brzy jsem pochopil, proč měl Gang zlomit mou víru v lidské dobro. Ještě než jsem nastoupil do policejního vozu, ztratil jsem okolo deseti procent své pokožky a krve.

„Jak to, že jsi tu sám?“ „Nelži, ty svině!“ „Co jsi s nimi udělal?“ vyslýchal mě. Žádné odpovědi, ať už pravdivé či lživé, nevěřil, ze všeho nejvíc mu však vadilo mlčení. Překvapilo mě, že jsem byl po pohození do auta ještě živý

Velmi podobné, byť spisovněji formulované, otázky mi kladl již pár dní po té soudce.

 Gang mě na proces skvěle připravil. Formální soud se totiž od jeho pěstního výslechu lišil pouze tím, že byl zdlouhavější a chladnější.

 Místo zuřivého křiku Gangova dotazoval se mě ledový hlas prokurátora a rány pěstí vystřídaly noci strávené v zamčeném sprchovém boxu, ve kterém dozorci nechali celou noc puštěnou ledovou vodu. 

 Ovšem bohyně Us nedopustila, abych byl zlomen. Několikrát mě vyslechli a nakonec usoudili, k mému velkému překvapení, že si nezasloužím být popraven.

Potrestali mě pouze doživotními nucenými pracemi v zahraničním pracovním táboře.

Až o mnoho let později jsem se dozvěděl, jak moc jsem za život vděčil tajnému policistovi, který se rozhodl pro útěk. Tomu dobrému člověku se utéct skutečně podařilo, a když už byl v bezpečí Kanady, kontaktoval své bývalé kamarády v Číně (kteří měli sice trochu jiné politické názory, ale samozřejmě byli rádi, že vyvázl) a vroucně je žádal, aby přimhouřili oko nade mnou, človíčkem, který mu k útěku dodal odvahy.

 Jeho prosby byly vyslyšeny a já tedy neskončil na popravišti nýbrž v pracovním táboře v Zambii.

 Mým úkolem bylo těžit měď. Vůbec jsem nechápal, jaký má ta práce smysl. Čína měla mědi na vlastním území víc než dost.

 Její podstata však byla prostá – Čína Zambii celkem k ničemu nepotřebovala, ale naši politici se rozhodli, že po Afričanech i tak budou výměnou za ekonomickou pomoc něco chtít, aby si zvykli být komunistickému režimu k dispozici.

A tak jsem poprvé za svůj život skutečně povýšil. Ze zaměstnance státu jsem se posunul na politického vězně. 

Po několika měsících strávených v obyčejném vězení jsem byl spolu s několika desítkami dalších vězňů letadlem transportován kamsi doprostřed džungle, kde se nás hned ujali opálení a ozbrojení krajané, kteří nás nahnali na korby náklaďáků, načež se rozdělili na strážce, kteří si vlezli s puškami za námi a na řidiče, kteří se s námi všem rozjeli do lomu velkého jak Peking.

Že mě čeká peklo, jsem poznal už během oné krátké jízdy. Prakticky hned po té, co auta sjela z přistávací plošiny na nezpevněnou cestu, všiml jsem si, že podél ní kráčí, opačným směrem než my, početná skupina černých rodin ověšených zavazadly a nemluvňaty. 

„Jejich vesnici jsme včera srovnali se zemí a na jejím území brzy začne těžba. Tihle lidé se teď nejspíš někde nedaleko usadí a za pět let se budou muset zase posunout,“ vysvětlil nám strážce korby, aniž by se ho kdokoliv na cokoliv dotazoval.   Jen pár minut po té, co to řekl, se z korby náklaďáků, který se kodrcal za námi, ozval křik.

Jeden vězeň vyskočil z vozu (jenž se po nezpevněné cestě kodrcal jen velmi pomalu) a nyní doufal, že stihne včas utéct a nebude zastřelen.

„To nebyl dobrý nápad,“ odtušil strážce naší korby, ale já jsem měl chvíli pocit, že mám důvod být na čínské ozbrojené složky opravdu hrdý, neboť se ke vzdorujícímu chovají stejně velkoryse, jako se řidič tanku zachoval ke mně. Nikdo totiž po dotyčném nezačal střílet.

Auta akorát zastavila a dozorci začali něco volat. Čínsky to tedy rozhodně neznělo. Na druhou stranu zněly jejich výkřiky poměrně podobně, a tak jsem se v chvilkovém záchvatu naivity odvážil doufat, že na něj volají něco jako „Sbohem!“ či „Rychleji!“ (Slovní zásoba pro vyjádření urážek a výhružek, které by dozorci jistě použily, kdyby uprchlíkův čin neschvalovali, je v každém jazyce tak velká, že by se jejich zvolání musela lišit výrazně víc).
	
 Má idealistická iluze však brzy skončila. Uprchlík prchal od silnice do lesa, kde naběhnul přímo do hloučku Afričanů.
	To byl jeho konec.  
  
O domov připravení domorodci se na něj vrhli, srazili ho k zemi, vytáhli nože a po chvíli už jeden kráčel klidným krokem zrovna k mému vozu svíraje v ruce uříznutou hlavu uprchlíka.
	
 Řidič si hlavu přebral a výměnou za ni podal Afričanovi balíček bankovek. Nakonec hlavu pohodil mezi nás na korbu a znovu se rozjel.

 „Bez lidí jako on, by to s místními bylo těžké,“ poznamenal dozorce. „díky penězům, které dostávají za likvidaci uprchlíků, nám odpouští, že jim doslova meleme zemi pod nohama. A pro nás je to levnější než je odškodňovat stavbou nových vesnic.“

 „Co jste to volali?“ otázal se někdo.

 „Zakázka,“odpověděl dozorce.

 „A k čemu jste tu vy?“ zeptal jsem se já.

 „Abychom stříleli po nich, kdyby se jim přestala líbit dosavadní podoba naší spolupráce.“

 Nejprve jsem se otřásl odporem, ale znechucení bylo již brzy nahrazeno radostným vzrušením. Život není o tom, aby byl lehký a pohodlný, ale o tom, že se člověk vytrvale postupuje po cestě, jíž si vybral. A tak jsem já, člověk toužící po svobodě pro čínský lid, právě tady, uprostřed krve, potu, násilí a obecně mnoha špatností, jež bylo možno napravit, přestal být jen ploužícím se stínem snažícím se nalézt svou hodnotu v hrátkách se slovíčky.  Ožil jsem a vděčně vzhlédl k Nebesům, abych jim za tento dar poděkoval. 
