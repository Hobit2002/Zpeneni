\chapter{}

Někdy na podzim 2019 jsem od Jieho dostal email následujícího znění: 

„Chavezie vyhyne, o tom nepochybuji. Upravili jsme jeden z netopýřích koronavirů a vyrobili tak odrůdu, která by měla ty mořské potvory hubit ve velkém. Jediný problém byl, že se současnou verzí mohou nakazit i lidé. Pro starší, popřípadě ty, jejichž imunita je už nahlodaná něčím jiným, by taková nákaza mohla být velmi často smrtelně nebezpečná. Snažili jsme se ji proto ještě upravit, aby v žádném případě nehrozilo, že se nějakým nedopatřením rozšíří mezi lidi.

Bohužel jsme nepočítali s výpravou několika šikovných idiotů z Vděčné země, kteří vpadli do výzkumné haly a vypustili většinu našich netopýrů. Bohužel právě na těchto netopýrech jsme zkoumali, co náš koronavirus dělá se složitějšími organismy a výsledky ani zdaleka nebyly uspokojivé.

Ti náboženští fanatici zkrátka nechali čínský lid napospas snadno přenosné a pro mnohé lidi zároveň smrtelně nebezpečné chorobě – z toho kouká minimálně epidemie a při té se do našich řek dostane spousta roušek a infikovaného zdravotnického materiálu. Když bude virus v řekách, dostane se i do moří a tam skoncuje s hrozbou chavézií, takže cíl dosažen, ale cena… ta bude šílená.

Rozhodl jsem se ukončit činnost svého výzkumného týmu. Čína už nemá, jak mi v boji s chavézií pomoct. Ale já jsem rozhodnutý pomoct jí v boji s naším lékem. Zdejší režim je sice hrozný, ale když čelíme hrozbám, jako je tato, musíme spolupracovat. V blízké době plánuji oficielně opustit Činu, ale ve skutečnosti v ní zůstat (nasednu sice na vlak, ale v průběhu jízdy jej zase opustím).

Už jsem si zařídil novou identitu wuhanského lékaře Li Wen-lianga a mám v plánu ji využít k tomu, abych co nejvíc varoval místní vědce a úředníky. Natvrdo přiznat problémy našeho týmu nehodlám, tím bych ohrozil životní příběh každého ze svých kolegů.“
