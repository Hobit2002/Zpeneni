\chapter{}

Umlil měl nejspíše pocit, že svým výkřikem všechno zachránil. Já ovšem žádný zázrak nepozoroval, akorát k maringotce se blížilo několik bagrů a nákladních vozidel, v každém dozorce s puškou.

Pokusil jsem se afrického revolucionáře ochránit a to způsobem sice nepříliš originálním zato již ozkoušeným. Postavil jsem se bagrům do cesty.

Většina mě prostě objela a pokračovala dál, jeden se mi však zabrzdit podařilo. Řidič mohl kroutit volantem podle libosti, ale nic mu to nepomohlo. Když si to uvědomil, vyskočil ven a rozhodl se mě pažbou pušky odehnat. 

Rozmáchl se poprvé a udeřil. Nastavil jsem letící zbrani lokty a vzápětí poznal, že to byl mizerný nápad. Větší bolest jsem od té doby, co mě vyslýchal Gang, nepocítil. Před druhou jsem už tedy raději uhnul. Třetí mě však navzdory mé snaze zasáhla do ramene a složila na zem. Dozorce mě čapnul a odtáhl z cesty svého vozu. 

Ještě než mě položil, skočil mu na záda jeden z domorodých dělníků, který v dolu pracoval za mzdu.

Statný dozorce ho hravě setřásl, ale druhý černý dělník mu mezitím vytrhl pušku a její hlaveň použil proti nejcitlivějším místům mužského těla. Tím byl náš společný sok vyřazen.
Mí kolegové naskočili do prázdného bagru a kamsi se s ním rozjeli. Já se postavil a vyrazil do tábora prozkoumat situaci.

Prvních několik minut jsem potkával jen vystresované dozorce, kteří běželi tu pro munici, tu pro obvazy, tu do nějaké skrýše a tu do boje. Všude kolem se pak poflakovali vězni, kterým sice revolta dobrovolných kolegů nevadila, ale báli se ji podpořit. 

Opravdu hrozný pohled se mi naskytl až u jídelny. V blízkosti stravovacího prostoru se válelo asi sedm těl zastřelených dozorců, které už naši revoltující kolegové obrali o zbraně i munici. Jelikož jsem se k smrtonosné budově nechtěl moc přibližovat, pokračoval jsem k parkovišti vozů určených pro přepravu lidí.   

Když jsem tam dorazil, naskytl se mi pohled na učiněnou zkázu. Na zemi se povalovalo okolo pětadvaceti těl, z nichž dvě třetiny náležely padlým dozorcům a zbytek povstalcům. Kolem a bohužel i na tom všem postávaly jak bagry, které dozorci využili jako improvizované tanky, tak i náklaďáky na transport čerstvých vězňů. Několik vozidel bylo zřejmě vybouraných a dalších pár hořelo.

Uprostřed toho stál Umlil a několik domorodých dělníků. Všichni měli buď dozorcům ukradené pušky, nebo lahve od piva pro důstojníky. Nicméně jelikož z každé lahve čněl knot, usoudil jsem, že rebelové si narychlo vyrobili molotovovy koktejly. 
Držel jsem se od těch vrahů v uctivé vzdálenosti a pozoroval, co s nimi naši nadřízení udělají. Dočkal jsem se. Vzdali se.

Po půlhodině, kdy tu a tam po roztroušené barikádě někdo vystřelil a kdy se čas od času z různých částí tábora ozval výkřik bolesti, strachu či triumfu, přišel za Umlilem klidným krokem a s rukama nad hlavu Caifu.

„Uspěli jste,“ zavolal už z dálky. „Jen velmi nerad bych ztratil ještě víc svých lidí.“
To jsem od Caifua nečekal, doposavad se vybarvil sice jako liberál, ale zároveň také jako zbabělec. Nyní se zdálo, že patří mezi ty děti bohyně Osn, které mají odvahu prohrát bitvu.

„Dejte nám zbraně a necháme vás být,“ řekl Umlil bez váhání.

„Kolik zbraní?“

„Všechny. Před odchodem do boje provedeme inventuru, a pokud zjistíme, že jste nám nějaké schovali, zastřelím vás.“

Caifu se poškrábal na hlavě a nervózně zasmál. Stejně jako já vůbec nechápal, jak to, že mu rozkazuje člověk, který ještě ráno čekal na smrt. Nicméně většina dozorců byla mrtvá a Caifu tušil, že smrt dalších by pro něj znamenala jistou popravu. Ztráta všech zbraní by však nebyla hodnocena o nic lépe. „Pokud mi všichni vaši současní příznivci neodpřísáhnou, že zbraně po akci zase vrátí a nenecháte mi tu nějaké rukojmí, dám zbytku dozorců pokyn, aby pokračovali v boji, dokud jedna ze stran zcela nevykrvácí,“ řekl a rozhodil rukama, aby poukázal na to, že ostraha tábora už je připravená.

Umlil se chvíli zamyslel a pak procedil mezi zuby: „Já s tyrany a hrdlořezy vyjednávat nebudu!“ Dvě sekundy na to už jsem Caiufua neviděl. Zambijec ho nejprve pěstním úderem do břicha složil k zemi a následně odtáhl za jeden z vozů.

Dozorci začali palbu. Rebelové ji oplatili, ale zpočátku se zdálo, že strážci pořádku, leckdy již ve střelbě cvičení, znovuzískají kontrolu nad táborem.

Povstalcům však nakonec pomohly dvě věci. První z nich byl bagr, který jsem nedávno zastavil ve víře, že tak dělám dobrou věc. Nyní tento vůz (bezpečně jsem ho poznal, neboť za sebou vláčel zbitého, skuhrajícího, původního řidiče) vpadl dozorcům do zad a přejel nejednoho střelce.

Ještě větší roli však sehrály domorodé posily, které se vyrojily z pralesa a chopily se zbraní po padlých rebelech. Místní stříleli mizerně, ale postupně dozorce udolali. Důl byl jejich.

„Hlídka se musí soustředit a bezustání kontrolovat každé místo, kde by se mohlo něco semlít, nikdy nesmí umdlít. Vy, či spíše vaši nyní již mrtví kolegové, jste byli mizerná hlídka.

Dopustili jste, aby se k mým černým bratřím, se kterými sdílím nejen chuť bojovat s apartheidem, dostala informace, že budu zastřelen a podcenili jste jejich ducha. Samozřejmě, že mě navštívili a domluvili se se mnou, že až budu v nouzi, hlasitě zavolám: ‚Je čas!‘ a oni se vzbouří.

Mí dobří bratři, které jste do té doby drželi dál od všech poprav a mučení, mě nezklamali. Předem nakradli munici, vyrobili molotovovy koktejly a získali pro boj s vámi i své příbuzné z okolního pralesa.

Když mi mnozí hrdinní vězni pomohli získat zbraň, cestu z maringotky a čas utéct před bagry, mohl jsem to všechno vidět. Nyní mám nabídku pro zdravotníka, tankmužíka, úplatkyodmítače a další, kteří mi pomohli. Můžete usmrtit zbývající ze svých věznitelů.“

Tímto projevem Umlil v bývalé štábní chatě uzavíral životy několika málo dozorců, kteří přežili povstání. Všichni stáli na židlích přímo pod jedním z nosných trámů, se kterým je spojovaly smyčky z ostnatého drátu, které měl každý na krku. My, vězni, jsme byli přítomni, jeho řeč poslouchali a ti z nás, kteří mu pomohli na svobodu, dostali možnost odkopnout židle zpod nohou dozorců a tím je popravit. Tu příležitost nevyužil nikdo.
 
„Žádný zájemce? Nevadí. My to zvládneme taky,“ pokrčil Umlil rameny a kopnul do nejbližší židle takovou silou, že v letu srazila ještě dvě další.

Navečer už byl tábor prázdný. Bojovní domorodci nám zabavili těžkou techniku, zbraně, medikamenty a Caiufua. 

Tehdy jsem zahořkl. „Jak se tohle mohlo stát?“ tázal jsem se sebe sama a nedokázal si odpovědět. To, co chtěli ve strachu z degradace udělat Caifu a Wuwang, bylo nanejvýš nechutné a že jsme se tomu postavili, bylo zcela správné. Ale jak to mohlo skončit tak hnusně?

Má víra v Konfuciovo učení mi napovídala, že kdo podporuje správné vztahy mezi jednotlivými složkami společnosti a uspívá v tom, přibližuje svět jeho ideální podobě. To se však nyní ani náhodou nestalo.
 
 Usoudil jsem, že jsem se musel někde přepočítat, že jsem skutečný vztah Umlila k nám nepochopil. Nyní mi však bylo jasné, že je to vrah a že si zaslouží, aby s ním bylo podle toho nakládáno.
 
Vzal jsem si trochu jídla, moskytiéru, naostřenou železnou tyč a vyrazil do pralesa hledat spravedlnost.
