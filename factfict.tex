\chapter*{Fakta a fikce}

Jsem si vědom toho, že kniha, která mixuje dohromady nedávné události s čirými výmysly, může ve čtenářích vyvolat pocit, že některé příhody a fenomény v ní popsané, jsou skutečné či alespoň silně skutečností inspirované, i když tomu tak vůbec není. Proto přikládám na konec knihy tento přehled, díky němuž každý čtenář spolehlivě rozezná, co bylo opsáno ze zpráv a co jak se sluší a patří vymyšleno.\vspace{0.5cm}

\textbf{Barack Obama} – Popis silných a slabých stránek Baracka Obamy vychází z pohledu, jaký na něj měl po společné večeři Steve Jobs. Čtenář by si tedy měl být vědom toho, že se k němu tato informace dostává ze čtvrté ruky a brát ji podle toho vážně.\vspace{0.5cm}

\textbf{Bill Gates a pandemie} – Podle mnohých konspirátorů americký miliardář pandemii rozpoutal, aby následně vydělal na prodeji vakcíny. Jedná se o blud. Jeho role byla místo toho přesně taková jako v této knize (až na to, že nespolupracoval s žádným Li Cu-quiánem) – podpořil vývoj nových vakcín a nechal je vyrábět ve velkém, ještě než byly schváleny, aby je lidstvo mělo k dispozici co nejdřív.\vspace{0.5cm}

\textbf{Brexit} – Podle pravdy (samozřejmě až na zapojení hlavních hrdinů do kampaně). Vůdci brexitového hnutí skutečně byli Cummings a Johnson, kteří zvítězili díky využití umělé inteligence,jež dokázala lidem ukázat „ten správný argument“ a přesvědčit, je, aby hlasovali pro odchod. Pravdivé je i to, že kampaň byla mimořádně vyhrocená a došlo během ní k vraždě protibrexitové političky.\vspace{0.5cm}

\textbf{COVID 2019} – Vše, co jsem o této nemoci napsal je až na jedinou výjimku pravda, onou výjimkou je umělý původ viru. Ačkoliv se vyskytly teorie, že virus unikl z laboratoří ve Wuchanu, vědecké kapacity se shodují v tom, že virus nebyl vytvořen člověkem.\vspace{0.5cm}

\textbf{Čínská Coca-Cola} – Čínské firmy a tajné služby bez skrupulí kradou západní technologie a know-how. Není mi však známo, že by vytvářely plagiáty jejich produktů a snažily se těžit z proslulosti jejich značky a čínskou továrnu je nutné brát i s jejím vojenským strategickým významem jakožto ryzí fikci.\vspace{0.5cm}

\textbf{Čínská eliminace nepohodlných} – Žádné důkazy svědčící o tom, že by Čína úkolovala své tajné služby zabíjením nepohodlných osob v zahraničí, jsem nedohledal. I tuto část příběhu je tedy nutné brát jako fikci. \vspace{0.5cm}

\textbf{Chavezie} – Úplná fikce. Pokud něco jako chavezie existuje a určité vědecké elity o ní ví, nejsme o této hrozbě, stejně jako v knize, informováni. Tento scénář je však ryze konspirační a v žádném případě se nesmí brát jakkoliv vážně. 
Co však bohužel z reality vychází, jsou klimatické změny, kvůli nimž se ve Zpěnění chavezie dostala do kontaktu s lidmi – oteplování a okyselování moří, které vede k úhynu mořských korálů. Tento jev je velkým problémem i bez chavezie, mořské korály jsou zdrojem potravy pro mnoho ryb a když o ně lidstvo přijde, moře mu už nebude ani zdaleka schopno poskytovat tolik potravy jako dřív.  \vspace{0.5cm}

\textbf{Donald Trump} – S výjimkou vztahu s Li Cu-quiánem vychází celý Trumpův příběh ze skutečnosti. Tento politik se v průběhu života hned několikrát zadlužil a již před vítězstvím v prezidentských volbách, tak  jako v knize, vykazoval všechny známky narcisticky porušené osobnosti.\vspace{0.5cm}

\textbf{Errol Musk a smaragdový důl} – Errol Musk, otec Elona Muska, výkonného ředitele firem SpaceX a Tesla, pravděpodobně nikdy žádný smaragdový důl nevlastnil ani v něm neměl podíl. Historka, která tvrdí opak, je sice na internetu poměrně rozšířená, ale Elon Musk ji popírá a o tomto vypravěčsky vděčném tématu není zmínka ani v jeho biografii od Ashleeho Vance. Faktem však je, že Errol Musk byl úspěšný podnikatel, který se nezdráhal své peníze utrácet.\vspace{0.5cm}
 
\textbf{Isimangaliso} – Stejně jako Li Cu-quián i jeho město je čirý výmysl, který se ani nesnaží odkazovat na nějaké skutečné místo. \vspace{0.5cm}

\textbf{Jihoafrické intervence na zambijské území} – Za apartheidu jihoafrická republika Zambii, která tento režim kritizovala, skutečně vojensky napadala. Tyto útoky však jen sotva byly tak uvolněné a jasný cíl postrádající jako v této knize.\vspace{0.5cm}

\textbf{Konfucianismus} – Tradiční čínské náboženství má v knize svůj základ na vzájemných vztazích mezi různě postavenými členy společenské hierarchie. Tento výklad vychází z knihy Křesťanství a náboženství Číny od Hanse Künga a Julie Ching a měl by odpovídat realitě. \vspace{0.5cm}

\textbf{Li Wen-liang} – Tento čínský lékař, který se dostal do konfliktu s režimem, neboť varoval před Covidem již v brzkém začátku pandemie, skutečně žil. Jeho smrt i následná hněvivá kritika nedostatku svobody na čínských sociálních sítích proběhly tak jako v knize. \vspace{0.5cm}

\textbf{Masakr na náměstí Nebeského klidu} – Vše co je o této události v knize napsáno (čínští politici a jejich názory, Pekingská studentská autonomní federace i způsob, jakým byly protesty rozehnány), by mělo být pravdivé – pochopitelně s výjimkou konkrétních dialogů a tvrzení, že Teng nechal protesty ukončit, protože se chtěl dožít jejich konce.\vspace{0.5cm}

\textbf{Mnohogenní} – Žádná mnohogenní armáda nikdy vycvičena nebyla a ačkoliv jsou hrátky s lidským genomem v Číně přípustnější než v demokratičtějších částech vyspělého světa, ke zvěrstvům tohoto rozsahu by zde mohlo jen sotva dojít. Jen vzpomeňme na He Jiankuiho, který upravil genom dvou děvčátek s cílem je obdařit imunitou proti HIV, a připravil se tímto činem o práci, velké množství peněz i o „svobodu“.\vspace{0.5cm}

\textbf{Německé klimatické ambice} – Ve srovnání s jinými velkými hráči jako Čínou, Japonskem a Spojenými státy v republikánských obdobích má Německo velké cíle, o premianta se však rozhodně nejedná. Země jako Indie, Francie či velká Británie toho pro ochranu klimatu dělají podstatně víc. Co se opatření na ochranu klimatu týče, patří Německo i v rámci EU mezi podprůměrné státy.\vspace{0.5cm}

\textbf{Tankerén} – Muž před tankem skutečně žil a jeho střet s obrněným vozem byl nafilmován (ovšem z velké dálky, takže veřejnost neví, co při něm bylo řečeno) a knižní popis této konfrontace mu přesně odpovídá. O dalších ani předchozích osudech čínského hrdiny nic nevíme. \vspace{0.5cm}  

\textbf{Těžba mědi v Zambii} – Pravda. V Zambii se skutečně měď intenzivně těží a Čína, která zde vlastní velké množství dolů (kde v minulosti několikrát došlo i k násilnému povstání zambijských dělníků), ji odtamtud importuje, jedná se však o práci státem podporovaných firem a nikoliv politických vězňů.\vspace{0.5cm}

\textbf{Vděčná země} – Naprostý výmysl bez jakékoliv vazby na realitu. Vzhledem k současnému a nedávnému dění na asijském kontinentu by leckterý čtenář mohl historii Vděčné země (včetně totálního kolapsu infrastruktury, který po pádu komunistické moci nastal) považovat za alegorii na (případné) dění v Hongkongu. Jednalo by se o úvahu rozumnou, nikoliv však správnou. 
