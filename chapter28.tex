\chapter{}

Do Vděčné země jsme přiletěli až o dva dny později. Vše potřebné jsme ovšem zjistili už dříve telefonicky.

Král Ping neměl ponětí ani o tom, že jeho ministr financí nejspíš způsobil smrt jeho ženy ani o tom, kde se zrovna nachází. 

V Africe se prý s Pingem přes Skype domluvil, že většinu vojáků demobilizuje a naučí je normálnímu životu a že proti Číně zakročí pouze několik málo, kteří byli vycvičení jakožto elitní přepadové komando. 

S těmi měl prý v plánu dočasně vyřadit z provozu vodní elektrárnu Tři soutěsky, a to aniž by byl při popřípadě po tomto činu chycen. Pinga přesvědčil, že jako demonstrace síly to postačí.

Vědět tolik, kolik věděl Ping, když s Jiem komunikoval, také bych podobné iniciativě přitakal. Bylo bezpochyby správné dát mnohogenní jednotce svobodu. Pokud ovšem Jie Vděčné zemi zabil duchovní i politickou matku a hlavního zahraničního patrona, nedalo se již dobro vidět za žádným z jeho počinů. Však také slibované komando do Šanghaje nikdy nedorazilo a ministr financí s Pingem úplně přestal komunikovat.

Když jsem Pingovi pověděl, co jsem se dozvěděl od Rusů a od Ganga, byl zděšen a zprvu nechtěl tomu, že zkázu království způsobil jeho vlastní ministr financí, ani věřit. Informace z různých zdrojů však do sebe zapadaly až příliš dobře a přijatelná alternativa se nenabízela.

„Vděčná země vznikla, aby byla dobrou duší planety Země. Pokud je někde na světě skupina lidí ohrožená druhým ‚holocaustem‘, musíme jí jakožto stát za každou cenu pomoct,“rozhodl se Ping a stejnými slovy již o pár dní později zahájil jednání s několika datovými vědci a inženýry Googlu.

Tyto chytré hlavy si následně sedly k bývalým policejním superpočítačům a s jejich pomocí provedli důkladnou analýzu satelitních snímků Číny, jejímž účelem bylo vytipování míst, na kterých by byl mohl v posledních měsících začít vývoj popřípadě svoz biologických zbraní.

Lokací, které mohly být tomuto účelu zasvěceny, našli hned několik, jako nejpravděpodobnější z nich však byl vyhodnocen přísně střežený objekt, který v posledním měsíci vyrostl, snad čirou náhodou, poblíž Tří soutěsek. Můj život byl poslední dobou natolik nabyt dobrodružstvím, že jsem se cítil být, zvláště v doprovodu Lydie a Wobucika, na sabotážní misi tím pravým.

Nelegální pouť Čínou byla pochopitelně o mnoho drsnější než pohodová tůra Zambií. První komplikace nás potkaly již ve Vděčné zemi. Jelikož čínská armáda celý městský stát obklíčila a kontrolovala všechny, kteří z něj vyjížděli, nepřicházelo v úvahu, abychom svou cestu začali na souši.

Pod rouškou tmy jsme tedy nasedli na narychlo a draze koupenou jachtu a vypluli. Námořní blokáda Šanghaje byla Číňanům zakázána Putinem, jehož ochranná opatření ještě posledních několik dní trvala, a tak nám v cestě nikdo nepřekážel.

Již příští poledne jsme loď velmi nepořádně zakotvili v Ning-po, ze kterého jsme pak rychle utekli do blízkých lesů. Již od prvního dne cesty nám byla v patách policie a my přitom potřebovali ke Třem soutěskám dorazit tak rychle! 

Ani v „divočině“ jsme se nezdržovali příliš dlouho, naším cílem bylo využít co nejdříve nějaký rychlovlak, který by nás dostal z nedalekého Hangzhou do Wu-chanu, odkud by nám k vodní elektrárně zbývalo jen pět set kilometrů. 

Jaká to ovšem smůla, že na nás téměř jistě už v Hangzhou čekala specielní, co do výzbroje a výcviku spíše vojenská než policejní jednotka. Velmi, jednou dokonce celý bdělý den vkuse, jsem se modlil, aby policisté přeci jen selhali. Prosba byla vyslyšena, Lydie začala přemýšlet.
	
„Jak moc místní policisté spoléhají na kamery a digitální rozpoznávání obličejů?“ zeptala se mě jednou večer.
	
„Absolutně, a bohužel jim to vychází. Kamery rozpoznají kohokoliv,“ odpověděl jsem jí nešťastně. Jenom se zasmála a o dva dny později jsem již zjistil proč.
	
Konečně jsme přišli až k městečku Hangzhou, ale ještě než jsme se vnořili do jeho rušných ulic, nasadili jsme si roušky, sluneční brýle a pečlivě si zakryli vlasy. Kamery nás tedy nerozpoznaly a co lepšího, ani jsme jim nepřišli podezřelí. Lidí s rouškami a brýlemi je plná Asie.
	
Tak jsme se ulicemi, které až podezřele často hlídal těžce ozbrojený policista (naštěstí Čína neměla sebemenší důvod předpokládat, že bychom do tohoto města vyrazili spíš než do jakéhokoliv jiného a policejní ostrahu tedy posílila jen mírně), dostali až ke stanici rychlovlaku.
	
Průchod městem byl proti tomu, co nás čekalo nyní, hračka. Aby se člověk dostal na čínský vlak, musel se podrobit kontrole srovnatelné s kontrolami letištními. Projít tím vším s rouškou a slunečními brýlemi nepřicházelo v úvahu. Svou identitu jsme však bohužel také přiznat nemohli. Já a Lydie jsme byli na seznamu vězňů a Wobucika by nejspíš rozpoznávací programy nijak neidentifikovaly. Zkrátka průšvih by provázel každého z nás. Jediné štěstí bylo, že jsme si to uvědomili už předem a ani bezpečnostní opatření nepokoušeli.
	
Namísto toho, jsme opustili nádraží, sedli si na lavičku a přemýšleli.
	
„Není to žádná pevnost. Kdybyste chtěli, klidně vás přes turnikety přendám a sám je přeskočím. Rozstřílet ochranku bude hračka,“ nabídl se nám Wobuciko.
	
„V žádném případě. Jestli tohle dobrodružství někdo smí nepřežít, pak jsem to já, ale nikoho jiného při něm zemřít nenechám,“ reagoval jsem okamžitě.
	
„Ale ta Wobucikova myšlenka nebyla špatná, akorát to chce zajistit, že si alternativní cesty nikdo nevšimne,“ podpořila Lydie vojáka a svá slova doplnila zamilovaným pohledem. 
	
„Taky bychom mohli ke Třem soutěskám doletět, akorát nám musí narůst křídla,“ reagoval jsem na její návrh podrážděně.

„Co by ne. Ale myslím, že uvrhnout nádraží do totálního chaosu, ve kterém snadno přelezeme turniket, bude snadnější,“ usmála se Lydie mé mrzutosti. Zbytek dne jsme strávili kupováním věcí, které nám měly v šíření chaosu pomoct.

Večer, když se setmělo, jsme usoudili, že čas akce nadešel. Znovu jsme vstoupili do nádražní budovy a posadili se zde na lavičky, abychom se „navečeřeli“. Ze začátku jsme skutečně jedli. Nejprve pizzu a pak třešně, koupili jsme si jich tolik, že se ona večeře pro všechny stala nejdražším jídlem našeho života, ale stálo to za to. Drtivě nejvíc třešní dostal Wobuciko a to nejen proto, že z nás byl největší a nejtěžší, nýbrž také proto, že měl nejlepší mušku a to i v disciplíně plivání.

Prvních dvanáct pecek poslal ladnými oblouky do koše, aby získal sebevědomí, a do pořádné práce se dal až pak. První jeho střela zasáhla do oka postarší dámu, která stála asi tři metry od nás. Dotyčná vyjekla a promnula si zasažené místo, to už ale Wobuciko zasáhl i její druhé kukadlo. Výkřik i mnutí se zopakovaly, byly však mnohem intenzivnější než napoprvé a upoutaly na sebe pozornost lidí, kteří zrovna postávali okolo.

Náš mnohogenní souputník byl jedním z mnoha, kdo se po druhém výkřiku staré paní zvedl, ale jedním z mála, kteří vyrazili přímo od ní, tedy nikoliv na pomoc ale na záchod. Po cestě Wobuciko  průběžně vyprazdňoval svá ústa plná pecek a další a další lidé jekali, překvapeně si sahali na oči a někteří i volali o pomoc, kterou se jim také personál skutečně snažil poskytnout. Co se tvorby chaosu týče, velmi se osvědčila jedna stařenka, kterou Afričan postřelil asi dvanáct metrů před záchodem. Ani ne sekundu po té, co byla zasažena, už z plna hrdla křičela: „Pomoc! Atentát! Terorismus! Ujgurové! Vražda! Trump!“ Tím k sobě svolala všechny dospělé lidi z okruhu deseti metrů, Wobucika nevyjímaje.

Zatímco však všichni přivolaní Číňané měli oči a ruce jen pro panickou stařenku, Wobuciko se věnoval dýmovnici, kterou ve změti těl nepozorovaně vytáhl, zapálil a hodil si ji přímo pod nohy. Lidé okamžitě začali utíkat, mnozí však přímo na místě popadali, a tak měl Wobuciko v kouři prvních dvacet vteřin společnost. 

Využil tento vzácný čas k tomu, aby naráz zapálil a po nádražní hale rozházel další tucet dýmovnic. Jelikož zatím sdílel oblak kouře s několika dalšími lidmi, nemohla si být kamery sledující ostraha jistá, že za chaosem stojí právě on.

Lidé nevěděli, co mají dělat. Někteří začali utíkat, jiní se dosud nevzpamatovali ze zásahu třešní, a tak se doposud svíjeli na zemi poutajíce k sobě pozornost altruističtějších soudruhů. Já, Lydie a Wobuciko jsme chaos využili k tomu, abychom se co nejrychleji dostali k turniketu zahalenému do zvlášť hustého dýmu.

Dlouho plánovaný přechod nakonec trval ani ne deset sekund. Wobuciko popadl do jedné ruky mě, do druhé Lydii, zvedl nás do vzduchu a přendal přes kovovou přepážku, kterou následně sám přeskočil. Tím bylo hotovo, deset minut po té jsme už stáli na nástupišti a dalších patnáct minut po té jsme nasedali do rychlovlaku a vyráželi směrem do Wu-chanu.

V Jüe-jangu jsme museli vlak opustit a počkat si na další, který nás odvezl do Wu-chanu. Odsud jsme se chtěli obyčejnými vlaky dopravit asi čtyři sta kilometrů až ke Třem soutěskám, Lydie si na internetu našla přímý spoj a ukázala nám, z jakého nádraží odjíždí.

Rychlovlaky bohužel nebyly jedinými bedlivě střeženými vozidly a považovali jsme za příliš riskantní opakovat tvorbu chaosu prostřednictvím dýmovnic a třešní. Zvolili jsme proto jednodušší cestu a vlezli deset minut před příjezdem vlaku na koleje asi tři sta metrů od nádraží a přiběhli mu po nich vlaku naproti. Když se rozjížděl, viděli jsme strážníky, jak přibíhají na nástupiště, ale byli to jen obyčejní měšťáci bez pravomoci a chuti kvůli nám zdržovat stovky lidí, a tak jsme mohli šťastně vyrazit a do čtyřiadvaceti hodin si už prohlížet gigantickou přehradu.

V den příjezdu jsme se utábořili v lese a přečkali zde noc. Ráno jsme pak vyrazili na místo, kde se měl nacházet podezřelý objekt. Google dokázal, že co šmírování týče, je na něj spoleh. Nepřekvapilo mě ovšem ani tak to, kde se základna nacházela, jako přesnost se kterou datoví vědci tohoto technologického giganta spočítali i trasy, po nichž chodili vojáci, kolem budovy hlídkující.

Tento výkon byl o to pozoruhodnější, že vojáků zde promenádovalo hned několik desítek. Bylo nadmíru jasné, že se uvnitř děje něco, co by Čína opravdu ráda utajila a nutno uznat, že na to šla dobře. I po té, co si Lydie lámala hlavu tři dny (a já taky, ale moje nápady nikdy nebyly zvlášť úspěšně), jsme nevěděli o žádné bezpečnostní díře.

Hlídky budovu vytrvale obcházely ve dne v noci. Veškerý kontakt s okolním světem objektu zajišťovala široká betonová cesta, po níž přijížděl autobus, který přivážel a odvážel dělníky (a čas od času také prostřídal pár vojáků), a kamiony, které dovnitř nejspíše dopravovaly materiál. 

Jedním z důvodů proč Lydiino lámání hlavy neneslo tolik ovoce jako dřív snad bylo i to, že ji ztratila pro Wobucika. Nejspíš si ji naklonil ještě v Africe, když ji vysvobodil z ruského zajetí, během oněch třech dnů však jejich láska vzplála naplno. 

První den se navzájem čas od času lehce dotkli a třetí den už bylo těžké poznat, kdo je kdo. Vzhledem k naší situaci byl tento románek krajně nezodpovědným, ale vzhledem k tomu, že Bůh nestvořil nic posvátnějšího než lásku mezi mužem a ženou, nemohl jsem jejich štěstí odsuzovat. Však nám taky nakonec pomohlo.

V noci z třetího na čtvrtý obhlížecí den leželi Lydie a Wobuciko zaklesnutí v sobě a novinářka svého partnera vychvalovala. „Ty máš tak velké svaly, na rukou, na nohou…“.  Snad už jsem byl dlouhou bezradností tak změkčilý, že jsem začal melancholicky osahávat Lydií vyjmenovávané údy na svém vlastním těle a pokaždé se ujistit, že mě by za ně chválit nemohla. Ruce, nohy i břicho jsem měl hubené.

„…a černé oči i hladké tváře“. Dotyk mých vlastních tváří, které po víc jak týdenním neholení hladkými nebyly ani náhodou, vytrhl mou maličkost z melancholie a následný výkřik „Heuréka!“ zas Lydii s Wobucikem z jejich milostného opojení.

„Jak moc jsme si podobní?“ ptal jsem se již o chvíli později svých druhů, zatímco jsem svítil mobilem střídavě na fotku na Gangově průkazce a střídavě na sebe.

Lydie chvíli přemýšlela, načež se otočila k Wobucikovi. „Vraž mu pěstí,“ vyzvala ho.

V zápětí jsem dostal ještě větší ránu do břicha, než jakou mě kdysi Jiří ztrestal za pasivitu. Lydie však stále ještě nebyla spokojená „Tam ne! Do nosu, ty můj goriláčku.“ i nyní ji Wobuciko poslechl.

Jen co se mi vytryskla první krev, otřela ji novinářka Gangovou průkazkou. „Už bych si nebyla jistá, že na ní nejsi,“ vysvětlila mi následně onen čin.

 „A kde jsem si ji ušpinil?“

 „Holt ne každý křesťan je mírumilovný, a tak tě onehdy jeden postřelil.“
