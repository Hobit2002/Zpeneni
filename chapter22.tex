\chapter{}

Chun shromáždila všechny povstalce a další vězně, kteří toho dne konvertovali, na dvorek a sdělila jim dva plány. První, ústupový, zněl asi takto: „Vlezte do kanálu a než ho opustíte, běžte jím, co nejdéle a co nejdále.“ Druhý plán se týkal toho, co potom a nebudu ho zde popisovat. Již příští kapitola pojedná o jeho realizaci. 

Tyto pokyny poslechli až na dvě výjimky všichni, ke kterým byly směřovány. První výjimku představovala nevelká skupina dobrovolníků, kteří odnesli neošetřené pacienty na své cely, aby tam o ně pečovali v bezpečí. Druhou výjimkou byla trojice Chun, Lydie a mě, která kanalizací prchala jen chvíli a už po jednom kilometru vylezla ven, aby zkusila najít zázemí u Chunina expřítele Pinga.  

Slovo „zázemí“ nutno používat opatrně, například v případě znepokojivě ledové vřelosti, se kterou je dříve veselý a srdečný Ping přijal, bychom se ho mohli zdržet. Ale co, sice vypadal, že by nás nejraději podřízl a pak vyhodila z okna, avšak nic kromě večeře nám neudělal. Ba co víc ke konci společného jídla se odvážil i k osobnímu vyznání. 

„Už tě chápu,“ řekl natolik Chun měkkým, tichým a třesoucím se hlasem, že pokud by čínská světice měla diktafon, jeho slova nahrála a nahrávku poslala do Holywoodu, začaly by se americké filmy točit v čínštině, aby se do nich dala Pingova slova vložit. 

„Vím, co jsi provedla. A vím, jak mohu být za své pohostinství potrestán, ale já bych si tak moc přál začít znovu.“

„Znovu nefunguje nikdy, je třeba jinak.“

„Ano, znovu a jinak. Když jsi mě opustila, rozhodl jsem se utopit žal v práci a byl za to povýšen. Teď mám víc peněz, méně času a méně spánku. Už rozumím tomu, jaké je to být večer totálně vysátý a respektoval bych to i u ostatních.“

Chun upřela oči do své misky a uvažovala, jak dlouho by spolu zvládli žít, než by ji policie odhalila. Byla si však vědoma toho, že Pingem předestřená životní cesta byla zcela neslučitelná s povstaleckým plánem. Mlčky váhala, jak podrobně mu má vysvětlovat svůj nesouhlas a z tohoto váhání ji vytrhnul až telefonát od papeže.
	
Zpráva o vylomenině, kterou křesťané ve věznici Tilanqiao provedli, už stihla obletět svět a oslovit i ty nejmocnější a nejslavnější. Lydii se teď snažilo dovolat (jakmile při vězeňské revoluci získala mobil a přístup na internet, okamžitě aktualizovala všechny své kontaktní údaje) téměř každé západní médium, každá evropská vláda, hlava každé církve… a ona jim esemeskami posílala čísla svých společníků. Brzo tak přišla o všechen kredit.

Každopádně telefonát od papeže se nesmí odmítnout. Lydie sebe sama nepovažovala za tak důležitou, aby na dotazy odpovídala a hovor předala raději Chun.

„Ano bratře, Bůh nás ochránil a my uspěli,“ jak přesně zněl, papežův dotaz nebylo slyšet.

„Nemějte o nás strach, nám nic nehrozí. Duch vane a nic nepřekoná jeho sílu, aby odválo naši víru. Naše duše nezabijí, a když zůstanou jen u těl, tak se nic nestane.“

„Stáhli jsme se, ale nevzdali jsme se. Počkejte sedm dní a…“ Lydie jí přistrčila druhý telefon, Donald Trump. Chun si ho přiložila k druhému uchu a pokračovala v hovoru „uvidíte zázrak, neváhejte přijet sem do Šanghaje a vzít sebou tolik věřících, kolik jen bude možné… Ano možná jsme blázně, ale blaze chudým v duchu… Ne snad? Jestli chcete zázraky, tak se přijeďte za týden podívat do Šanghaje,“

Poslední větu slyšel i pravoslavný patriarcha volající na mobil, který Lydie Chun držela pod ústy.

„Ne opravdu se nebojíme. Vy všichni jste křesťané, tak jak můžete pořád omílat téma strachu? Čeho se mám bát? Leda vlastní zloby, pýchy a lenosti. Tenhle svět není křesťanský, ani zdaleka ne, a nebude jím, dokud bude ve středu našeho myšlení náš vlastí prospěch. Obraťme se Kristu bratří a přestaňme se tak zabývat jen sami sebou. Já nejsem nic, dokud tu jsem především pro sebe. Teprve to, že mi záleží víc na svých bližních…“ 

Při těch slovech máchla rukou a tím promarnila příležitost změnit politickou situaci v USA, jelikož Donald Trump její slova o nesobeckosti neslyšel.
 
 „...teprve to mi umožňuje být součástí něčeho aspoň trochu stálého. Žádný strach nemůže převážit tu nutkavou potřebu přeměnit Boži milosrdenství, kterého se nám dostalo v radostnou chválu, obětavou službu a změnu k lepšímu.“

Nikdo z posluchačů a těch bylo ke konci telefonátu skoro deset, se neubránil slzám. Papež plakal dojetím, patriarcha zoufalstvím, kam se křesťanství dostalo, Trump štěstím, které pociťoval při představě Číny v problémech, Putin výčitkami, které jej zavalily jako tsunami lítostivých duchů, Johnson smíchem, ředitel BBC radostí z představy, kolik lidí bude sledovat jeho vysílání, když předem ohlásí vysílání této promluvy a stejně tak i ředitel SkyTeam při představě kolik lidí si koupí letenku do Šanghaje. 

Plakal i Ping. „Myslel jsem si, že už ti rozumím, ale ty ses mezitím úplně změnila. Vyprávěj mi prosím, jak se tvůj život dostal až sem. Pracoval jsem šestnáct hodin, ale tohle si ujít nenechám“ poprosil Chun, když se po třech hodinách nepřetržitého volání konečně vyčerpaně zhroutila.

‚On mi ale zůstal stejný. Přes všechna slova porozumění mou únavu nadále totálně ignoruje‘ pomyslela si Chun. Aby však za sedm dní mohlo dojít k zázraku, bylo třeba se od něčeho odpíchnout a Ping jako začátek nebyl špatný.
Po jídle, když všichni ostatní vyrazili do postelí, sedla si Chun s Pingem ke stolu a začala mu vyprávět, co jsem jí dříve řekl já.

Po večeři a mnoha historických telefonátech to šlo ráz na ráz. Policii se podařilo nás sledováním našich telefonů přibližně lokalizovat, a tak už o půlnoci před domem parkovalo několik policejních aut. Naštěstí si Ping, který se teprve nedávno vyšvihl z dělníka na manažera, nemohl dovolit bydlení na lepším místě než v paneláku, kde se pod jeho bytem nacházelo přibližně osm a nad ním asi deset jiných. Vertikální polohu GPS určit nedokázala. 

A tak, zrovna když Chun Pingovi vyprávěla, jak nastavovala druhou tvář obuškům, přerušil ji dupot na schodech a křik vyvolaný tím, že policisté vtrhli do některého z bytů.

Po chvíli ticha se Ping poškrábal na hlavě: „To asi hledají vás, ale jak zjistili, že tu jste?“

Chun si okamžitě velmi živě představila své bývalé kolegy od policie, jako neviditelné duchy poletující městem a šmírující ty obyvatele, kteří si doposud zachovali vlastní duši, ale pak se přiměla vycházet ze svých zkušeností: 

„Mobily nás prozradily.“

„Teď jsou asi u Čchenů, mají stejné příjmení jako ty,“ usoudil Ping.

„Musíme se jít udat, nemůžeme tě vystavovat nebezpečí,“ smutně si povzdechla Chun.

„Snad by stačilo zbavit se telefonů,“ nadhodil Ping.

Jak navrhnul, tak učinili. Chun, Lydie i já jsme mu dali své mobily, on si je strčil do kapes kabátu a policistům za zády sjel výtahem do prvního patra, kde probudil svého kamaráda. Z postele vytažený se tvářil velmi nevrle a ještě více nechápavě, když ho Ping požádal, aby mohl vejít dovnitř. Síla obou těchto emocí v něm ještě mnohonásobně vzrostla, když Ping rázně přišel k oknu, otevřel ho a vyskočil z něj ven. Bylo to tak divné, že čerstvě probuzený usoudil, že spí, zavřel okno a vrátil se zpátky do postele.

Ping, který se tímto kouskem vyhnul policejní kontrole u vstupu do domu, doběhl na vlak, jímž se nechal dovézt až k továrně, kde pracoval. Naše mobilní telefony svých hostů odložil pod sedadlo. Na pracovišti svou ochotou dřít přes čas nadřízené velmi potěšil, a tak nad rámec standardní dvanáctihodinové pracovní směny ještě čtyři hodiny vyřizoval administrativní záležitosti.
\vspace{0.75cm}

Ráno roj policejních helikoptér, několik set po zuby ozbrojených těžkooděnců doplněných ještě těžší technikou zablokovalo Šanghajskou vlakovou dopravu. Vysokorychlostní vůz musel zastavit, přes čtyři stovky pasažérů si mohlo jít trhat okvětní lístky ze sedmikrásky a přitom se ptát: „budu vyhozen za pozdní příchod do práce?“ „nebudu vyhozen?“ a elitní policejní komando vpadlo do vlaku, kde přístroje odhalily mobilní telefony povstalců.

Vyšetřovatel doufal, že povstalce lapí. Kvůli jejich rebélii včera v Tilanqiau kulkou vlastního nadřízeného padl muž, do kterého byl detektiv zamilovaný. Kdyby se mu nedostalo té cti chytat právě ty teroristy, kteří všechno zavinili, zůstal by doma a na svou tajnou lásku vzpomínal. Jeho svébytná mysl však byla milosrdná a dopřála mu tento den prožít s představou zesnulého, která byla téměř stejně živá jako dotyčný před dvěma dny. Když vbíhal do vlaku, jeho láska letěla za ním a hecovala ho, aby byl, až se s rebely střetne, neúprosný.

A teď tohle! Čtyři telefony na zemi a nikdo k nim. Už se chystal vytáhnout zbraň a ve vzteku je rozstřílet, když náhle jeden zazvonil.

„Jsi to stále ještě ty?“ otázal jsem se. (Volal jsem tehdy na svůj mobil z telefonní budky, abych se ujistil, že už se ho Ping zbavil.)

„Ano,“odpověděl policista doufaje, že čirou náhodou volá nějaký ze zločinců.

„Ne“ jestli jsem si něco opravdu dobře pamatoval, pak hlasy a obličeje. Nyní jsem bezpečně poznal, že na druhé straně není Ping „ale tím spíše potřebuješ slyšet tu radostnou zvěst.“ Má slova prý byla tak nadšená, že se detektivu samotnému rozbušilo srdce vzrušením. „Miluji Tě, protože Ježíš Tě miluje.“

Vzdor profesionálnímu odstupu, vzdor obrovské bolesti a nenávisti, kterou byl sžírán, pocítil vyšetřovatel po mých slovech příjemné uklidnění.

O křesťanství toho věděl asi tolik jako o skutečném významu slov „svoboda a demokracie“ a vůbec tedy netušil, že láska má více významů.  Má slova však pro něj byla o to více povzbuzující a to dokonce tak, že se vedle vidiny jeho mrtvého milého objevila i druhá vidina krásného archanděla, jehož vyšetřovatel ztotožnil se mnou.

Do něj se někdo zamiloval? A hned dva muži? To bylo poprvé za celý jeho život. On sám byl sice zamilovaný do svých známých už mnohokrát, ale tyto city nikdy nikdo neopětoval. Oficielně už sice měl manželku, ale jednalo se o sňatek uzavřený pouze proto, aby jeho konzervativní rodiče neodhalili, že je homosexuál. 

Stále pro mrtvého truchlil, ale pocit, že je milován, smutek zvládal překvapivě efektivně kompenzovat. Škoda jen, že jeden z těch dvou milovníků patřil do vězení. Naštěstí tu stále byl ten Ježíš.

„Můj milovaný bratře, on pro Tebe zemřel a já jsem ochoten učinit v příhodnou chvíli totéž. Obrať se, ukonči tu nenávist, vyčkej sedm dní a přijde zázrak, o jakém se ti dřív ani nesnilo. Milovaný bratře, jen kvůli této zvěsti nás pronásledujete, ale proč? My vás milujeme, přidejte se k nám…“Vyšetřovatel hovor típl. ‚Blázen‘ pomyslel si, ale usoudil, že týden si na onen zázrak, zřejmě nějaké dost divné rande, počká.

Ještě dramatičtěji hovor prožíval policista z odposlechů, který měl o něco citlivější duši než vyšetřovatel. Bodejť by ne, celé mládí si přál být psychoterapeutem, který naslouchá lidem s problémy, ale jelikož tuto profesi čínské úřady považovaly za zbytečnou, nezbývalo mu než svou touhu, naslouchat ukojit skrze policejní odposlechy. Mé krédo na něj silně zapůsobilo, rozeslal tedy mou řeč všem známým (kteří už od něj předchozí den obdrželi sérii Chuniných telefonátů) a večer, před tím než byl zatčen, se poprvé účastnil bohoslužeb.
