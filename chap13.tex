\chapter{}
Roky před tím, než to všechno začalo, býval jsem docela obyčejný záchranář žijící v nehezkém městě na severovýchodě Číny. Bydlel jsem v malém a levném bytě (a to ačkoliv jsem jinak dřel jako mezek), dával si pozor na svá slova a snažil se nijak nevyčnívat z davu. Tak nějak plynul můj život od osmnácti do pětatřiceti.

Rok před svým setkáním s Jiřím jsem se jednou přímo v nemocnici zhroutil vyčerpáním. Sedmnáct let nepřetržité práce si vybralo svou daň, ale bylo v zápětí velmi štědře odměněno. Když jsem tak ležel na zemi, ujala se mě jedna mladá sestřička, která se zrovna nacházela poblíž.
 
Když jsem se probral, zjistili jsme, že se oba cítíme osamělí a spoustou práce odříznutí od jiných lidí. Toto vědomí nás dalo dohromady. Po měsíci chození jsme se vzali a trochu si tak pomohli. Bohužel jen velmi krátce. Už po svatební cestě, kterou jsme strávili převážně na plážích Šanghaje, se mé novomanželce udělalo velmi zle.

Odvezl jsem ji do nemocnice, ale nikdo nedokázal ani diagnostikovat její problém. Přiznávám, že ani já sám se nikdy nesetkal s takovými symptomy, které mělo její onemocnění. Když jsem se tři dny po propuknutí její choroby a půl dne před její smrtí bavil s lékařem, řekl mi, že má žena nemá téměř žádné bílé krvinky (nicméně virus HIV v jejím těle nenašli), že ztrácí schopnost vstřebávat do těla vodu, že je silně meningeální a že trpí spoustou dalších příznaků, z nichž každý by nahrával trochu jiné chorobě. Patogen, který za tím vším stál, ovšem nikdo nenašel. A tak má jediná láska umřela.
\vspace{0.75cm}


Duševně jsem se zhroutil. Jestli jsem chtěl něco, pak to, abych se zmotivoval k sebevraždě. Ale ani to se mi nedařilo. Byl jsem zcela apatický, a pouhé zvednutí ruky mě stálo přemáhání. Měsíc jsem nepřišel do práce a ani svou nepřítomnost nadřízeným nevysvětlil. Byly dny, kdy jsem nejedl ani nepil, ale pouze zíral do stropu a v hlavě měl bílo. Má duše zkrátka věděla, že kdybych měl emočně strávit to, co mě potkalo, zničilo by mě to, a tak se raději vypnula.

Z tohoto stavu mě vytrhnul až Jie. Vůbec netuším, co toho hada vedlo k tomu, že se mi snažil pomoct, ale je fakt, že nebýt jeho, ležel bych ještě teď na posteli a pozoroval pavučiny. 
	
Teď trochu uceleněji.

Během jedné z bezesných nocí začal někdo zvonit na můj byt. Samozřejmě, že jsem mu neotevřel. Po několika minutách mačkání zvonku toho nechal a já uslyšel, jak něco šramotí v zámku. Tento zvuk trval asi dvě minuty a pak zase přešel ve zvonění, tentokrát však velmi krátké. Už chvíli po zvonku se ozval lidský hlas.

„Přicházím, abych vám vrátil štěstí!“

‚Co ten ví o mém neštěstí?‘ pomyslel jsem si a ležel dál.

„Přicházím, abych vás přiblížil vaší ženě!“

Tohle už zabralo. Namáhavě jsem se zvedl z lože a došoural se otevřít dveře.

„Mohu dovnitř?“ otázal se mě muž, který za nimi stál.

Lehce jsem pokývl hlavou a on vešel. „Už pár týdnů vás pozoruji a zjišťuji, že jste zničený,“ oznámil mi. „To by mě podle mých policejních nadřízených vůbec nemělo zajímat. Stačí, že jste se nenakazil nemocí své ženy a že ji tedy nejspíš nebudete předávat dál (při těch slovech se lehce ovšem zcela nepochopitelně rošťácky pousmál). Já jsem ovšem perfekcionista a chci vám pomoct.“

Tupě jsem na něj zíral.

„Představte si, že je nějaký Bůh a řekněte mi, jaký by měl být, aby byl nejlepší možný.“

Dlouho jsem neodpověděl nic a pak se zmohl na pouhé:„Jestli nějaký Bůh je, pak to, co se děje, je jeho vůle a jelikož tím, co se děje, trpím, je to můj nepřítel,“ 

„Proč se svazovat touto logikou? Nejen, že není jediná možná, nýbrž je taky ošklivá; a víra, to je umění a umění, to je, navzdory moderním umělcům, kteří si sebe sama spletli s novináři a filosofy, krása. Jaká jsou ta nejkrásnější nebesa, kterým byste byl schopen věřit?“

Zamyslel jsem se a dokonce v sobě po dlouhých týdnech probudil emoce, ale ještě než jsem si dokázal vybavit něco krásného, rozpomněl jsem se na svou ženu a kolaps, před kterým se má duše chránila apatií, propuknul naplno. S pláčem jsem se smotal do klubíčka.

Můj host mě chvíli smutně pozoroval, ale zřejmě nevěděl jak reagovat. „Řekněte, čím vám mohu pomoct a já to udělám,“ řekl velmi nešikovně.

„Mně nemůžete pomoct,“ odpověděl jsem nechtě nic, než aby odešel.

„Kdybyste chtěl uvěřit, že se svou ženou ještě setkáte, dovolte mi, abych za vámi v neděli přišel i s několika přáteli.“

Už jsem byl zcela vyčerpaný a tak jsem přestal vzlykat a znovu se dlouze odmlčel. Nabídka nezvaného hosta mi připadala mimořádně divná, ale i v mé jinak zablokované mysli se našlo dost přemýšlivosti na to, abych si uvědomil, že je to možná jediná cesta, jak se vyhrabat z postupné zkázy.

„Přijďte“ dostal jsem ze sebe.

Jie se slabě usmál, pomohl mi zpátky do postole a odešel.
\vspace{0.75cm}

Jak ten Jidáš slíbil, tak učinil, v neděli se přímo v mém bytě konaly bohoslužby čínské podzemní církve a stejně tak tomu bylo i týden příští, popříští a popopříští.

A nejen to. Když členové křesťanské komunity zjistili, v jakém jsem stavu, začali ke mně domů chodit, aby mi pomohli s úklidem a hlavně, aby mi poskytli společnost. Krom toho mi při těchto příležitostech vždy přinesli nějaké pořádné jídlo.

V neposlední řadě jsem také dostal od Jieho několik knížek a odkaz na blog jakéhosi afrického filosofa. Prošel jsem si obojí a byl studiem velmi povzbuzen do dalšího života.

Brzy jsem díky křesťanské komunitě začal ožívat a to nejen tělesně, sociálně, ekonomicky a duševně, nýbrž také spirituálně.

V Číně se veřejně nikdy o věcech jako posmrtném osudu duše nemluvilo a nyní jsem se začal několikrát do týdne setkávat s lidmi, kteří byli zcela přesvědčení, že až skončí jejich pozemský život, který většinou za moc nestál, půjdou ke svému dobrému Bohu do ráje a zde se setkají i se všemi, kteří je na zemi opustili. Tomuto Bohu prý, alespoň podle mých spolubratrů, nezáleželo na ničem jiném než komunitě, kterou v jeho království budovali již zemřelí a kam po smrti přišla každá lidská bytost. V jeho společenství prý sice přetrvávala bolest a nepohodlí nikoliv však smrt a agonie, a tak byli ti, kdo se v životě oddávali trýznění svých bližních, odsouzeni k samotně a nemožnosti seberealizace a to až do doby, kdy se napravili. Tento život měl být přípravou, během níž si člověk měl osvojit laskavost, poctivost, důvěřivost a zodpovědnost za své okolí, které prý vyvářely základ posmrtných vztahů. 

Pochopitelně, že se mi ta víra velmi líbila. Vůbec jsem netušil, co všechno dokazuje pravdivost této víry, ale jak jsem onehdy četl v moudrém internetovém blogu Jiem propagovaného myslitele, u víry si člověk s nějakou pravdivostí nesmí lámat hlavu, stačí, když není nesmyslná. 
\vspace{0.75cm}

Uplynul rok a já už zase chodil do práce a uměl se o sebe postarat. Smutek ze smrti mé ženy mě neopustil, ale má duše usoudila, že bych měl přijmout víru v naše budoucí setkání a já tedy začal věnovat až dvě hodiny denně modlitbám.

Každé ráno, nezávisle na počasí a svém aktuálním zdravotním stavu, jsem vylezl na střechu panelového domu a za pomalého a mechanického cvičení jógy (které vypadalo rozhodně méně podezřele než sepnuté ruce a klek či jakákoliv jiná častá modlitební póza) vzýval Boha.

Prosil jsem ho, abych zvládal svou práci, aby policie nepřišla na mé členství v podzemní církvi, aby se mí milí uzdravili ze svých chorob a hlavně aby ono slibované Nebe bylo opravdu skutečné a já se zde potkal se svou ženou. Tak šťastný, abych taky děkoval, jsem nebyl.

Onen africký filosof, jehož jméno už jsem zapomněl, mě sice inspiroval, abych víru přijal, ale mé pochybnosti o tom, zda náhodou není bludná, nevyvrátil. Tolik jsem si přál, aby biblické příběhy a stejně tak i přinejmenším ty novodobé legendy o světcích, byly pravdivé. 

To, že v nich občas nefungovaly fyzikální zákony, jsem přešel jako nepodstatné. Mnohem víc mi vrtalo hlavou, zda je možné, aby lidé byli skutečně tak věrní, odvážní a laskaví, aby Boží království skutečně bylo oním slibovaným rájem. Na základě své zkušenosti s lidmi jsem si těžko mohl odpovědět, že ano.

Když jsem se s těmito pochybnostmi svěřil při bohoslužbách svým bratřím, krčili rameny a bylo vidět, že je nic takového netrápí. Akorát Jie reagoval jinak: „Což o to, nějaký nový František z Assisi či nová Matka Tereza, by přišli vhod, ale ještě víc potřebujeme rytíře a bojovníka.“

Byl jsem rád, že je naše touha alespoň z části společná. Nicméně jsem jeho pohled nepřebral a nadále se modlil: „Pane Bože, seznam mě prosím s někým Tebou požehnaným.“

Tato prosba byla asi první, na kterou Nebeský Král odpověděl.

Jedné noci mi u postele zazvonil telefon. Zvedl jsem ho a bezpečně poznal Jieho hlas.

„Zdravím Dongu, vzpomínáš, jak jsi onehdy říkal, že toužíš po světci a já odpověděl, že bych chtěl raději hrdinu?“

Něco jsem rozespale zabručel.

„Našel jsem člověka, který je obojím. Ale bojím se, že mi bude zabit.“

„To by byla škoda,“ odpověděl jsem.

„To tedy. Ale já už mu moc pomoc nemůžu,“

„A já snad ano?“

„ Ano, ty ano Až tenhle hovor skončí, pošlu ti GPS souřadnice místa, kde by měl být vojenský transportér uvízlý v řece Xiliao, ty tam přijedeš a snad potkáš živého Evropana, kterého dopravíš k sobě domů. Zítra se u tebe stavím a vyzvednu ho.“

Aniž bych si uvědomil, že se tak sám stávám jak světcem, tak hrdinou, rychle jsem vyskočil z postele a aniž bych se oblékl, naskočil do auta a vyrazil na místo, jehož souřadnice mi už Jie stihl poslat.
