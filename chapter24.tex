\chapter{}

Blížil se začátek čtvrtého kvartálu a Chun si přes všechen počáteční optimismus nebyla jistá, jestli se Vděčná země dožije jeho konce. Putin sice pohrozil Číně, že pokud Vděčnou zemi napadne, odstřihne od elektřiny popřípadě námořní blokádou zabrání v jejím zásobování, pomstí ji celým ruským jaderným arsenálem. Přesto se v blízkosti Šanghaje shromáždil téměř milión čínských vojáků, kteří aspoň zastřelili každého, kdo hranice Vděčné země překročil a kteří by, kdyby dostali rozkaz, byli schopni i bez atomových zbraní vymazat z povrchu zemského celé město do dvou hodin.
	
„Modlím se, aby nastal další zázrak. Vím ale, že sám nepřijde. Vůbec ovšem netuším, jak ho trochu popohnat,“ svěřila se mi Chun ,když skončilo první zasedání vlády (konalo se v prostorách bývalé věznice Tilanqiao, které Chun vybrala za své královské sídlo).

„Stejně jako doposud. Pojmi to jako výzvu, která je mezi tebou a Bohem. Udržet se může být nekonečně těžké, ale věř, že ti Bůh bude dodávat sílu jít dál tak dlouho, jak dlouho po ní budeš toužit. Víra je to jediné, co rozhoduje o našem životě. Krom toho měj na paměti, že budujeme Kristovu komunitu a té by, stejně jako všem jiným krásným věcem, pohodlí jedině škodilo.“ Tyto myšlenky nebyly moje, vyčetl jsem je z díla nějakého afrického filosofa.	

„Souhlásím. Však se také o sílu a vytrvalost modlím sedm hodin denně. Co mi chybí, jsou nápady, a těmi jsem dosud obdařena nebyla. Jistě, vynasnažím se dobře odvést svou práci, ale to přeci nemůže být všechno!“ posteskla si Chun.

„Nyní jsi politička a to docela vysoce postavená. Putin není jediný politik, se kterým si pohovoříš jinak než telefonicky. Zajisté se setkáš i se Si Ťing-pchingem či Donaldem Trumpem. Musíš na ně zapůsobit, musíš je obrátit! “

Tato vize Chun zas na nějaký čas vytrhla z reality a ona měla chvíli pocit, že je celá její poradní síň plná představitelů vojenských a ekonomických velmocí. 

\vspace{0.75cm}
\textbf{Podle Jieho deníku}
\vspace{0.75cm}

{\itshape
Gang se z mrákot probral na nejvýše položené terase Ťin Mao. Kromě něho tu byl ještě muž, který mu připadal povědomý. Chvíli si ho ze země, na které ležel, prohlížel a pak se mu rozsvítilo. Jeho společník, jenž se zrovna díval kamsi dolů, byl mezi strážci pořádku relativně známý policejní analytik a navíc jeho bývalý kolega v zahraniční rozvědce, Jie Zhe. Zda je však tento úspěšný policista stále věrný režimu, či zda se nechal zmámit křesťanským blouzněním, už Gang netušil.

Rozhodl se tedy, že prostě odejde. Zvedl se, došel ke dveřím a zjistil, že jsou zamčené. Kopl tedy do nich, ale pak ho napadlo, že jestli ho Jie zamknul na terase, nebude nejspíš na jeho straně. A za poslední týden se Gang velmi rychle naučil zabít každého, kdo byl byť jen trochu nevěrný čínskému režimu.

S úvahou ‚Zůstal tu, protože se mylně domnívá, že se mnou může poměřovat síly. Hlupák. Kdo jednou promarnil čas intelektuální prací, nemá ve střetu se mnou šanci,‘ pomyslel si a zašátral za opaskem po pistoli. Nenašel ji.

„Co mátu v plánu, Gangu Yen?“ otázal se Jie, jehož už zřejmě pohled dolů omrzel.

„Zlikvidovat všechny křesťanské rebely v téhle budově,“

„Pro takovou marnou práci je vás škoda. Svět už se zbláznil.“

„Mluvte prosím aspoň trochu k věci. Chcete snad označit mou snahu za mrháním času?“

„Ano, jen se…“ zpochybněním důležitosti své práce byl Gang dopálen. Vrhnul se na Jieho a jeho předpoklad, že analytik mu nebude v přímé konfrontaci schopen jakkoliv vzdorovat se potvrdil. Stačila jedna rána pěstí střední síly a drzý pomlouvač už ležel na zemi. Gang do něj párkrát kopnul a Jie se stočil do klubíčka, jako kdyby ho napadl pes. 

„Takže, já podle vás nepoznám, která práce dává smysl a která ne?“ otázal se Gang, když usoudil, že argument, kterým dokazoval svou pravdu, byl už dostatečně silný.

Jie jen něco zahuhlal. Gang si k němu tedy přikleknul, chytil ho za ramena a vztyčil. „Tak jak je na tom smysluplnost mojí práce?“ zopakoval svůj dotaz.

Jie mu odpověděl pepřovým sprejem, který si vytáhl z kapsy a schoval do dlaně, během toho, co do něj Gang kopal.
„Vím, že jsem slabší než průměrná dospívající dívka. Posledních dvacet let života jsem plně investoval do snahy pochopit tenhle svět a to je prvním z důvodů, proč byste mi neměl vymlacovat duši z těla ale naslouchat.

Když se Gang po nějaké té minutě dostal do stavu, kdy byl schopen znovu používat oči, zjistil, že oním druhým důvodem byla jeho vlastní pistole, kterou mu Jie sebral, když byl ještě v bezvědomí,  a kterou na něho nyní mířil.

„Vy byste svou práci dělal excelentně, kdybyste si byl ochoten připustit, že ji excelentně neděláte. Podívejte se dolů!“. 

Gang se tedy podíval a hle. Ulice byly plné jásajících lidí a totálního chaosu. „Tak a teď si otevřete WeChat a shlédněte první video s tím chlapem, kterého jste chtěl před pár hodinami zastřelit, na které narazíte.“  I nyní jej Gang uposlechnul a zhlédnul nahrávku, která celý rozruch vyvolala.

„Co byste s tím mohl udělat?“ otázal se ho Jie.

„Zabít Chun a vzít těm povstalcům jak vůdkyni, tak naději.“

„Vypadají snad, jako by je někdo vedl? A naděje se pro ně stala životním paradigmatem. Spíš je přesvědčíte, že žijí v Matrixu, než abyste je uvrhnul do zoufalství.“

„Tohle musí skončit! Klidně sám během několika let vystřílím celé tohle město, ale nedovolím, aby mě tenhle mor přežil.“

„Máte pravdu, vystřílet celou Šanghaj a nejen ji, je v tuto chvíli nejrozumnější řešení celé krize a já vím o armádě, která nám s tím pomůže.“

„Jistě, vůbec nechápu, proč už ČLOA dávno nezaútočila.“

„Putin Číně pohrozil, že kdyby nás napadla, použije proti ní ruské atomové zbraně. Já ale vím o dosud utajovaných geneticky modifikovaných jednotkách, nebylo by těžké, poslat je, aby zaútočily na Šanghaj. Pokud bychom cestou vyplenili ještě pár čínských sídel, nikdo by si netroufl předpokládat, že likvidace tohoto bordelu byla právě čínskou iniciativou.“

Gang se na chvíli zamyslel a pak přikývl: „Není to ideální. Čína si nezaslouží žádnou újmu, ale jdu do toho. Seženu si ještě pár věrných chlapů, vy nám řeknete, kde ty jednotky najdeme a my si pak pro ně dojdeme.“

Nejhorší na tom bylo, že Jie v celém mém příběhu ještě nikdy nelhal tak málo.}
\vspace{0.75cm}

„Putin se dlouho u moci neudrží. Rusům se jeho politika nelíbí. Už to není moderní Alexandr Dobyvatel, kterého tak zbožňovali, nýbrž Woodrow Wilson 21. století. Bohužel, svoboda předběhla moc pouze na jeho osobním, nikoliv na národním hodnotovém žebříčku. Je to pouze otázka času, než někdo svrhne Putina a následně skoncuje s veškerou podporou, kterou nyní Rusko Vděčné zemi poskytuje,“ pronesla Lydie pochmurně, když se na chodbě vládní věznice potkala s Chun.
Královna, beztak neveselá, se zamračila.„Co s tím můžu dělat?“

„Byla jsem zatčena a obviněna ze špionáže, protože jsem objevila vojenskou jednotku, se kterou bychom se ubránili každé konvenční armádě,“ odpověděla Lydie.

Chun se na chvíli zamyslela a pak rezolutně odvětila: „Ne. Dokud jsem královnou, nebude přitakáno žádnému krveprolití. Krom toho, pokud je ta armáda čínská, tak nám asi nebude pomáhat.“

„Mohla by. Je to synteticky vyrobené vojsko, genetické inženýrství už je pěkných pár let mnohem vyvinutější než veřejnost tuší. Číňané zvládli už v devadesátých letech vyrobit supervojáky, které pěstují na utajeném cvičišti v Africe. Tito vojáci povětšinou netuší nic o světě kolem, zato jsou asi jedinými Číňany, které vláda vede k náboženské víře. Už odmala o ně pečují důstojníci, kteří jim vštěpují fanatickou úctu k čínské vojenské a stranické hierarchii. Kdo má dost frčků na uniformě, za tím se vrhnou i do pekel. A nutno říct, že se Číně povedli natolik, že to špatně skončí spíš pro peklo než pro ně.

Pokud chceme zabránit krveprolití, mohl by se tam někdo vypravit a přinejmenším je někam schovat. Sehnat mu uniformu čínského vojáka snad nebude obtížné.“

„Ty ovšem musíš zůstat tady. Vděčná země potřebuje ministryni zahraničí.“

Lydie se zašklebila, od té cesty si slibovala mnohem víc než jen záchranu Vděčné země. Doufala, že díky ní zažije další skvělý příběh, který jí nejen přinese novinářskou slávu, ale zároveň jí poskytne pocit životního naplnění.

„Mohla bych potřebnou komunikaci vyřizovat online, mám elektronický podpis, takže ani podepisování smluv by nebyl problém,“ nadhodila, ač věděla, že se jedná o špatný nápad. Internet poskytoval člověku všechno, jen ne soukromí.

„Tohle jsou těžké časy, potřebuji mít své ministry poblíž, jste mi nedocenitelnou psychickou podporou,“ řekla Chun, jež kybernetickému světu vůbec nerozuměla.

Lydie usoudila, že pokud je královna ochotna argumentovat i takhle, nemá žádná diskuze smysl.
\vspace{0.75cm}

„Nechtěl byste být ministrem zahraničí?“ otázala se navečer Jieho.

„Pročpak?“ 

„Potřebuji odjet do Afriky.“

„Tak to se asi přidáte ke svým bývalým dozorcům.“

„Prosím?“

„Vděčná země by měla dříve či později problém s fanatickými příznivci minulého režimu, kteří se našli především na místech, jež je vychovávaly k nenávisti, tedy mezi policisty a vězeňskými dozorci. Kdybychom s nimi něco neudělali, stali by se z nich teroristé. Proto jsem je dal dohromady a namluvil jim, že jsem na jejich straně a že zakládám právě onu teroristickou organizaci, která by jinak samovolně vznikla později. 

Tahle parta už tuší, že je zanedlouho pošlu do Afriky, aby tam přebrali kontrolu nad mnohogenními vojáky. Jejich cíl je tedy shodný s Vaším, lišíte se v tom, že oni mají v plánu vojáky použít proti Vděčné zemi.“

„A jak chcete zamezit tomu, že uskuteční to, co uskutečnit chtějí?“

„Včas je eliminuji… a jak tak nad tím přemýšlím, vlastně byste to mohla udělat Vy.“

„Ano, pokud bych cestovala spolu s nimi, byla by to asi nejlepší možnost. Neumím si však představit, že by mě mezi sebe přijali. Oni vědí, že ze mě sled událostí udělal jak špiónku, tak ministryni země, kterou nesnáší.“

„Zato ovšem neví, že já vím, kde přesně se Čínská armáda nové doby, nachází. Prozradil jsem jim pouze to, že jsem si vědom její existence.“

„To znamená?“ otázala se ho Lydie.

Jie se na chvíli zamyslel a pak začal pomalu vysvětlovat nový plán, který už Lydie přijala.

„Dobře tedy. Půjdu do toho s vědomím, že se mě pokusí zabít. Akorát mi ještě není jasné, proč by měli věřit tobě, ministrovi Vděčné země.“

„Jejich důvěru si získám. Tenhle svět to trochu zabolí, ale bolest je ze všech podob zla tou nejlaskavější.“
