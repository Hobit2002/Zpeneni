\chapter{}
15. 4. 1989 Chu Jao-pang zemřel.  Smrt reformního komunisty byla pro mě, pro Mei i pro milióny dalších lidí po celé Číně zároveň smrtí naděje na změnu shora.

Začalo přituhovat. Tak jako mnoho jiných i já si byl vědom toho, že pokud režimu neřekneme: „Ne!“, může zase přijít totalitní maoistické peklo, ve kterém se po nás bude akorát chtít, abychom si krom vymytí mozků nechali sedřít kůži z těla ve firmách prominentních podnikatelů.

Když se na jaře roku 1989 začali Pekingu shromažďovat protestující studenti a dělníci, samozřejmě jsme u toho nechyběli. S malým dítětem jsme ale sotva mohli trávit na náměstí tolik času jako studenti, které dosud rodina nesvazovala, a tak byla naše účast všelijaká, z čehož jsem měl hodně špatný pocit.

Ne že bych začal litovat svého rozhodnutí založit rodinu, ale bohyně Mars nebyla spokojená s tím, jak málo toho pro dosažení svého snu o svobodě dělám, a dávala mi to skrze mé svědomí najevo.

Musel jsem pracovat. Má cesta vedla kolem náměstí Nebeského klidu, v jehož blízkosti jsem seděl za bankovní přepážkou. Vždy, když jsem procházel kolem studentských davů shromážděných na náměstí, píchla mě bohyně Mars do srdce a připomněla mi tak, že já, který jsem se rozhodl být režimu nepřítelem, sedím v jedné z jeho bank.

Neviděl jsem ovšem východiska. Jít místo práce protestovat, být vyhozen nejprve z ní a pak i z domu? To by mi asi Mei nikdy neodpustila (režim nesnášela, ale její antipatie měly svůj základ v přesvědčení, že ve svobodě by se mohla mít mnohem lépe) a kdo ví, jak by na to doplatilo naše dítě.

Můj vnitřní konflikt byl tím vyhrocenější, čím hlušší se komunistická vláda jevila. Cítil jsem, že je potřeba, abych se přidal a studentská masa už byla nepřehlédnutelná a neignorovatelná. 

Začal jsem se tedy demonstrací pravidelně účastnit alespoň každý večer po práci. Přicházel jsem kvůli tomu domů až pozdě, ale tuto míru aktivismu Mei stále ještě podporovala. Ani protesty mě však nenaplňovaly. Chyběla na nich totiž přítomnost bohyně Mars. Skoro všem demonstrantům totiž vadilo především to, že museli žít ve stálém strachu, zda neřekli či neudělali něco protirežimního, což bylo nepohodlné. Bohyně Mars však nežehnala lidské touze po pohodlí a naopak přála, těm, kteří se nebáli činit svůj život těžkým.

V roce 1989 jsem však bohyni Mars ještě neznal a nebyl schopen si takto shrnout své znechucení z demonstrací. Jaké štěstí, že jsem potkal proroka, který mé pocity vystihl slovy tak moudrými, že by se podle nich dalo vychovat dítě.

„Je to tu možná hezké a monumentální, ovšem zároveň zcela beznadějné. Jsme dav a dav je vždy přilepen k zemi. Všichni chceme přesně to samé – svobodu, ale chybí nám opravdu velké aspirace bez ní neuskutečnitelné. Dobré změny v dějinách, ať už to byla ekonomická reforma v téhle zemi, založení Indie nebo zpřístupnění počítačů obyčejným lidem, potřebují několik málo odhodlaných vizionářů, kteří touží po dosažení svého cíle a raději umřou, než aby se ho vzdali. Davy jako ten náš jsou vždy jen nástroj, samy o sobě způsobí leda tak chaos.

Jestli sem teď vrazí policisté a začnou nás mlátit, nezbude z našeho hnutí nic. Chceme pohodlnější život, a tak se, když zjistíme, že cesta k němu je velmi nepohodlná, vzdáme.  I já odejdu a už se nevrátím, leda, že by se do té doby objevil někdo výjimečný, to bych pak klidně nasadil život za to, aby dokonal, co dokonat chce.“

Pravdivost oněch slov byla otestována už následující týden.
	
Začátkem červa už byly protesty dva měsíce dlouhé a stařičký Teng Siao-pching, faktický vládce Číny, se rozhodl zajistit, že je přežije. Tisíce mladých lidí tuto možnost bohužel nedostaly.

Milionovou demonstrace rozehnala, či spíše rozstřílela armáda. 

Nechodil jsem v těch dnech do práce. Venku bylo nebezpečno a s tím, co se dělo, bychom stejně nic nenadělali.

Po třech dnech nám ovšem doma došly zásoby jídla a já se musel vydat na nákup. Do obchodu jsem přišel, obchodem jsem prošel a z obchodu jsem odešel jako vždy ale po cestě zpátky, když jsem přecházel přes silnici, uviděl jsem, jak ke mně jede kolona tanků.

Myslím, že tehdy ke mně bohyně Mars poprvé promluvila přímo. „Rozhodl ses jít do války se systémem. Doposud jsi však pro realizaci tohoto snu neudělal nic. Změň to! Máš příležitost, jaká se nebude opakovat! Dobře přeci víš, že komunistická pokřivenost vztahů je smrtící.“

S touto jí vnuknutou myšlenkou jsem se zastavil uprostřed silnice a počkal, až tanky přijedou blíž. 

Stalo se. To co následovalo, se zapsalo do historie naší rodiny i do dějin celého světa.  
	
Tanky přijely až ke mně a pak se zastavily. První se mě několikrát pokusil objet, ale já ho nenechal. 

Když řidič pochopil, že se s tou svou mašinkou může vrtět, jak chce a stejně mu to nepomůže, vzdal to. 

Neotálel jsem, vyšvihl se na stroj a pokusil se otevřít poklop, abych mohl trochu promluvit do duše jeho řidiči.

Povedlo se.

„Co jste to provedli? Vy jste měli být strážci, ne vrahy!“ zvolal jsem.

„My víme, a proto chceme co nejdřív opustit místo, kde jsme byli donuceni se vrahy stát,“ odpověděl voják. „A teď už prosím slezte a pusťte nás, jinak to s vámi špatně skončí a kvůli nám už špatně skončilo lidí až moc.“

Tak tohle jsem nečekal. „Dobře,“ řekl jsem a začal slézat „přečtěte si Konfucia, vojáku.“ rozloučil jsem se už definitivně.

„Přečtu,“ usmál se řidič.
	
 Seskočil jsem dolů a chvíli měl velmi dobrý pocit. Udělal jsem dobrou věc a potkal překvapivě dobrého člověka. Zrovna jsem se začal usmívat, když mi došlo, že k radosti není důvod.

Co na tom, že jsou lidé dobří, když z nich režim dělá vrahy? Ne, zatím se nic nevyřešilo. Znovu jsem skočil před rozjíždějící se tanky.

„Já myslel, že už jsme se domluvili?“ zalamentoval voják.

„Neberte si to osobně. Stojím jen před tankem, vy klidně vylezte a pokračujte pěšky,“ poradil jsem mu zcela vážně a tankista nejspíš usoudil, že jsem blázen.

„Mohu se přidat?“ ozvalo se náhle zezadu. Otočil jsem se a uviděl, jak ke mně zezadu na kole přijel moudrý demonstrant.

„Raději odvezte nákup mé ženě, já budu asi zatčen,“ požádal jsem ho.

A vskutku. Sotva jsem to dořekl, už mě zezadu popadli dva svalnatí pořízci a odtáhli mě pryč.

Tanky se zase rozjely. 