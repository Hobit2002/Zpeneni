\chapter{}

Jiří si vzal obrovskou půjčku a i s penězi vyrazil do blízkosti rusko-čínských hranic, zde si vydobyl obdiv u několika pologramotných a věčně podnapilých automechaniků, když s jejich pomocí převedl do praxe jeden ze svých nejodvážnějších plánů – stavbu zařízení, se kterým by se mu podařilo střeženou hranici přeletět.

Několik měsíců strávil v chudé a nudné vesničce, kam mu byly dováženy různé součástky a on je podle četných schémat, které si před cestou opatřil a nechal vysvětlit u svých přátel techniků, s pomocí již zmíněných zástupců ruského proletariátu, dával dohromady.

A přibližně v době, kdy upadl do té desetiny české populace, o níž se zajímali exekutoři, nechal se pozdě v noci vynést domorodcem řízenou helikoptérou přímo k rusko-čínské hranici. Když měli oba pocit, že výš a blíž už se nedostanou, vyskočil Jiří z vrtulníku a na vlastním rogalu plachtil nocí nikým neviděn mnoho kilometrů do čínského vnitrozemí.
Mohlo by se zdát samozřejmé, že když rogalo přistálo, už dobrodruha nikam dopravit nemohlo. 

Chyba lávky! Jiří měl v plánu se s jeho pomocí dopravit ze severovýchodní čínské divočiny přes Šanghaj až na samý jihovýchod, tedy oblasti Kantonu, Fo-šanu, Tung-kuanu a dalších bohatých a fakticky v jeden gigantický celek slitých velkoměst, kde tak jako ve všech velkoměstech, nebylo centimetru čtverečního, který by nemonitorovaly alespoň dvě kamery. Jiří si všechna rizika, která pro něj z moderních bezpečnostních opatření vyplývala, uvědomoval, a své rogalo konstruoval tak, aby mu pomohlo se s nimi vyrovnat. 

Jakmile se jeho nohy dotkly země, oddělil kovový rám od plachty, kterou následně nožem rozřízl na několik částí. Rozhodně se nejednalo o části náhodné, Jiří své rogalo vyrobil z pečlivě vybíraného fešného dámského oblečení.

Ještě v Rusku střihal blůzy, sukně i punčochy a sešíval je do plachty. Nyní udělal pravý opak. Plachtu rozřezal zpátky na jednotlivé kusy šatstva a ty pak jehlou a nití vrátil do stavu, ve kterém se daly nosit a nevypadaly nijak zle (tedy alespoň za tmy). 
 
Již brzy měl díky nim nápadníků víc než dost.
\vspace{0.75cm}

Jednoho večera si po cestě z města Beijixiang jeden bohatý čínský kapitalista, který se zrovna vracel ve svém luxusním nablýskaném voze z dovolené, všiml, jak se u silnice promenáduje spoře oděná a o to bohatěji nalíčená žena. Usoudil, že vášnivá noc by mohla jeho pobyt v severočínské divočině parádně zakončit, a tak u domnělé prostitutky zastavil, otevřel dveře a vyzval ji, ať vleze dovnitř.

Prostitutka se nijak neošívala, skočila do vozu, popadla boháče za ramena a vyhodila ho z jeho vlastního auta ven. Ještě než se přepadený zmohl na jakýkoliv vzdor, zabouchla dveře, šlápla na plyn a odjela s jeho autem do tmy.

Mladý šanghajský majitel několika továren na trnové koruny, růžence, levná kadidla a další cetky, které všelijací drobní podnikatelé po celém světě nabízeli zbožným křesťanským turistům, byl hluboce otřesen. Už mnohokrát ve svém životě si velmi živě představoval setkání s ďáblem a po těchto představách míval delší pocity hrůzy, žádný z nich si však nezadal s tím, co cítil nyní.

Žádný div, že tedy takřka vyjekl hrůzou, když si všiml, že mu ta lehká děva a tvrdá loupežnice ještě před odjezdem vrazila mezi nohy poměrně objemný balíček.  Okamžitě ho odhodil do příkopu a rozběhl se zpátky do města. 
\vspace{0.75cm}

Výrobce křesťanských cetek velmi prohloupil. Jiří by se nikdy nesnížil na úroveň sprostého zloděje aut, který za sebou nechává samé naštvané a zchudlé. Jeho zcela klíčovou zásadou bylo bohaté odškodnění všech postižených.

Balíček, který vyděšený továrník zahodil, obsahoval v bankovkách milion dolarů, který si Jiří před svou cestou půjčil od různých věřitelů. To ovšem nebylo všechno. 

Když Jiří vyrážel do Číny s plánem cestovat v autech, které v přestrojení za prostitutku ukradne, překypoval sebejistotou a byl skálopevně přesvědčený o tom, že se stane ostře stíhaným avšak nepolapitelným fantomem, který policii přiměje k tomu, aby vypsala obrovskou odměnu každému, kdo o něm zjistí aspoň nějaké informace. Proto k penězům přidal také kontakt na své rodiče a přátele s poznámkou: „Když na ně půjdete chytře, řeknou vám o mně vše.“	
\vspace{0.75cm}

V následujících dnech a týdnech se příběh za prostitutku převlečeného Čecha, který si byl dobře vědom toho, že jakmile ukradne auto, začne ho okamžitě hledat policie, a osamělého muže jedoucího nocí a toužícího po sexu mnohokrát zopakoval.
Vždy jednou za pár dní zaparkoval Jiří ukradené auto u nějakého výrazného objektu, převlékl a přelíčil se, vymáchal černou paruku ve vodě, čímž zajistil, že pustila trochu barvy a vyrazil k nejbližší benzínce, kasinu či jinému prostoru, kde setkání s nevěstkou nikoho nepřekvapilo.

Tam si počíhal na styku žádoucího chlapa jedoucího v autě o něco málo levnějším, než bylo to, kterým cestoval dosud a standardním způsobem mu jeho vůz zabavil. Aby však dotyčný nepřišel ke škodě, předal mu polohu a klíče auta dražšího, jímž cestoval dosud.

 S čerstvě ukradeným vozem pak několik hodin jel, aby se schoval ve stínu a tam předělal poznávací značku (z prvního ukradeného vozu dokonce seškrábal barvu a vyměnil logo výrobce, neboť předpokládal, že právě tento vůz bude hledán mimořádně důkladně).

Jiří byl na sebe i svůj plán pyšný. Cítil se jako člověk, který zvládá jednat ctně i úspěšně. Tyto emoce nebyly zcela opodstatněné. Již brzy se ukázalo, že spoustu věcí úplně nedomyslel. Jednou z největších chyb bylo, že si nevzal dost žiletek, a tak už do měsíce neměl množství make-upu potřebné k tomu, aby za světla nebylo vidět, že mu z měkoučké dámské pleti vyráží vousy. Prostitutka je ovšem bytostí noci, a tak se mu tato nedůslednost pro jiné z jeho chyb ani nestihla stát osudovou.

Možná naštěstí pro Jiřího ale velmi naneštěstí pro celé zbývající lidstvo u policie pracoval nejchytřejší tvor, jakého jsem kdy potkal a tento, duší spíše štír a démon než člověk, dostal možnost si případ loupežné prostitutky nastudovat.
\vspace{0.75cm}

Jednou večer Jiří jako obvykle stopoval na benzínce, když před ním zastavilo mimořádně drahé auto. K takovým situacím už párkrát došlo a Jiří musel vždy bohaté zájemce odmítnout, jelikož neměl nic, čím by je za krádež vozu mohl odškodnit. Tento byl ovšem zvlášť neodbytný.

„ Ihned nastupte, nebo si vrazím do prstu tuhle jehlu!“ přikázal mu řidič anglicky, jakmile otevřel dveře, ukazuje přitom na jednu ze tří jehel, které se povalovaly na jeho palubní desce.

Falešná prostitutka se zmateně zasmála a řidič pohotově vzal jednu z jehel a skutečně si ji do bříška prstu vrazil. Trochu přitom zkřivil tvář a ukázal šokovanému mladíkovi pramének krve stékající po jeho dlani. „Ihned nastupte, nebo si všechny tyto jehly vrazím do stehna!“

Jiří jen zděšeně zíral, ale do auta k tomuhle masochistovi se mu ani náhodou nechtělo. Sebetrýznič na jeho nečinnost zareagoval realizací svého slibu. Noha začala krvácet a obličej řidiče se zkřivil bolestí. Obchodní jednání tím ovšem neskončilo.

„Okamžitě naskočte, nebo se zastřelím!“ zvolal zájemce, vytáhl zpod sedadla pistol a přiložil si ji ke spánku. Český skaut měl jasno v tom, že ničí smrt zavinit nechce, a tak bez meškání naskočil a zabouchnul za sebou dveře.

„Nezklamal jste, chlape. Máte štěstí, že možná budu na vaší straně a na rozdíl od vás si nevytvářím žádné sebevražedné zásady. Bez mé pomoci by vám vaše vlastní ctnosti dříve či později zabránily v uskutečnění cíle mnohem hodnotnějšího než sebectnostnější lidský život,“ uvítal Jiřího řidič vesele, jen co se rozjeli a on si prohlédl vousy svého hosta. „Tak, a za chvíli zastavíme na nějakém místě, kde mě ošetříte, já jsem totiž na všechny praktické záležitosti hrozně nešikovný.“
\vspace{0.75cm}

A od oné noci už Jiří nekradl.

Dostal se do spárů Jieho, člověka, který obelhal a zradil všechny, s nimiž měl co dočinění.

Oné noci se provinil pouze proti Číně, která ho zaměstnávala jakožto policejního analytika a dopřávala mu takové důvěry (což o to, Jie byl bezpochyby mimořádně schopný), že mohl nahlížet do takřka všech materiálů, které měla čínská policie k dispozici.

Když se doslechl o prostitutce vyhazující chlapy z aut, nebylo pro něj těžké si nastudovat veškerou vypracovanou dokumentaci.

Netroufám si vykládat duševní pochody toho lotra, ale vycítil v Jiřím nějaký potenciál. Pravděpodobně potenciál být bombou.

„Co máte v plánu až vás v Šangahji vyložím?“ tázal se ho prý jednou.

„Svést nějakého policistu, opít ho ještě než by mohl stát o styk a pak od něj zjistit, v jakém vězení končí špióni a vlastizrádci. Následně si ono vězení vyhlédnout, najít do něj vedoucí vodovodní potrubí a vodu kontaminovat, to hned na několika místech, aby jedna oprava náhodou všechno nevyřešila. Až budou vězni odvážení pryč, tak se nějak převléknout za policistu najít Lydii, unést ji a v přístavu se se zásobou potravin schovat do dopravního kontejneru k nějakému zboží a nechat odvézt na západ.“

Jieho ten plán velmi pobavil, neboť koutky jeho úst se lehce nadzdvihly. „Je to vysoká hra a to se mi líbí, ale bohužel má své díry. Zboží posílané na západ je přeci jen trochu kontrolované a a policejní uniforma člověku přístup k nejostřeji hlídaným vězňům nezaručí. Krom toho se zapomínáte holit a zajímalo by mě, zda zvládnete mluvit tak, aby vás člověk považoval za čínskou ženu.“

„Jak si přeješ, miláčku“ pronesl Jiří vysokým hlasem jednu z frází, jíž se už před cestou naučil.

„Dobře! Naštěstí má čínština tolik dialektů, že ta slova můžete zkomolit sebevíc a nikomu to nepřijde podezřelé. Nicméně pokud Vám to nebude vadit, pomůžu Vám Vaši misi provést trochu bezpečněji.“

 Jiřího zaplavila vlna nadšení, získat na svou stranu policistu a to ještě navíc takhle vysoce postaveného! Kdyby byl věřící, strávil by děkovnou modlitbou klidně celý den, ale on, který byl v tomto ohledu typický Čech – agnostik – se ze svého štěstí pouze radoval. Jie mu však již brzy zchladil hlavu.
 
„Dnes v noci se budu muset účastnit porady na vojenské základně Xuèxīng de xīgài, Vás pochopitelně nechám v lese, pár desítek kilometrů na sever od ní. Zůstaňte u silnice a při zpáteční cestě vás vyzvednu.“
