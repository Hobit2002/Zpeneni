\chapter{}
Jie vyrazil do Číny, aby zde přetočil zde vývoj dějin, já s Fetuem jsme se pokusili o totéž.

Možná se čtenáři těchto řádků zdá, že chavezia ohrožovala jen pár lidí žijících poblíž teplých moří a rozhodně nestálo za to kvůli ní spouštět akce světového významu. Bohužel stálo, těch nebezpečí bylo o dost víc.

Výzkumníci zjistili, že nebezpečných breberek ve světovém oceánu rychle přibývá a podle jejich odhadu by se do roku 2030 při dosavadním tempu úbytku korálových útesů a oteplování moří, mohl stát oceán natolik nakaženým, že by archea  otrávila každou ryby vytaženou z vody a to tak moc, že by ani sebelepší tepelná úprava k detoxikaci masa nestačila.

Byla tu hrozba, že by nám moře mohlo přestat sloužit jako zdroj potravy a to by byla největší katastrofa v dějinách lidstva.

Aby se této pohromě předešlo, bylo by třeba oteplování oceánu zastavit, či alespoň maximálně zpomalit a to se nám zdálo nemožné.

Většina populace by musela přejít do módu uhlíkové neutrality. Takový počin by ovšem zároveň znamenal velký ekonomický krok zpět – tedy něco, k čemu se téměř žádný politik nedokázal odhodlat (a přitom něco, co leckterý z vlastní hlouposti udělal).

I tak jsem to s nimi zkusil a pozval Obamu, Merkelovou a několik dalších představitelů vyspělých zemí na soukromé jednání, kde jsem jim nebezpečí Chavezie představil. Ačkoliv ho vzali velmi vážně, končil jsem schůzi s velmi smíšenými pocity.

Zvláště americký prezident Barack Obama, ve mně vyvolal směs obdivu, výčitek a znechucení. Onen obdiv si vysloužil detailní znalostí americké politiky. Přesně věděl, co je nezbytné k dosažení které změny a vždy byl schopen rychle a rozumně vysvětlit, proč nic z toho, co by bylo potřeba, neprosadí. A právě tento přístup mě znechucoval. Věděl jsem, že Obama dobře chápe, jakou hrozbu chavezia představuje a moc rád by před ní ochránil nejen Spojené státy, ale celý svět, zároveň však nevěřil, že by republikáni byli schopni přitakat jakékoliv významnější ochraně životního prostředí, neb by se jim zdálo, že se tak děje na úkor průmyslu. Uměl jsem si docela dobře představit, jak moc bude strach z chavezie zkombinovaný s pocitem bezmoci amerického prezidenta mučit a bylo mi líto, že jsem ho odsoudil k tak špatným pocitům.

Docela odlišně zareagovala Angela Merkelová. Za celou schůzi toho sice moc neřekla, ale když už se setkání chýlilo ke konci, stačilo jí docela strohé prohlášení: „Zvládneme to. Svět uhlíkovou neutralitu potřebuje i z jiných důvodů než chavezie a myslím, že mé zemi a s ní i celé Evropské unii k jejímu dosažení pár desítek let postačí.“

Jinak ovšem politici, ať už se jednalo o představitele Francie, Kanady, Itálie nebo Velké Británie, nic obdivuhodného nepředvedli. Nikdo z nich nejen že se neodvážil tak jako Merkelová přislíbit alespoň nějakou snahu, ale ani neprojevil tak bystrou mysl jako Obama. Jednání jsem tedy končil s pocitem, že bez politické revoluce lidská civilizace nepřežije. 
 
Když jsem o jednání později mluvil s Fetuem, došli jsme k závěru, že musíme do čel mocných institucí, států a korporací, dosadit vůdce, kterým nebude vadit, že hodně lidí naštvou. Bez podnikavých šílenců už to dále nešlo. Bylo třeba najít politiky, podnikatele i manažery, jimž bude lhostejné, že kvůli nim lidé zchudnou a kteří budou dost charismatičtí na to, aby sobě svěřené přesvědčili, že začala nová éra, v níž se nelze držet starých hodnot pohodlí, rostoucího HDP a poživačného individualismu, ale že je nutné především všemi prostředky dostat svět z krize, kam jej tyto hodnoty dostaly.

Aby člověk ovlivnil svět tak moc, jak jsme ho chtěli ovlivnit my, potřeboval nástroj, kterým by mohl oslovit miliony lidí. My díky Jiemu a jeho softwaru na získávání zaměstnanců pro mé firmy ten nástroj měli.

A tak to začalo. Dali jsme se dohromady s výzkumným týmem Cambridge Analytica a upravili Jieho systém tak, aby pomocí analýzy sociálních sítí dokázal přesvědčit kteréhokoliv kyberaktivního člověka o čemkoliv.

Po pár měsících práce byl cíl dosažen. Mocná umělá inteligence začala vyhledávat všeliké světové volby, vyhodnocovat, kteří kandidáti jsou nejodvážnější a následně začít kampaň ovlivňovat v jejich prospěch.

Později jsem zjistil, že si umělá inteligence párkrát spletla svět byznysu se světem politiky, což vedlo například k tomu, že v jakési České republice velmi pomohla zcela neschopnému a už vůbec ne odvážnému zemědělskému magnátovi vyhrát parlamentní volby a stát se premiérem.

V podstatnějších případech jsme na software dohlíželi, a tak k podobným chybám nedocházelo. Není žádným tajemstvím, že jsme tímto způsobem vyvedli Británii z Evropské unie a dostali Donalda Trumpa do Bílého domu.

Hořce jsem litoval předčasné smrti Steva Jobse, byl jsem totiž přesvědčen, že on by se jakožto charismatický vůdce, dříč a notorický přehlížeč cizích pocitů do Bílého domu hodil nejvíc. Ale co naplat, ani nezlomná vůle mrtvé nevzkřísí, a tak nezbylo než zvolit plán b.

Donalda Trumpa už roky před jeho kandidaturou znal každý, já samozřejmě taky. Ač jsem nikdy nesledoval jeho televizní show, velmi jsem si ho vážil. Byl to muž podnikavý, odvážný, obdařený pevnou vůlí a schopností prosadit si právě to, co chce. Looseři a jimi ovlivnění lidé tyto vlastnosti mnohdy neoceňují, neb je lidem úspěšným nevědomě závidí, ale já nebyl looser, a to i přesto, že mě život nesčetněkrát srazil na kolena.

Právě v tom jsem si byl s Donaldem Trumpem blízký. Když v roce 2004 vyhlásil osobní bankrot, neb byl ve stomilionových dluzích, pozval jsem ho na návštěvu do svého sídla. Strávili jsme den jednáním a navečer byli domluvení, že v Isimangalisu budou do roka postaveny dvě desítky luxusních podniků pro ubytovávání a bavení obchodních a politických hostů. Po této zakázce jsme se do sebe zamilovali. Opravdu! V následujících letech jsem od něj obdržel několik krásných dopisů (a byly to opravdu skvělé dopisy), mnohem důležitějším však byl pro náš vztah rok 2009, kdy Donald zase zbankrotoval a starý příběh se zopakoval.

Napotřetí jsem si ho pozval začátkem roku 2015. I tentokrát jsem mu chtěl pomoct v cestě za jeho sny.

„Nechtěl bys být americkým prezidentem, Donalde?“ zeptal jsem se ho.

„V celé historii se na to nikdo nehodil víc než já,“ odvětil můj přítel, kterého jsem shledal za nejlepší možnou Jobsovu náhradu.

„Myslím, že mám technologii, která ti s tím pomůže,“

„Jen se neboj mi rovnou říct, co chceš na oplátku,“

„Uhlíkovou neutralitu Spojených států,“

„Ty věříš vědcům? Uhlíková neutralita by měla zmírnit klimatickou změnu, ale proč o ni usilovat, když žádná klimatická změna není? Ano, občas je někde sucho a jinde hoří, ale nemyslím si, že věda ví proč. Já se zaměřím hlavně na to, aby Amerika ukázala, že jsou její lokty stále ještě ostré a spálila zbytečné vazby, které utvořila“ odbyl mě Trump a stočil konverzaci od moci k obvyklému tématu peněz.

Hovor plynul a plynul, až byl přerušen obědem. Tuto pauzu, jsem využil, abych se poradil s Fetuem, zda má smysl Trumpa v prezidentské kandidatuře podpořit, i když popírá klimatickou změnu a chystá se hrát na notu amerického šovinismu.

Fetu vzácně odpověděl bez dlouhého přemýšlení: „Když jsem před měsícem navštívil Británii, abych zde pokřtil svou knihu ‚Bláznovstvím k dominanci‘, potkal mě zde jeden mimořádně moudrý muž.

Stalo se to na samém konci mé cesty. Rozhodl jsem se věnovat den prohlídce Národní galerie a právě tam mě dotyčný zastihl.

‚Už jsem Vás přečetl,‘ oslovil mě ‚a rád bych Vás upozornil na několik chybných úvah, jichž jste se dopustil‘

‚Budu rád,‘ odpověděl jsem.

‚Co fungovalo v Africe, těžko bude fungovat v Evropě. Zdejší přebyrokratizovaný režim si nás rozmazlil. Domorodec znající tvrdost divočiny ví, co je odvaha a co je dobro. Když se ocitne v krizi, vzmůže se k obojímu.

Současný Evropan je na tom o dost hůř, pohodlný byrokratický režim z nás snímá zodpovědnost a my degenerujeme. Nejsme už způsobilí k velkým činům.‘

‚A co myslíte, že by vás k nim mohlo přivést?‘

‚Velký návrat osobní zodpovědnosti,‘ odpověděl můj kritik pohotově ‚Británie musí například odejít z Evropské unie, té zrůdné instituce, která vede lidi po celé Evropě k slepému spoléhání na Německo.‘

‚Svět má ovšem velké problémy, pane…‘

‚Cummingsi.‘

‚…pane Cummingsi. Hrozí ekologická katastrofa a lidé samotní se neodhodlají proti ní něco udělat.‘

‚Teď ne, ale až se naučí převzít za sebe zodpovědnost, vrhnou se pro záchranu svých rodin i do pekel.‘

Tím náš hovor skončil a musím s tím, co Dominic Cummings říkal souhlasit. Výchova lidí musí předcházet záchranu světa. Amerika potřebuje kapitalistického republikánského prezidenta, který lidi naučí, aby se spoléhali především sami na sebe, popřípadě na komunity, které však budou aktivně budovat. Jen nepohodlí malého státu lidi donutí, aby pocítili skutečnou zodpovědnost za to, na čem jim záleží.“

Bylo mi potěšením udělat ze svého dlouholetého přítele prezidenta Spojených států. Ne že by jeho volební vítězství byla zásluha pouze mě a vylepšeného Jieho systému, Donald udělal pro své zvolení také maximum. Nechal si pomoct od Rusů a v průběhu kampaně použil proti svým kandidátům každý argument, který ho napadl. Od dob války ve Vietnamu Amerika nezažila agresivnější volby.

Prezidentské volby 2016 však nebyly jediným demokratickým procesem, do nějž jsme se vložili. Naším druhým velkým rozhodnutím byla podpora britského brexitového hnutí.

Fetu mě nejprve seznámil s Cummingsem a Johnsonem, abych se ujistil, že pokud chci u moci odvážné lidi, pak jsou zcela ideálními adepty. Ani jeden sice na ochranu životního prostředí moc nedal, ale bylo jasné, že pod jejich rukama se Britové brzy naučí postarat o sebe samé, neboť stát toho pro ně udělá jen minimum.  

Nutno přiznat, že referendum se nám trochu vymklo z rukou. Spousta nenávisti vedoucí dokonce až k vraždě političky… Některé věci ovšem dodavatelé softwaru ovlivní jen těžko.
Každopádně jsme i tady uspěli a uvrhli Británii do velmi výchovného chaosu, završeného Johnsonovým dosazením do premiérského křesla. Spojené království se učilo být připraveno na všechno.

Našimi dalšími počiny byla také například předvolební podpora Emmanuela Macrona (Ze všech politiků, kterým jsme pomohli k moci, byl právě on tím nejlepším. Nejen že mu nechyběla kuráž a vůle, nýbrž měl také na rozdíl od Trumpa a Johnsona rozum), či obrovské investice do mimořádně ambiciózního podnikání nejstaršího syna mého bývalého investora Errola.
Zkrátka je jistě vidět, že jsme pro zocelení a zpružnění lidstva udělali maximum. A ono to stále nebylo dost! 

Psal se rok 2019 a rozhodně se nedalo říct, že by v něm bohyně Schod působila více než dřív. To by totiž překypoval silou, energií a odhodláním, on se však dusil v trpkosti, rozpadu a neveselých klimatických i politických vyhlídkách. 

„Co jsme to provedli?“ zalamentoval jsem jednoho večera před Fetuem. „Trump, Johnson a další měli svět povzbudit a oni ho místo toho přiotrávili! Aspoň, že Macron se chová docela obstojně.“

„Nedivím se, od tvého kamaráda jsem si nic významného nesliboval, ale svůj účel splnil. Lidé jsou otrávenější než kdy dřív. Pokud od druhé světové války opravdu někdy nedůvěřovali těm nad sebou, pak je to nyní. A čím méně důvěřují mocným, tím spíš sami vezmou věci do vlastních rukou,“ odvětil Fetu.
