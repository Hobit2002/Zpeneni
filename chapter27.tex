\chapter{}

„Víš, co je to zmrzlina?“

„Ne.“

„Sladká pochoutka, která nejen že většinou výtečně chutná, nýbrž také chladí.“

„To může být zajímavé.“

„To si piš. Stejně jako sociální sítě, metro, knihkupectví, eskalátory, mrakodrapy, autonomní auta a další věci, které zatím vůbec neznáš.“

„Proč mi to říkáš?“

„Protože ti chci dát šanci se s tím vším setkat. Jie tě sice pověřil, abys mě hlídal, ale zároveň ti dal svobodu. Víš, co to znamená? Že mě můžeš doprovodit do velkého světa.“

Na ta slovu pevný stisk ruky mého strážníka povolil a já se mohl v rámci možností pohodlně uvelebit v korbě jednoho z náklaďáků, které do vesničky každé tři dny přivážely jídlo, pití, toaletní papír a další užitečné věci.

Vlezl jsem do ní (po domluvě s řidičem) očekávaje, že mě odveze pryč z mnohogenní osady buď přímo za Jiem nebo někým, kdo bude chápat, co exministr financí dělá. S sebou jsem si vzal jen Gangův policejní průkaz. Předpokládal jsem, že pokud bych před sebou měl cestu do Číny (což by mě vůbec nepřekvapilo) mohl by mi přijít vhod (pokud bych si tedy nechal narůst pořádné vousy a nebylo poznat, že mé tváře vypadají jinak než ty Gangovy).

Sotva jsem nasednul, vrhnul se na mě voják pověřený mým dozorem a chytnul mě za nohu. Naštěstí stačila výše popsaná konverzace a zase jsem byl volný.

Mohutný a silný černoch se posadil vedle mě. „Měl bych si ověřit, že to co dělám je správné,“ prohlásil nervózně.

„Já pojedu za Jiem, takže se ho budeš moct zeptat,“ odpověděl jsem mu a pak se auto rozjelo. Cestu jsem strávil vyptáváním na Wobucikovo (tak jsem si svého druha až doposud nazývaného výhradně jako 1004R99 pojmenoval) dětství a dospívání ve vojenské základně.

Už jen sama skutečnost, že jsem byl první, který mu říkal jménem a nikoliv výrobním číslem, svědčila o tom, že daň za vylepšené geny byla opravdu vysoká.

Můj spolucestující si nikdy nehrál, nikdy nikomu neřekl „ne“, nikdy v reálném životě nepoznal vztah muže a ženy… pedagogická katastrofa.

Ještě než jsem se dostal k jeho životním plánům (ač se těžko dalo očekávat, že by nějaké měl), auto zastavilo a my museli vystoupit.

Ocitli jsme se uprostřed města, a kam jsme se otočili, tam se tyčily nějaké mrakodrapy. Ulice byli plné lidí všech barev pleti. Řidič vylezl také, zamknul vůz a odešel na oběd do McDonaldu.

Wobuciko byl z toho všeho naprosto zmatený a jelikož ho nikdy nikdo nevedl byť k sebemenší asertivitě, jen se rozhlížel kolem sebe. I já se cítil trochu bezradně, doufal jsem, že mě náklaďáky dovezou do jednoho odlehlého komplexu nějaké tajné služby, ale tohle mě překvapilo. 

Zkusil jsem tedy štěstí ve skladišti, které se nacházelo přímo u parkoviště. Kousek jsem popošel, našel oficielní vchod a za chvíli už byl vyhazován vrátným. Nebránil jsem se.„Opravdu nechci dělat potíže, ale komu tahle budova patří?“ zeptal jsem se ho akorát. 

„Otci celého tohoto města. Muží, bez jehož peněz by tu nestála jediná budova,“

„Jieho?“

„Jakého Jieho?“

„Tak koho?“ na konci onoho rozhovoru jsem získal adresu vily, ve které měl onen dobrodinec bydlet. Okamžitě jsem tam vyrazil, Wobuciko mi byl v patách.

Celé město se rozprostíralo okolo již vytěženého smaragdového dolu, v jehož bezprostřední blízkosti se nikdo z bezpečnostních důvodů neodvážil stavět domy, a tak se zde nacházela úzká elipsa divočiny. Vila legendárního miliardáře se nacházela hned vedle tohoto pruhu, a když jsem si na svém mobilu prohlédl mapy, usoudil jsem, že nejkratší cesta od skladiště k vile povede právě přes houštinu.

„Ruce vzhůru!“ ozvalo se nám za zády, když nás od cíle dělilo posledních tři sta metrů.

Poslechl jsem a Wobuciko taky. Z křoví vylezli dva agenti, muž a žena. Již brzy se ukázalo, že jsou to Rusové.

„Co tu děláte?“ otázali se nás.

Pravdivě jsem jim odpověděl.
	
 „Bránit vám v tom nejspíš nebudeme. Záleží, co nám řeknete. Chceme znát všechny informace, které se týkají osob nějak spřízněných s Jiem…“

„Právě na toho se jdu zeptat!“ zvolal jsem. „Tenhle filantrop totiž zásobuje vesničku demobilizovaných vojáků, kterou Jie založil.“

„Řekněte nám vše, co o ní víte. Jinak vás nepustíme dál,“ Rusové na to.

Vyhověl jsem jim, ale bohužel to nijak nepomohlo. „Takže vrchní představitelé Vděčné země? Nezlobte se, ale to vás opravdu nemůžeme pustit dál.“ Na to slova vytáhli želízka a provazy, aby nás spoutali.

„Ne abyste kladli odpor, jsme ozbro…“ ani výstrahu nestihli doříct a Wobuciko už oběma sevřel hrdla, zvedl je do vzduchu (každého jednou rukou) a pak s nimi třískal o sebe, dokud nepustili zbraně. Chvíli trvalo, než pochopili, co se od nich očekává, a tak ze začátku dokonce párkrát vystřelili, jejich rány však až na jednu výjimku skončily v okolních stromech. Ona výjimka však zasáhla Wobucika do boku.

„Co s nimi mám dělat?“ otázal se mě můj spolucestovatel, když byl s pacifikací hotov.

„Zatím nic, raději nějak nalož se svým zraněním,“ požádal jsem ho.

Wobuciko tedy položil Rusy na zem, sednul si na ně a pak si vyhrnul triko. Užasl jsem. Jeho bok nebyl tvořen normálním masem, nýbrž pásem tlusté a zrohovatělé pokožky. Očividně měl ochranou funkci a plnil ji dobře. Kulka díky němu nezasáhla žádné orgány a Wobuciko si ji do minuty zcela bezbolestně vytáhl pinzetou.  

Jen co byla kulka venku, přišel čas věnovat se Rusům úpícím pod jeho mohutným tělem.

„Chápu, že se vám naše jednání s vámi nelíbilo, ale prosím nemstěte se, dělali jsme to pro světový mír,“ začala nás prosit přimáčknutá agentka.

„Jak náš zájem o Jieho ohrožoval světový mír?“

„Návštěva ho ohrozit nemusela, ale Vděčná země možná a Jie bezpochyby ano. Jako její představitelé byste o tom mohli něco vědět. A proto jsme vás chtěli vyzpovídat.“

„Nenapadá mě nic, čím by má nová vlast, mohla být nebezpečná.“

„Už je to nějaký čas, co s námi váš ministr financí, tehdy vládní zmocněnec pro budování Systému sociálního kreditu, komunikoval ohledně zapojení Rusů do tohoto projektu. Právě od tohoto jednání ho sledujeme a víme, že se několik dní před tragickou smrtí Vladimíra Putina setkal s pilotem, který jej tehdy zradil, a že se toto setkání zopakovalo tak brzy po atentátu, jak jen to bylo možné. Značně ho tedy podezříváme, že měl v celé události prsty. A možná nejen on, nýbrž celá vláda Vděčné země. A pak je tu ještě jedno video zachycené špionážním dronem.“

„Jaké?“

„Jie se v něm s prezidentem Si Ťing-pchingem domlouvá na tom, že když položí dva osly na oltář, dostane prostředky k tomu, aby za pomocí biologických zbraní provedl holocaust.“

Vykulil jsem oči, tohle pro mě byla naprostá novinka. „Proč by Vděčná země usilovala o nějaký holocaust? Zvláště pak holocaust za cenu Putinova života? Vždyť nás chránil, neměli jsme důvod jej likvidovat,“ podivil jsem se.

„Upřímně, vůbec nevíme, o co vám v té Vděčné zemi jde. Minimálně ministr Jie, který má na Vděčnou zemi pomalu určující vliv, se zdá být hráčem své vlastní hry…“ vysvětlila Ruska.

„A vy si myslíte, že o tom ten miliardář ví něco víc?“ otázal jsem se jich.

„Dost možná. Je nám známo, že se těsně před šanghajskou revolucí setkal s Jiem.“

Tím bylo rozhodnuto. Řekl jsem Wobucikovi, aby Rusy přivázal ke stromu a sám vyrazil kupředu směrem k vile.

Ještě než jsem tam dorazil, můj souputník mě dohnal. V podpaždí si nesl Lydii. „Přišla sem asi o dvě hodiny dřív než my. Neměla ovšem ochranku, takže ji chytli, svázali a schovali. Slyšel jsem, jak se sto metrů od nás převaluje v křoví, tak jsem ji osvobodil,“ dodal na vysvětlenou.

Rozhodně jsem mu jeho rozhodnutí nezazlíval, akorát mě zajímal příběh, který Lydiino chycení předcházel. Odpověděla mi, ještě než jsme zazvonili na miliardářskou vilu. 

Jieho prý sledovala ve dne v noci. Ocitla se uprostřed dějin a to navíc dějin nesmírně dramatických a tedy i na vrcholu své novinářské kariéry. Nehodlala si tedy nechat ujít byť jen jediný čin jednoho z hlavních aktérů. Došla až tak daleko, že do Jieho chatky instalovala odposlouchávací zařízení (vzala si ho s sebou do Afriky, aby mohla odposlouchávat hovory přívrženců komunismu), když mu tedy ve tři ráno zazvonil budík, vzbudila se i ona a stihla vyběhnout dostatečně včas, aby viděla, kterak ten hadiplž autem odjíždí kamsi pryč.  

Rychle se oblékla a vylezla ven. Věděla, že celý areál je přes noc hlídaný (a to právě proto, aby neutekla). Několik strážných, kteří po vesničce patrolovali, se navíc mohli pyšnit výborným sluchem, zrakem i čichem. A právě jejich skvělý čich novinářku zachránil.

Jen co opustila svou chatku, zamířila do přístřeší, ve které klimbal její denní hlídač. Rychle si sundala své šaty a místo nich si nasadila jeho kalhoty, boty a triko. Na rozloučenou pak na jeho práh vylila polovinu jedné ze svých voňavek.

A pak vyrazila. Dávala si velký pozor, aby ji žádný hlídač neviděl. Zato ovšem důkladně dbala na to, aby byla slyšet. Hlídky o ní samozřejmě věděly, vůbec však netušily, že je to ona. Její kroky zněly jako ty jejich, její oblečení už na dálku čpělo nikterak podezřelým zápachem mnohogenních a naopak její vlastní pach vycházel z jedné z chatek. Nikoho tak ani nenapadlo, že by za ní přišel a prohlédl si ji.

Lydie takto opustila tábor a dostala se na betonovou cestu, po které vyrazila směrem z pralesa. Když se rozednělo, uslyšela již z dálky hučení motorů. Přišel čas na taxík. Sáhla tedy do kapsy, vytáhla poloprázdnou voňavku a vypila, co v ní zbývalo.

Okamžitě se jí udělalo velmi zle. Vypjala však všechnu svou vůli a nedovolila si zvracet. Její záchrana nyní byla v rukách lékaře. 

Když kolem začali projíždět první řidiči, všimli si, jak leží u cesty a prosebně mává. Když jí konečně jeden zastavil a ptal se, co se děje, řekla mu pravdu. „Přiotrávila jsem se a potřebuji vypumpovat žaludek.“

Řidiči se to zdálo zvláštní, ale její přání vyslyšel. Naložil ji, otočil se a ještě než se Slunce přehouplo přes nadhlavník, vycházela Lydie z nemocnice v Isimangalisu. Okamžitě zamířila do parku, kde ji ovšem chytli a svázali Rusové.

Díky Wobucikovi jsme však nyní mohli úspěšně zazvonit na dveře vily a oba čekat na majitele domu. Naštěstí toto čekání nebylo dlouhé, již po chvíli se dveře samy otevřely a my mohli vejít.

Uvítal nás docela obyčejně vyhlížející Číňan. Když jsme se ho zeptali, zda něco tuší o vztahu mezi Jiem a Putinem, odpověděl, že nevím o ničem zvláštním, ale že měli určitě podobnou duši.

„Čím?“ otázala se ihned Lydie.

„Smrtelně nebezpečným šílenstvím,“ odpověděl miliardář s lehkým smutným úsměvem.

„Jak to myslíte?“

„Oběma je přirozené usilovat o vlastní zánik, což velmi cením.“ 

Víc už nám neřekl, snažili jsme se z něj dostat víc, ale odpovídal jen ve velmi podobně znějících hádankách. Jistě, mohli jsme se víc snažit, ale nebyl čas. Okamžitě jsme vyrazili na letiště do Livingstonu, abychom chytli první let do Vděčné země.
